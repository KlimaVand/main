\typeout{- START ----------------------------------- Preamble.tex --}
\documentclass[11pt,twocolumn]{article}
\usepackage[english]{babel}
\usepackage{verbatim}
\usepackage{t1enc}
\usepackage[xdvi]{epsfig}
\usepackage[xdvi]{graphicx}
\usepackage{latexsym}
\usepackage{lastpage}
\usepackage{longtable}
\usepackage{tabularx}
\usepackage{theorem}
\usepackage{verbatim}
\usepackage{varioref}
\usepackage{amsmath}
\usepackage{amssymb}
\usepackage{alltt}
\usepackage{array}
\usepackage{longtable}
\usepackage{lscape}
\usepackage{rotfloat}
\usepackage{pdfpages}
\usepackage[plainpages=false]{hyperref}
%\usepackage{wrapfig}
%\usepackage{xcolor}
\usepackage{listings}
%\usepackage{a4page}
\usepackage{mdwlist}
%\usepackage{appendix}
\usepackage{colortbl}

\lstloadlanguages{SQL,PHP,bash}

\lstdefinelanguage{CSharp}
{
 morecomment = [l]{//},
 morecomment = [l]{///},
 morecomment = [s]{/*}{*/},
 morestring=[b]",
 sensitive = true,
 morekeywords = {abstract,  event,  new,  struct,
   as,  explicit,  null,  switch,
   base,  extern,  object,  this,
   bool,  false,  operator,  throw,
   break,  finally,  out,  true,
   byte,  fixed,  override,  try,
   case,  float,  params,  typeof,
   catch,  for,  private,  uint,
   char,  foreach,  protected,  ulong,
   checked,  goto,  public,  unchecked,
   class,  if,  readonly,  unsafe,
   const,  implicit,  ref,  ushort,
   continue,  in,  return,  using,
   decimal,  int,  sbyte,  virtual,
   default,  interface,  sealed,  volatile,
   delegate,  internal,  short,  void,
   do,  is,  sizeof,  while,
   double,  lock,  stackalloc,
   else,  long,  static,
   enum,  namespace,  string,
   RemoteCall, Position, Sample, 
   CommandType, SqlDataReader, commandtext,
   Parse, ToString, SqlCommand,
   SqlConnection}
}

\lstdefinelanguage{CSharpPlus}
{
 morecomment = [l]{//},
 morecomment = [l]{///},
 morecomment = [s]{/*}{*/},
 morestring=[b]",
 sensitive = true,
 morekeywords = {abstract,  event,  new,  struct,
   as,  explicit,  null,  switch,
   base,  extern,  object,  this,
   bool,  false,  operator,  throw,
   break,  finally,  out,  true,
   byte,  fixed,  override,  try,
   case,  float,  params,  typeof,
   catch,  for,  private,  uint,
   char,  foreach,  protected,  ulong,
   checked,  goto,  public,  unchecked,
   class,  if,  readonly,  unsafe,
   const,  implicit,  ref,  ushort,
   continue,  in,  return,  using,
   decimal,  int,  sbyte,  virtual,
   default,  interface,  sealed,  volatile,
   delegate,  internal,  short,  void,
   do,  is,  sizeof,  while,
   double,  lock,  stackalloc,
   else,  long,  static,
   enum,  namespace,  string, RemoteCall, Position, Sample, CommandType}
}
\lstset{language=CSharp}
\lstset{numbers=left}
\lstset{basicstyle=\ttfamily\scriptsize}
\lstset{tabsize=3}
\lstset{stringstyle=\ttfamily}
\lstset{showstringspaces=false}
\lstset{breaklines=true}
\lstset{captionpos=b}
\lstset{frame=ltrb}

\usepackage{epic}
\usepackage{eepic}
\usepackage{url}
\usepackage{multicol}
\usepackage{fancyhdr}
\usepackage{float}
\usepackage{times}
\usepackage{makeidx}
%\usepackage{algpseudocode}
%\usepackage{algorithmicx}
%\usepackage{algorithm}
%\usepackage[dvips]{color}
\usepackage{rotating}
\usepackage{pst-node}
\usepackage{pst-coil}
\usepackage[Lenny]{fncychap}
\usepackage{ifthen}
\usepackage{supertabular}

\newenvironment{pseudo}
  {
    \footnotesize	% 10pt
    \tt
    \baselineskip 10pt
    \parskip8pt
    \vspace\smallskipamount
    \begin{alltt}
    \frenchspacing
    \obeylines
  }{
    \par
    \end{alltt}
  }

\usepackage{color}

%% Control the fonts and formatting used in the table of contents.
\usepackage[titles]{tocloft}
%% Use Helvetica-Narrow Bold for Chapter entries
%\renewcommand{\cftchapfont}{%
%  \fontsize{11}{13}\usefont{OT1}{phv}{bc}{n}\selectfont}

\usepackage{relsize} %% Allow relative font size specifications (e.g. \smaller, \larger).

%% Make sure that the bibliography is listed in the table of contents,
%% but that the table of contents itself is not.
\usepackage[nottoc]{tocbibind}

\oddsidemargin 	=    0.0cm
\evensidemargin = 	 0.0cm
\parskip        =    1em
\parindent      =    0em %Determines how far the first line of a paragraph is indented.
\baselineskip   =    3ex

\renewcommand{\topfraction}{.85}
\renewcommand{\bottomfraction}{.7}
\renewcommand{\textfraction}{.15}
\renewcommand{\floatpagefraction}{.66}
\renewcommand{\dbltopfraction}{.66}
\renewcommand{\dblfloatpagefraction}{.66}
\bibliographystyle{alpha}

%\pagestyle{fancy}
%\renewcommand{\chaptermark}[1]{\markboth{#1}{}}
%\renewcommand{\sectionmark}[1]{\markright{\thesection\ #1}}
%\fancyhf{} \fancyhead[LE,RO]{\bfseries\thepage}
%\fancyhead[LO]{\bfseries\rightmark}
%\fancyhead[RE]{\bfseries\leftmark}
%\renewcommand{\headrulewidth}{0.5pt}
%\renewcommand{\footrulewidth}{0pt}
%\addtolength{\headheight}{2.5pt}
%\fancypagestyle{plain}{\fancyhead{}\renewcommand{\headrulewidth}{0pt}}

\typeout{- SLUT ----------------------------------- Preamble.tex --}

\textwidth 6in
\setcounter{tocdepth}{2} 
