%% Based on a TeXnicCenter-Template by Tino Weinkauf.
%%%%%%%%%%%%%%%%%%%%%%%%%%%%%%%%%%%%%%%%%%%%%%%%%%%%%%%%%%%%%

%%%%%%%%%%%%%%%%%%%%%%%%%%%%%%%%%%%%%%%%%%%%%%%%%%%%%%%%%%%%%
%% HEADER
%%%%%%%%%%%%%%%%%%%%%%%%%%%%%%%%%%%%%%%%%%%%%%%%%%%%%%%%%%%%%
\documentclass[a4paper,twoside,10pt]{report}
% Alternative Options:
%	Paper Size: a4paper / a5paper / b5paper / letterpaper / legalpaper / executivepaper
% Duplex: oneside / twoside
% Base Font Size: 10pt / 11pt / 12pt


%% Language %%%%%%%%%%%%%%%%%%%%%%%%%%%%%%%%%%%%%%%%%%%%%%%%%
\usepackage[USenglish]{babel} %francais, polish, spanish, ...
\usepackage[T1]{fontenc}
\usepackage[ansinew]{inputenc}

\usepackage{lmodern} %Type1-font for non-english texts and characters


%% Packages for Graphics & Figures %%%%%%%%%%%%%%%%%%%%%%%%%%
\usepackage{graphicx} %%For loading graphic files
%\usepackage{subfig} %%Subfigures inside a figure
%\usepackage{pst-all} %%PSTricks - not useable with pdfLaTeX

%% Please note:
%% Images can be included using \includegraphics{Dateiname}
%% resp. using the dialog in the Insert menu.
%% 
%% The mode "LaTeX => PDF" allows the following formats:
%%   .jpg  .png  .pdf  .mps
%% 
%% The modes "LaTeX => DVI", "LaTeX => PS" und "LaTeX => PS => PDF"
%% allow the following formats:
%%   .eps  .ps  .bmp  .pict  .pntg


%% Math Packages %%%%%%%%%%%%%%%%%%%%%%%%%%%%%%%%%%%%%%%%%%%%
\usepackage{amsmath}
\usepackage{amsthm}
\usepackage{amsfonts}


%% Line Spacing %%%%%%%%%%%%%%%%%%%%%%%%%%%%%%%%%%%%%%%%%%%%%
%\usepackage{setspace}
%\singlespacing        %% 1-spacing (default)
%\onehalfspacing       %% 1,5-spacing
%\doublespacing        %% 2-spacing


%% Other Packages %%%%%%%%%%%%%%%%%%%%%%%%%%%%%%%%%%%%%%%%%%%
%\usepackage{a4wide} %%Smaller margins = more text per page.
%\usepackage{fancyhdr} %%Fancy headings
%\usepackage{longtable} %%For tables, that exceed one page


%%%%%%%%%%%%%%%%%%%%%%%%%%%%%%%%%%%%%%%%%%%%%%%%%%%%%%%%%%%%%
%% Remarks
%%%%%%%%%%%%%%%%%%%%%%%%%%%%%%%%%%%%%%%%%%%%%%%%%%%%%%%%%%%%%
%
% TODO:
% 1. Edit the used packages and their options (see above).
% 2. If you want, add a BibTeX-File to the project
%    (e.g., 'literature.bib').
% 3. Happy TeXing!
%
%%%%%%%%%%%%%%%%%%%%%%%%%%%%%%%%%%%%%%%%%%%%%%%%%%%%%%%%%%%%%

%%%%%%%%%%%%%%%%%%%%%%%%%%%%%%%%%%%%%%%%%%%%%%%%%%%%%%%%%%%%%
%% Options / Modifications
%%%%%%%%%%%%%%%%%%%%%%%%%%%%%%%%%%%%%%%%%%%%%%%%%%%%%%%%%%%%%

%\input{options} %You need a file 'options.tex' for this
%% ==> TeXnicCenter supplies some possible option files
%% ==> with its templates (File | New from Template...).



%%%%%%%%%%%%%%%%%%%%%%%%%%%%%%%%%%%%%%%%%%%%%%%%%%%%%%%%%%%%%
%% DOCUMENT
%%%%%%%%%%%%%%%%%%%%%%%%%%%%%%%%%%%%%%%%%%%%%%%%%%%%%%%%%%%%%
\begin{document}

\pagestyle{empty} %No headings for the first pages.





\cleardoublepage %The first chapter should start on an odd page.

\pagestyle{plain} %Now display headings: headings / fancy / ...



%% Chapters %%%%%%%%%%%%%%%%%%%%%%%%%%%%%%%%%%%%%%%%%%%%%%%%%
%% ==> Write your text here or include other files.

%\input{intro} %You need a file 'intro.tex' for this.


%%%%%%%%%%%%%%%%%%%%%%%%%%%%%%%%%%%%%%%%%%%%%%%%%%%%%%%%%%%%%
%% ==> Some hints are following:



\section{Include phosphor}\label{umlauts}
This small program takes phosphor from FeedTable2004\_modDC\_NJH.txt and replace it in feedItems-file.
The path to feedItems-file should be given as an argument and FeedTable2004\_modDC\_NJH.txt should be next to the executable .



\section{FarmToTabel}\label{references}
This program generate a table overview based on farm files that is written in farms.txt.


\section{XMLdataToTabXLS}\label{dividing}
Writing a table with name, Description, Units and Value. The program takes 2 arguments (input- and output-files)

\section{testProgram}\label{dividing}
Compare 2 xml files to see if the content matches. There is a setup file called input.txt where the file-name should be in.
\section{ChangeInformationToXML}\label{dividing}
This program replaces a single value in an xml document. It does take 4 arguments: name of file, path down the xml-tree, name of item that should be replaced and the new value.
\section{puttingInformatinIntoIntoAxmlFile}\label{dividing}
This program adds an existing xml-tree to another xml-tree. It does take 3 arguments: path to file where then new content should be added to, Path to where the content should be added, path to a file that contains the new content.
\section{addingColour}\label{dividing}
This program adds attribute from xml tree in another tree. The program does take 2 arguments: The first argument is the destiny of the changes. The second arguments are the source of the changes.
\end{document}

