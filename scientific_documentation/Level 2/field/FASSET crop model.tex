\documentclass[%parskiphalf,%numbers=noendperiod
]{scrartcl}
\usepackage{amsmath}
\usepackage[latin1]{inputenc}
\usepackage[english]{babel}
\usepackage[T1]{fontenc} 
\usepackage{graphicx,parskip}
\usepackage{booktabs,longtable}
\usepackage{lmodern}
\usepackage[round]{natbib}

\newcommand\mymarginpar[1]{\marginpar {\flushleft\bfseries\scriptsize #1}}

\title{FASSET Crop Model}

\author{}
\date{\today}

\usepackage{lmodern}


\usepackage[backref]{hyperref}

\begin{document}

\maketitle
\tableofcontents
\newpage

\section{Plant}

\begin{tabular}{lll}
DS        & Phase & Start of phase \\ \hline
-2      & Not sown &  \\
$[-10[$  & sown but not emerged & Sown \\
$[0 1[$  & vegetative growth  & Emerged \\
$[12[$   & reproductive & Anthesis \\
$[23[$   & Ripening & End of Grain filling \\
3       & & Ripe 
\end{tabular}

\subsection{Phenology}

%Add 2 phenology classes together if they are not to different in the Developing stage 
%
%Add(Phenology, frac)
%
%\begin{align*}
%	DS  =& (1-frac)  \cdot  DS + frac  \cdot  aPhenology.DS \\
%   TempSumForLeaf  =& (1-frac)  \cdot  TempSumForLeaf \\
%                     &    + frac  \cdot  aPhenology.TempSumForLeaf \\
%   TempSumAfterFirstMarch = &(1-frac) \cdot  TempSumAfterFirstMarch \\
%                           &  + frac \\
%                           &  \cdot  aPhenology.TempSumAfterFirstMarch
%\end{align*}

\subsubsection*{Update(AirTemp,SoilTemp,DayLength)}
Update DS - development stage, DSIncrease, TempSumForLeaf and TempSumAfterFirstMarch.
\mymarginpar{TS0, TS1, TS2, TS3, TB0, TB1, TB2, TB3, DMAX, DB} 
   
   if $DS<0$  (Not emerged)
   
   \quad then $DS = DS + \tfrac{ \max(SoilTemp-TB0,0)}{TS0}$ \quad 
    
   else  \quad  (Emerged)
  
  \quad 	$TempSumForLeaf = TempSumForLeaf + \max(0,AirTemp)$

\quad 	if $month=3$ and $GetDay()=1$ then $TempSumAfterFirstMarch = 0$
 
 \quad  $ TempSumAfterFirstMarch = TempSumAfterFirstMarch + \max(0,AirTemp)$

 \quad  $DL=\max\left(0,\min\left(1,\tfrac{DayLength-DB}{DMAX-DB}\right)\right)$
   
 \quad   DSIncrease = DS

\quad
$DS = 
\begin{cases}
DS + DL \cdot \tfrac{\max(AirTemp-TB1,0)}{TS1} &  \text{if } DS < 1 \text{ (Until flowering)} \\
DS + \tfrac{\max(AirTemp-TB2,0)}{TS2} &   \text{if } DS < 2  \text{ (Until end of grain filling)} \\
DS + \tfrac{\max(AirTemp-TB3,0)}{TS3} &	\text{if } TS3 > 0 \text{ and } DS < 3 \text{ (Until ripe)} \\
3 & \text{otherwise (Ripe)}	
\end{cases}$

if $DS > 3$ then	$DS = 3$

$DSIncrease = DS-DSIncrease$  (Increase in DS this day)

   dates for critical growth developmental stages
   
   if $DS > 0$ and not $EmergencePassed$ then
   
   \quad	DateOfEmergence = date, EmergencePassed = true

   if $DS > 1$ and not $FloweringPassed$ then
    
   \quad 	DateOfFlowering = date, FloweringPassed = true

   if $DS > 2$ and not $EndOfGrainFillingPassed$ then

  \quad  	DateOfEndGrainFill = date, EndOfGrainFillingPassed = true 
   
   if $DS > 3$ and not $RipenessPassed$ then
    
    \quad 	DateOfRipeness = date, RipenessPassed = true


\minisec{Phenology.Sow()}
	DS = -1 , 
   TempSumForLeaf  = 
   TempSumAfterFirstMarch  = 0

\subsection{Sow(SeedDensDryMatt,aRelativeDensity,NitrogenInSeed)}
 An account of the DM lost via GerminationConversion must be added, in order
 to keep total balance of DM/carbon! \mymarginpar{GerminationConversion, InitialRootDepth} 
 %   
   \begin{align*} 
   Phenology.Sow() & \\		
	SeedDM           & = SeedDensDryMatt  \cdot  GerminationConversion \\
   InitialSeedDM   &  = SeedDM   \\
	RootDepth        & = InitialRootDepth 
	\end{align*}
TotalRootLength  = 
	DMRoot    = 
	DMVegTop   =  
	DMStorage  =  
	$CAI_g$  = 
	FillFlag  = 0
	
	%	
	$Nitrogen.SetBoth(NitrogenInSeed,0)$,	
  $RelativeDensity = aRelativeDensity$,
 
   $Nbudget.SetInput(NitrogenInSeed)$, 

   $DMbudget.SetInput(SeedDensDryMatt  \cdot  GerminationConversion)$ 
   
   $Nbudget.SetOutput(0), N15budget.SetOutput(0), DMbudget.SetOutput(0)$ 


\minisec{GetfracToAnthesis()}
the DS remaining before anthesis. $GetfracToAnthesis()  = \max(\min(1 - DS,1),0)$

\subsection{Daily DM growth - Update(ActivePar)}   \mymarginpar{LinearLeafPhase}    
   extinction coefficient for photosynthetic active radiation \mymarginpar{k}

   $\mathbf{GiveExtCoeff} = k$
   
   Rg = ActivePar 

   (Rg is something like $radiation  \cdot  exp(-GiveExtCoeff)  \cdot  GiveExtCoeff  \cdot  LAIinInterval$)

   temp = theClimate.tmean,   
  $DailyDMGrowth  = 
   \Delta DMTop  = 0$
     
   if $Phenology.Sown()$ and $not Phenology.Ripe()$ % and $not terminated$
  
 \quad     if $totalLAIStand>0$ then    \mymarginpar{ColonialisationRate}
      \begin{align*}
  maxSpacefrac  = &\tfrac{CAI_g+CAI_y}{totalLAIStand} \\
  fracOfSpace  = &\max(0,\min(maxSpacefrac, \\
                   & \quad  fracOfSpace  + \max(0,temp)  \cdot  ColonialisationRate)) 
      \end{align*}                        
  $soilTemp = aSoil.GetTemp(200)$, 
     $DayLength =  theClimate.PhotoPeriod()$           
           
  Phenology.Update(temp,soilTemp,DayLength)     (updates phenology)

   if $Phenology.Sown()$ then		
  
  \quad $TempSumRoot = TempSumRoot + \max(0,temp)$

  \quad if Phenology.Emerged()

  \quad \quad $PlantHeight = 
      MaxPlantHeight  \cdot  (1-Phenology.GetfracToAnthesis())$  

  \quad $PlantHeight = 
      \min(PlantHeight,MaxPlantHeight) $


%\quad 
     % \begin{align*}
		  %DeltaDM() & = DeltaDM()er() \\
 \quad $DailyDMGrowth  = \mathbf{\Delta DM}$ 
            
 \quad  $AccumulatedDMProduction = AccumulatedDMProduction + \mathbf{\Delta DM} $
      %\end{align*}

 \quad DMbudget.AddInput($\mathbf{\Delta DM} $)
 
 \quad if $Phenology.TempSumForLeaf \le Phenology.LinearLeafPhase$ then
\mymarginpar{Topfrac,LinearLeafPhase}
     
 \quad\quad (the same termal time as linear leaf growth) 
\begin{align*} 
DMTransfer & = \min(2  \cdot  Topfrac  \cdot  InitialSeedDM 
  \cdot   \tfrac{\max(0,temp)}{Phenology.LinearLeafPhase},SeedDM) \\
DMVegTop &= DMVegTop  + 0.5  \cdot  DMTransfer \\
DMRoot  &= DMRoot   + 0.5  \cdot  DMTransfer \\
SeedDM &= SeedDM - DMTransfer
\end{align*}
\quad TransferDMToRoot($\mathbf{\Delta DM} $)
      
\quad $CalcRootGrowth()$
      
\quad TransferDMToStorage($\mathbf{\Delta DM} $)
     \begin{align*} 
      DMVegTop & = DMVegTop + \mathbf{\Delta DM} \\
      \Delta DMTop & = \mathbf{\Delta DM}
      \end{align*}                             
%\quad      Used for output
\quad      CalcLAI()
           
% DailyDMGrowth()
 
  
\subsubsection{TransferDMToStorage($\Delta DM$)}
 New limitation to the transfer of DM to storage according
   to the concept of Justes et al. 1994. Annals of Botany 74:397-407.
   \mymarginpar{StoreForFilling, FillFactor}

the DS frac between start and end of grain fill \mymarginpar{DS\_StartFill}

 $frac = \begin{cases}
    \tfrac{ DSIncrease}{2-DS\_StartFill} & \text{if } DS \le 2 \\
   0 & \text{otherwise}
    \end{cases}$

$\mathbf{Phenology.GetfracOfGrainFill()}  = \max(\min(frac,1),0)$

  $Store  = \min(0.9,StoreForFilling), Fill  = \min(0.9,FillFactor) $
  
	if $Phenology.Anthesis()$ and $FillFlag=0$ then   
	
  \quad 	if $TransferableStorage < 0.00000001$ then  $TransferableStorage = DMVegTop  \cdot  Store $
%
  \begin{align*}
  TransferableStorage & = TransferableStorage +  \Delta DM  \cdot  Fill \\
  TransferedDMToStorage & = TransferableStorage
  \end{align*}
 %  
   if Phenology.GrainFillStart() then
   
   \quad	FillFlag = 1 
   	
   \quad   if $fNitrogen() > 0$ then (linear fill of DM stored during lag phase)  \mymarginpar{ConversionCoefficient}
          \begin{align*}
          transfer  & =  \min(DMVegTop, \max(0,TransferableStorage \\
                    & \quad  \cdot  Phenology.GetfracOfGrainFill()))  \\
         DMVegTop  & =  DMVegTop - transfer \\
         DMStorage  & =  DMStorage + transfer +   \Delta DM  \cdot  Fill  \cdot  ConversionCoefficient         
         \end{align*}  
   \quad \quad if $DMVegTop < TransferableStorage$ then $TransferableStorage = 0.9  \cdot   DMVegTop$ 
    	
   \quad \quad    $  DMbudget.AddOutput(( \Delta DM  \cdot  Fill)  \cdot  (1-ConversionCoefficient)) $     
    
    \quad \quad      $\Delta DM  =  \Delta DM  \cdot  (1-Fill)$ 

     \quad \quad     $TransferedDMToStorage  = TransferedDMToStorage - transfer $
     
 
   if $Phenology.GrainFillEnd()$ and $TransferedDMToStorage>0$ then 

   \quad  (Check to see if all transferable storage is transfered)
   %
     \begin{align*}
      transfer &=\max(0,TransferedDMToStorage) \\
   	DMStorage  &=  	DMStorage + transfer \\
      DMVegTop & = DMVegTop -  transfer \\
      TransferedDMToStorage & =  0 
      \end{align*}
%%%%%%%%%%%%%%% transfere DM to root %%%%%%%%%%%%%%%%%%%%%%%%%%%%%%%%%%%	
\subsubsection{TransferDMToRoot($\Delta DM$)} 
\citep{berntsen2005simulation}
this function makes no sense at highly stressed plants

fracToRoot \mymarginpar{MinAllocToRoot, MaxAllocToRoot, DS\_StartFill} 

$ = \begin{cases}
	MaxAllocToRoot & \text{if not Emerged()} \\  
	%
	MinAllocToRoot + (MaxAllocToRoot - MinAllocToRoot) & \\	
	 \cdot  GetfracToAnthesis() & \text{if Emerged() and not Anthesis()} \\
 %
   \max(0,MinAllocToRoot  \cdot  
    (DS\_StartFill - DS)) &  \text{if Anthesis() and not GrainFillEnd()} \\
 %
   MinAllocToRoot & \text{otherwise}
	\end{cases}	
	$	



  \begin{align*}
  \Delta Root & = \max(0, \Delta DM  \cdot  fracToRoot) \\
   RootTranslocation & = RootTranslocation + \Delta Root \\
     \Delta DM & =   \Delta DM- \Delta Root  \\
	DMRoot & = DMRoot +\Delta Root 
	\end{align*}	
   $RootDecay(\Delta Root)$ 
  
   
\subsubsection{RootDecay(RootTranslocation)} \mymarginpar{DecayRoot, RhizoDepositfrac, DS\_StartFill}
\citep{berntsen2005simulation}
Rhizodeposition and root decay, t - Temp \textdegree C 


$TempEffect(t)
    = \begin{cases}
     \exp(0.47-0.027  \cdot  t+0.00193  \cdot  t^2) & \text{if } t > 20 \\
     0.1  \cdot  t & \text{if } t>0 \text{ and } t \le 20 \\   
     0 & \text{otherwise}
   \end{cases}       
   $
   
		if not Phenology.Ripe() then 	$aNitrogenInRoot =  NitrogenInRoot()$ 	
		
		if rootMatter then rootMatter.Setamount(0) 

		SoilTemp = aSoil.GetTemp(200)

		if $DMRoot>0$ and $temp>0$ and $Phenology.Emerged()$ then
	     \begin{align*}
     	DMdecay  & =  DMRoot  \cdot  TempEffect(SoilTemp)  \cdot  DecayRoot \\
     	           & \quad 
    +  RootTranslocation  \cdot  RhizoDepositfrac \\
			DMdecay & =  \min(DMRoot,DMdecay)
			\end{align*}
%
		if $DMdecay>0$ then					
		$ NConcentration =  \tfrac{aNitrogenInRoot}{DMRoot} $
				
		\quad		if $NConcentration.n < MinN\_Root$ then			        					
		
		\quad\quad	$NConcentration = NConcentration  \cdot  \tfrac{MinN\_Root}{NConcentration.n} $
\begin{align*}
Ndecay & = NConcentration  \cdot  DMdecay \\
DMRoot  & = DMRoot - DMdecay \\
NitrogenTotal & = NitrogenTotal - Ndecay
\end{align*}
\quad			$	rootMatter.Setamount(\tfrac{DMdecay}{rootMatter.GetDM()})$
			
\quad			$rootMatter.SetorgN\_content(\frac{Ndecay}{\tfrac{DMdecay}{rootMatter.GetDM()}})$
				
\quad			Nbudget.AddOutput(Ndecay.n), N15budget.AddOutput(Ndecay.n15), 

\quad DMbudget.AddOutput(DMdecay)
				\begin{align*}
				AccumulatedRootDeposit  &= AccumulatedRootDeposit  + DMdecay \\
				AccumulatedRootNDeposit &= AccumulatedRootNDeposit + Ndecay
				\end{align*}   
   

   



\minisec{CO2Effect}
\citep{olesen2000sensitivity} \mymarginpar{CO2Effect}

$CO2PhotEffect  =  \begin{cases}
CO2Effect  \cdot  \exp\left(0.4537-\tfrac{170.97}{climate.GetCO2Concentration()}\right) & \text{ if no } C4Photosynt. \\
1 & \text{ otherwise}
\end{cases}$ 

%\minisec{fTW(temp)} 
Similar to \citep{hansen1990npo}. DIFFERENT FORMULA IN \citep{berntsen2005simulation} 
\mymarginpar{MinDMtemp, MaxDMtemp}

$fTW(temp)	= 	
		\begin{cases}
		0 & \text{if } temp \le MinDMtemp \\
		1 & \text{if } temp \ge MaxDMtemp \\
		\tfrac{Temp-MinDMtemp}{MaxDMtemp-MinDMtemp} &   \text{otherwise}
		\end{cases}$   



%%%%%%%%%% Delta DM %%%%%%%%%%%%%%%%%%%%%%%%%%%%%%%%%%%%%%%%%%%%%%%%%%%%%%%%%%%%%%%%%%
     
\subsubsection{$\Delta DM$} \mymarginpar{MaxRadUseEff, 
 CO2Effect, PhotoSyntActPart}
   Alters the dry matter production relative to CO2 concentration
    unaltered at $CO2 = 377ppm$.  CO2 function parameters only estimated for WINTER WHEAT.
   \citep{olesen2002comparison,berntsen2005simulation}	
	\begin{align*}
	% CO2conc & = theClimate.GetCO2Concentration() \\
 % CO2PhotEffect & = 1 \\
  %\intertext{if no C4Photosynthesis then}
  %CO2PhotEffect & =  CO2Effect  \cdot  %\exp\left(0.4537-\tfrac{170.97}{CO2conc}\right) \\
     InterceptedRadiation & = InterceptedRadiation + Rg  \cdot  PhotoSyntActPart \\
    PAR\_Reduction & = \min(1,1-0.0445  \cdot  (Rg  \cdot  PhotoSyntActPart-5))  \\
	\Delta DM & = Rg  \cdot  PhotoSyntActPart  \cdot  MaxRadUseEff   \cdot  CO2PhotEffect \\ 
	          & \quad   \cdot  TranspRatio    \cdot  fNitrogenCurve()   \cdot  fTW()   \cdot  PAR\_Reduction 	
  \end{align*}
  
\minisec{OptimalDeltaDM()}  
\citep{berntsen2005simulation}
 Calculates the optimal production. Used for root growth 
\begin{align*}
Optimal\Delta DM()  & =  Rg  \cdot  PhotoSyntActPart  \cdot  MaxRadUseEff 
      \cdot  CO2PhotEffect  \cdot  fTW() 
     \end{align*} 
 
\minisec{PotentialDM(radiation,temp)} 
 Get really rough estimate of potential growth. Assumes complete interception of radiation
 Used in grassland management 
 \begin{align*} 
 PotentialDM(rad,temp) & = rad  \cdot  PhotoSyntActPart  \cdot  MaxRadUseEff \\
        & \quad  \cdot  CO2PhotEffect  \cdot  fTW(temp) 
\end{align*}

\minisec{fNitrogenCurve()}  \mymarginpar{NCurveParam}
\citep{olesen2002crop} , ScaleFactor  = 1.05 
	\begin{align*}	
	Nstress & = ScaleFactor  \cdot  (1-\exp(NCurveParam  \cdot  fNitrogen())) \\
	fNitrogenCurve() & = \max(0,\min(1,Nstress)) 
	\end{align*} 
	
ONLY FOR WINTER WHEAT? DIFFERENT IN \citep{berntsen2005simulation} 	
%%%%%%%%%%%%%%%%%%%%%%%%%%%%%%%%%%%%%%%%%%%%%%%%%%%%%%%%%%%%%%%%%%%%%%%%%%%%% 
 
\subsection{LAI}
 
the DS remaining before flag ligule \mymarginpar{DS\_Flagligule}

$GetfracToFlagLigule() = \begin{cases}
   	 \max\left(0,\tfrac{DS}{DS\_Flagligule}\right) & \text{if } DS < DS\_Flagligule \\
  \min\left(2,1+\tfrac{DS-DS\_Flagligule}{2-DS\_Flagligule}\right) & \text{otherwise} 
\end{cases}$

leaf scenesence factor. 2 as assuming that crop is yellow at end of grain filling.
\begin{align*}
LeafSenesence() & = \tfrac{DSIncrease}{2-1} 
\end{align*}

\subsubsection*{CalcLAIndices()}  \mymarginpar{InitialCAI, CoeffGreenIdx, LAIDMRatio, LAINitrogenRatio, MinimumSenescense, Conversfactor, GLAImax, LinearLeafPhase} 
  \citep{olesen2002crop,olesen2004simulation}

 if $CAI_g>GLAI_{max}$ then 
    $GLAI_{max}=CAI_g$ \quad \quad    
  % state variable that holds the maximum achieved green LAI
  
   if $Phenology.Emerged()$ then
   
   \quad	if $Phenology.TempSumForLeaf \le Phenology.LinearLeafPhase$ then
    
\quad\quad$CAI_g=InitialCAI  \cdot \tfrac{Phenology.TempSumForLeaf}{Phenology.LinearLeafPhase} $          
  \quad (value from Porter)
    
   \quad   else
   
   \quad\quad   $	DS\_num = 1.3 $
      	
     \quad\quad 	if $CropName = "WinterWheat"$ then 	$DS\_num =1 $
         
    \quad\quad  if $Phenology.DS<DS\_num$ then       
				\begin{align*}
	%			 reducedGrowth  & =  Phenology.GetfracToFlagLigule() \quad (Correct later)  \\  
	%			 reducedGrowth  & =  \tfrac{1}{1+Phenology.GetfracToFlagLigule()^{10}} \\
        % & \\
         expLeafExt & =  CoeffGreenIdx  \cdot  CAI_g  \cdot  \max(0,temp)\\
         expL &  =  expLeafExt  \cdot  \min\left(1,\tfrac{TranspRatio}{0.6}\right)  \cdot   \tfrac{1}{1+Phenology.GetfracToFlagLigule()^{10}}  \\ 
         dmL  &  =  LAIDMRatio  \cdot  DMVegTop - CAI_g \\
         NL   &  =    LAINitrogenRatio  \cdot  NitrogenInVegTop()  - CAI_g \\
         pot\Delta GLAI  & =  \max(0,GLAI_{max}Crop-CAI_g) \\
         & \\         
         \mathbf{\Delta GLAI} & =  \min\left(expL, dmL, NL, pot\Delta GLAI\right) \quad 
\text{ \citep{olesen2002crop,olesen2004simulation}}\\
         CAI_g & =  CAI_g + \Delta GLAI 
        \end{align*}
     %   
      
 if $Phenology.LeafSenesence() > 0$ or  $temp < 0$ then
         %
    
   $senescense = \begin{cases} \tfrac{Phenology.LeafSenesence()}{1-MinimumSenescense} & \text{ if } temp \ge 0 \\ 
  \tfrac{0.05}{1-MinimumSenescense} & \text{ if } temp < 0 
 \end{cases} $
  \begin{align*}
   \Delta CAIg  = & -\max(0,senescense  \cdot  GLAI_{max}   \cdot  (1-TranspRatio  \cdot  MinimumSenescense))
  \end{align*}
            if $CAI_y \ge 0 $ or  $CAI_g \ge 0 $ then $ CAI_g = CAI_g + \Delta CAIg $         
                       	
           if $CAI_g < 0$ then 

              \quad $\Delta CAIg = \Delta CAIg - CAI_g $,            
               $CAI_g = 0$, $CAI_y = CAI_y- \Delta CAIg  \cdot  Conversfactor$
             

\minisec{LAfrac(height)} \mymarginpar{LAIDistributionPar}
   leaf area frac below a certain height h.  
   \citep{olesen2004simulation}.  $a=LAIDistributionPar$
  
  $ LAfrac(h) = 
   \begin{cases}
   0 & \text{if } PlantHeight =0 \\
   1 & \text{if }  h \ge PlantHeight \\
    (a+1)  \cdot  (a+2) 
          \cdot  \tfrac{h^{a+1}}{PlantHeight^{a+1}}  
          \cdot    \left( \tfrac{1}{a+1}-\tfrac{h}{(a+2)  \cdot  PlantHeight}\right) &
          \text{ otherwise}
   \end{cases}     $

SMALL DIFFERENT TO FORMULA IN \citep{olesen2004simulation}


\subsubsection*{TotalLAI(height, thickness)}
the total leaf area index in a certain layer of the canopy

return $(CAI_g+CAI_y)  \cdot  (LAfrac(height+thickness)  - LAfrac(height))$


\subsubsection*{GreenLAI(height,thickness)}
the green leaf area in a specific layer in canopy
 assuming that the yellow leaves are situated in the lower part of the canopy
 below the green leaves. h - height.    
    \begin{align*}    
   UpperArea  & =  \max(0,(CAI_g+CAI_y)  \cdot  LAfrac(h+thickness)-CAI_y) \\
  LowerArea  & =  \max(0,(CAI_g+CAI_y)   \cdot  LAfrac(h)-CAI_y)    
   \end{align*} 
return $UpperArea-LowerArea$   
  
  %  
%soil information, NumOfLayers and MaxRootDepth
%


\subsection{Nitrogen}
\minisec{GiveNitrogenResponseCO2()}
Effect of CO2 on the minimum and maximum concentrations of Nitrogen
\begin{align*}
  CO2 & = theClimate.GetCO2Concentration() \\
GiveNitrogenResponseCO2() & = \max(1,0.6+\tfrac{141}{CO2})   
\end{align*}  

\minisec{Corrected N Store, Corrected N Root} \mymarginpar{MinN\_Root, MaxN\_Root, MinN\_Store MaxN\_Store}
The min/max concentration of N in the storage organs/roots as affected by CO2

NCO2 = GiveNitrogenResponseCO2() 
\begin{align*} 
MinN\_Store & = NCO2  \cdot  MinN\_Store & MaxN\_Store & = NCO2  \cdot  MaxN\_Store \\ 
MinN\_Root & = NCO2  \cdot  MinN\_Root & MaxN\_Root & = NCO2  \cdot  MaxN\_Root \\ 
\end{align*}

\minisec{NMin(), Nmax()}
Calculate the min/max amount Nitrogen that the crop needs  [\% g N in VegTop/$m^2$]	
\begin{align*}
Nmin() = & MinN\_Root  \cdot  DMRoot + MinN\_Store  \cdot  DMStorage
           + \tfrac{NpctMinVegTop()}{100}   DMVegTop \\
 Nmax() = &     MaxN\_Root  \cdot  DMRoot  + MaxN\_Store  \cdot  DMStorage 
          +  \tfrac{NpctMaxVegTop()}{100}  DMVegTop
	\end{align*}
	
	
\subsubsection*{Parameter for Nitrogen Curves}

\begin{tabular}{lrcrc}
                           & Nmin              & valid [t/ha] & Nmax &  valid [t/ha] \\ \hline
Winter Wheat  \citep{justes1994determination} & $2.2   DM^{-0.44}$  &  0.5-1.4  & $8.3   DM^{-0.44}$ & 1.5 - 14\\
Maize         \citep{plenet1999relationships} & $2.05  DM^{-0.56}$  &  0-22     & $6.30  DM^{-0.42}$ & 0-22 \\
Maize         \citep{bleken2009spn} & $2.7  DM^{-0.56}$   &   1-?     & $3     DM^{-0.42}$ & 0.75 - ? \\              
Winter Rape \citep{colnenne1998determination} & $2.07  DM^{-0.17}$  &  0.3-6.9  & $6.18  DM^{-0.21}$ & 0.3-6.9 \\
Ryegrass      \citep{marino2004nitrogen} & $1.458 DM^{-0.231}$ &    ?       & $6.145 DM^{-0.418}$ & ?
\end{tabular}             
      
	

\subsubsection*{NpctMaxVegTop(), NpctMinVegTop()} 
\mymarginpar{PowDM, NPctMax, NPctMin, ReducedNInYellow}
  $ NpctMaxVegTop() =
  \begin{cases}
  NPctMax  \cdot  \frac{1.5}{fracofspace}^{PowDM}  & \text{ if $DMVegTop \le 150$}  \\   
  	 NPctMax  \cdot  \left(\tfrac{DMVegTop}{100  \cdot  fracOfSpace}\right)^{PowDM}   & \text{ if $DMVegTop > 150$} 
	\end{cases}$
	
	$NpctMaxVegTop() = NpctMaxVegTop()  \cdot  GiveNitrogenResponseCO2()$
 
  $ NpctMinVegTop() =
  \begin{cases}
  NPctMin  \cdot  \frac{1.5}{fracofspace}^{PowDM}  & \text{ if $DMVegTop \le 150$}  \\   
  	 NPctMin  \cdot  \left(\tfrac{DMVegTop}{100  \cdot  fracOfSpace}\right)^{PowDM}  & \text{ if $DMVegTop > 150$} 
	\end{cases}$
 
  $NpctMinVegTop() = NpctMinVegTop()  \cdot  GiveNitrogenResponseCO2()$
  
  if $CAI_y > 0$ then  
  $NpctMaxVegTop =  NpctMaxVegTop  \cdot  \tfrac{ReducedNInYellow  \cdot  CAI_y + CAI_g}{CAI_y + CAI_g} $

  if $CAI_y > 0$ then  
  $NpctMinVegTop =  NpctMinVegTop  \cdot  \tfrac{ReducedNInYellow  \cdot  CAI_y + CAI_g}{CAI_y + CAI_g} $


%Follows the concept of Justes et al. 1994. Annals of Botany 74:397-407.
%the exsistance of yellow parts reduces the concentration of N in the vegtop
%\begin{align*}
%  NpctMinVegTop() & =  NpctVal  \cdot  NCO2 \\
%  NpctVal & = NPctMin  \cdot  1.5^{PowDM} \\
%\intertext{if $DMVegTop>150$ then} % \quad (for the green leafs and stem)}
%	NCO2 & = GiveNitrogenResponseCO2() \\
%  NpctVal & = NPctMin  \cdot  \left(\tfrac{DMVegTop}{100  \cdot  fracOfSpace}\right)^{PowDM} 
%\intertext{if $CAI_y > 0$ then}
% RedFac & = \tfrac{ReducedNInYellow  \cdot  CAI_y+CAI_g}
%                        {CAI_y+CAI_g} \\
% NpctVal & = NpctVal  \cdot  RedFac 
%\end{align*}

%GiveRootLengthList() = RootLengthList
\subsubsection*{NitrogenDemand()} \mymarginpar{DS\_StartFill}
 	
 	if Phenology.GrainFillStart() 
 	
 	then 	$NitrogenDemand = \tfrac{Phenology.DS-2}{Phenology.DS\_StartFill-2}  \cdot  NitrogenDemand $ 		
 	
 	else 	$NitrogenDemand= max(0,Nmax()-NitrogenTotal)$ 
  
\subsubsection*{NitrogenAndTransp(aNitrogenUptake,aTranspRatio)}
Adding Nitrogen uptake to the crop and replace Transp Ratio,
   TranspRatio = aTranspRatio
   
   if $TranspRatio<0.5$ and $CAI_y+CAI_g<0.1$ then TranspRatio = 1
          
   (corrects problem with Transp the day of germination)
     
   if $aNitrogenUptake.n > 1e-15$ then
   \begin{align*}
      NitrogenTotal & = NitrogenTotal + aNitrogenUptake \\
      AccumulatedNProduction & = AccumulatedNProduction + aNitrogenUptake 
   \end{align*}
   %   
      Nbudget.AddInput(aNitrogenUptake.n,                  
    N15budget.AddInput(aNitrogenUptake.n15)

   
  
  
\subsubsection*{NitrogenInRoot(), NitrogenInStorage(), NitrogenInVegTop(), GiveTotalNitrogen() }
		\begin{align*}		
  NitrogenInRoot & =  (MinN\_Root +fNitrogen()  \cdot  (MaxN\_Root -MinN\_Root))  \cdot  DMRoot   \\
  NitrogenInStorage & =  (MinN\_Store + fNitrogen()  \cdot  (MaxN\_Store-MinN\_Store))  \cdot   DMStorage  
  \end{align*}  
%
  %$ N15Ratio =  Nitrogen.Get15NRatio()$
  %$ SetBoth(NContent,N15Ratio  \cdot  NContent) $ 
  %$SetBoth(NContent,N15Ratio  \cdot  NContent) $
  %
Calculate how much Nitrogen that are in the top plant that is not planed for usage
	\begin{align*}
	NitrogenInVegTop() & = TotalNitrogen-NitrogenInRoot()-NitrogenInStorage()  
  \end{align*}   
  $GiveTotalNitrogen() = NitrogenTotal$ 
 
\subsubsection*{fNitrogen()}
\citep{olesen2002crop} nitrogen nutrition index (NNI). 
returns a factor that describe how much N is 
in each part of the plant
	
  if not $Phenology.Ripe()$ and $Nmax()>Nmin()$ then     
 
   return  %\max\left(0,\min\left(1,
$\tfrac{NitrogenTotal-Nmin()}{Nmax()-Nmin()}$
                 % \right)\right)   
 
\subsection{Root}

%\minisec{RootPenetrationReduction()}

\minisec{CalcRootDensAtSurf(RootDistrParm,RootDepth,RootDensAtBottom)}
$ a =  \tfrac{1-\exp(-RootDistrParm  \cdot  RootDepth)}{RootDistrParm} 
 + 0.5  \cdot  0.3 RootDepth  \exp(-RootDistrParm  \cdot  RootDepth) $    
    
return $\begin{cases} 
\tfrac{TotalRootLength}{a} & \text{if } a \ne 0 \\
RootDensAtBottom & \text{otherwise}
\end{cases}$
  	
%%% Calc Root Growth %%%%% 
\subsubsection{CalcRootGrowth()} \mymarginpar{RootPentrRate, 
SpecificRootLength, RootDistrParm, RootDensAtBottom, TempRootMin, MaxRootDepth} 
\citep{berntsen2005simulation}

$clay   = aSoil.GetSoilProfile().GetClayContent(250,500) $
 
$\mathbf{RootPenetrationReduction()} = \min\left(1,\max\left(0.5,0.5+(clay-0.02)  \cdot  \tfrac{0.5}{0.08-0.02}\right)\right) $

DIFFERENT FORMULA IN \citep{berntsen2005simulation}

if Phenology.Sown()
 
\quad		if $temp>TempRootMin$
		\begin{align*}
   RootDepth & =  \min(MaxRootDepth,  RootDepth + RootPentrRate \\ 
            & \quad  \cdot  RootPenetrationReduction()  \cdot  (temp-TempRootMin)) 
    \end{align*}
    %
    \begin{align*}       
	TotalRootLength  =& DMRoot  \cdot  SpecificRootLength \\
	RootDensAtSurf  =& CalcRootDensAtSurf(RootDistrParm,RootDepth,RootDensAtBottom) \\
  RootDensAtBottom  =& RootDensAtSurf  \cdot  \exp(-RootDistrParm  \cdot  RootDepth)
 \end{align*}
 CalcRootDistribution() 
 
 

%
  


	
\minisec{CalcRootDistribution()}  
Updating root each in each layer of the soil.
 $startdepth=0$,
 $LThickness = LayerThickness$
 	
for all $i=0 \dots NumOfLayers-1$ do
        
  \quad    if $LThickness_i<0$ then
      $ LThickness_i= \tfrac{aSoil.GetLThickness(i)}{1000} $
      
\quad$ RootLengthList_i= 
   \begin{cases}
 RootLengthInInterval(startdepth,LThickness_i) & \text{if $LThickness_i>0$} \\
       0 & \text{otherwise}
     \end{cases}
     $     
          
\quad    $startdepth = startdepth+ LThickness_i$	

\minisec{RootLengthInInterval(startdepth,thickness)}
SIMILAR TO \citep{berntsen2005simulation}
s - startdepth, t - thickness in meter,  Length1  =  Length2  = 0

    if not ($Phenology.Sown()$ and $RootDepth > 0$) then return $0$

    \textbf{Length in exponential curve}
    \begin{align*} 
     ExtraRootDepth & = 0.3 \\
     EndDepth1 & = \min(RootDepth,s+t)  \\
     EndDepth2 & = \min((1+ExtraRootDepth)  \cdot  RootDepth,s+t) \\
     s2 & = \max(s,RootDepth)      
    \end{align*}
	if $s<RootDepth $ then  \mymarginpar{MinimumRootDens}		
		\begin{align*}
			Length1 = & RootDensAtSurf \tfrac{\exp(-RootDistrParm  \cdot  s) 
			          -   \exp(-RootDistrParm  \cdot  EndDepth1)}{RootDistrParm} \\
			          &   +  (EndDepth1-s)  \cdot  MinimumRootDens 
	  \end{align*}
%
    \textbf{Length in linear declining curve}
       
   	if $s + t > RootDepth$ and $s < (1+ExtraRootDepth)  \cdot  RootDepth$ then
				
		\quad  (assuming that root systems extends 30\% further than rootdepth)
       \begin{align*}
  DensAtBottom & = RootDensAtSurf  \cdot  (\exp(-RootDistrParm  \cdot  RootDepth))\\
  Dens1 & = DensAtBottom  \cdot  \left(1-\tfrac{s2-RootDepth}
  {ExtraRootDepth  \cdot  RootDepth}\right) \\  
 Dens2 & = DensAtBottom  \cdot  \left(1-\tfrac{EndDepth2-RootDepth}
 {ExtraRootDepth  \cdot  RootDepth}\right)  \\  	  
         Length2 & = 0.5  \cdot  (EndDepth2-s2)  \cdot  (Dens1+Dens2)
         \end{align*}  
  \textbf{return Length1+Length2}
  
%%%%%%%%%%%%%%%%%%%%%%%%%%%%%%%%%%%%%%%%%%%%%%%%%%%%%%%%%%%%%%%%%%%%%%%%%%%%%%%%%%%%%%%%%%%%%%%%%%%%%%%%%%%%%%%%%%% 
\subsubsection*{GiveRootInInterval(startDep, thick, input)}

%Input is either DMRoot or NitrogenInRoot
  
 TopOfLayer =
  BottomOfLayer  =
  Sum  =
	0
	
	%

 $Total = \sum_{i=0}^{MaxSoilLayers - 1} RootLengthList_i$

	%for all $i$ in $0 \dots MaxSoilLayers - 1$ do $Total = Total + RootLengthList_i $

	for all $i$ in $0 \dots MaxSoilLayers - 1$ do   	
   	\begin{align*}
      	TopOfLayer    & = BottomOfLayer \\
      	BottomOfLayer & = TopOfLayer+0.05
   	\end{align*}   
  if $TopOfLayer > startDep$   
  and $BottomOfLayer < startDep+thick$ then
   
          
  \quad	$Sum += RootLengthList_i$  (Simple case) 
      	
    	else if not $TopOfLayer < startDep$ and $	BottomOfLayer> startDep+thick $ then
      	 	
   \quad                    
            $Sum +=  \tfrac{BottomOfLayer-startDep}{0.05}  \cdot  RootLengthList_i$
(Only bottom of layer is in interval)            

    
      	else if $TopOfLayer <  startDep$  and not $BottomOfLayer> startDep+thick$ then
            	
     \quad                          
               $Sum +=  \tfrac{startDep+thick-TopOfLayer}{0.05}  \cdot  RootLengthList_i$
  (Only top of layer is in interval)
               
return $Sum  \cdot   \tfrac{input}{Total}$

\subsubsection*{TargetFunction(RootDistrParm,RootDepth,TotalRootLength,RootDensAtBottom)}
	return $\tfrac{TotalRootLength  \cdot  RootDistrParm}{1-\exp(-RootDistrParm  \cdot  RootDepth)}
		-\tfrac{RootDensAtBottom}{\exp(-RootDistrParm  \cdot  RootDepth)} $




\subsubsection*{CalcFormParameter(RootDistrParm,RootDepth, TotalRootLength,RootDensAtBottom)}

	iter =0,
	maxiter  =10000, 
	MaxDeviation =0.00001,
	move =  $RootDistrParm  \cdot  0.01 $

	if $move<0$ then $move=-move$
	 
	while $fabs(TargetFunction(RootDistrParm,\dots)) > MaxDeviation$ and $iter<maxiter$ 
	
	do $iter++$

do ($RootDistrParm = RootDistrParm+move$,  $iter++$)	
		
while $TargetFunction(RootDistrParm+move, \dots) < 0$ and $iter < maxiter$  

do  $move=move  \cdot  0.5 $
		
do 
 if $RootDistrParm-move>0$ then $ RootDistrParm = RootDistrParm-move $

 \quad $iter++$

while $TargetFunction(RootDistrParm+move, \dots) < 0$

\quad  and $RootDistrParm-move > 0$  and $iter < maxiter $ 

do
		$move = move  \cdot  0.5 $

	
\subsection{Others}	
  
\subsubsection*{IrrigationDemand()}  
\mymarginpar{WaterDeficitVegGrowth, WaterDeficitLagPhase, WaterDeficitGrainFill, LinearLeafPhase}
Returns true or false. Soil water deficit = 1 - Waterstatus, 
$MaxWDeficit = 1$
	
 if Phenology.Emerged() and 

\quad $Phenology.TempSumForLeaf \ge Phenology.LinearLeafPhase$ then
 
\quad if not Phenology.Anthesis() 

\quad  then  
$MaxWDeficit = \begin{cases}
 	1 & \text{if } month \ge 10 \text{ or } month \le 3 \\
 WaterDeficitVegGrowth & \text{otherwise}
   	\end{cases} $     	
        
\quad else $MaxWDeficit = 
         	\begin{cases}
         	WaterDeficitLagPhase & \text{if not } Phenology.GrainFillStart()  \\       
         	WaterDeficitGrainFill & \text{otherwise}
       	\end{cases} 
       	$ 
   
\quad   if aSoil and $MaxWDeficit<1$ and $RootDepth>0$ then
  
\quad\quad   	$WaterStatus= 
   	\tfrac{aSoil.GetAvailWater(0,RootDepth  \cdot  1000)-aSoil.GetWiltCapacity(0,RootDepth  \cdot  1000)}
   	{aSoil.GetFieldCapacity(0,RootDepth  \cdot  1000)-aSoil.GetWiltCapacity(0,RootDepth  \cdot  1000)}
     $   
      
$IrrigationDemand() = (1-WaterStatus) > MaxWDeficit  $

  
\subsubsection*{StartBudget()}
	Nbudget.SetInput(Nitrogen.n), N15budget.SetInput(Nitrogen.n15)
	
  DMbudget.SetInput(DMVegTop+DMRoot+DMStorage)
  
  Nbudget.SetOutput(0), N15budget.SetOutput(0), DMbudget.SetOutput(0)


\subsubsection*{EndBudget(NRemain,DMRemain)}
 $NRemain=Nitrogen.n,  N15Remain = Nitrogen.n15$

$DMRemain=GiveDMVegTop()+DMRoot + DMStorage+SeedDM$ 

EndBudget(NRemain,DMRemain) = N15budget.Balance(N15Remain)


   
   
\subsubsection*{Harvest(Storage, Straw)} 
\mymarginpar{fracNminInVegTop}
harvest the current crop. The DM and straw 
is stored in the input parameters. Nitroget is updated. 
The plant is still alive.

   CutOrHarvested = true 
      
     converting from DM to fresh weight
	  
  $Storage.Setamount\left(\tfrac{DMStorage}{Storage.GetDM()}\right)$,   $Straw.Setamount\left(\tfrac{DMVegTop}{Straw.GetDM()}\right) $
  
   partition between compartments,  fN =  fNitrogen()
  \begin{align*}  
  TotalRootN  & = (MinN\_Root +fN  \cdot  (MaxN\_Root-MinN\_Root))   \cdot  DMRoot \\
  TotStorageN & = (MinN\_Store +fN  \cdot  (MaxN\_Store-MinN\_Store))   \cdot  DMStorage \\
   N15Ratio   & =  Nitrogen.Get15NRatio()  \\
   TotalStrawN & =  \left(\tfrac{NpctMinVegTop()}{100} + fN  \cdot  \tfrac{NpctMaxVegTop()-NpctMinVegTop()}{100}\right)  \cdot  DMVegTop \\
   dNrest & = NitrogenTotal - TotalRootN - TotStorageN - TotalStrawN 
\end{align*}
   rest of N is allocated to root.
   if Nrest positive: there is some unallocated N in the plant
   
   if $Nrest<-0.01$ or $Nrest>0.01$ then    
      TotalRootN = TotalRootN+Nrest 
   
   $ RootN.SetBoth(TotalRootN,N15Ratio  \cdot  TotalRootN) $
   
  $ StorageN.SetBoth(TotStorageN,N15Ratio  \cdot  TotStorageN)$
    
   StrawN = Nitrogen - RootN - StorageN 
   
   if $Storage.GetAmount()>0$ then      
   $	Storage.SetorgN\_content\left(\tfrac{StorageN}{Storage.GetAmount()}\right) $
   	
  if $Straw.GetAmount()>0$
  
  \quad $	Straw.SetNO3\_content\left(
        \tfrac{StrawN}{Straw.GetAmount()}  \cdot  (0.5  \cdot  fracNminInVegTop)\right)$
        
   \quad  $ Straw.SetNH4\_content\left(
        \tfrac{StrawN}{Straw.GetAmount()}  \cdot  (0.5  \cdot  fracNminInVegTop)\right)$
        
  \quad  $  Straw.SetorgN\_content\left(
        \tfrac{StrawN}{Straw.GetAmount()}  \cdot  (1-fracNminInVegTop)\right)$
 
$	NitrogenTotal = NitrogenTotal-StorageN-StrawN$

   Nbudget.AddOutput($StorageN.n+StrawN.n$) 
                     
  N15budget.AddOutput($StorageN.n15+StrawN.n15$)
                     
 DMbudget.AddOutput($DMVegTop+DMStorage$)   

	$DMVegTop   = 
	DMStorage   = 
	CAI_g  = 
	CAI_y  = 0 $
	
   $PlantHeight = 0$ (Modify to real cutting height!),   
 $ topMatter = NULL$ 
	
  EndBudget(NRemain,DMRemain) 
  
\subsubsection*{Terminate(Straw,DeadRoot,RootLengthList)} 
\mymarginpar{fracNminInVegTop, fracNminInRoots}
Kills the plant. The length and amount of the root and amount of straw is returned
 
   if $DMVegTop+DMStorage>0$ then
    $ Straw.Setamount\left(\tfrac{DMVegTop+DMStorage}{Straw.GetDM()}\right) $

	  $ StrawN = NitrogenTotal - NitrogenInRoot() $
	   
	  if $Straw.GetAmount()>0$ then
	   
		\quad	  $ Straw.SetNO3\_content\left(
			     \tfrac{StrawN}{Straw.GetAmount()}  \cdot  (0.5  \cdot  fracNminInVegTop)\right) $
			     
     \quad  $  Straw.SetNH4\_content\left(
           \tfrac{StrawN}{Straw.GetAmount()}  \cdot  (0.5  \cdot  fracNminInVegTop)\right) $
           
     \quad $   Straw.SetorgN\_content\left( 
           \tfrac{StrawN}{Straw.GetAmount()}  \cdot  (1-fracNminInVegTop)\right) $
 	  	  
 	 $  NitrogenTotal = NitrogenTotal-StrawN $
 
   $RootLengthList= GiveRootLengthList()$
    
   $DeadRoot.Setamount\left(\tfrac{DMRoot}{DeadRoot.GetDM()}\right) $
   
   if $DMRoot>0$  then 
  
 \quad $ 	DeadRoot.SetNH4\_content(
   	  \tfrac{NitrogenTotal}{DeadRoot.GetAmount()}  \cdot  0.5  \cdot  fracNminInRoots) $
 
  \quad $ 	DeadRoot.SetNO3\_content(
   	  \tfrac{NitrogenTotal}{DeadRoot.GetAmount()}  \cdot  0.5  \cdot  fracNminInRoots) $
 
 \quad $ 	DeadRoot.SetorgN\_content(
   	   \tfrac{NitrogenTotal}{DeadRoot.GetAmount()}  \cdot  (1 - fracNminInRoots)) $
 
Nbudget.AddOutput(Nitrogen.n),                
N15budget.AddOutput(Nitrogen.n15)

DMbudget.AddOutput(DMRoot) 	

	RootDensAtSurf = 
	TotalRootLength  = 
	RootDepth      = 
	DMRoot   = 
   PlantHeight     = 0
   
   terminated      = true  



\textbf{SetRUEfactor(f)}
Scales the radiation use efficiency according to crop rotation and
pesticide use. Calling this function should remain optional.

$MaxRadUseEff = MaxRadUseEff  \cdot  f$

Calculate how much water evaporates \citep{olesen2000sensitivity}.
   
$\mathbf{GiveEvapFactor()} = 1+0.02  \cdot  (CAI_g+CAI_y) $

$\mathbf{InterceptionCapacity()} = InterceptCoeff  \cdot  (CAI_g+CAI_y)$  \mymarginpar{InterceptCoeff}

\minisec{AssignRootParameters(CropSoil)}
  \begin{align*}
   aSoil       & = CropSoil \\
   NumOfLayers & = \min(aSoil.GetLayers(), MaxSoilLayers) \\
	MaxRootDepth & = \min\left(\tfrac{aSoil.GetMaxRootDepth()}{1000},MaxRootDepthCrop\right)
	\end{align*}

\minisec{MaxTransferable()}
$Nsurplus = NitrogenTotal-Nmin()$, $MaxTransferable() = \tfrac{Nsurplus}{MinN\_Store}$   

\textbf{SetDSAfterCut()}: DS = 0.05 


\section{Plant - Clover}


\subsection{cropClover(const cropClover acrop) : cropRyegrass(acrop)}

  $MaxFixPerDM=acrop.MaxFixPerDM$

  $DMCostPerN=acrop.DMCostPerN$

  $NFixationThisDay=acrop.NFixationThisDay$


\subsection{Add(crop aCrop, frac)}

   $cropRyegrass.Add(aCrop,frac)$

   $cropClover c = (cropClover) aCrop$

   $MaxFixPerDM=(1-frac)  \cdot  MaxFixPerDM + frac  \cdot  c.MaxFixPerDM$

   $DMCostPerN=(1-frac)  \cdot  DMCostPerN + frac  \cdot  c.DMCostPerN$

   $NFixationThisDay=(1-frac)  \cdot  NFixationThisDay + frac  \cdot  c.NFixationThisDay$


\subsection{CalcPotentialNFixation(dDryMatt)}

   (estimated that fix first starts after 20 days with 10 \textdegree C (Crush: Nitrogen fixation))

   if $Phenology.TempSumForLeaf<200$ then  return $0$                          
 	

   $fTemp = 1$  
                                          
   (Temp response from \citet[]{wu1999simulation})

   Response of N fixation to soil Temp  \citep{liu2011models}
  
   $ftemp = \begin{cases}
             \max\left(0,min\left(1,\tfrac{30-temp}{4}\right)\right) & \text{ if temp > 26} \\
             \max\left(0,min\left(1,\tfrac{temp-9}{4}\right)\right)  & \text{ otherwise}
            \end{cases}$

   return $dDryMatt  \cdot  MaxFixPerDM  \cdot  fTemp$


\subsection{NitrogenAndTranspiration(aNitrogenUptake, aTranspirationRatio)}


   $TranspirationRatio = aTranspirationRatio$

   if $TranspirationRatio<0.5$  and  $YellowCAI+GreenCAI<0.01$  then

   \quad (corrects problem with transpiration the day of germination)

   \quad $TranspirationRatio = 1$

   if $aNitrogenUptake.n>1e-15$ then

     \quad  $Nitrogen += aNitrogenUptake$, $AccumulatedNProduction += aNitrogenUptake$

     \quad $Nbudget.AddInput((aNitrogenUptake).n)$,     
      $N15budget.AddInput((aNitrogenUptake).n15)$
   
   if $aNitrogenUptake.n<0$ then

      \quad warn("cropClover.NitrogenAndTranspiration - nitrogen uptake is negative")
   
   $dDM =  \Delta DMer()$

   $PotNFix   = CalcPotentialNFixation(dDM)$

   $NFixationThisDay = min(NitrogenDemand(),PotNFix)$

   if $NFixationThisDay>0$ then
 $Nfix.SetBoth(NFixationThisDay,0)$                   

\quad (IMPROVE THIS TO ACCOUNT FOR REAL DISTRIBUTION OF N15!)

\quad $Nitrogen = Nitrogen + Nfix$

    \quad   $DMCost = NFixationThisDay  \cdot  DMCostPerN$

     \quad  $DMRoot   -= DMCost  \cdot  fracToRoot()$

      \quad $AgeClassTop_0 -= DMCost  \cdot  (1-fracToRoot())$

     \quad  $DMbudget.AddOutput(DMCost)$
   
\quad $AccumulatedNProduction += Nfix$

   \quad $Nbudget.AddInput((Nfix).n)$,  
  $N15budget.AddInput((Nfix).n15)$

   \quad $AccumulatedNFixation += NFixationThisDay$
  
   else

\quad $NFixationThisDay=0$


\section{Plant - Ryegrass}


%cropRyegrass.cropRyegrass(const char aName, const int aIndex, const base aOwner, string cropName )
%             :crop(aName,aIndex,aOwner, cropName)
%
% //  For complete nomenclature see base class 'crop'

%   PlantItemName  = "RYEGRASS"         //  No primarly harvest item
%   StrawItemName  = "RYEGRASS"         //  These names maps to products.dat !
%   WinterSeed     = true               //  Is the present crop a wintercrop.
%   GrowthContinue = true
%   HasBeenCut = false
%	grazableDM=0
%   grazableN = 0

 %  // ---- Parameters -----------------------------------------------------------


% 	// Dry matter accumulation


%   commonData  \cdot  data=globalSettings.CropInformation
%   data.FindItem("\minDayLength",&\minDayLength)

%	//  DM allocation
%   //  Parameters below adjusted so that the total carbon flow matches data from Paustian et al. 1990. J. Appl. Ecol., 27, 60-84

%   data.FindItem("RespirationRoot",&RespirationRoot) //  Root respiration given as a daily frac of DM at 10 deg. Celsius, Calibrated to Burreh?jvej data
   
%data.FindItem("RespirationTop",&RespirationTop)   //  Top respiration given as a daily frac of DM at 10 deg. Celsius

%   data.FindItem("TransferRate",&TransferRate) //  frac (Temp-dependend) of the four age pools that is transferred to subsequent frac

%   data.FindItem("BulkDensity",&BulkDensity)   //  g DM/m3 standing crop - guesstimated

%   data.FindItem("Recyclefrac",&Recyclefrac)


%   //  LAI and canopy structure

%   data.FindItem("InitialLAIIncrease",&InitialLAIIncrease)

 %  data.FindItem("liveOMD",&liveOMD)       	//  Digestibility of first 3 age classes

%   data.FindItem("deadOMD",&deadOMD)       	//  Digestibility of last age class

%   data.FindItem("RateOfRipeReSeed",&RateOfRipeReSeed)



 %    		// Rate of storage reseeding

%   //  Relative availability of ages classes, according to the values for
%   //  Thornleys age classes of leaves.
%   //  Note that Thornleys age class avalabilities for sheth + stem all are unity!!!
%
%   $uptakeWeight_0 = 1$, $uptakeWeight_1 = 0.75$, $uptakeWeight_2 = 0.5$, $uptakeWeight_3 = 0.25$
%
%   Nbudget.SetNames(CropName,"N")
%
%   DMbudget.SetNames(CropName,"DM")
%
%   //  State variables ----------------------------------------------------------
%   TempSumAfterCut = 0                                             //  Temp sum after last cut
%   CutDelay = 0    
%                                                       //  The Temp sum nessecary for obtaining 0.5 LAI after a cut
%   for(int i=0i<4i++)
%   
%   \quad grazedDM_i=0
%
%   \quad AgeClassTop_i=0
%   
%   ReSeedDM = 0



%cropRyegrass.cropRyegrass(const cropRyegrass& acrop)
%   : crop(acrop)
%{
%   BulkDensity            = acrop.BulkDensity
%   RespirationTop         = acrop.RespirationTop
%   RespirationRoot        = acrop.RespirationRoot
%   TransferRate           = acrop.TransferRate
%   Recyclefrac        = acrop.Recyclefrac
%   TempSumAfterCut = acrop.TempSumAfterCut
%   CutDelay               = acrop.CutDelay
%   liveOMD                = acrop.liveOMD
%   deadOMD                = acrop.deadOMD
%   grazableDM             = acrop.grazableDM
%   grazableN              = acrop.grazableN
%   for (int i=0i<4i++)
%   {
%      AgeClassTop_i      = acrop.AgeClassTop_i%	  
% grazedDM_i         = acrop.grazedDM_i
%      uptakeWeight_i     = acrop.uptakeWeight_i
%   }
% }



\subsection{Add(aCrop, frac)}

   crop.Add(aCrop,frac)

  $ cropRyegrass  \cdot  c = (cropRyegrass \cdot )aCrop$

  $ BulkDensity            =(1-frac) \cdot BulkDensity + frac \cdot c.BulkDensity$
  
 $RespirationTop         =(1-frac) \cdot RespirationTop + frac \cdot c.RespirationTop$

  $ RespirationRoot        =(1-frac) \cdot RespirationRoot + frac \cdot c.RespirationRoot$

  $ TransferRate           =(1-frac) \cdot TransferRate + frac \cdot c.TransferRate$

   $Recyclefrac        =(1-frac) \cdot Recyclefrac + frac \cdot c.Recyclefrac$

   $TempSumAfterCut =(1-frac) \cdot TempSumAfterCut + frac \cdot c.TempSumAfterCut$

  $ CutDelay               =(1-frac) \cdot CutDelay + frac \cdot c.CutDelay$
 
  for (int i=0i<4i++)

\quad       $AgeClassTop_i=(1-frac) \cdot AgeClassTop_i+frac \cdot c.AgeClassTop_i$

\quad 	   $grazedDM_i=(1-frac) \cdot grazedDM_i+frac \cdot c.grazedDM_i$

 \quad     $ uptakeWeight_i     = (1-frac) \cdot uptakeWeight_i+frac \cdot c.uptakeWeight_i$
   



%cropRyegrass  \cdot  cropRyegrass.clone()
%{
%   cropRyegrass \cdot  aClone = new cropRyegrass( \cdot this)
%   return aClone
%}





\subsection{GrazeOrCut(decomposable Hay,fracLeft,graze)}

\citep[Eq. 12]{berntsen2005simulation}

An externally given cutting height should replace fracLeft !!!
JBE implemented this 

   warn("cropRyegrass.GrazeOrCut should never be called - use cropRyegrass.Cut instead.")
  
 $CutOrHarvested = true$

   $PlantHeight = PlantHeight \cdot fracLeft$

   if $GiveDMVegTop() \le 0$ then

\quad      warn("cropRyegrass.Cut/Graze - vegetative top must be bigger than zero here")

   Hay.Setname(StrawItemName),
   theProducts.GiveProductInformation(Hay)

   Hay.RemoveWater()

   $aDMInVegTop = GiveDMVegTop()$

   $nitrogen aNitrogenInVegTop = NitrogenInVegTop()$

   if $aDMInVegTop > 0$ then
  
\quad 		Hay.Setamount($aDMInVegTop \cdot (1-fracLeft)$)
      
\quad $Hay.SetorgN\_content(\tfrac{aNitrogenInVegTop}{aDMInVegTop \cdot (1-fracLeft)}$)   
   
   else
   
\quad 	warn("RyeGrass. Nothing to cut/graze")


   $WeightedAverage = 0 $                      

 //  JB implemented the selectively grazing 6/9/2002
  
 $IntakeSelectivity [] = \{0.4,0.3,0.2,0.1\}$

\#ifdef BIOMODTEST

if $ not graze$ then
   
  \quad    $ TempSumAfterCut = 0   $                 
      
\quad  $CutDelay = 300 \cdot DMTotalStandVegTop  \cdot \tfrac{1-fracLeft}{350} $     
\citep[Eq. 12]{berntsen2005simulation}


      
\quad  for (int i=0i<4i++) $IntakeSelectivity\_i = 0.25$
   
\#endif
  
      $nitrogen HayN =  Hay.GetorgN_content()  \cdot  Hay.GetAmount()$

   $Nitrogen = Nitrogen - HayN$

  $ RemoveDM = (1-fracLeft)  \cdot  GiveDMVegTop()$       //  DM to remove
	
Removed = RemoveDM
	
while $Removed>1e-10$ do
    

  \quad 	$WeightedAverage += \sum_{i=0}^3 IntakeSelectivity_i \cdot AgeClassTop_i$

   \quad 	for (int i=0i<4i++)
   	
  \quad  \quad       $DMRemove = \min\left(AgeClassTop_i,RemoveDM \cdot IntakeSelectivity_i 
  \cdot \tfrac{AgeClassTop_i}{WeightedAverage}\right)$

   \quad \quad 		$AgeClassTop_i -= DMRemove$

  \quad  \quad       $Removed -= DMRemove$
   	
   \quad    $RemoveDM = Removed$
  

   $DMVegTop=GiveDMVegTop() $      //  updates DMvegtop

   $CalcLeafAreaIndices()$

   Nbudget.AddOutput(HayN.n), N15budget.AddOutput(HayN.n15),

   DMbudget.AddOutput(Hay.GetAmount())
  
   EndBudget(NRemain,DMRemain),
   HasBeenCut=true

   Phenology.SetDSAfterCut()









\subsection{RootDecay(RootTranslocation)}

   crop.RootDecay(RootTranslocation),
   RootRespiration = 0

   //  Root respiration

   if $DMRoot>0$  and  $temp>0$ then
   
  \quad     SoilTemp=aSoil.GetTemp(200)

   \quad    $RootRespiration  = DMRoot \cdot RespirationRoot \cdot TempEffect(SoilTemp)$

   \quad    DMRoot -= RootRespiration

   \quad    DMbudget.AddOutput(RootRespiration)

   \quad    AccumulatedRootRespiration += RootRespiration

    \quad   AccumulatedDMProduction -= RootRespiration         

//  Respiration are not included in this budget!

   \quad    RootTranslocation -=  RootRespiration   
   
   minimumN = Nmin(), maximumN = Nmax()

   if $DMStorage + DMRoot + DMVegTop > 0$  and  $Nitrogen.n > 0.5$ then
  
 	if $minimumN > Nitrogen.n  \cdot  1.25$ then 

   	\quad\quad	warn("crop.RootDecay - not enough nitrogen in plant")
   	
   	\quad  if $maximumN < Nitrogen.n  \cdot  0.75$ then 	

	 \quad\quad 	warn("crop.RootDecay - too much nitrogen in plant")
    
  





\subsection{TopRespiration()}
\citep[Eq. 11]{berntsen2005simulation}

   $topRespiration=0$



   RespirationClass[] = \{1,1,0.75,0.5\}      

	for (int i=0i<4i++)
   

   \quad   $RespCoeff = RespirationClass_i \cdot RespirationTop \cdot TempEffect(temp)$

   \quad   $r=AgeClassTop_i \cdot RespCoeff$

    \quad  $AgeClassTop_i -= r$,
    $topRespiration += r$
  
   DMbudget.AddOutput(topRespiration)

   AccumulatedDMProduction -= topRespiration

   minimumN = Nmin(), maximumN = Nmax()

   if $DMStorage + DMRoot + DMVegTop > 0 $
and $Nitrogen.n > 0.5$ then
  
   \quad if $minimumN > Nitrogen.n  \cdot  1.25$ then
   	

   \quad\quad   	warn("crop.TopRespiration - not enough nitrogen in plant")
   	
  \quad 	if $maximumN < Nitrogen.n  \cdot  0.75$
   	
 
\quad\quad	   	warn("crop.TopRespiration - too much nitrogen in plant")
   	
  

   return topRespiration



\subsection{UpdateAgeClasses(deltaDM)}



   ReducFac = 1

   grazableDM =GetAvailableDM(0)

   $	grazableN = \begin{cases}
           GetAvailableN(0) & \text{ if } grazableDM>0 \\
   	                 0  & \text{ otherwise}
   	            \end{cases}$

   if $YellowCAI+GreenCAI > 0$ then

   \quad 	$ReducFac = \tfrac{ReducedNInYellow \cdot YellowCAI+GreenCAI}{YellowCAI+GreenCAI}$
   
   aNitrogenInVegTop = NitrogenInVegTop()

   aDMVegTop = GiveDMVegTop()

if $topMatter \ne NULL$ then delete topMatter, topMatter=NULL
  
   $T=TransferRate \cdot TempEffect(temp)$

  $ transfer_0=\Delta DM$,
   $transfer_1=2 \cdot T \cdot AgeClassTop_0$

   $transfer_2=T \cdot AgeClassTop_1$,
  $ transfer_3=T \cdot AgeClassTop_2$

   RF = Recyclefrac

   if Phenology.Anthesis() then $RF = 0.5 \cdot Recyclefrac$


   $senescenceDM= (1-RF) \cdot T \cdot AgeClassTop_3$

   if $Phenology.DS > 1.5$ then $senescenceDM = T \cdot GiveDMVegTop()$

   $recycledDM=RF \cdot T \cdot AgeClassTop_3$

  $ transfer_0+=recycledDM  $

   if $senescenceDM>0$ then

 \quad  	$NConcentrationInYellow = ReducedNInYellow \cdot \tfrac{aNitrogenInVegTop.n}{aDMVegTop/ReducFac} $


\quad  // 		$TotalLostN = NConcentrationInYellow \cdot (1-RF) \cdot senescenceDM$

\quad 		$TotalLostN = NConcentrationInYellow \cdot senescenceDM$

 \quad      $ senescenceN.SetBoth(TotalLostN \cdot (1-Nitrogen.Get15NRatio()),TotalLostN \cdot Nitrogen.Get15NRatio())$

 \quad      if $senescenceN.n<0$ then
   	

 \quad   \quad    warn("cropRyeGrass. negative N loss"),  senescenceN.Clear()
     

 \quad      //  Transfer DM and N senesced to "topMatter"

\quad       topMatter=new decomposable

 \quad      topMatter.Setname("RYEGRASS")

\quad       theProducts.GiveProductInformation(topMatter)

 \quad    	topMatter.Setamount($0.01 \cdot \tfrac{senescenceDM}{topMatter.GetDM()}$) 

//  Conversion from g/m2 to t/ha

 \quad     $ topMatter.SetorgN\_content(\tfrac{senescenceN}{\tfrac{senescenceDM}{topMatter.GetDM()}})$

 \quad     $ AccumulatedTopDeposit += senescenceDM $    

\quad //  Accumulated top deposition (state variable)

 \quad   	$AccumulatedTopNDeposit = AccumulatedTopNDeposit + senescenceN$


  $ AgeClassTop_0+=transfer_0-transfer_1 - grazedDM_0$
  
 if $AgeClassTop_0<0$ then

\quad      warn("cropRyeGrass.UpdateAgeClasses - grazing (or transfer) has resultet in negative DM in age class 0")
 
\quad        $AgeClassTop_0=1$ //  Hack!
 

  $ AgeClassTop_1+=transfer_1-transfer_2 - grazedDM_1$

   if $AgeClassTop_1<0$ then

 \quad        warn("cropRyeGrass.UpdateAgeClasses - grazing (or transfer) has resultet in negative DM in age class 1")

 \quad       $ AgeClassTop_1=1$ //  Hack!

  $ AgeClassTop_2+=transfer_2-transfer_3 - grazedDM_2$

   if $AgeClassTop_2<0$ then

  \quad       warn("cropRyeGrass.UpdateAgeClasses - grazing (or transfer) has resultet in negative DM in age class 2")

\quad         $AgeClassTop_2=1$ //  Hack!

   $AgeClassTop_3+=transfer_3$

$   transferFromAgeClasses = recycledDM + senescenceDM + grazedDM_3$

   if $transferFromAgeClasses>0$ then

 \quad     	RemoveDMFromOldestAgeClass(transferFromAgeClasses)

   if $AgeClassTop_3<0$ then

 \quad        warn("cropRyeGrass.UpdateAgeClasses - grazing (or transfer) has resultet in negative DM in age class 3")

 \quad       $ AgeClassTop_3=1 $ //  Hack!

   // Budget for grazing losses - NJH 13.12.2000

	$totDMGrazed = grazedDM_0 + grazedDM_1 + grazedDM_2 + grazedDM_3$

   NC = 0

   if $grazableDM>1E-3$ then 
       $ NC = \tfrac{grazableN}{grazableDM}$

  $TotalgrazedN = NC \cdot totDMGrazed $//  To correct for changed conditions relative to corresponding prior step

   grazedN.SetBoth($TotalgrazedN \cdot (1-Nitrogen.Get15NRatio())$,

   \quad  $TotalgrazedN \cdot Nitrogen.Get15NRatio()$)

   nitrogen lostN=senescenceN + grazedN,
   lostDM=senescenceDM + totDMGrazed

   Nitrogen = Nitrogen - lostN

   Nbudget.AddOutput(lostN.n),
   N15budget.AddOutput(lostN.n15)

   DMbudget.AddOutput(lostDM)

   if (Nitrogen.n<0)

 \quad        warn("cropRyeGrass.UpdateAgeClasses - grazing has resultet in negative N")

  \quad      $ Nitrogen.n=0.1 $ //  Hack!

   /* zero grazedDM and grazedN 

   Now reset in ClearTemporaryVariables, called from grazing\_manager\_class

   for(int i=0i<4i++)  $grazedDM_i = 0$

    */

   EndBudget(N,DM)

   return $\Delta DM-lostDM$



\subsection{GiveDMVegTop()}
   $retVal = \sum_{i=0}^3 AgeClassTop_i$

   if $retVal<0$ then
      warn("cropRyegrass.GiveDMVegTop - negative dry matter")

   return retVal

\subsection{CalcLeafAreaIndices()}

\textbf{DeadDMTop()} = $ 0.5 \cdot AgeClassTop_3$

   if $Phenology.Emerged()$  and  $GiveDMVegTop()-DeadDMTop()>0$ then
	
\quad   	   if $Phenology.TempSumForLeaf \le Phenology.LinearLeafPhase$ then

 \quad  \quad      $GreenCAI =  InitialCAI \cdot RelativeDensity$ 

\quad  \quad \quad  \quad   $\tfrac{Phenology.TempSumForLeaf}{Phenology.LinearLeafPhase} $  (value from Porter)
    
  else
     

\quad    \quad     if HasBeenCut  or  UnderSown then  

\quad    \quad    \quad   //  if the grass has not been cut and is not undersown => more vigorous growth

 \quad   \quad   \quad  $ GreenCAI  = pow(LAIDMRatio \cdot (GiveDMVegTop()-DeadDMTop()),0.726)  $ 
  
 \quad   \quad   \quad // this power is from Lemaire: Diagnosis of the n status in crops pp65.

\quad     \quad          else     
                        
\quad     \quad  //  if the grass has not been cut and is not undersown => more vigorous growth

  \quad   \quad   \quad            $ GreenCAI  = InitialLAIIncrease $

\quad   \quad   \quad $\cdot pow(LAIDMRatio \cdot (GiveDMVegTop()-DeadDMTop()),0.726) $    //  estimate

  

   else
\quad   	   $GreenCAI = 0$

   //  JB removed the out-commenting of the cut delay

   if $TempSumAfterCut<CutDelay$  and  $DMTotalStandVegTop>0$ then

 \quad     	$GreenCAI = 0.5 \cdot \tfrac{TempSumAfterCut}{CutDelay}$

\quad \quad  $\cdot GiveDMVegTop()/DMTotalStandVegTop$

   $YellowCAI = LAIDMRatio \cdot DeadDMTop()$


\subsection{Update(ActivePar)}

   DailyDMGrowth = 0

//    rootMatter = NULL    NJH pinged this out March 2009
	
Rg 	= ActivePar
  
 temp = theClimate.tmean

	if $Phenology.Sown()$ then                            //  Never stops
  
      if $aSoil = NULL$ then
         error("cropRyegrass.Update - Invalid soil pointer")

   	soilTemp=aSoil.GetTemp(20)

      DayLength = theClimate.PhotoPeriod()

 //      if $ not Phenology.Anthesis()$  or  $ not UnderSown$ then
      	Phenology.Update(temp,soilTemp,DayLength)      //  updates phenology


		$TempSumRoot += \max(0,temp)$

     $ TempSumAfterCut += \max(0,temp)$

		UpdateHeight()

     $ \Delta DM=\Delta DM()$

      $DailyDMGrowth = \Delta DM$

   	$DMbudget.AddInput(\Delta DM)  $            

      $AccumulatedDMProduction += \Delta DM$

      if $Phenology.TempSumForLeaf \le 200$ then
     
   \quad        $ DMTransfer=2 \cdot Topfrac \cdot InitialSeedDM \cdot \tfrac{\max(0,temp)}{Phenology.LinearLeafPhase}$
   
 \quad       $ DMTransfer = \min(DMTransfer,SeedDM)$

   \quad        $ SeedDM -= DMTransfer$

   \quad         if $ReSeedDM >0$ then
        
    \quad   \quad        	$ReSeedDMTransfer = 2 \cdot Topfrac \cdot InitialSeedDM 
\cdot \tfrac{\max(0,temp)}{Phenology.LinearLeafPhase}$

    \quad   \quad        	$ReSeedDMTransfer = \min(ReSeedDMTransfer,ReSeedDM)$

     \quad   \quad         $ DMTransfer += ReSeedDMTransfer$

         
   \quad        $ AgeClassTop_0 += 0.5  DMTransfer$

   \quad      $   DMRoot  += 0.5  DMTransfer$
     

      $TransferDMToRoot(\Delta DM)$

      CalcRootGrowth()

      	$TransferDMToStorage(\Delta DM)$

      $DeltaDMTop = UpdateAgeClasses(\Delta DM)$

      Respiration = TopRespiration()

      DeltaDMTop -= Respiration,  DailyDMGrowth -= Respiration

      DMVegTop=GiveDMVegTop()

      if (Phenology.Ripe()  or  DayLength < minDayLength) then

     \quad 	RipeSenescence()    //  Still some debugging to do!

      CalcLeafAreaIndices()


	
   return DailyDMGrowth


\subsection{RipeSenescence()}
Ripe senescence including DS setback and loss of grains to topmatter

   Phenology.SetDSAfterCut()

   if $DMStorage> 0$ then
   
   \quad	nitrogen aNitrogenInStorage = NitrogenInStorage(),
   aDMStorage = GiveDMStorage()

\quad		$RSeneStorage=\min(RateOfRipeReSeed \cdot TempEffect(temp) \cdot aDMStorage,$

  \quad\quad                  $aDMStorage)$

   \quad	if $RSeneStorage>0$ then
   	

   \quad  \quad 	DMStorage -= RSeneStorage 

     \quad  \quad    ReSeedDM += RSeneStorage //  Transfer DM to ReSeedDM
  

     % // Hvad mere??


\subsection{UpdateHeight()}


JBE's original formulation, seems to work very well with undersown crops

$PlantHeight = \begin{cases}
                PlantHeight + \tfrac{\max(0,temp)}{1000} \cdot maxPlantHeight & \text{ if Phenology.Emerged()} \\
                0 & \text{ otherwise}
           \end{cases}$


 $PlantHeight = \min\left(\min\left(PlantHeight,maxPlantHeight\right),\tfrac{DMTotalStandVegTop}{BulkDensity}\right) $

 But modifiy if DM becomes limiting. Transitions between undersown crops and removal of previous main crop might need later attention !!!

\subsection{TransferDMToStorage(dDryMatt)}


	if Phenology.Anthesis()  and  FillFlag = 0 then
  
   \quad 	if $TransferableStorage<1E-8$ then

    \quad \quad   	$TransferableStorage = GiveDMVegTop() \cdot StoreForFilling$

     \quad \quad  $TransferableStorage +=  \cdot dDryMatt \cdot FillFactor$

    \quad \quad   TransferedDMToStorage = TransferableStorage



   
   if $Phenology.GrainFillStart()$ then
 
   \quad 	FillFlag = 1

    \quad   if $fNitrogen() > 0$ then
    
    \quad \quad   	$transfer = \max(0,TransferableStorage \cdot Phenology.GetfracOfGrainFill())$     

\quad \quad  //  linear fill of DM stored during lag phase
    
  \quad \quad    RemoveDMFromOldestAgeClass(transfer)
   
 \quad \quad     $ DMStorage += transfer+ \cdot dDryMatt \cdot FillFactor \cdot ConversionCoefficient$


    \quad \quad		minimumN = Nmin(),
   		maximumN = Nmax()

     \quad \quad  	if $minimumN > Nitrogen.n + 0.01$ then    

  \quad \quad  // Check nitrogen status and transfer back
       
      \quad \quad  \quad     $ transferBack = \max(0,\min(transfer,\tfrac{minimumN - Nitrogen.n}{minN\_Store})$

     \quad \quad \quad     	RemoveDMFromOldestAgeClass(-transferBack)

       \quad \quad \quad   	DMStorage -= transferBack, transfer = transfer - transferBack
      
\quad \quad   
   		if $maximumN < Nitrogen.n - 0.01$     // Check nitrogen status and transfer back
   		
    \quad \quad \quad         $transferBack = \max(0,\min(transfer,\tfrac{Nitrogen.n - maximumN}{maxN\_Store})$
  
   \quad \quad \quad     	RemoveDMFromOldestAgeClass(-transferBack)
 
    \quad \quad \quad     	DMStorage -= transferBack, transfer = transfer - transferBack

 	     	
      \quad \quad   $ DMbudget.AddOutput((dDryMatt \cdot FillFactor) \cdot (1-ConversionCoefficient))$
   
  \quad \quad     $dDryMatt =  dDryMatt \cdot (1-FillFactor)$
    
  \quad \quad    TransferedDMToStorage -= transfer



 if (Phenology.GrainFillEnd()  and  TransferedDMToStorage>0)     
       
 \quad //  Check to see if all transferable storage is transfered
  
\quad      $transfer=\max(0,TransferedDMToStorage)$

\quad   	DMStorage += transfer

\quad      RemoveDMFromOldestAgeClass(transfer)

 \quad     minimumN = Nmin(), 	maximumN = Nmax()

\quad      if $minimumN > Nitrogen.n + 0.01$  then     // Check nitrogen status and transfer back
     
\quad\quad        $ transferBack = \max(0,\min(transfer,\tfrac{minimumN - Nitrogen.n}{minN\_Store})$

\quad\quad      	RemoveDMFromOldestAgeClass(-transferBack)

 \quad\quad       	DMStorage -= transferBack,  transfer = transfer - transferBack
     
  \quad 	if $maximumN < Nitrogen.n - 0.01$ then   // Check nitrogen status and transfer back
   	
 \quad\quad       $ transferBack = \max(0,\min(transfer,\tfrac{Nitrogen.n - maximumN}{maxN\_Store})$

 \quad\quad      $	RemoveDMFromOldestAgeClass(-transferBack)$

  \quad\quad      $DMStorage -= transferBack$,
       $transfer = transfer - transferBack$
      
\quad      TransferedDMToStorage = 0






\subsection{GiveEvapFactor()}
   return $1$

%\subsection{GiveCropHeight()}
%   return $PlantHeight$



\subsection{RemoveDMFromOldestAgeClass(transferFromAgeClass)}
Removes DM from oldest ageclass

    remove=0

   for(int i=0i<4i++)
   
   \quad $remove=\min(AgeClassTop[3-i],transferFromAgeClass)$

    \quad  AgeClassTop[3-i] -= remove, 
    transferFromAgeClass -= remove
   




\subsection{SetGrazed(grazingheight, DMGrazed)}
Adds an animal's contribution to the patch's grazed material
Each animal calls this routine if it grazes this patch

   $grazableDM = GetAvailableDM(0)$

   $proportionCut=\tfrac{DMGrazed}{grazableDM}$

   $checkDM=0$

   for(int i=0i<4i++)
   
      \quad (Material is removed by grazing in proportion to its contribution to grazable DM)

       \quad if $grazableDM>0$ then
      
      	 \quad \quad $grazedDM_i+=proportionCut  \cdot  AgeClassTop_i  \cdot  uptakeWeight_i$

          \quad \quad $checkDM+=grazedDM_i$

   


   

\subsection{ClearTemporaryVariables()}
   for(int i=0i<4i++) do $grazedDM_i=0$




\subsection{Grazing(grazedDM)}  
	   nitrogen cutN = 0

	   initialDM=GiveDMVegTop()+DMStorage

	  $ proportionCut = \tfrac{grazedDM}{initialDM}$

	   if $proportionCut>1$ then error("cropRyegrass.Grazing - Proportion cut > 1")

	   nitrogen  aNitrogenInVegTop = NitrogenInVegTop()
	 
  for(int i=0i<4i++)
	 
	 \quad     $ tempDM = proportionCut \cdot AgeClassTop_i$

	  \quad     $AgeClassTop_i-= tempDM $  // cut material
	 
      $cutN = aNitrogenInVegTop \cdot proportionCut$

      Nitrogen = Nitrogen-cutN

	   if $DMStorage > 0$ then
	  
	    \quad	  nitrogen aNitrogenInStorage=NitrogenInStorage().n

	  \quad  	 $ DMStorage \cdot =(1-proportionCut)$

	  \quad     $aNitrogenInStorage.n \cdot =(1-proportionCut)$

	  \quad     Nitrogen.n -=  aNitrogenInStorage.n
	  


\subsection{Getter}

\minisec{GetGrazedDM()}
amount of DM grazed during the last period, then zero the values

  return  $\sum_{i=0}^3 grazedDM_i$


\minisec{GetAvailableDigestibleDM(residualDM)}

   $retVal=0$, $residualDM/=4$, $tempDM=0$

   for(int i=0i<4i++)
   
   \quad $tempDM=AgeClassTop_i  \cdot  uptakeWeight_i - residualDM$

    \quad	if $tempDM>0$ then
     
     \quad      \quad if $i<3 $

     \quad      \quad  \quad then $retVal += tempDM  \cdot  liveOMD$

       \quad    \quad  \quad else $retVal += tempDM  \cdot  deadOMD$
   
   
   return $retVal$


\minisec{feedItem cropRyegrass.GetAvailability(cutHeight, int animalType)}
Calculates grazable DM, N,C. Loads information on the available material into myProduct
Returns amount in g fresh weight/sq metre

$feedItem myProduct = new feedItem((feedItem)theProducts.GetProductElement(465))  $

set product to grass, 100kgN, cut after 2 weeks (N content will be modified)

myProduct.Setamount(0)

   OMD=
   tempDigest = 
   grazableDM =
   grazableN = 0

   $N\_conc = NitrogenInVegTop().n/GiveDMVegTop()$

   $liveOMD = 0.4 + 15.4  \cdot  (N\_conc-0.006)$

   $liveOMD = \begin{cases}
              0.8 & \text{if $liveOMD>0.8$} \\
              deadOMD  & \text {if $liveOMD<deadOMD$}
             \end{cases}$


   $grazableDM=GetAvailableDM(0)$

   $tempDigest= GetAvailableDigestibleDM(0)$

   if (grazableDM>0)
   	
       \quad    grazableN =GetAvailableN(0)

       \quad    $myProduct.Setamount(\tfrac{grazableDM}{myProduct.GetDM()})$

        \quad   $myProduct.SetorgN\_content(grazableN/myProduct.GetAmount())$

  
        
         \quad     if $animalType=1$ then
           
         \quad   \quad     $ OMD=\tfrac{tempDigest}{grazableDM}$,
               myProduct.SetOMD(OMD),
           myProduct.CalcME()

         \quad    \quad     myProduct.CalcFE(),
          myProduct.CalcFill(2)

        \quad       \quad   $MEPerKg = 0.16 \cdot myProduct.GetOMD() \cdot 100 - 1.8$  //  from SCA 1990

        \quad    \quad      $myProduct.SetME(MEPerKg  \cdot  myProduct.GetDM())$

       \quad         
       \quad      

       \quad       else 

 \quad    \quad    error("cropRyegrass.GetAvailability - not an allowed animal type")

      \quad           
       

	return myProduct





\minisec{GetAvailableStandingDM(cutHeight,bool useUptakeWeight)}

Returns DM above the 'cut height'


   $grazableDM = 0$

   for(int i=0i<4i++)
   
    \quad  $temp\_height=PlantHeight$, $cutH=\\max(cutHeight,0.01)$

    \quad $proportionCut= \begin{cases}
               \tfrac{temp\_height - cutH}{temp\_height} &   \text{ if } temp\_height>cutH \\
               0 & \text{ otherwise}        \end{cases}$

    \quad    w=1

    \quad if useUptakeWeight then  $ w=uptakeWeight_i$

    \quad  $tempDM=proportionCut  \cdot  AgeClassTop_i  \cdot  w$
     
    \quad   $grazableDM +=tempDM$
 
   return $grazableDM$


\minisec{GetAvailableDM(residualDM)}

   $retVal=0$, $residualDM/=4$, $tempDM=0$

   for(int i=0i<4i++)
  
   \quad $tempDM=AgeClassTop_i  \cdot  uptakeWeight_i - residualDM$

   \quad if $tempDM>0$ then $retVal+=tempDM$
  
   return $retVal$



\minisec{GetAvailableN(residualDM)}
 
   availDM = GetAvailableDM(residualDM)

   return $\tfrac{availDM}{GiveDMVegTop()}  \cdot  NitrogenInStorage().n + NitrogenInVegTop().n$


\subsection{Cut(plantItem cutPlantMaterial, cutheight)}

Cut and return a plantItem. returns cut material in g/sq metre


   if $cut\_height<0$ then $cut\_height=0.01$

   if $GiveDMVegTop() \le 0$ then

   \quad   warn("cropRyegrass.Cut - vegetative top must be bigger than zero here")

/* 
   \quad  grazableDM = 0

   for(int i=0i<4i++)

    \quad    $temp\_height=PlantHeight$,
       $cutH=\max(cutHeight,0.01)$

     \quad   if $temp\_height>cutH$ then

    \quad  \quad     	$proportionCut= \tfrac{temp\_height - cutH}{temp\_height}$

    \quad    else

   \quad  \quad     	$proportionCut = 0$

    \quad    $tempDM=proportionCut  \cdot  AgeClassTop_i  \cdot  uptakeWeight_i$

 \quad 	  $ grazableDM +=tempDM$
  
   return grazableDM
*/


   if $cutPlantMaterial.GetName()_0 = ''$ then //  
This might need expansion in the future. For FarmN purpose this will probably do. !!!
   
\quad      cutPlantMaterial.Setname(PlantItemName)

\quad       theProducts.GiveProductInformation(cutPlantMaterial)
   
 cutDM = 
   cutN = 0

   initialDM=GiveDMVegTop()

   nitrogen  aNitrogenInVegTop = NitrogenInVegTop()

   for(int i=0i<4i++)
   
\quad       temp\_height=PlantHeight

\quad       if $temp\_height>cut\_height$

\quad \quad       	$proportionCut = \tfrac{temp\_height-cut\_height}{temp\_height}$

\quad       else

\quad \quad       	$proportionCut = 0$

\quad       $tempDM = proportionCut \cdot AgeClassTop_i$

\quad       $AgeClassTop_i-= tempDM $  // cut material

\quad 	   $cutDM += tempDM$

   if $cutDM>0$ then
  	
\quad       $cutN = aNitrogenInVegTop \cdot \tfrac{cutDM}{initialDM}$

 \quad      Nitrogen = Nitrogen-cutN
 
   if $DMStorage > 0$ then

\quad    	nitrogen aNitrogenInStorage=NitrogenInStorage().n

 \quad      cutDM += DMStorage,
      cutN.n += aNitrogenInStorage.n

 \quad      DMStorage=0,
      Nitrogen.n -=  aNitrogenInStorage.n


   if (cutPlantMaterial.GetDM()<0.01)
   
  \quad     error("cropRyegrass.Cut - dry matter content is below 1 %")

   \quad    cutPlantMaterial.SetDM(0.10) //  In order to be able to debug
  



   cutPlantMaterial.Setamount($\tfrac{cutDM}{cutPlantMaterial.GetDM()}$)     // adjust for DM content

   if $cutPlantMaterial.GetAmount()>0$ then

\quad 	  $ cutPlantMaterial.SetorgN\_content(\tfrac{cutN}{cutPlantMaterial.GetAmount()})$
 
  Nbudget.AddOutput(cutN.n),     
   DMbudget.AddOutput(cutDM)

   if $GiveDMVegTop() \le 0$ then

    \quad  warn("cropRyegrass.Cut - zero or negative amount of dry matter in veg. top")

   HasBeenCut = true,
   Phenology.SetDSAfterCut()

\subsection{Harvest(decomposable Storage, decomposable Straw)}

   CutOrHarvested = true,
   CuttingHeight = 0.1

   $RatioStrawCut = \tfrac{PlantHeight-CuttingHeight}{PlantHeight}$

   //    Storage.Setname(PlantItemName)

   //    if $PlantItemName \ne ""$ then
      Storage.Setname("GRASSSEED")
  
 //    if $PlantItemName \ne ""$ then
     
   	\quad   theProducts.GiveProductInformation(Storage)

   	\quad   Storage.Setamount($\tfrac{DMStorage}{Storage.GetDM()}$)
    

   //    Straw.Setname(StrawItemName)

      Straw.Setname("GRASSSEEDSTRAW"),
   theProducts.GiveProductInformation(Straw)

$   Straw.Setamount(GiveDMVegTop() \cdot \tfrac{RatioStrawCut}{Straw.GetDM()})$
  

   check amount of nitrogen at harvest, minimumN = Nmin(), maximumN = Nmax()

   if $minimumN > Nitrogen.n  \cdot  1.25$ then 
   
      
\quad    	warn("crop.Harvest - not enough nitrogen in plant at harvest")
   

   if $maximumN < Nitrogen.n  \cdot  0.75$ then 
  
  
\quad 	  warn("crop.Harvest - too much nitrogen in plant at harvest")
   
   // partition between compartments

   fN = fNitrogen()

  	$TotalRootN  = (minN\_Root+fN \cdot (maxN\_Root-minN\_Root)) \cdot DMRoot$

   $TotStorageN = (minN\_Store+fN \cdot (maxN\_Store-minN\_Store)) \cdot DMStorage$

   $N15Ratio    = Nitrogen.Get15NRatio()$

   TotalStrawN =  $(\tfrac{NpctminVegTop()}{100}+fN \cdot \tfrac{NpctmaxVegTop()-NpctminVegTop()}{100}) \cdot GiveDMVegTop()$

  $ Nrest = Nitrogen.n - TotalRootN - TotStorageN - TotalStrawN$

   if $Nrest<-0.01$ or $Nrest>0.01$ then
  
  \quad     warn("crop.Harvest - Nitrogen content adjusted")
  
\quad  	$RedFac = \tfrac{Nitrogen.n+Nrest}{Nitrogen.n}$

\quad   		$TotalRootN = RedFac \cdot TotalRootN$

\quad    	$TotStorageN = RedFac \cdot TotStorageN$
 


   RootN.SetBoth(TotalRootN,$4N15Ratio \cdot TotalRootN$)

   StorageN.SetBoth(TotStorageN,$N15Ratio \cdot TotStorageN$)

   StrawN = Nitrogen - RootN - StorageN

   if $Storage.GetAmount()>0$ then
   	$Storage.SetorgN\_content(\tfrac{StorageN}{Storage.GetAmount()}$)

   if $Straw.GetAmount()>0$ then
		$Straw.SetorgN\_content(\tfrac{StrawN}{Straw.GetAmount()}$)

	Nitrogen = Nitrogen-StorageN -StrawN

	if $ not Phenology.Ripe()$ then
      warn("crop.Harvest - crop not ready for harvest")

   if $GiveDMVegTop()<0$  or  $DMStorage<0$  or $StorageN.n<0$  or  $StrawN.n<0$  or  $Nitrogen.n<0$ then
   
\quad	   warn("crop. Harvest values should not be negative")

 \quad     $StorageN.n = \max(0,StorageN.n)$,
    $StrawN.n = \max(0,StrawN.n)$


   Nbudget.AddOutput(StorageN.n+StrawN.n),
   N15budget.AddOutput(StorageN.n15+StrawN.n15)                   

   DMbudget.AddOutput($GiveDMVegTop() \cdot RatioStrawCut+DMStorage$)

   for (int i=0i<4i++)

   \quad 	$AgeClassTop_i -= RatioStrawCut \cdot AgeClassTop_i$

DMStorage   = 0, 
GreenCAI = 0.1,
YellowCAI= 0.2

   PlantHeight        = 0.1,
   topMatter          = NULL


   EndBudget(NRemain,DMRemain),   Phenology.SetDSAfterCut(),    HasBeenCut = true


\subsection{Terminate(decomposable Straw,decomposable DeadRoot, RootLengthList)}
checks to make sure grazing has not occurred on the day of ploughing

   for(int i=0i<4i++)

    \quad   if $grazedDM_i>0$ then

     \quad      \quad error("cropRyegrass.Terminate - terminating a crop on same day as it was grazed")
   
   $DMRoot += ReSeedDM$, $ReSeedDM = 0$

   crop.Terminate(Straw,DeadRoot,RootLengthList)

\subsection{bool cropRyegrass.EndBudget(NRemain, DMRemain)}

	bool retVal=true

  NRemain=Nitrogen.n, N15Remain = Nitrogen.n15

  DMRemain=GiveDMVegTop()+DMRoot+DMStorage+SeedDM+ReSeedDM

    

   if $ not Nbudget.Balance(NRemain)$ or $ not DMbudget.Balance(DMRemain) $

  \quad or $ not N15budget.Balance(N15Remain)$ then retVal=false
 

   return retVal


\bibliography{literature}
  \bibliographystyle{plainnat}

\end{document}