\documentclass[%parskiphalf,%numbers=noendperiod
]{scrartcl}
\usepackage{amsmath}
\usepackage[latin1]{inputenc}
\usepackage[english]{babel}
\usepackage[T1]{fontenc} 
\usepackage{graphicx,parskip}
\usepackage{booktabs,longtable}
\usepackage{lmodern}
\usepackage[round]{natbib}

\newcommand\mymarginpar[1]{\marginpar {\flushleft\bfseries\scriptsize #1}}

\title{FASSET Soil Model}

\author{}
\date{\today}

\usepackage{lmodern}

\usepackage[backref]{hyperref}


\begin{document}

\maketitle
\tableofcontents
\newpage

Shortforms: Amm-Ammonium, Nit-Nitrat


\section{pF\_Curve}

\subsection{pF\_Curve.pF\_Curve(intPool) : base()}
  aPF\_Curve = new pF\_Curve\_Implementation(intPool)

\subsection{pF\_Curve.pF\_Curve(pF\_Curve pFSource)}
  aPF\_Curve = pFSource.aPF\_Curve, aPF\_Curve.AddOnePFCurve()


\section{Water and nitrogen in soil}

\subsection{soil.soil(char Pname,Idx,base owner) : base(Pname,Idx,owner)}

 AccumulatedWaterDrainage   = 
   AccumulatedNitDrainage = 
   evapotransp        = 
   soilEvap            = 
   CO2Evolution               = 0

   nitLeached.Clear(),
   ammLeached.Clear(),
   denitrification.Clear(),
   nitrification.Clear()

   EvapDepth = 200,
   totalTransp        = 0,
   NitrogenNetMineralisation.Clear(),
   nitrogenUptake=NULL,
   Nbudget.SetNames("soil","N"),
   WaterBudget.SetNames("soil","water")
\begin{align*}
   Surface & = new soilSurface &
   Profile & = new soilProfile("SoilProfile",0,this) 
\end{align*}
%\subsection{% soil.soil(soil& Soil) : base(Soil)}
%       EvapDepth          = Soil.EvapDepth
%       evapotransp        = Soil.evapotransp
%       soilEvap           = Soil.soilEvap
%       CO2Evolution              = Soil.CO2Evolution
%       nitLeached            = Soil.nitLeached
%       ammLeached           = Soil.ammLeached
%       denitrification           = Soil.denitrification
%       nitrification             = Soil.nitrification
%       totalTransp        = Soil.totalTransp
%       NitrogenNetMineralisation = Soil.NitrogenNetMineralisation
% 
%       AccumulatedWaterDrainage  = Soil.AccumulatedWaterDrainage
%       AccumulatedNitDrainage= Soil.AccumulatedNitDrainage
%       nitrogenUptake            = Soil.nitrogenUptake
% 
%       Nbudget	  	              = budget(Soil.Nbudget)
%       WaterBudget	              = budget(Soil.WaterBudget)
%       WaterBudget.Reset()      History ignored!
%       Surface                   = new soilSurface( \cdot Soil.Surface)
%       Profile                   = new soilProfile( \cdot Soil.Profile)
%       Profile.SetOwner(this)
%       Nbudget.Reset()      History ignored!
 

%\subsection{% soil.Initialize(char soilFileName)}
% 
%    SetCritical()
%    cout << "Initialising soil no " << GetLongNameInNumbers() << " with the file " << soilFileName << endl      a problem with patches can occure at this point
%    OpenInputFile(soilFileName)
%    Profile.Initialize(file)
%    if (theOutputData.DetailsData.getSoilDetails()=true)
%    {
% 
%       string fileName=theSystemData.getOutputDirectory()+"soil"+GetLongNameInNumbers()+".xls"
% 
%\subsection{%      	f.open(fileName.c\_str(),ios.out)}
%       f.precision(4)
%       f << "date"
% 
%       for(unsigned i=0i<theOutputData.WaterData.size()i++)
%          f << "\twater"<< theOutputData.WaterData_{i}.getStartDepth() << "-"<<theOutputData.WaterData_{i}.getEndDepth() 
%    
%       for(unsigned i=0i<theOutputData.N\minData.size()i++)
%          f << "\tN\min"<< theOutputData.N\minData_{i}.getStartDepth() << "-"<< theOutputData.N\minData_{i}.getEndDepth()
%          f << "\tNO3-"<< theOutputData.N\minData_{i}.getStartDepth() << "-"<< theOutputData.N\minData_{i}.getEndDepth()
%          f << "\tNH4-"<< theOutputData.N\minData_{i}.getStartDepth()  << "-"<< theOutputData.N\minData_{i}.getEndDepth()
%          f << "\tTotal15N-"<< theOutputData.N\minData_{i}.getStartDepth() << "-"<< theOutputData.N\minData_{i}.getEndDepth()
%       
%       for(unsigned i=0i<theOutputData.SoilOrgCData.size()i++)
%            {
% 
%               f << "\tSMB-C-"<< theOutputData.SoilOrgCData_{i}.getStartDepth() << "-"<<  theOutputData.SoilOrgCData_{i}.getEndDepth()
% 
%                f <<"\tNOM-C-"<< theOutputData.SoilOrgCData_{i}.getStartDepth() << "-"<<  theOutputData.SoilOrgCData_{i}.getEndDepth()
%                f <<"\tIOM-C-"<< theOutputData.SoilOrgCData_{i}.getStartDepth() << "-"<<  theOutputData.SoilOrgCData_{i}.getEndDepth()
%                f <<"\tAOM1-C-"<< theOutputData.SoilOrgCData_{i}.getStartDepth() << "-"<<  theOutputData.SoilOrgCData_{i}.getEndDepth()
%                f <<"\tAOM2-C-"<< theOutputData.SoilOrgCData_{i}.getStartDepth() << "-"<<  theOutputData.SoilOrgCData_{i}.getEndDepth()
%                f <<"\tSMB1-C-"<< theOutputData.SoilOrgCData_{i}.getStartDepth() << "-"<<  theOutputData.SoilOrgCData_{i}.getEndDepth()
%                f <<"\tSMB2-C-"<< theOutputData.SoilOrgCData_{i}.getStartDepth() << "-"<<  theOutputData.SoilOrgCData_{i}.getEndDepth()
%                f <<"\tSMR-C-"<< theOutputData.SoilOrgCData_{i}.getStartDepth() << "-"<<  theOutputData.SoilOrgCData_{i}.getEndDepth()
% 
%            }
%            for(unsigned i=0i<theOutputData.SoilOrgNData.size()i++)
%            {
% 
%          	  f << "\tSMB-N-"<< theOutputData.SoilOrgNData_{i}.getStartDepth() << "-"<<  theOutputData.SoilOrgNData_{i}.getEndDepth()<<"[m/g]"
%          	  f << "\tSMB-N15-"<< theOutputData.SoilOrgNData_{i}.getStartDepth() << "-"<<  theOutputData.SoilOrgNData_{i}.getEndDepth()<<"[m/g]"
%          	    << "\tAOM1 N-"<< theOutputData.SoilOrgNData_{i}.getStartDepth() << "-"<<  theOutputData.SoilOrgNData_{i}.getEndDepth()<<"[m/g]"
%          	    << "\tAOM2 N-"<< theOutputData.SoilOrgNData_{i}.getStartDepth() << "-"<<  theOutputData.SoilOrgNData_{i}.getEndDepth()<<"[m/g]"
%          	    << "\tSMB1 N-"<< theOutputData.SoilOrgNData_{i}.getStartDepth() << "-"<<  theOutputData.SoilOrgNData_{i}.getEndDepth()<<"[m/g]"
%          	    << "\tSMB2 N-"<< theOutputData.SoilOrgNData_{i}.getStartDepth() << "-"<<  theOutputData.SoilOrgNData_{i}.getEndDepth()<<"[m/g]"
%          	    << "\tSMR N-"<< theOutputData.SoilOrgNData_{i}.getStartDepth() << "-"<<  theOutputData.SoilOrgNData_{i}.getEndDepth()<<"[m/g]"
%          	    << "\tNOM n15-"<< theOutputData.SoilOrgNData_{i}.getStartDepth() << "-"<<  theOutputData.SoilOrgNData_{i}.getEndDepth()<<"[m/g]"
%          	    << "\tIOM n15-"<< theOutputData.SoilOrgNData_{i}.getStartDepth() << "-"<<  theOutputData.SoilOrgNData_{i}.getEndDepth()<<"[m/g]"
%          	    << "\tAOM1 n15-"<< theOutputData.SoilOrgNData_{i}.getStartDepth() << "-"<<  theOutputData.SoilOrgNData_{i}.getEndDepth()<<"[m/g]"
%          	    << "\tAOM2 n15-"<< theOutputData.SoilOrgNData_{i}.getStartDepth() << "-"<<  theOutputData.SoilOrgNData_{i}.getEndDepth()<<"[m/g]"
%          	    << "\tSMB1 n15-"<< theOutputData.SoilOrgNData_{i}.getStartDepth() << "-"<<  theOutputData.SoilOrgNData_{i}.getEndDepth()<<"[m/g]"
%          	    << "\tSMB2 n15-"<< theOutputData.SoilOrgNData_{i}.getStartDepth() << "-"<<  theOutputData.SoilOrgNData_{i}.getEndDepth()<<"[m/g]"
%          	    << "\tSMR n15-"<< theOutputData.SoilOrgNData_{i}.getStartDepth() << "-"<<  theOutputData.SoilOrgNData_{i}.getEndDepth()<<"[m/g]"
%          	    <<"\tMOM n15-"<< theOutputData.SoilOrgNData_{i}.getStartDepth() << "-"<<  theOutputData.SoilOrgNData_{i}.getEndDepth()<<"[m/g]"
%            }
%       for(unsigned i=0i<theOutputData.PoreWaterData.size()i++)
%       {
% 
%          f << "\t" << theOutputData.PoreWaterData_{i}.getDepth()/10 <<"cmNit-N"
%          << "\t" << theOutputData.PoreWaterData_{i}.getDepth()/10 <<"cmAm-N"
%          << "\t" << theOutputData.PoreWaterData_{i}.getDepth()/10 <<"cmN"
%        	<< "\t" << theOutputData.PoreWaterData_{i}.getDepth()/10 <<"cmCL"
%          << "\t" << theOutputData.PoreWaterData_{i}.getDepth()/10 <<"cmN15"
% 
%       }
% 
%       for(unsigned i=0i<theOutputData.TempData.size()i++)
%       {
%     	  f << "\tt" <<theOutputData.TempData_{i}.getDepth()
%       }
% 
%       f  << "\t Amm n 0-250 "
%         << "\t nit n 0-250"
%         << "\t Amm n15 0-250 "
%         << "\t nit n15 0-250"
%       f << "\tnetMineralisation"
%     	<< "\tN2O-N"
%     	<< "\tN2O-D"
%     	<< "\tN2O"
%     	<< "\tN2Oratio"
%     	<< "\tDrainage"
%     	<< "\tDrainage-N"
%     	<< "\tSoilEvap"
%     	<< "\tPlantET"
%       f << "\tDenitrification"
%     	<< "\tNitri"
%     	<<"\tCO2"
% 
%       f << "\tSoilsurface-C"
%     	<<"\tSoilSurface-OrgN"
%     	<<"\tSoilSurface.TotN"
%       f << endl
%    }
%    CloseInputFile()
% }


"nitrogenUptake" not included

\subsection{soil.Update(surplus, airTemp, EPotToSoil, snowContent, linkList<rootStructure>  rootInfo)}
Updates the soil corresponding to the passing of one day.

\begin{tabular}{ll}
   EPotToSoil     & potential evapotransp allocated to soil \\
   surplus       & water added to soil [mm/d] \\
   airTemp & air temp [C] \\
   snowContent    & snow content \\
   roots          & root descriptor list
\end{tabular}

   WaterBudget.AddInput(surplus),
   evapotransp=0

   for all i in 0..rootInfo.NumOfNodes()-1 do      Convert root radius from mm to m
 
   \quad  $rootInfo.ElementAtNumber(i).rootRadius=\tfrac{rootInfo.ElementAtNumber(i).rootRadius}{1000}$

 
  Profile.UpdateFinDiffTemp(airTemp,snowContent)
       (Update soil temp)
  
    %  Profile.UpdateTemp(airTemp,snowContent)      Old version

        Update soil organic matter
\begin{align*}
   nitrogen startMinN & = GetMinNitrogen() \\
   CO2Evolution & = Profile.UpdateOrganicMatter() \\
   NitrogenNetMineralisation & = NitrogenNetMineralisation \\
                             & \quad     +     GetMinNitrogen()-startMinN 
\end{align*}
        Update nit by denitrification and nitrification
   \begin{align*}
nitrification & = Profile.UpdateNitrification() \\
   denitrification & = Profile.UpdateDenitrification(CO2Evolution)
\end{align*}
   Nbudget.AddOutput(denitrification.n+Profile.GetN2OFromDenitrification().n)
  
 Nbudget.AddOutput(Profile.GetN2OFromNitrification().n)

        Update soil Evap and nitrogen movement to soil
   \begin{align*}
   relativeSurfaceWater & = \min(1,
     \tfrac{Profile.GetAvailWater(0.,EvapDepth)}
                                    { Profile.GetAvailCapacity(0.,EvapDepth))} \\
   maxEvap & = Profile.GetMaximumEvap()
   \end{align*}
   Surface.Update(surplus, EPotToSoil, relativeSurfaceWater, maxEvap, 

    \quad \quad \quad Profile.GetEvapContent(),airTemp)
   
   soilEvap = Surface.GetEvap(),
   evapotransp  = evapotransp + soilEvap
   
   Profile.SubtractEvap(soilEvap),
   WaterBudget.AddOutput(soilEvap)

   Profile.SubtractTransp(rootInfo), 
   totalTransp = 0

   for all i in 0..rootInfo.NumOfNodes()-1 do  

    totalTransp  = totalTransp  + rootInfo.ElementAtNumber(i).actualTransp
 
   evapotransp = evapotransp + totalTransp,
   WaterBudget.AddOutput(totalTransp)
\begin{align*}
   surfaceNit & = Surface.ExtractNitDisolved() \\
   urfaceAmm & = Surface.ExtractAmmDisolved()
\end{align*}
   surplus = surplus +  Surface.ExtractWater()      

Code review 2011 Check whether it is properly figured out
or whether all the water and N disappear pa once!

   Chloride = Surface.ExtractChloride()
  
      Water infiltration and nitrogen leaching

   Profile.UpdateInfiltrationSWAT(surplus,surfaceNit,surfaceAmm,

\quad nitLeached,ammLeached,Chloride)

        Deposit possibly "unused" water and N on soil surface
 
       Code review Kan det nogensinde ske????!!!
 
  Surface.AddWater(surplus) 

 Surface.AddNitDissolved(surfaceNit),
  Surface.AddAmmDissolved(surfaceAmm)
 
       Perform surface run-off
  
 Surface.SurfaceRunOff()
 
       Nitrogen uptake by the plants
  
 nitrogenUptake = Profile.SubtractNitrogenUptake(rootInfo)
  
    %  Perform budgeting
  
 WaterBudget.AddOutput(Profile.GetDrainage($GetMaximumDepth())+Surface.GetRunOff()$)
  
 if $Month=4 $ and  $Day=1$ then
 \quad  $AccumulatedWaterDrainage = AccumulatedNitDrainage = 0$
   \begin{align*}
   AccumulatedWaterDrainage   & = AccumulatedWaterDrainage   + Profile.GetDrainage(1000) \\
   AccumulatedNitDrainage & = AccumulatedNitDrainage + Profile.GetNitLeaching(1000).n
   \end{align*}
   for all i in 0..rootInfo.NumOfNodes()-1 do 
	   $Nbudget.AddOutput(nitrogenUptake_{i}.n)$
 
  Nbudget.AddOutput($Surface.GetRunOff\_N().n$)

   Nbudget.AddOutput(GetNitLeaching(GetMaximumDepth()).n

  \quad \quad \quad +GetAmmLeaching(GetMaximumDepth()).n)

        EndBudget(NRemain,WaterRemain),  
  WriteOutput()




\subsection{Add}

\minisec{soil.Add(soil addSoil, f)}
\begin{align*}
   EvapDepth         &  = (1-f) \cdot EvapDepth+f \cdot addSoil.EvapDepth \\  
   evapotransp      &  = (1-f) \cdot evapotransp+f \cdot addSoil.evapotransp \\  
 CO2Evolution             & = (1-f) \cdot CO2Evolution+f \cdot addSoil.CO2Evolution \\  
 nitLeached         &   = nitLeached \cdot (1-f)+addSoil.nitLeached \cdot f \\ 
  ammLeached        &   = ammLeached \cdot (1-f)+addSoil.ammLeached \cdot f \\  
 denitrification         &  = denitrification \cdot (1-f)+addSoil.denitrification \cdot f \\   
 nitrification          &   = nitrification \cdot (1-f)+addSoil.nitrification \cdot f \\
   NitrogenNetMineralisation & = NitrogenNetMineralisation \cdot (1-f) \\
                               & \quad     + addSoil.NitrogenNetMineralisation \cdot f \\
   totalTransp      &  = (1-f) \cdot totalTransp+f \cdot addSoil.totalTransp \\
  soilEvap       &    = (1-f) \cdot soilEvap+f \cdot addSoil.soilEvap \\
 AccumulatedWaterDrainage &  = (1-f) \cdot AccumulatedWaterDrainage \\
                            & \quad  +f \cdot addSoil.AccumulatedWaterDrainage \\
   AccumulatedNitDrainage &= (1-f) \cdot AccumulatedNitDrainage \\
                           & \quad     +f \cdot addSoil.AccumulatedNitDrainage
\end{align*}
   Surface.Add(addSoil.GetSoilSurface(),f), 
  Profile.Add(addSoil.GetSoilProfile(),f)
  
 Nbudget.Add(addSoil.Nbudget,f),
 WaterBudget.Add(addSoil.WaterBudget,f)


\minisec{soil.AddNutrient(nitrogen surfNit, 
 nitrogen surfAmm,  nitrogen soilNit,  nitrogen soilAmm)}
Adds nitrogen to soil

\begin{tabular}{ll}
   surfNit/surfAmm   & Nit/Amm to be added to soil surface [g N/m] \\   
   soilNit/soilAmm   & Nit/Amm to be added to top of soil profile [g N/m]  
\end{tabular}
  
 Surface.AddSolidNitrogen(surfNit,surfAmm),  
 Profile.AddNutrient(soilNit,soilAmm)
 
  Nbudget.AddInput(surfNit.n+surfAmm.n+soilNit.n+soilAmm.n)

\minisec{soil.AddDecomposable(decomposable  produc,  rootLengthList)}
  
 produc.RemoveWater()                                               To cope with dry matters less than 1
  
	if $produc.GetAmount()>0$ then
 	
    
  if $produc.GetorgN\_content().n<0$ then
%\subsection{   		%theMessage.FatalError("soil.AddDecomposable - negative amount of organic N added")}
  		%    if ((rootLengthList)  and  ( \cdot rootLengthList\le 0))
%\subsection{   	%    	theMessage.FatalError("soil.AddDecomposable - rootLengthList can not contain zero in first element")}
  	
\quad	 organicProduct p = new organicProduct

  	\quad	p.SetName($produc.GetName()$)

  	\quad	p.SetCarbon($produc.GetC\_content() \cdot produc.GetAmount() \cdot 100$)

  	\quad	p.SetNitrogen($produc.GetorgN\_content() \cdot produc.GetAmount() \cdot 100$)

  	\quad	p.SetNO3\_content($produc.GetNO3\_content() \cdot produc.GetAmount() \cdot 100$)

  	\quad	p.SetNH4\_content($produc.GetNH4\_content() \cdot produc.GetAmount() \cdot 100$)

  	\quad	p.SetkAOM1($produc.GetkAOM1()$),
  		p.SetkAOM2($produc.GetkAOM2()$)

   \quad   p.SetReduceTurnover(produc.GetReduceTurnover())

  	\quad	if $produc.GetAOM1\_frac()>0$ then
   		p.SetAOM1\_frac($produc.GetAOM1\_frac()$)

  	\quad	if $produc.GetAOM2\_frac()>0$ then
   		p.SetAOM2\_frac($produc.GetAOM2\_frac()$)

           N allocations
       sum=0

  \quad   if $produc.GetAOM1\_Nfrac()>0$  
or  $produc.GetAOM2\_Nfrac()>0$ then      
   
\quad\quad		p.SetNFraction(0,produc.GetAOM1\_Nfrac())           	     AOM1
     
  \quad\quad  $sum=sum+produc.GetAOM1\_Nfrac()$
   
	\quad\quad	p.SetNFraction(1,produc.GetAOM2\_Nfrac())           	     AOM2
    
 \quad\quad   $sum=sum+produc.GetAOM2\_Nfrac()$
   
   \quad\quad  p.SetNFraction(2,1-sum)   (NOM)
  
%      if (sum<0  or  sum>1)
%\subsection{ %         theMessage.FatalError("soil.AddDecomposable - inconsistency in values for fAOM1 and/or fAOM2")}
     
  AddOrganicProduct(p,rootLengthList),
  Nbudget.AddInput(p.GetTotalNitrogen().n)

  delete p

\minisec{soil.AddOrganicProduct(organicProduct  product,   rootLengthList)}

  organicProduct p = new organicProduct(  product)

 if $rootLengthList=NULL$ then Surface.AddProduct(p)
 
 else Profile.AddProduct(p, rootLengthList)

delete p

\subsection{soil.Incorporate(depth, frac, bool tillage, char tillagetype)}
  Incorporates organic matter from the surface and roots into the soil
(depth: mm, frac: 0-1)
 
if $depth>0$  and  $frac>0$ then
   
\quad	   if(OpenInputFileString("Tillage.dat"))      Only comes in effect when tillage.dat is present
		
  \quad\quad
    if tillage then    p1=0.1, p2=0.5
    
   \quad\quad\quad TurnoverFactor=2
  \quad\quad\quad 		FindSection(tillagetype)
   
   \quad\quad\quad	GetParameter("KillSMBFraction",p1),    
                  	GetParameter("MoveToFasterFraction",p2)
     
 \quad\quad\quad	GetParameter("TurnoverFactor",TurnoverFactor)

  \quad\quad\quad       if p1 > 0 then  
       	KillSMB(p1,depth)
     
  \quad\quad\quad  	if p2 > 0 then     
     	ModifyAOM(p2,TurnoverFactor,depth)

  \quad\quad    CloseInputFile()
	
  \quad    n = Profile.GetNumberOfLayers()
   
   %LengthList[300],
   
$       p = LengthList_0,   
   CurrentDepth = 0,   
    organicProduct prod$
   
   for all i in 0..n-1 do     
      
   \quad LayerThickness = Profile.GetLayerThickness(i+1)
    
   \quad $CurrentDepth += LayerThickness$,       
   $LengthList_{i} = 0.$
    
    if $CurrentDepth\le depth$ then
  
    \quad      $LengthList_{i} = LayerThickness$
    
     else if $CurrentDepth-LayerThickness<depth$ then
    
   \quad      $LengthList_{i} = depth-(CurrentDepth-LayerThickness)$  
   
   $n = Surface.GetNumberOfProducts()$
   
   for all j in 0..n-1 do    
  
   \quad    $prod = Surface.RemoveProduct(j,frac)$
  
    \quad   if $prod.GetAmount().c>0$ then   
    	 $AddOrganicProduct(prod,LengthList_0)$
   
     delete prod     
   



\minisec{soil.KillSMB(Fraction,Depth)}
  Fraction (0-1) : frac of SMB killed, Depth: mm

for all i in 0..Profile.GetNumberOfLayers()-1 do
 
\quad if $Profile.GetLayer(i).GetStartDepth()<Depth$ then
   (Actually a simplification !!!)

  \quad\quad  $Profile.GetLayer(i).KillSMB(Fraction)$


\minisec{soil.ModifyAOM(Fraction,TurnoverFactor, Depth)}
Fraction (0-1) : frac of AOM transferred to pools with a modified turnover rate
TurnoverFactor : Factor for modification of turnover rate
Depth: mm

  for all i in 0..Profile.GetNumberOfLayers()-1 do
  
 \quad if $Profile.GetLayer(i).GetStartDepth()<Depth$ then    
  (Actually a simplification !!!)
    
 \quad\quad Profile.GetLayer(i).ModifyAOM(Fraction,TurnoverFactor)
    
\subsection{Budget}

\minisec{soil.StartBudget()}
   
nitrogen N = GetTotalNitrogen(),   
Nbudget.SetInput(N.n)
  
 WaterBudget.SetInput(GetWater(0,GetMaximumDepth())+GetSurfaceWater())
   
Profile.StartBudget()

\minisec{bool soil.EndBudget(   NRemain,  WaterRemain)}
  
 bool retVal=true,
     NRemain = GetTotalNitrogen().n
  
 if !Nbudget.Balance(NRemain) then  retVal=false
   
 WaterRemain = GetWater(0,GetMaximumDepth())+GetSurfaceWater()
  
 if !WaterBudget.Balance(WaterRemain) then  retVal=false
   
Profile.EndBudget()
 
  return retVal


%\subsection{% soil.WriteOutput()}
% 
%  	if(theOutputData.DetailsData.getSoilDetails()=true)
% 	{
%  		f << theTime.GetYear() << "-" << theTime.GetMonth() << "-" << theTime.GetDay()
%  		for(unsigned i=0i<theOutputData.WaterData.size()i++)
% 	 	{
% 	 		f << "\t" << GetWater(theOutputData.WaterData_{i}.getStartDepth(), theOutputData.WaterData_{i}.getEndDepth()-theOutputData.WaterData_{i}.getStartDepth())
% 	 	}
%  		for(unsigned i=0i<theOutputData.N\minData.size()i++)
%  		{
%  			StartDepth =theOutputData.N\minData_{i}.getStartDepth()
%  			Thickness = theOutputData.N\minData_{i}.getEndDepth()-StartDepth
% 	 		f << "\t" << GetMinNitrogen(StartDepth, Thickness).n
% 	 		f << "\t" << GetNit(StartDepth, Thickness).n
% 	 		f << "\t" << GetAmm(StartDepth, Thickness).n
% 	 		f << "\t" << GetTotalNitrogen(StartDepth, Thickness).n15
% 	 	}
% 
%  		for(unsigned i=0i<theOutputData.SoilOrgCData.size()i++)
%  		{
%  			StartDepth =theOutputData.SoilOrgCData_{i}.getStartDepth()
%  			Thickness = theOutputData.SoilOrgCData_{i}.getEndDepth()-StartDepth
%  			SoilMass = GetSoilMass(StartDepth,Thickness)/10000000     BMP 29.10.2008 !!!!!!
% 	 		f << "\t" << (GetPoolCarbon("SMB1",StartDepth,Thickness) + GetPoolCarbon("SMB2",StartDepth,Thickness))/SoilMass
% 	 		f.precision(5)
% 	 		f  << "\t" << GetPoolCarbon("NOM",StartDepth,Thickness)
% 	 		f.precision(4)
% 	 		f  << "\t" << GetPoolCarbon("IOM",StartDepth,Thickness)
% 			   << "\t" << GetPoolCarbon("AOM1",StartDepth,Thickness)
% 	 		   << "\t" << GetPoolCarbon("AOM2",StartDepth,Thickness)
% 	 		   << "\t" << GetPoolCarbon("SMB1",StartDepth,Thickness)
% 	 		   << "\t" << GetPoolCarbon("SMB2",StartDepth,Thickness)
% 	 		   << "\t" << GetPoolCarbon("SMR",StartDepth,Thickness)
% 	 	}
%  		for(unsigned i=0i<theOutputData.SoilOrgNData.size()i++)
%  		{
%  			StartDepth =theOutputData.SoilOrgNData_{i}.getStartDepth()
%  			Thickness = theOutputData.SoilOrgNData_{i}.getEndDepth()-StartDepth
%  			SoilMass = GetSoilMass(StartDepth,Thickness)/10000000     BMP 29.10.2008 !!!!!!
%  			f << "\t" << (GetPoolNitrogen("SMB1",StartDepth,Thickness).n + GetPoolNitrogen("SMB2",StartDepth,Thickness).n)/SoilMass
%  			f << "\t" << (GetPoolNitrogen("SMB1",StartDepth,Thickness).n15 + GetPoolNitrogen("SMB2",StartDepth,Thickness).n15)/SoilMass
%  		   f  << "\t" << GetPoolNitrogen("AOM1",StartDepth,Thickness).n
%  		      << "\t" << GetPoolNitrogen("AOM2",StartDepth,Thickness).n
%  		      << "\t" << GetPoolNitrogen("SMB1",StartDepth,Thickness).n
%  		      << "\t" << GetPoolNitrogen("SMB2",StartDepth,Thickness).n
%  		      << "\t" << GetPoolNitrogen("SMR",StartDepth,Thickness).n
%  	      f.precision(5)
%  		   f  << "\t" << GetPoolNitrogen("NOM",StartDepth,Thickness).n
%  		   f.precision(4)
%  		   f  << "\t" << GetPoolNitrogen("IOM",StartDepth,Thickness).n
%  		      << "\t" << GetPoolNitrogen("AOM1",StartDepth,Thickness).n15
%  		      << "\t" << GetPoolNitrogen("AOM2",StartDepth,Thickness).n15
%  		      << "\t" << GetPoolNitrogen("SMB1",StartDepth,Thickness).n15
%  		      << "\t" << GetPoolNitrogen("SMB2",StartDepth,Thickness).n15
%  		      << "\t" << GetPoolNitrogen("SMR",StartDepth,Thickness).n15
%  		      << "\t" << GetPoolNitrogen("NOM",StartDepth,Thickness).n15
% 	 	}
% 
%  		for(unsigned i=0i<theOutputData.PoreWaterData.size()i++)
%  		{
%  			depth = theOutputData.PoreWaterData_{i}.getDepth()
%  	 		f << "\t" << 1000 \cdot GetNit(depth-5,10).n/GetWater(depth-5,10)
%          	<< "\t" << 1000 \cdot GetAmm(depth-5,10).n/GetWater(depth-5,10)
%             << "\t" << 1000 \cdot (GetNit(depth-5,10).n+GetAmm(depth-5,10).n)/GetWater(depth-5,10)              Convert to mg/l
%          	<< "\t" << 1000 \cdot Profile.GetChloride(depth-5,10)/GetWater(depth-5,10)         Convert to mg/l
%             << "\t" << 1e5 \cdot GetNit(depth-5,10).n15/GetWater(depth-5,10)
% 
%   		}
% 
%       for(unsigned i=0i<theOutputData.TempData.size()i++)
%  		{
% 
%  			depth = theOutputData.TempData_{i}.getDepth()
%  	 		f << "\t" << Profile.GetTemp(depth)
% 
%   		}
% 
%       f  << "\t" << GetAmm(0,250).n
%          << "\t" << GetNit(0,250).n
% 	      << "\t" << GetAmm(0,250).n15
%          << "\t" << GetNit(0,250).n15
% 
% 
%       N2ORatio = 1
%       if (Profile.GetN2OFromNitrification().n+Profile.GetN2OFromDenitrification().n>1e-10)
%       	N2ORatio = denitrification.n/(Profile.GetN2OFromNitrification().n+Profile.GetN2OFromDenitrification().n)
% 
% 
% 	     f << "\t" << NitrogenNetMineralisation.n
%          << "\t" << Profile.GetN2OFromNitrification().n
% 	      << "\t" << Profile.GetN2OFromDenitrification().n
% 	     	<< "\t" << Profile.GetN2OFromNitrification().n+Profile.GetN2OFromDenitrification().n
%          << "\t" << N2ORatio
%          << "\t" << AccumulatedWaterDrainage
%          << "\t" << AccumulatedNitDrainage
%          << "\t" << soilEvap
%          << "\t" << totalTransp
%          << "\t" << denitrification.n
%          << "\t" << nitrification.n
%          << "\t" << CO2Evolution
%       f << "\t" << Surface.GetCarbon()
%     	<< "\t" << Surface.GetOrganicNitrogen().n
%     	<< "\t" <<  Surface.GetTotalNitrogen().n
%       f << endl
%    }
% }



\subsection{Getter}

\minisec{nitrogen soil.GetNitrogenNetMineralisation()} 
  nitrogen Mineralisation = NitrogenNetMineralisation

NitrogenNetMineralisation.Clear()
    
return Mineralisation


\minisec{nitrogen soil.GetAmm(), GetNit(), GetMinNitrogen(), GetMinNitrogen(startDep, thick), GetTotalNitrogen(), 
GetTotalNitrogen(startDep, thick), GetTotalNitrogen(depth)}
return $Surface.GetAmmSolid()+Surface.GetAmmDisolved()
                       +Profile.GetAmm()$

return Surface.GetNitSolid()+Surface.GetNitDisolved()+Profile.GetNit()

return GetAmm()+GetNit()

return GetAmm(startDep,thick)+GetNit(startDep,thick)

return Surface.GetTotalNitrogen()+Profile.GetTotalNitrogen()

return GetMinNitrogen(startDep,thick) + Profile.GetOrganicNitrogen(startDep,thick)

return Surface.GetTotalNitrogen()+ GetTotalNitrogen(0, depth)




\section{Soil surface}

\subsection{soilSurface.soilSurface(): base()}
   %firstProduct=NULL     lastProduct=NULL
   %ammSolid.Clear()   nitSolid.Clear()
   %ammDisolved.Clear()   nitDisolved.Clear()
   %ammoniaVolatilisation.Clear()   RunOff\_N.Clear()
  
   nitrogenDisolvedConstant = 0.05, 
   nitrogenLeachedConstant  = 0.05    

Evap  = 
   temp    =   water        = 
   RunOff            = 
   Chloride                  = 0




\subsection{soilSurface.soilSurface(soilSurface surface) : base(surface)}
Copy constructor
surface - Source object
 %firstProduct = NULL
 %lastProduct = NULL
 %if (&surface)
\begin{align*}
  ammSolid          &  = surface.ammSolid \\
  nitSolid            & = surface.nitSolid \\
  ammDisolved       &  = surface.ammDisolved \\
  nitDisolved        &  = surface.nitDisolved \\
  nitrogenDisolvedConstant&  = surface.nitrogenDisolvedConstant \\ 
  nitrogenLeachedConstant  &= surface.nitrogenLeachedConstant \\
  Evap            &  = surface.Evap \\
  ammoniaVolatilisation   & = surface.ammoniaVolatilisation \\
  temp            &  = surface.temp \\
  water                 &   = surface.water \\
  RunOff                &   = surface.RunOff \\
  RunOff\_N              &   = surface.RunOff\_N \\
  Chloride             &    = surface.Chloride
\end{align*}  
$organicProduct  current$, $organicProduct tempProduct$

  $current = surface.firstProduct$

  while $current$ do

\quad	tempProduct = new organicProduct(current)

\quad	AddProduct(tempProduct), current = current.Next(), delete tempProduct


%\subsection{% soilSurface.Add(soilSurface \cdot  surface, frac)}
%   ammSolid            = ammSolid \cdot (1-frac) + surface.ammSolid \cdot frac
%   nitSolid             = nitSolid \cdot (1-frac) + surface.nitSolid \cdot frac
%   ammDisolved         = ammDisolved \cdot (1-frac) + surface.ammDisolved \cdot frac
%   nitDisolved          = nitDisolved \cdot (1-frac) + surface.nitDisolved \cdot frac
%   ammoniaVolatilisation    = ammoniaVolatilisation \cdot (1-frac) + surface.ammoniaVolatilisation \cdot frac
%   RunOff\_N                 = RunOff\_N \cdot (1-frac)+surface.RunOff\_N \cdot frac
% 
%   nitrogenDisolvedConstant = (1-frac) \cdot nitrogenDisolvedConstant+frac \cdot surface.nitrogenDisolvedConstant
%   nitrogenLeachedConstant  = (1-frac) \cdot nitrogenLeachedConstant+frac \cdot surface.nitrogenLeachedConstant
%   Evap              = (1-frac) \cdot Evap+frac \cdot surface.Evap
%   temp              = (1-frac) \cdot temp+frac \cdot surface.temp
%   water                    = (1-frac) \cdot water+frac \cdot surface.water
%   RunOff                   = (1-frac) \cdot RunOff+frac \cdot surface.RunOff
%   Chloride                 = (1-frac) \cdot Chloride+frac \cdot surface.Chloride
% 
%    organicProduct current
%   current = firstProduct                                first adjust original products
%   while (current)
%    current.Scale(1-frac)
% 	current = current.Next()
%   current = surface.firstProduct                      add products from surface
%   while (current)
%    current.Scale(frac)
% 	AddProduct(current)
% 	current = current.Next()



\subsection{Update}

\minisec{soilSurface.Update( surplus,   ePotToSoil,  relativeSurfaceWater,   maxEvap,   EvapContent,   airTemp)}
   Updates the soil surface corresponding to the passing of one day.

\begin{tabular}{ll}
   surplus            & Water to be added (and drained) from soil surface [mm/d] \\
   ePotToSoil         & Potential evapotransp from soil [mm/d] \\
   sumWaterContent    & Content of plant available water in the Evap depth [mm] \\
   sumWaterCapacity   & Capacity for plant availble water in the Evap depth [mm] \\
   airTemp     & Air temp [�C]
\end{tabular}

temp = airTemp

 	     Calculate soil Evap E
%  	     Old method
%  	/* 
% 	if (RelWater>1)
% 	  E=1
%  	else
%   if (RelWater>0.33)   E=0.9+0.1 \cdot (RelWater-0.33)/0.67
%   else
%    if (RelWater>0.22)  E=0.27+0.63 \cdot (RelWater-0.22)/0.11
%    else
%     if (RelWater>0.2) E=0.05+0.22 \cdot (RelWater-0.2)/0.02
%     else
%       E=0.05 \cdot RelWater/0.2
%   */
      Plauborg and Olesen (MVTOOL) and Plauborg (Ph.D. dissertation)
	
        $E = \begin{cases}
              1 & \text{ if } ePotToSoil \le EvapContent \\
              0.65 \cdot relativeSurfaceWater & \text{ otherwise}
             \end{cases}$

      $  Evap = \min(maxEvap,E \cdot ePotToSoil)$

     Update mineral nitrogen contents
        \begin{align*}
 	disolved  & = \min(1,nitrogenDisolvedConstant \cdot relativeSurfaceWater \\
               & +
            nitrogenLeachedConstant \cdot ( surplus)) \\
 	nitDisolved  & = nitDisolved + nitSolid \cdot disolved \\
 	nitSolid     & = nitSolid - nitSolid \cdot disolved 
       \end{align*}
         \begin{align*}
 	disolved & = \min(1,nitrogenDisolvedConstant \cdot relativeSurfaceWater \\
                 &  + nitrogenLeachedConstant \cdot ( surplus)) \\
 	ammDisolved & = ammDisolved + ammSolid \cdot disolved \\
 	ammSolid   & = ammSolid - ammSolid \cdot disolved 
       \end{align*}

\minisec{soilSurface.SurfaceRunOff()}
   maxWater =25,  
   RunOff = 0,   
   RunOff\_N.Clear()

   if $water>maxWater$ then  
      \begin{align*}
      f & =\frac{\max(0,(water-maxWater)) \cdot 0.1}{water} \\
      nitDisolved  & = nitDisolved + nitSolid \cdot 0.5 \\
      ammDisolved & = ammDisolved + ammSolid \cdot 0.5 \\
      RunOff   & = f \cdot water \\
      RunOff\_N & = (nitDisolved+ammDisolved) \cdot f \\
      water    & = water - f \cdot water \\
      nitDisolved & = nitDisolved - nitDisolved \cdot f \\
      ammDisolved & = ammDisolved - ammDisolved \cdot f \\
      nitDisolved & = nitDisolved + nitSolid \cdot 0.5 \\
      ammDisolved & = ammDisolved + ammSolid \cdot 0.5 
      \end{align*}
      nitSolid.Clear, 
      ammSolid.Clear()  
%    if(ammDisolved.n<-1E-6  or  nitDisolved.n<-1E-6  or  ammSolid.n<-1E-6  or  nitSolid.n<-1E-6)
%\subsection{%    	theMessage.WarningWithDisplay("SoilSurface.SurfaceRunOff - negative n concs")}

\subsection{Extract}

\minisec{soilSurface.ExtractWater(), ExtractChloride()}
 extract  = water,  water  = 0,    return extract

 extract  = Chloride,  Chloride  = 0,  return extract

\minisec{nitrogen soilSurface.ExtractAmmDisolved(), ExtractNitDisolved()}  
 Extract amm:
   nitrogen am=ammDisolved, ammDisolved.Clear(), return am

Extract nit:
   nitrogen n = nitDisolved, nitDisolved.Clear(), return n



\subsection{Add}

\minisec{soilSurface.AddSolidNitrogen(nitrogen addNit, nitrogen addAmm), AddDisolvedNitrogen(nitrogen addNit, nitrogen addAmm)}
Adds nitrogen to the soil surface.
   addNit - Nit to be added [$g N/m^2$], 
   addAmm - Amm to be added [$g N/m^2$]
   \begin{align*}
   nitSolid  & = nitSolid + addNit &  ammSolid & = ammSolid + addAmm \\
  %
%    if(ammDisolved.n<-1E-6  or  nitDisolved.n<-1E-6  or  ammSolid.n<-1E-6  or  nitSolid.n<-1E-6)
%\subsection{% 	   theMessage.WarningWithDisplay("SoilSurface.AddNitrogen - negative n concs")}
   nitDisolved & = nitDisolved + addNit &
   ammDisolved & = ammDisolved + addAmm 
   \end{align*}
%    if(ammDisolved.n<-1E-6  or  nitDisolved.n<-1E-6  or  ammSolid.n<-1E-6  or  nitSolid.n<-1E-6)
%\subsection{% 	   theMessage.WarningWithDisplay("SoilSurface.AddNitrogen - negative n concs")}


\minisec{soilSurface.AddProduct(organicProduct  product)}
Adds an organic (or inorganic) product to the the soil surface.
   product - Organic product object

	organicProduct p = new organicProduct(product)

 	AddDisolvedNitrogen(p.GetNit(), p.GetAmm())

 	p.ClearNO3\_content(), p.ClearNH4\_content()

 	if !firstProduct
        then firstProduct = lastProduct = p

 	%else 	
  	bool poolResemblance = false

  	organicProduct  aProduct = firstProduct

  \quad		while aProduct  do
		
  \quad		 \quad if (!poolResemblance  and  (aProduct.ProductSimilarity(p)))

   \quad	\quad	\quad	aProduct.AddNext(p),
   	poolResemblance = true,   		
        aProduct = aProduct.Next()
  		
  \quad		if !poolResemblance then
 		
\quad	  		\quad bool productAdded = lastProduct.AddNext(p)

\quad	  		\quad  if productAdded then lastProduct = p


\subsection{Remove}


\minisec{organicProduct soilSurface.RemoveProduct(No, frac)}
Removes part of an organic product from the soil surface. The removed product
is returned as an object in the function call.
   No  - Product number in list to remove (0..),
   frac - Fraction of product to be removed

  organicProduct p = firstProduct,
  organicProduct np = NULL
  
  for all i in 0..No-1 do

  \quad   if p then 
            p = p.Next(),
      np = p.GetFraction(frac),
 	p.Scale(1-frac) 	
 
  return np





\subsection{Getter}

\minisec{nitrogen soilSurface.GetTotalNitrogen()}
  return GetAmmSolid()+GetNitSolid()+GetAmmDisolved()+
     GetNitDisolved()+GetOrganicNitrogen()


\minisec{soilSurface.GetCarbon(), GetOrganicNitrogen(),GetTotalWater(), GetNumberOfProducts()}  
   organicProduct current = firstProduct,  sum = 0.
   
   while $current \ne NULL$ do

     \quad sum = sum + current.GetCarbon() | GetOrganicNitrogen() | GetTotalWater() | 1

     \quad current  = current .Next()  
 
  return sum


%\subsection{nitrogen soilSurface.}   
%   organicProduct  p = firstProduct, nitrogen sum, sum.Clear()
   
%   while (p) do   sum = sum +  p = p.Next()
   
 %  return sum




%\subsection{soilSurface.}   
  % n=0,  organicProduct current = firstProduct
   
 %  while $current  \ne  NULL$ do  n++,  current = current.Next()
   
%   return n








\section{Soil Profile}

\citep{DaisyDescription} 

Density ($\rho$), specific heat capacity (c), and thermal conductivity (K) of different soil
constituents. Water and air at 10�C (de Vries, 1963).

\begin{tabular}{lrrr}
Soil constituent & $\rho$ & c & K \\ 
               & $kg/m^3$ & J/(kg\textdegree C) & W/(m \textdegree C) \\ \hline
Quarts         & 2660 & 750 & 8.8 \\
Clay minerals  & 2650 & 750 & 2.9 \\
Organic matter & 1300 & 1920 & 0.25 \\
Water          & 1000 & 4192 & 0.57 \\
Ice            & 920 & 2050 & 2.20 \\
Air            & 1.25 & 1005 & 0.025
\end{tabular}

Column of soil layers stored as a list
\begin{align*}
LatentHeatMelting & = 334400  [J/kg] \\
HeatCapacityWater & = 4192   [J/(kg dgC)] \\
HeatCapacitySolid & = 750    [J/(kg dgC)] \\
HeatCapacityIce & = 2050      [J/(kg dgC)] \\
WaterDensity & =1000 [kg/m^3] \\
IceDensity & =920    [kg/m^3] \\
VolHeatCapacityIce & =HeatCapacityIce \cdot IceDensity [J/(m^3 dgC)] \\
VolHeatCapacityWater & =HeatCapacityWater \cdot WaterDensity [J/(m^3 dgC)] \\
snowConductivity & = 2.86e-6 \cdot 300 \cdot 300  \qquad from DAISY \\
specCapacity & = LatentHeatMelting \cdot WaterDensity [J/m^3]
\end{align*}

\subsection{soilProfile.soilProfile(char Pname,Idx,base  owner) : base(Pname,Idx,owner)}
% last=first = NULL
 
  for all i in 0..MaxPlants-1 do $nitrogenUptake_{i}.Clear()$

  for all i in 0..MaxSoilLayers-1 do   
      \begin{align*}
      soilLayerArray_{i} & =NULL &
      distanceToPreviousLayer_{i} & =0 \\
      distanceToNextLayer_{i} & =0   &
      temp_{i} & =0
      \end{align*}  
           infiltration = 
   depthDampning = 
 maxRootDepth         = 
 NumOfSoilLayers      = 0 

 EvapExtCoef          = 0.01,
 EvapCapacity  = 10, 
 EvapContent  = 0, 

$   minTemp_0  =-10, 
   minTemp_1=-10, 
   minTemp[2] =-10, 
   minTemp[3] =-3, 
   minTemp[4] =0, 
   minTemp[5]=5, 
   minTemp[6] =7, 
   minTemp[7] =8 (because of start-up problem)
   minTemp[8] =8, 
   minTemp[9] =3, 
   minTemp[10] =-5, 
   minTemp[11] =-10 $

   lastErrorYear =-1,
   calculateAverage =true



\subsection{soilProfile.soilProfile(soilProfile Profile) : base(Profile)}
   \begin{align*}
   NumOfSoilLayers    &= Profile.NumOfSoilLayers \\
   EvapExtCoef        &= Profile.EvapExtCoef \\
   EvapCapacity &= Profile.EvapCapacity \\
   EvapContent & = Profile.EvapContent \\
   maxRootDepth       & = Profile.maxRootDepth
   \end{align*}
   %first = NULL
   %last  = NULL
  for all i in 0..MaxPlants-1 do	$nitrogenUptake_{i}.Clear()$
  
   i = 0
  
   if Profile  then

   \quad	soilLayer current, soilLayer tempLayer, current = Profile.first
     
   while current do
       
   \quad       tempLayer = new soilLayer(current),
          AddItem(tempLayer)

    \quad     $ soilLayerArray_{i}=tempLayer$,
        $  soilLayerArray_{i}.SetOwner(this)$

  \quad        current = current.Next()

   \quad       i++     
      
      for all i in 0..ExtraLayers-1 do     \quad (extra layers)

    \quad  $soilLayerArray_{NumOfSoilLayers+i} = \text{ new } soilLayer( Profile.soilLayerArray_{NumOfSoilLayers+i}) 
     $

    

    \quad $ soilLayerArray_{NumOfSoilLayers+i}.SetOwner(this)$
  
    for (i=0i<(NumOfSoilLayers+ExtraLayers)i++)
      \begin{align*}   
    distanceToPreviousLayer_{i} & = Profile.distanceToPreviousLayer_{i} \\
         distanceToNextLayer_{i} &    = Profile.distanceToNextLayer_{i} \\
         temp_{i}          &   = Profile.temp_{i}      
    \end{align*}
Distances between layers considered equal. This assumption is not tested though.


\subsection{Add}


\minisec{soilProfile.Add(soilProfile   Profile,frac)}
  % if (NumOfSoilLayers \ne Profile.NumOfSoilLayers)
%\subsection{ %  	theMessage.FatalError("soilProfile.profiles with different number of layers can't be added!")}

   for all i in 0..MaxPlants-1 do	$nitrogenUptake_{i}.Clear()$
\begin{align*}
   EvapExtCoef         &= (1-frac) \cdot EvapExtCoef+frac \cdot Profile.EvapExtCoef  \\
   EvapCapacity &= (1-frac) \cdot EvapCapacity+frac \cdot Profile.EvapCapacity \\
   EvapContent  &= (1-frac) \cdot EvapContent+frac \cdot Profile.EvapContent \\
   maxRootDepth        &= (1-frac) \cdot maxRootDepth+frac \cdot Profile.maxRootDepth
\end{align*}
for all i in 0..(NumOfSoilLayers+ExtraLayers)-1 do 

     \quad  $soilLayerArray_{i}.Add(Profile.soilLayerArray_{i},frac)$

    \quad  $ temp_{i}  = (1-frac) \cdot temp_{i}+frac \cdot Profile.temp_{i}$
 
\minisec{soilProfile.AddNutrient(nitrogen soilNit, nitrogen soilAmm)}
Adds nitrogen to the first layer in the profile.
   addNit   - Nit to be added [$g N/m^2$],
   addAmm   - Amm to be added [$g N/m^2$]

% if (first  \ne  NULL)
  first.AddNutrient(soilNit,soilAmm)


\minisec{soilProfile.AddItem(soilLayer  data)}
Adds a soil layer to the profile, 
   data  - Soil layer object

 if first = NULL 
then first = data
else last.AddNext(data)

 last = data

  
\minisec{soilProfile.AddProduct(organicProduct product,  rootLengthList)   }
      n = GetNumberOfLayers(), 
      L = rootLengthList, 
      sumL  = 0

      for all i in 0..n-1 do 
       \quad  $ sumL =   sumL + \cdot (L++)$

    $  soilLayer current = first$,
      $ organicProduct prod = (organicProduct  \cdot ) product $

     To awarning

       organicProduct p, L = rootLengthList,  k=0 (Remove later!)

      while current do

  \quad      if  $L>0$ then

     \quad\quad  p = prod.GetFraction($\frac{L}{sumL}$), 
        	current.AddProduct(p), delete p

       \quad 	current = current.Next()

        \quad	L++

         k++




\subsection{soilProfile.Initialize(fstream  f)}
   Setfile(f), startDepth = 0,
   SetCritical(), FindSection("SoilParameters"),
   GetParameter("Nodes",NumOfSoilLayers),
   nodeThickNess=100,
   UnsetCritical()

   GetParameter("NodeThickness",nodeThickNess)

   JBC[] = \{0.03,0.03,0.025,0.02,0.02,0.015,0.015,0.01,0.01,0.01\} 

       modify the dependence upon the clay,
      perhaps a mistake

   JB = 5,
   EvapExtCoef = JBC[JB-1]

   %soilLayer aTempSoilLayer

   for all i in 0..NumOfSoilLayers-1 do

    \quad   aTempSoilLayer = new soilLayer("soilLayer",i,this)

   \quad    aTempSoilLayer.Initialize(startDepth,nodeThickNess,f)

  \quad     AddItem(aTempSoilLayer)

  \quad     $soilLayerArray_{i}=aTempSoilLayer$,
      $startDepth += nodeThickNess$
 
   AssignMaxRootDepth()
%    if (startDepth<maxRootDepth)
%\subsection{%    	theMessage.FatalError("soilProfile.Initialize - The depth of defined layers is smaller than the root depth")}
 
 for all i in 0..NumOfSoilLayers-1 do     

       Calculates the distance between layers and converts to meter
      
       default temp
  
     $ temp_{i}  = 8,                  
      aTempSoilLayer  = soilLayerArray_{i}$
       
      $	distanceToPreviousLayer_{i} = \begin{cases}
          \tfrac{ aTempSoilLayer.GetThickness()}{2000}   & \text{ if } i = 0 \\
          \tfrac{aTempSoilLayer.GetThickness()+soilLayerArray_{i-1}.GetThickness()}{2000} & \text{ otherwise}
      \end{cases}$  
    

     $	distanceToNextLayer_{i} = $

$\begin{cases}
        \tfrac{ aTempSoilLayer.GetThickness()}{1000 } &   \text{ if } i=NumOfSoilLayers-1 \\
     \tfrac{aTempSoilLayer.GetThickness()+soilLayerArray_{i+1}.GetThickness()}{2000} & \text{ otherwise}
   \end{cases}$
  
  for all i in 0..ExtraLayers-1 do        

  \quad      extra layers to ensure no heat flow at bottom (boundary condition)
   
  \quad      $soilLayerArray_{NumOfSoilLayers+i}  = 
   \text{new } soilLayer(soilLayerArray_{NumOfSoilLayers-1})$  

  \quad $  temp_{i+NumOfSoilLayers}  = 8 $

   \quad  $distanceToNextLayer[NumOfSoilLayers + i]    =$

\quad\quad$ 
distanceToNextLayer_{NumOfSoilLayers-1} \cdot (i+1) \cdot 10   $
    
   \quad   $soilLayerArray_{NumOfSoilLayers+i}.SetStartDepth($

\quad \quad 
$ soilLayerArray[NumOfSoilLayers+i-1].GetStartDepth()$

\quad \quad  \quad $+  1000 \cdot distanceToNextLayer_{i+NumOfSoilLayers-1})$ 
  
   $ distanceToPreviousLayer_{i+NumOfSoilLayers}  = distanceToNextLayer_{i+NumOfSoilLayers-1}$
      
   $soilLayerArray_{NumOfSoilLayers+i}.SetThickness($

\quad  \quad $distanceToNextLayer_{i+NumOfSoilLayers} \cdot 1000)$
  
  $ soilLayerArray_{NumOfSoilLayers+i}.SetFieldCapacity(\tfrac{soilLayerArray[
      NumOfSoilLayers-1].GetFieldCapacity()}{     soilLayerArray_{NumOfSoilLayers-1}.GetThickness()} \cdot 
                                                          soilLayerArray_{NumOfSoilLayers+i}.GetThickness())
  $

  \quad    $soilLayerArray_{NumOfSoilLayers+i}.SetWater(soilLayerArray_{NumOfSoilLayers+i}.GetFieldCapacity())
  $

   f=NULL

   for (i=0i<(NumOfSoilLayers+ExtraLayers)i++)
   \begin{align*}
   	DistPreviousLayerInv_{i} & = \tfrac{1}{distanceToPreviousLayer_{i}} \\
      DistNextLayerInv_{i}    &  = \tfrac{1}{distanceToNextLayer_{i}}
  \end{align*} 
   if theControlParameters.GetWriteSoilTemp() then

SoilTempFile.open("soiltemp.txt",ios.out)



\minisec{soilProfile.AssignMaxRootDepth()}
   $c1=0.06, 
   c2=0.15, 
   d1=600, 
   d2=2000$,
   UnsetCritical()

   if !GetParameter("MaxRootDepth", maxRootDepth) then

   \quad  Clay = GetClayContent(500,200)
   
  \quad $ maxRootDepth = \begin{cases}
                    d1 & \text{ if } Clay<c1 \\
                    \min\left(d2,d1+(d2-d1) \cdot  \tfrac{Clay-c1}{c2-c1}\right) & \text{ otherwise}
                   \end{cases}$


%FARM-N pig farm 96074: calls a NULL here at some po???!!!








\subsection{Subtract}



\minisec{soilProfile.SubtractEvap(soilEvap)}
Subtracts Evap [mm] from profile.
  

 /*  Newest, but presently discarded method

 soilLayer current = first

 while $soilEvap>0$ do
  \begin{align*}
subtractAmount & = \min(soilEvap,current.GetAvailWater()) \\
   soilEvap & = soilEvap - subtractAmount \\
   current.SubtractEvap(subtractAmount) \\
current & = current.Next()
\end{align*}
  */

      Original method - extended with the EvapContent part
\begin{align*}
 soilLayer current & = first \\
 directEvap & =\min(\min(soilEvap,EvapContent),soilLayerArray_0.GetAvailWater()) \\
 soilEvap & =  soilEvap- directEvap \\
 EvapContent & =EvapContent-directEvap
\end{align*}
$ soilLayerArray_0.SubtractEvap(directEvap)$

 if $soilEvap>0$ then 
 sum = GetMaximumEvap()

 \quad  while $current  \ne  NULL$  and  $sum>0$ do
      \begin{align*}
      a & = %     current.GetThickness() \cdot 
        \max(0,current.GetAvailWater()) \cdot \exp(-EvapExtCoef current.GetCenterDepth()) \\
     &  current.SubtractEvap(soilEvap \cdot \tfrac{a}{sum}) \\
      current & = current.Next()
 \end{align*}



\minisec{soilProfile.SubtractTransp(linkList<rootStructure>   rootList)}
Subtracts estimated transp from the water in the layer.

   $waterBefore = soilLayerArray_0.GetAvailWater()$

%    $PlantDemand[MaxPlants]$
% 
%  $  WaterUptakeAbility[MaxSoilLayers \cdot MaxPlants]$
% 
%   $ TotalWaterUptakeAbility[MaxSoilLayers \cdot MaxPlants]$
% 
%    %WaterResource_{MaxSoilLayers}
% 
%  $  PlantUptake[MaxPlants \cdot MaxSoilLayers]$

   for all i in 0..$MaxPlants \cdot MaxSoilLayers-1$ do  
	   $PlantUptake_{i}=0 $  

   AttackerNumber = rootList.NumOfNodes()
%    if (AttackerNumber>MaxPlants)  
%       cout << GetLongName() << endl
%\subsection{% 		theMessage.FatalError("soilProfile.SubtractTransp \maximum number of crops exceeded.")}
  
 for all i in 0..AttackerNumber-1 do

   \quad	$PlantDemand_{i} = \max(0,rootList.ElementAtNumber(i).transpDemand)$

  $ rootStructure  \cdot TotalRoot = new rootStructure$

  % EstimateTotalRoot(TotalRoot,rootList)

   ResourceNumber  = 0, 
   soilLayer currentSoilLayer = first

   while $currentSoilLayer  \ne  NULL$ do

  \quad   currentSoilLayer = currentSoilLayer.Next(),
   ResourceNumber++	
  
   TotalPlantUptake[MaxPlants]

  for all j in 0..AttackerNumber-1 do
   	$TotalPlantUptake_{j} = 0$

   linkList <rootStructure>.PS currentRoot

   rootList.PeekHead(currentRoot),
   PlantIdx = 0

   while $currentRoot  \ne  NULL$  do

   \quad SoilLayerIdx = 0, 
      soilLayer currentSoilLayer = first

      while $currentSoilLayer  \ne  NULL$ do
        
      \quad $	WaterUptakeAbility[SoilLayerIdx+PlantIdx \cdot ResourceNumber] = $
               
 \quad \quad $currentSoilLayer.MaxTransp(currentRoot.element) $
        \begin{align*}   
         currentSoilLayer&= currentSoilLayer.Next() \\
         SoilLayerIdx & =   SoilLayerIdx+ 1
       \end{align*}      
      rootList.OneStep(currentRoot)

      PlantIdx++, 
   currentSoilLayer  = first,
   SoilLayerIdx = 0

   while $currentSoilLayer  \ne  NULL$  do
      \begin{align*}
      TotalWaterUptakeAbility_{SoilLayerIdx} & = currentSoilLayer.MaxTransp(TotalRoot) \\
      WaterResource_{SoilLayerIdx} & = \min(currentSoilLayer.GetTotalWater() \\
              & \quad  -currentSoilLayer.GetIce(),\\
              & \quad TotalWaterUptakeAbility_{SoilLayerIdx}) \\
      currentSoilLayer & = currentSoilLayer.Next()     \\ 
       SoilLayerIdx & = SoilLayerIdx+ 1
     \end{align*}
   ResourceCompetition(ResourceNumber, AttackerNumber, PlantDemand, $WaterUptakeAbility_0$, $WaterResource_0$, $PlantUptake_0$)

  $ delete [sizeof(double) \cdot MaxSoilLayers] TotalRoot.rootLengthList  $
    
   delete TotalRoot

   if $rootList \ne NULL$  then (Perform only if plants has roots) 

   \quad   $soilLayer currentSoilLayer = first$,
     SoilLayerIdx = 0
     
  \quad    while $currentSoilLayer  \ne  NULL$ do
 
  \quad\quad    	subtractTransp = 0

   \quad\quad      for all j in 0..AttackerNumber-1 do         
           \begin{align*}
            transpRate & = PlantUptake[SoilLayerIdx+j \cdot ResourceNumber] \\
            transpRate & = \min(transpRate, \\
                      & \quad currentSoilLayer.MaxTransp(rootList.ElementAtNumber(j))) \\
            subtractTransp & = subtractTransp + transpRate \\
            TotalPlantUptake_{j} & = TotalPlantUptake_{j} + transpRate     
           \end{align*}
  \quad       currentSoilLayer.SubtractTransp(subtractTransp)

 \quad       currentSoilLayer = currentSoilLayer.Next(),
        SoilLayerIdx++
      
   for all i in 0..AttackerNumber-1 do  
   	$rootList.ElementAtNumber(i).actualTransp = TotalPlantUptake_{i}$
   
  %   if (TotalPlantUptake_{i}<0)
%\subsection{    %  	theMessage.WarningWithDisplay("soilProfile.SubtractTransp negative ",TotalPlantUptake_{i})}
%       if (TotalPlantUptake_{i}-rootList.ElementAtNumber(i).transpDemand>1e-5)
%          cout << theTime << endl
%          cout << "Layer: " << i << "  TotalPlantUptake: " << TotalPlantUptake_{i} << "  RootList\_transpDemand: " << rootList.ElementAtNumber(i).transpDemand
%\subsection{%       	theMessage.WarningWithDisplay("soilProfile.SubtractTransp exceeds demand by ",TotalPlantUptake_{i}-rootList.ElementAtNumber(i).transpDemand)}
      
   if (waterBefore>1E-5) then
   	$EvapContent   = \tfrac{soilLayerArray_0.GetAvailWater()}{waterBefore}$




\minisec{nitrogen soilProfile.SubtractNitrogenUptake(linkList<rootStructure>  rootList)}
  Subtracts estimated nitrogen uptake by the plants from the mineral
nitrogen pools in the layer.

%    PlantDemand[MaxPlants]
% 
%    $NitrogenUptakeAbility[2 \cdot MaxSoilLayers \cdot MaxPlants]  $
% 
%    $NitrogenResource[2 \cdot MaxSoilLayers]$
% 
%    $PlantUptake[MaxPlants \cdot 2 \cdot MaxSoilLayers]$
% 
%    $Iteration=MaxPlants \cdot 2 \cdot MaxSoilLayers$

   for($i=0i<Iterationi++$)   
	  $ PlantUptake_{i}=0 $  

   %nitrogen TotalPlantUptake[MaxPlants]

   AttackerNumber = rootList.NumOfNodes()

   for all i in 0..AttackerNumber-1 do

  \quad 	$PlantDemand_{i} = rootList.ElementAtNumber(i).NitrogenDemand$
	
   \quad    for all i in 0..AttackerNumber-1 do
   \quad\quad	$TotalPlantUptake_{j}.Clear()$

   rootStructure TotalRoot = new rootStructure

   EstimateTotalRoot(TotalRoot,rootList)

   ResourceNumber = 0, 
   soilLayer currentSoilLayer  = first

   while $currentSoilLayer  \ne  NULL$ do
   
   \quad   currentSoilLayer = currentSoilLayer.Next(),
    ResourceNumber++
  
   $ResourceNumber = ResourceNumber \cdot 2   $

                                   2 as both nit and amm are substrates

   linkList <rootStructure>.PS currentRoot

   rootList.PeekHead(currentRoot),
   PlantIdx = 0

   while $currentRoot  \ne  NULL$  do
     
	\quad SoilLayerIdx  = 0 ,
      soilLayer currentSoilLayer = first
      
      \quad  while $currentSoilLayer  \ne  NULL$  do 
        
      	\quad\quad$NitrogenUptakeAbility[2 \cdot SoilLayerIdx+PlantIdx \cdot ResourceNumber] $

         \quad\quad\quad    $   = currentSoilLayer.MaxNitrogenFlux(false,currentRoot.element) $

	\quad\quad$NitrogenUptakeAbility[2 \cdot SoilLayerIdx+PlantIdx \cdot ResourceNumber+1] $

           \quad\quad\quad     $ = currentSoilLayer.MaxNitrogenFlux(true ,currentRoot.element) $
   
\begin{align*}       
 currentSoilLayer & = currentSoilLayer.Next() \\
         SoilLayerIdx &= SoilLayerIdx  + 1
        \end{align*}      
      rootList.OneStep(currentRoot)

      PlantIdx++
  

   SoilLayerIdx = 0, 
   currentSoilLayer = first, 
   static tmp  =0, 
   tmp  = tmp + 1

   while $currentSoilLayer  \ne  NULL$  do
      \begin{align*}
       TotalAmmUptake & = currentSoilLayer.MaxNitrogenFlux(false,TotalRoot) \\
      LayerAmm & = currentSoilLayer.GetAmm().n \\
      &  currentSoilLayer.FracAmmInWater() \\
   	NitrogenResource[2 \cdot SoilLayerIdx]  & = \min(LayerAmm,TotalAmmUptake)
      \end{align*}
   	if($tmp=0$)   	
   		cout<<currentSoilLayer.GetAmm().n<<endl   	
   \begin{align*}
      TotalNitUptake & = currentSoilLayer.MaxNitrogenFlux(true,TotalRoot) \\
      LayerNit & = currentSoilLayer.GetNit().n \\
   	NitrogenResource[2 \cdot SoilLayerIdx+1] & = \min(LayerNit,TotalNitUptake) \\
      currentSoilLayer &  = currentSoilLayer.Next() \\
      SoilLayerIdx & = SoilLayerIdx+ 1
  \end{align*}  
   ResourceCompetition(ResourceNumber, AttackerNumber, PlantDemand, $NitrogenUptakeAbility_0$, $NitrogenResource_0$, $PlantUptake_0$)

      $ delete [sizeof(double) \cdot MaxSoilLayers] TotalRoot.rootLengthList $

     
   delete [] TotalRoot.rootLengthList,  
   delete TotalRoot

   if $rootList \ne NULL$ then (Perform only if plants has roots)
 	
        \quad  soilLayer currentSoilLayer = first, 
   	SoilLayerIdx  = 0
        
   	\quad while $currentSoilLayer  \ne  NULL$ do
   	
        \quad\quad   for all j in 0..AttackerNumber-1 do   
            \begin{align*}
             NH4LayerUptake & = currentSoilLayer.SubtractAmmUptake( \\
               & \quad PlantUptake[2 \cdot SoilLayerIdx+j \cdot ResourceNumber]) \\
            NO3LayerUptake & = currentSoilLayer.SubtractNitUptake( \\
               & \quad PlantUptake[2 \cdot SoilLayerIdx+j \cdot ResourceNumber+1]) \\
            TotalPlantUptake_{j} & = TotalPlantUptake_{j} + NH4LayerUptake + NO3LayerUptake
  	    \end{align*}		
        \begin{align*}
         currentSoilLayer & = currentSoilLayer.Next() \\
         SoilLayerIdx & = SoilLayerIdx + 1
       \end{align*} 	
   for all i in 0..AttackerNumber-1 do
   	$nitrogenUptake_{i} = TotalPlantUptake_{i}$

   return nitrogenUptake

\subsection{soilProfile.EstimateTotalRoot(rootStructure TotalRoot,linkList<rootStructure> rootList)}
   Generates average total rootDens

   TotalRoot.rootRadius = TotalRoot.rootpF = TotalRoot.NitUptakeRate 

= TotalRoot.AmmUptakeRate=   
   TotalRoot.MinimumSoilNit  = 
   TotalRoot.MinimumSoilAmm  = 0

  $ TotalRoot.rootLengthList$  = new double$[MaxSoilLayers]$  (LEAK !!!)
 
   for ($j=0j<MaxSoilLayersj++$) do $TotalRoot.rootLengthList_{j} = 0$

   % TotalPlantRootLength[MaxPlants]

   TotalRootLength = 0, 
   AttackerNumber = 0

   if rootList then AttackerNumber = rootList.NumOfNodes()

   for all i in 0..AttackerNumber-1 do   
      \begin{align*}
      currentRoot & = rootList.ElementAtNumber(i) \\
      TotalPlantRootLength_{i} & = 0
      \end{align*}
      for all j in 0..MaxSoilLayers-1 do
   		$TotalPlantRootLength_{i} += currentRoot.rootLengthList_{j}$

     $ TotalRootLength += TotalPlantRootLength_{i}$
   
   if $TotalRootLength>0$ then 

\quad	   for all i in  0..AttackerNumber-1 do 	
  
\quad\quad	   	$currentRoot = rootList.ElementAtNumber(i)$
	     
\quad\quad	$ TotalRoot.rootRadius +=\tfrac{TotalPlantRootLength_{i}}{TotalRootLength} currentRoot.rootRadius$
	   	
\quad\quad	$TotalRoot.rootpF +=\tfrac{TotalPlantRootLength_{i}}{TotalRootLength} currentRoot.rootpF$
		
\quad\quad	$TotalRoot.NitUptakeRate +=\tfrac{TotalPlantRootLength_{i}}{TotalRootLength} currentRoot.NitUptakeRate$
		
\quad\quad	$TotalRoot.AmmUptakeRate +=\tfrac{TotalPlantRootLength_{i}}{TotalRootLength} currentRoot.AmmUptakeRate$
		
\quad\quad	$TotalRoot.MinimumSoilNit +=\tfrac{TotalPlantRootLength_{i}}{TotalRootLength} currentRoot.MinimumSoilNit$
		
\quad\quad	$TotalRoot.MinimumSoilAmm +=\tfrac{TotalPlantRootLength_{i}}{TotalRootLength} currentRoot.MinimumSoilAmm $

	\quad      for all j in 0..MaxSoilLayers-1 do

	 	$TotalRoot.rootLengthList_{j} += currentRoot.rootLengthList_{j}$
	  
  


\subsection{soilProfile.ResourceCompetition(ResourceNumber, AttackerNumber, double  Demand, double  UptakeAbility, double  Resource, double ResourceRemoved)}
   % PossibleUptake[MaxPlants]
   %$MaxResourceUptake[2 \cdot MaxSoilLayers]$

   for all j in 0..AttackerNumber-1 do $PossibleUptake_{j} = 0$

   for all i in 0..ResourceNumber-1 do $MaxResourceUptake_{i} = 0$

   for all j in 0..AttackerNumber-1 do

   \quad	for all i in 0..ResourceNumber-1 do
    	$ResourceRemoved_{i+j \cdot ResourceNumber} = 0$

   bool DemandFullfilled = false

   while !DemandFullfilled   do

  \quad    DemandFullfilled = true

  \quad  for all j in 0..AttackerNumber-1 do

 \quad\quad	   	for all i in 0..ResourceNumber-1 do

 \quad\quad\quad	      	$MaxResourceUptake_{i} += UptakeAbility_{i+j \cdot ResourceNumber}$

   	for all j in 0..AttackerNumber-1 do
    
	\quad   	for all i in 0..ResourceNumber-1 do

         \quad\quad   if $MaxResourceUptake_{i}>0$ then

            \quad\quad\quad	$PossibleUptake_{j} += UptakeAbility_{i+j \cdot ResourceNumber} \cdot 
\min(1,\tfrac{Resource_{i}}{MaxResourceUptake_{i}}$
        
            \quad\quad $DemandFullfilled = DemandFullfilled \& (PossibleUptake_{j}>Demand_{j})$
    

      if DemandFullfilled  then 
   
      \quad\quad	for all j in 0..AttackerNumber-1 do

	\quad\quad\quad  for all i in 0..ResourceNumber-1 do

           \quad\quad\quad\quad 	if $PossibleUptake_{j}>0$ and  $MaxResourceUptake_{i}>0$ and  $Demand_{j}>0$ then
	          $ResourceRemoved_{i+j \cdot ResourceNumber} = \tfrac{Demand_{j}}{PossibleUptake_{j}} \cdot UptakeAbility_{i+j \cdot ResourceNumber} \cdot 
\min(1,\tfrac{Resource_{i}}{MaxResourceUptake_{i}})$
      else    
 
      	bool singleDemand = false
for all i in 0..AttackerNumber-1 do
      
	\quad   if (($PossibleUptake_{j}>Demand_{j}$)  and  ($Demand_{j}>0$))

           \quad\quad  (single plant demand can be fullfilled!)
          
          \quad\quad  	for all i in 0..ResourceNumber-1 do   
         
           \quad\quad\quad 		if  $PossibleUptake_{j}>0$  and  $MaxResourceUptake_{i}>0$ then

          \quad\quad\quad  \quad $ResourceRemoved_{i+j \cdot ResourceNumber} = \tfrac{Demand_{j}}{PossibleUptake_{j}} \cdot UptakeAbility_{i+j \cdot ResourceNumber} \cdot \min(1,\tfrac{Resource_{i}}{MaxResourceUptake_{i}})$
               
         \quad\quad\quad\quad $  UptakeAbility_{i+j \cdot ResourceNumber} = 0$
              
           \quad\quad   singleDemand = true

         \quad     $ Demand_{j} = 0$
         
      	

         if !singleDemand then
        (all plants uptake are below demand !!)

       \quad\quad     for all j in 0..AttackerNumber-1 do

       \quad\quad\quad        for ($i=0 i<ResourceNumber i++$)

         \quad\quad\quad\quad      
  if $PossibleUptake_{j}>0$  and $MaxResourceUptake_{i}>0$  and  
  $Demand_{j}>0$ 

         \quad\quad\quad\quad\quad  then
    $ResourceRemoved_{i+j \cdot ResourceNumber} = UptakeAbility_{i+j \cdot ResourceNumber}$
          
       $ \cdot \min(1,\tfrac{Resource_{i}}{MaxResourceUptake_{i}})$
         
   DemandFullfilled = true
         
\subsection{Update}   

\minisec{soilProfile.UpdateT(  soilTemp,airTemp,depth,meanAirTemp)}   
Calculates soil temp using a simple algorithm.
Returned values for periods without frost are good estimates, returned
values for periods with frost are fair estimates as frost and snow influence
are not accounted for in a fully satisfactory fashion using this simple method.
Konstants are presently hand-fitted (8.3.1999).   
  airTemp  - mean air temp of the current day (Celsius),
   depth  - soil depth in mm,  
meanAirTemp - yearly mean air temp for the site (Celsius)

  $ k1 =  30000,
   k2 = -0.00033, 
   k3 =  0.006,
   k4 =  0.12,  
   a1 = \tfrac{k1}{k1+(depth^2)} $
 
   if $soilTemp<k3 \cdot depth$ then $a1=a1 \cdot k4$
   \begin{align*}
   dampenFactor &=\exp(k2 \cdot depth) \\
   Tdampened &=((airTemp-meanAirTemp) \cdot dampenFactor) \\
             & +meanAirTemp   \\
             &   Dampen amplitude. Tdampened is target value for this depth \\
    \cdot soilTemp &= \cdot soilTemp+a1 \cdot (Tdampened- \cdot soilTemp) \\
             &     Pursue target value
  \end{align*}




\minisec{soilProfile.UpdateOrganicMatter(), UpdateNitrification()}
Updates organic matter pools in profile.
Performs nitrification and updates contents of ammonia and nit.
 returns Nitrification [$g N/m^2$/d]

   CO2Evolution  = 0,  soilLayer current  = first

   while $current  \ne  NULL$ do   

      \quad CO2Evolution =  CO2Evolution + current.UpdateOrganicMatter() |
          UpdateNitrification()

      \quad  current  = current.Next()
   
   return CO2Evolution



    
\minisec{nitrogen soilProfile.UpdateDenitrification(TotalCO2Emission)}
Performs denitrification and updates contents of nit.
 returns Nitrification [$g N/m^2$/d] 

   nitrogen sum, soilLayer current = first

   while $current \ne NULL$ do

    \quad  current.UpdateN2OFromDenitrification(TotalCO2Emission)
      
    \quad  sum  = sum  current.UpdateN2Production(), 
      current  = current.Next()
    
   return sum


% \subsection{nitrogen soilProfile.GetPoolNitrogen(char Name), GetPoolNitrogen(char Name,startDep,thick),GetPoolCarbon(char Name,startDep,thick)}
%    nitrogen Content, soilLayer current = first
% 
%    while $current  \ne  NULL$
%    
%       \quad Content  = Content + current.GetPoolNitrogen(Name) |  
%            GetPoolNitrogen(char Name,startDep,thick) | GetPoolCarbon(char Name,startDep,thick)
%       
%        \quad current  = current.Next()
%    
%    return Content





\subsection{Getter}


\minisec{nitrogen soilProfile.GetAmm(s,t), GetNit(s,t),
GetN2OFromNitrification(), GetN2OFromDenitrification(), GetChloride(s,t),
GetAmm(), GetNit(), GetCarbon(), GetOrganicNitrogen(), GetOrganicNitrogen(s,t),
GetOrganicCarbon(s,t), GetSoilMass(s,t), GetPoolNitrogen(Name), 
           GetPoolNitrogen(Name,s,t), GetPoolCarbon(Name,s,t)}
   nitrogen Content, soilLayer current = first, s - startDep, t - thick

   while $current  \ne  NULL$ do
     
      \quad Content = Content + current.GetAmm(s,t) |
GetNit(s,t) | GetN2OFromNitrification() | GetN2OFromDenitrification() |
GetChloride(s,t) | GetAmm() | GetNit() | GetCarbon() | GetOrganicNitrogen()
| GetOrganicNitrogen(s,t) | GetOrganicCarbon(s,t) | GetSoilMass(s,t) | GetPoolNitrogen(Name) |  
           GetPoolNitrogen(Name,s,t) | GetPoolCarbon(Name,s,t)

     \quad current = current.Next()
    
   return Content


\minisec{soilProfile.GetTotalWater(s,t), GetAvailWater(s,t),
GetAvailCapacity(s,t), GetFieldCapacity(s,t), GetWiltCapacity(s,t)
}
Calculates (soil water content |                 
 plant available soil water content  | 
 capacity for plant available water |   
 field capacity for water | wilting capacity for) for a portion of the profile.

\begin{tabular}{ll}
   s - startDep   & Start depth of layer to be calculated for [mm] \\
   t - thick     & Thickness of layer to be calculated for [mm] \\
   returns    & Capacity for plant available water [mm]
\end{tabular}

 Content  = 0, 
 soilLayer current = first

 while $current  \ne  NULL$ do
 
\quad Content = Content + current.GetTotalWater(s,t) | 
 GetAvailWater(s,t) |
 GetAvailCapacity(s, t) | 
 GetFieldCapacity(s,t) | 
 GetWiltCapacity(s,t) 
  
\quad current  = current.Next()

 return Content




\minisec{soilProfile.GetClayContent(startdepth, thick)}
   $Content = TotalWeight=0,
   soilLayer current=first$

   while $current  \ne  NULL$ do
     
    \quad
      top=current.GetStartDepth(),
      th=current.GetThickness()
    
    \quad  if $top+th>startdepth$  and  $top<startdepth+thick$ then    
            (Test if within range)
       
     \quad $ slice = \begin{cases}
                 \max(0,top+th-startdepth)  & \text{ if } (top<startdepth) \\
                 \max(0,startdepth+thick-top) & \text{ if } ((top+th)>(startdepth+thick)) \\
                 th & \text{ otherwise}
                \end{cases}$   
 \begin{align*}
        TotalWeight&=  TotalWeight+ slice current.GetDryBulkDensity() \\
        Content&= Content+ slice current.GetDryBulkDensity() current.GetClayContent() \\   
      current & = current.Next()  
\end{align*} 
   return $\tfrac{Content}{TotalWeight}$  


\minisec{soilProfile.GetMaximumDepth()}
   %if (soilLayerArray_{NumOfSoilLayers-1}=NULL)
%\subsection{   %   theMessage.FatalError("soilProfile.GetMaximumDepth - array points at NULL")}
   %else
      return $soilLayerArray_{NumOfSoilLayers-1}.GetEndDepth()$
   %return 0

\minisec{nitrogen soilProfile.GetTotalNitrogen()}
   return  GetAmm()+GetNit()+GetOrganicNitrogen()


\minisec{nitrogen soilProfile.GetNitLeaching(depth),GetAmmLeaching(depth)}  
    bool found = false, 
   soilLayer current = first
  
   while $ not found$  and  $current  \ne NULL$ do

 \quad     if ($depth  \ge  current.GetStartDepth()$  
\quad \quad \quad and  $depth\le current.GetStartDepth()+current.GetThickness()$)
	    
    \quad  \quad   then  $  N = current.GetNitLeaching() | GetAmmLeaching(depth)$, 
    $found = true$

  \quad  $  current = current.Next() $  

   return N

\minisec{soilProfile.GetNumberOfLayers()}  
   n  = 0,  soilLayer current  = first

   while $current  \ne  NULL$  do   
      do n = n + 1,
      current = current.Next()
      
   return n

\minisec{soilProfile.GetLayerThickness(No)}   
   n = 0, thick  = 0, soilLayer current  = first
 
   while $current  \ne  NULL$ do
    
      \quad if n=No then thick = current.GetThickness(), n = n + 1, current = current.Next()
    
   return thick

\minisec{soilProfile.GetTemp(depth)}
The temp of a given soil layer is defined at the top of the layers

  i=0

  while ($soilLayerArray_{i}.GetStartDepth()\le depth$  and $ i<NumOfSoilLayers$)

   \quad  i++, if $i=NumOfSoilLayers-1$
     then  return $ soilLayerArray_{i}.GetTemp()$

$ a=depth-soilLayerArray_{i-1}.GetStartDepth()$,

$   b=soilLayerArray_{i}.GetStartDepth()-depth   $

 return $soilLayerArray_{i-1}.GetTemp() \cdot \tfrac{ b}{a+b}
          +soilLayerArray_{i}.GetTemp() \cdot \tfrac{a}{a+b}$

\minisec{soilProfile.GetDrainage(depth)}
  if $depth\le \tfrac{soilLayerArray_0.GetThickness()}{2}$ then
		return $soilLayerArray_0.GetWaterFlux()$

  if $depth  \ge  (soilLayerArray_{NumOfSoilLayers-1}.GetStartDepth()$

\quad	\quad \quad \quad$ +\tfrac{soilLayerArray_{NumOfSoilLayers-1}.GetThickness()}{2})$ 

\quad \quad then
		return $soilLayerArray_{NumOfSoilLayers-1}.GetWaterFlux()$

	i=0

	while $soilLayerArray_{i}.GetStartDepth()+\tfrac{soilLayerArray_{i}.GetThickness()}{2} <depth$ do

	\quad	i++,
  if $i=NumOfSoilLayers-1$ then
	return $soilLayerArray_{i}.GetWaterFlux()$

   return $soilLayerArray_{i-1}.GetWaterFlux()$





\minisec{soilProfile.GetMaximumEvap()}
Estimated maximum Evap from whole profiles

   soilLayer current  = first, sum  = 0

   while $current  \ne  NULL$ do
    \begin{align*}
     sum & = sum + \max(0,current.GetAvailWater()) \\
            &  \cdot \exp(-EvapExtCoef current.GetCenterDepth()) \\  %current.GetThickness() \cdot 
     current & = current.Next()
    \end{align*}
   return sum


\subsection{Budget}
  
\minisec{soilProfile.StartBudget()}
  soilLayer current = first

  while $current  \ne  NULL$  do current.StartBudget(), current = cu

\minisec{soilProfile.EndBudget()}

soilLayer current = first

while $current  \ne  NULL$ do
  current.EndBudget(),
 current = current.Next()



\section{Soil layer}


define $\pi$   3.14159265358979323846
\begin{align*}
HeatCapacityFreezingWater & = 334400 [J/kg] \\
HeatCapacityWater  &  = 4192  [J/kgC] \\
HeatCapacityIce & = 2000   [J/kgC] \\
HeatCapacitySolid  & = 750   [J/kgC] \\
WaterDensity   & = 1000   [kg/m^3]
\end{align*}

\subsection{Update}



\minisec{soilLayer.UpdateOrganicMatter()}
Update organic matter pools in layer.
   \begin{align*}
   TillageEffectTemp & = \max(0,TillageEffectTemp-0.05) \\
   NewTemp & = \min(37.0,TillageEffectTemp+temp) \\
%
   pFValue & = pF.GetpF(\tfrac{waterContent}{thickness})
\end{align*}
\begin{align*}
   nitrogen nit  & = nit\_mob + nit\_imob &
   nitrogen amm & = amm\_mob + amm\_imob   
\end{align*}
 nitfrac = ammfrac =    unprotecFrac = 0

   if $nit.n>0$ then $  nitfrac = \max(0,\min(1,\tfrac{nit\_mob}{nit})$
    then
  
  if $amm\_mob.n>0$  and  $amm.n >0$ then
   $  ammfrac = \max(0,\min(1,\tfrac{amm\_mob}{amm})$

        Deals with protected NH4



   if $amm.n>0$ then
   	$unprotecFrac = \tfrac{AccesibleAmm()}{amm.n}$
  \begin{align*}
    nitrogen ProtectedNH4\_mob & = amm\_mob \cdot (1-unprotecFrac) \\
   nitrogen ProtectedNH4\_imob & = amm\_imob \cdot (1-unprotecFrac) \\
   amm & = amm \cdot unprotecFrac
 \end{align*}
   \begin{align*}
  CO2Evolution & = OrganicMatter.Update(NewTemp,pFValue,nit,amm) \\
  nitfrac & = FractionMobileWater() \\
  ammfrac & = nitfrac \\
  nit\_mob  & = nit \cdot nitfrac \\
  nit\_imob & = nit \cdot (1-nitfrac) \\
  amm\_mob & = amm \cdot ammfrac+ProtectedNH4\_mob \\ 
  amm\_imob & = amm \cdot (1-ammfrac)+ProtectedNH4\_imob
 \end{align*}
  return CO2Evolution





\minisec{nitrogen soilLayer.UpdateNitrification()}
\citep{chatskikh2005simulation}
Perform nitrification and update contents of ammonia and nit.
returns Nitrification [$g N/m^2$/d]


   %if (amm\_imob.n<-1e-10  or  amm\_mob.n<-1e-10  or  amm\_imob.n15<-1e-10  or  amm\_mob.n15<-1e-10)
%\subsection{   %	theMessage.WarningWithDisplay("SoilLayer.UpdateNitrification NH4 less than zero")}
  %	nitrogen AddN
 
   ratio =    N15RatioAmm = 1
   
   if $(nit\_imob+nit\_mob).n>0$ then
   	$ratio = \tfrac{ nit\_mob}{nit\_imob+nit\_mob}$


   if $amm\_imob.n+amm\_mob.n>0$ then
  	$N15RatioAmm = \max(0,\min(1,
         \tfrac{amm\_imob.n15+amm\_mob.n15}{amm\_imob.n+amm\_mob.n}))$

       M-N kinetics DAISY parameters

%   /*  $NH4Conc = \tfrac{(amm\_imob+amm\_mob).n}{thickness/10}/10000 $           
%
%       unit g-N/cm-3
%  \begin{align*}
%   MNRate & = 5e-5 \cdot \tfrac{NH4Conc}{NH4Conc+5e-5} 
%             \cdot \tfrac{thickness} \cdot 10000 \\
%   Add & = MNRate % \cdot 
%  \end{align*}
%                TempNitrificationEffect(),
%                WaterNitrificationEffect() */

	    Old DAISY nitrification model
\begin{align*}  
	FirstOrderRate & = nitrificationRate \cdot AccesibleAmm() \\
   Add & = FirstOrderRate \cdot  TempNitrificationEffect() \cdot  
          WaterNitrificationEffect() 
\end{align*}
   AddN.SetBoth(Add,$Add \cdot N15RatioAmm$)
\begin{align*}
   N2Oratio & = RatioN2OEmissionNitri() \\
	ratio & = FractionMobileWater() \\
   nit\_mob & = nit\_mob + AddN \cdot ratio \cdot (1-N2Oratio) \\
   nit\_imob & = nit\_imob + AddN \cdot (1-ratio) \cdot (1-N2Oratio) 
\end{align*}
   $N2OFromNitrification = AddN \cdot N2Oratio$

    Remove amm,
   ratio = 1

   if $amm\_imob.n+amm\_mob.n>0$ then

   \quad	$ratio = \max(0,\min(1,\tfrac{amm\_mob.n}{amm\_imob.n+amm\_mob.n}))$
\begin{align*}
   amm\_mob.n & = amm\_mob.n - AddN.n \cdot ratio \\
   amm\_imob.n & = amm\_imob.n - AddN.n \cdot (1-ratio) 
\end{align*}        
   ratio = 1

   if $amm\_imob.n15+amm\_mob.n15>0$ then

   \quad	$ratio = \max(0,\min(1,\tfrac{amm\_mob.n15}{amm\_imob.n15+amm\_mob.n15}))$
   \begin{align*}
amm\_mob.n15  = amm\_mob.n15 - AddN.n15 \cdot ratio \\
   amm\_imob.n15 = amm\_imob.n15 - AddN.n15 \cdot (1-ratio)
\end{align*}
  %if (amm\_imob.n<-1e-10  or  amm\_mob.n<-1e-10  or  amm\_imob.n15<-1e-10  or  amm\_mob.n15<-1e-10)
%\subsection{  % 	theMessage.WarningWithDisplay("SoilLayer.UpdateNitrification NH4 less than zero")}
   Nbudget.AddOutput(N2OFromNitrification.n),
   N15budget.AddOutput(N2OFromNitrification.n15)

   return $ AddN \cdot (1-N2Oratio)$


\minisec{soilLayer.RatioN2OEmissionNitri()}
Perform N2O emission from nitrification. %returns emission

  % N2Oratio, Approach by Ingwersen et al, 1999
  \citep{chatskikh2005simulation} %???
  \begin{align*}
  tFactor & = \max(0,\min(1,\exp(-0.5 \cdot (\tfrac{temp}{2 \cdot 3.14 \cdot 2.72}-2)^2))) \\
  WaterFactor & = \max(0,\min(1,GetWaterFilledPorosity()))
\end{align*}

% Combined emission factor from er Ambus (2001), anual effect as from N2Oratio = 0.28-0.48 (Maag and Vinther, 1996) 
% N2Oratio = 0.006
\begin{align*}
 N2Oratio & = 0.047 \cdot tFactor \cdot WaterFactor          
  \end{align*}
return N2Oratio

\minisec{soilLayer.TempNitrificationEffect()}  
Calculates the effect of temp on nitrification.
returns relative effect. 
\citep{berntsen2005simulation,hansen1990npo} (DAISY page 123)

return $ \begin{cases}
       0 & \text{ if } temp \le 1 \\
 0.125 \cdot (temp-1.) & \text{ if } temp \le 5 \\
0.1 \cdot temp & \text{ if } temp\le 20 \\
\exp(0.4657-0.027259 \cdot temp+0.0019315 \cdot temp^2) & \text{ otherwise}
    \end{cases}$   

\minisec{soilLayer.WaterNitrificationEffect() } 
Calculates the effect of soil water content on nitrification.
returns relative effect
  
$pf = pF.GetpF(\tfrac{waterContent}{thickness})$

return $ \begin{cases}
     0 & \text{ if } pf\le 0 \text{ or } pf  \ge  5 \\
\tfrac{pf}{1.5} & \text{ if } pf<1.5 \\
1 & \text{ if } (pf<2.5) \\
\tfrac{5-pf}{2.5}  & \text{ otherwise}
    \end{cases} $


\minisec{soilLayer.UpdateN2OFromDenitrification(TotalCO2Emission)}
\citep{chatskikh2005simulation}
Perform denitrification and update contents of nit.
returns Denitrification [$g N/m^2$/d]

  % nitrogen NRemove

     Water effect on denitrificaiton, data presented in Parton et al. (2000)

  $ WaterEffect = \max\left(0,\min\left(1,0.0116+
   \tfrac{1.36}{1+\exp(-(GetWaterFilledPorosity()-0.815)/0.0896)}\right)\right)$

        N effect on denitrificaiton, based on loam soil
   \begin{align*}
   NitConc & = \tfrac{(nit\_mob.n+nit\_imob.n)}{thickness} \cdot \tfrac{1000}{dryBulkDensity} \cdot 1e6    
     \text{ (unit: mg/kg) } \\
   NitEffect & = \max(0,\min(1,1.17 \cdot \tfrac{NitConc}{32.7+NitConc}))
  \end{align*}
        Temp effect on denitrificaiton
   \begin{align*}
   tempAdjust & = 1/\exp(-3.432+0.168 \cdot 10 \cdot (1-0.5 \cdot \tfrac{10}{36.9})) \\
   TempEffect & = tempAdjust \cdot \exp(-3.432+0.168 \cdot temp \cdot 
(1-0.5 \cdot \tfrac{temp}{36.9}))
\end{align*}
        kd estimation

	$DenPotClay = 0.286+0.0028 \cdot clayContent \cdot 100  $

    clay effect is based on Drury et al., 1991  (Based on linear regression of data - JBE)

        Actual denitrification
\begin{align*}
	Denitrification & = DenPotClay \cdot PotentialCarbonTurnover() \\
                         & \quad      \cdot WaterEffect \cdot NitEffect \cdot TempEffect \\
   Denitrification & = \min(Denitrification,0.9999 \cdot (nit\_mob.n+nit\_imob.n))
\end{align*}
   if $Denitrification>0$ then   
    N15RatioNit= 1

  if $nit\_imob.n+nit\_mob.n>0$ then

   \quad $N15RatioNit = \max(0,\min(1,\tfrac{nit\_imob.n15+nit\_mob.n15}{nit\_imob.n+nit\_mob.n}))$

      NRemove.SetBoth(Denitrification,$N15RatioNit \cdot Denitrification$)

      N from nit pools

	\quad   ratio = 1.

     	if $nit\_imob.n+nit\_mob.n>0$ then

      	\quad $ratio = \max(0,\min(1,\tfrac{nit\_mob.n}{nit\_imob.n+nit\_mob.n}))$
\begin{align*}
     	nit\_mob.n & = nit\_mob.n - ratio \cdot Denitrification \\
     	nit\_imob.n & = nit\_imob.n - (1-ratio) \cdot Denitrification
\end{align*}
           15N from nit pools

      \quad ratio = 1.

     	if $nit\_imob.n15+nit\_mob.n15>0$ then
      
	\quad $ratio = \max(0,\min(1,\tfrac{nit\_imob.n15}{nit\_imob.n15+nit\_mob.n15}))$     	
\begin{align*}
         nit\_imob.n15 & = nit\_imob.n15 - N15RatioNit \cdot ratio \cdot Denitrification \\     	
         nit\_mob.n15 & =  nit\_mob.n15 - N15RatioNit \cdot (1-ratio) \cdot Denitrification
\end{align*}
      %if(nit\_mob.n<-0 or  nit\_imob.n<-0)
%\subsection{   	%	theMessage.WarningWithDisplay("soilLayer.Denitrification - negative nit concs")}
%
%	
%
   Nbudget.AddOutput(NRemove.n),
   N15budget.AddOutput(NRemove.n15),
   N2OFromDenitrification = NRemove


\minisec{soilLayer.PotentialCarbonTurnover()}
Potential carbon turnover used in denitrification model

 return OrganicMatter.GetCarbonAvailability()



\minisec{nitrogen soilLayer.UpdateN2Production()}
\citep{chatskikh2005simulation}
Perform estimation of N2 production from N2O pool.
%returns - emission

	nitrogen TotalN2OProduction = N2OFromDenitrification + N2OFromNitrification

        W-factor (deficit of oxygen), based on Nommik (1956) and Wijler and Delwiche (1954)
        inversed water factor
   \begin{align*}
   fWater & = 1-(0.0116+\tfrac{1.36}{1+\exp(-\tfrac{GetWaterFilledPorosity()-0.815}{0.0896})}) \\
   fWater & = \max(0,\min(1,fWater))
   \end{align*}
        Diffusion model substitution the profile N2O also will be included!!!

   $fTemp = \tfrac{1 }{1+\exp(-0.64+0.08 \cdot temp) }$   %  Finn's temp factor

        Updated for loam soil to the clayContent range, corr. with data from Letey et al. (1980)
   
$fClay = 1.26 \cdot \exp(-1.16 \cdot clayContent)-0.249$

        Soil N20 emission profile for loam soil % , based on Yoh et al. (1999)
   \begin{align*}
   depth & = \tfrac{GetCenterDepth()}{1000} \\
   fProfile & = \max(0,\min(1,1.0008-0.0343 \cdot depth-3.1816 \cdot depth^2))
  \end{align*}
        N2 evolution and distribution of emitted N2O between the sources
\begin{align*}
   N2ORatio & = fClay \cdot fWater \cdot fTemp \cdot fProfile      N2O reduction corrected!
\end{align*}
\begin{align*}
   nitrogen N2OEmission & = TotalN2OProduction \cdot N2ORatio \\
   N2OFromDenitrification & = N2OFromDenitrification \cdot N2ORatio \\
   N2OFromNitrification & = N2OFromNitrification \cdot N2ORatio
\end{align*}
\begin{align*}
   Denitrification = TotalN2OProduction \cdot (1-N2ORatio)
\end{align*}
   return $ TotalN2OProduction \cdot (1-N2ORatio) $

     notice this sum only includes N2 from denitrification



\subsection{soilLayer.Initialize(startDep,thick,fstream f)}
   Setfile(f),   
   $startDepth   = startDep, 
   thickness    = thick, 
   center  = startDep+0.5 \cdot thick, 
   foundLayer  =false$
   
   SetCritical(), GetSectionNumbers("SoilLayer",first,n),  i=first,
   topOfLayer=0

   while not foundLayer  and  i<first+n do

      \quad FindSection("SoilLayer",i),
      totalThickness,
      GetParameter("Thickness",totalThickness)

   %   if ($totalThickness\le 0$)

%\subsection{      	Terminate("soilLayer.Initialize - Thickness must be above zero")}

      foundLayer=($center  \ge  topOfLayer$)  and  ($center\le (topOfLayer+totalThickness)$)

      if foundLayer then
     
      \quad   UnsetCritical(),
      soilType=5

      \quad   GetParameter("SoilType",soilType),
         GetParameter("Ammonium",amm\_imob.n)

    \quad    $ amm\_imob=amm\_imob \cdot \tfrac{thickness}{totalThickness}$, 
        $amm\_mob.n        = 0$

    \quad     GetParameter("Nitrate",nit\_imob.n)

    \quad     $nit\_imob=nit\_imob \cdot \tfrac{thickness}{totalThickness}$, 
      $nit\_mob.Clear()$    


   \quad      GetParameter("SiltContent",siltContent)

  \quad       if not GetParameter("ClayContent",clayContent) then
        
  \quad  JBC[] = {0.025,0.025,0.075,0.075,0.125,0.125,0.20,0.35,0.50,0.10}
         
  \quad JB = 5

    \quad        if $soilType  \ge  1$  and $ soilType\le 10$ then
          	JB = soilType

   \quad         clayContent= JBC[JB-1]   
     
  \quad       if !GetParameter("DrainageConstant",drainageConstant) 
  then 

Function from \citep{addiscott1991simulation}
%        Function from Addiscot Whitmore 1991, Soil Use. Man., 94-102 
          \begin{align*}
            drainageConstant&=1.0271-3.02 \cdot clayContent^2      \\               
            drainageConstant&=\min(1,\max(0.1,drainageConstant))   \text{ Cut at [0.11.0]} \\
            drainageConstant&=\min(1,drainageConstant \cdot 15.462 \cdot thickness^{-0.7}    
         \end{align*}
    Fitted to Fig. 2c, \citep{addiscott1991simulation} 

   \quad       if !GetParameter("HoldbackConstant",holdbackConstant) then
       holdbackConstant=0.1

   \quad         if $startDep>199$ then $holdbackConstant=0.3$

  \quad\quad       if $!GetParameter("DryBulkDensity",dryBulkDensity)$ then
          	dryBulkDensity=1.5
        \begin{align*}
         porosity & = 1-\tfrac{dryBulkDensity}{2.65} \\
         dryBulkDensity & =dryBulkDensity \cdot 1E6 \\
         carbon & =0
         \end{align*}
         GetParameter("CarbonPct",carbon)

	$bool TopSoil = (startDep<250)$
      
   $pF = new pF\_Curve(0) $

     Interpolation method set to zero. Correct ?
        
 pF.ReadParameters(f,i,dryBulkDensity,clayContent,siltContent,carbon,TopSoil)     

          Extra parameters added for support of Mualem van Genucten model
      
  
 pF\_FC = 2

         if $soilType=1$    then pF\_FC = 1.8
\begin{align*}
         fieldCapacity & = pF.GetRelativeWater(pF\_FC) \cdot thickness \\
         waterContent & = fieldCapacity \\
         MaxContent & = porosity \cdot thickness
\end{align*}
         if  $fieldCapacity>MaxContent$ then
         	$porosity = pF.GetRelativeWater(0)  $  (HACK?)
        
         if $porosity\le 0.9 \cdot pF.GetRelativeWater(0)$ then
         	$porosity = pF.GetRelativeWater(0)  $  (HACK?)
         \begin{align*}
         wiltCapacity & = pF.GetRelativeWater(4.2) \cdot thickness \\
         nitrificationRate   & = 0.1 \\
         tortuosityLimit     & = 0.1 \\
         carbon &=carbon \cdot dryBulkDensity \cdot \tfrac{thickness}{10E4} \\
         char sfn[80] &="cnmodel.dat"
         \end{align*}
         GetParameter("SomFileName",sfn)

         OrganicMatter.Initialize(clayContent,carbon,f,i,sfn)

            topOfLayer = topOfLayer+totalThickness

      i++
   

   if not foundLayer then

  \quad  Terminate("soilLayer.Initialize - no appropriate layer found")

   \quad Setfile(NULL),
   DeleteInputFile()


\subsection{soilLayer.soilLayer(char aName, aIdx, base aOwner): base(aName,aIdx,aOwner)}
   %next    = NULL
   %pF     = NULL

   OrganicMatter     = new organicMatter("OrganicMatter",0,this)

   ammLeaching.Clear(), 
   nitLeaching.Clear(),
   nit\_mob.Clear(),
   nit\_imob.Clear(),
   amm\_mob.Clear(),
   amm\_imob.Clear(),
   N2OFromNitrification.Clear(),
   N2OFromDenitrification.Clear(),
   Denitrification.Clear()
   
   temp = 8,
   equilibrated   = false,

  waterContent             = 
   waterFlux              =    
   iceContent              = 
   TillageEffectTemp  = 
    clayContent              = 
   Chloride\_mob             = 
   Chloride\_imob            = 
   drainageConstant        = 
   holdbackConstant       = 
   dryBulkDensity           = 
   porosity               = 
   fieldCapacity         = 
   startDepth           = 
   thickness              = 
   wiltCapacity           = 
   nitrificationRate      = 
   tortuosityLimit        = 
   waterUptake             = 
   CO2Evolution            = 
   mobility                = 
   mobility1             = 0

   Nbudget.SetNames("soillayer","N"),
   N15budget.SetNames("soillayer","N15"),
   WaterBudget.SetNames("soillayer","water")


%\subsection{% soilLayer.soilLayer(soilLayer& Layer): base(Layer)}
%  next = NULL
%    TillageEffectTemp = Layer.TillageEffectTemp
%   waterContent             = Layer.waterContent
%   nit\_mob              = Layer.nit\_mob
%   nit\_imob             = Layer.nit\_imob
%   amm\_mob             = Layer.amm\_mob
%   amm\_imob            = Layer.amm\_imob
%   temp              = Layer.temp
%   drainageConstant         = Layer.drainageConstant
%   holdbackConstant         = Layer.holdbackConstant
%   dryBulkDensity           = Layer.dryBulkDensity
%   porosity                 = Layer.porosity
%   fieldCapacity            = Layer.fieldCapacity
%   startDepth               = Layer.startDepth
%   thickness                = Layer.thickness
%   wiltCapacity             = Layer.wiltCapacity
%   clayContent              = Layer.clayContent
%   nitrificationRate        = Layer.nitrificationRate
%   tortuosityLimit          = Layer.tortuosityLimit
%   waterUptake              = Layer.waterUptake
%   waterFlux                = Layer.waterFlux
%   CO2Evolution             = Layer.CO2Evolution
%   ammLeaching         = Layer.ammLeaching
%   nitLeaching          = Layer.nitLeaching
%   iceContent               = Layer.iceContent
%   equilibrated             = Layer.equilibrated
%   Chloride\_mob             = Layer.Chloride\_mob
%   Chloride\_imob            = Layer.Chloride\_imob
%   mobility                 = Layer.mobility
%   mobility1                = Layer.mobility1
%   N2OFromNitrification     = Layer.N2OFromNitrification
%   N2OFromDenitrification   = Layer.N2OFromDenitrification
%   Denitrification          = Layer.Denitrification
%   OrganicMatter            = new organicMatter( \cdot Layer.OrganicMatter)
%   OrganicMatter.SetOwner(this)
%   pF                       = new pF\_Curve( \cdot Layer.pF)
%   Nbudget			         = budget(Layer.Nbudget)
%   Nbudget.Reset()              History ignored!
%   N15budget			         = budget(Layer.N15budget)
%   N15budget.Reset()            History ignored!
%   WaterBudget 	            = budget(Layer.WaterBudget)
%   WaterBudget.Reset()          History ignored!
 

% 
% pF curves are assumed to be equal. This assumption is not tested though.
% 
%\subsection{% soilLayer.Add(soilLayer \cdot  Layer, frac)}
%   TillageEffectTemp= TillageEffectTemp \cdot (1-frac)+Layer.TillageEffectTemp \cdot frac
%   nit\_mob 				= nit\_mob \cdot (1-frac)+Layer.nit\_mob \cdot frac
%   nit\_imob 			= nit\_imob \cdot (1-frac)+Layer.nit\_imob \cdot frac
%   amm\_mob 			= amm\_mob \cdot (1-frac)+Layer.amm\_mob \cdot frac
%   amm\_imob 			= amm\_imob \cdot (1-frac)+Layer.amm\_imob \cdot frac
%   ammLeaching 		= ammLeaching \cdot (1-frac)+Layer.ammLeaching \cdot frac
%   nitLeaching 		= nitLeaching \cdot (1-frac)+Layer.nitLeaching \cdot frac
%   N2OFromNitrification  = N2OFromNitrification \cdot (1-frac)+Layer.N2OFromNitrification \cdot frac
%   N2OFromDenitrification= N2OFromDenitrification \cdot (1-frac)+Layer.N2OFromDenitrification \cdot frac
%   Denitrification       = Denitrification \cdot (1-frac)+Layer.Denitrification \cdot frac
% 
%   waterContent 			= (1-frac) \cdot waterContent+frac \cdot Layer.waterContent
%   temp 				= (1-frac) \cdot temp+frac \cdot Layer.temp
%   drainageConstant 		= (1-frac) \cdot drainageConstant+frac \cdot Layer.drainageConstant
%   holdbackConstant 		= (1-frac) \cdot holdbackConstant+frac \cdot Layer.holdbackConstant
%   dryBulkDensity 			= (1-frac) \cdot dryBulkDensity+frac \cdot Layer.dryBulkDensity
%   porosity 					= (1-frac) \cdot porosity+frac \cdot Layer.porosity
%   fieldCapacity 			= (1-frac) \cdot fieldCapacity+frac \cdot Layer.fieldCapacity
%   startDepth 				= (1-frac) \cdot startDepth+frac \cdot Layer.startDepth
%   thickness 				= (1-frac) \cdot thickness+frac \cdot Layer.thickness
%   wiltCapacity         	= (1-frac) \cdot wiltCapacity+frac \cdot Layer.wiltCapacity
%   clayContent          	= (1-frac) \cdot clayContent+frac \cdot Layer.clayContent
%   nitrificationRate    	= (1-frac) \cdot nitrificationRate+frac \cdot Layer.nitrificationRate
%   tortuosityLimit 		= (1-frac) \cdot tortuosityLimit+frac \cdot Layer.tortuosityLimit
% 
%   waterUptake = (1-frac) \cdot waterUptake+frac \cdot Layer.waterUptake
%   waterFlux = (1-frac) \cdot waterFlux+frac \cdot Layer.waterFlux
%   CO2Evolution = (1-frac) \cdot CO2Evolution+frac \cdot Layer.CO2Evolution
%   iceContent= (1-frac) \cdot iceContent+frac \cdot Layer.iceContent
%   Chloride\_mob    = (1-frac) \cdot Chloride\_mob+frac \cdot Layer.Chloride\_mob
%   Chloride\_imob   = (1-frac) \cdot Chloride\_imob+frac \cdot Layer.Chloride\_imob
%   mobility   = (1-frac) \cdot mobility+frac \cdot Layer.mobility
%   mobility1  = (1-frac) \cdot mobility1+frac \cdot Layer.mobility1
% 
%   Nbudget.Add(Layer.Nbudget,frac)
%   N15budget.Add(Layer.N15budget,frac)
%   WaterBudget.Add(Layer.WaterBudget,frac)
%   OrganicMatter.Add(Layer.OrganicMatter,frac)





\subsection{soilLayer.SetIceContent(aIceContent)   }

  $ iceContent = \max(0,\min(waterContent,aIceContent))$

\subsection{Add}

\minisec{soilLayer.AddNext(soilLayer layer)}

 next = layer


\minisec{soilLayer.AddNutrient(nitrogen addNit, nitrogen addAmm)}
Adds nitrogen to layer.
addNit- Nit to be added [$g N/m^2$].
addAmm - Amm to be added [$g N/m^2$]

    if $waterContent > fieldCapacity$ then

\quad $nit\_mob  = nit\_mob + addNit$, $amm\_mob  = amm\_mob + addAmm$
    
 else 

\quad $nit\_imob = nit\_imob + addNit$,   
$amm\_imob  = amm\_imob + addAmm$    
      
   Nbudget.AddInput(addNit.n+addAmm.n),
   N15budget.AddInput(addNit.n15+addAmm.n15)


\minisec{soilLayer.AddWater(surplus, nitrogen nitLeached, nitrogen ammLeached,
    ChlorideLeached}
Perform infiltration of water, nit and amm in layer.
surplus         - Water to be added to current (next) layer [mm],
nitLeached/ammLeached  - Nit/Amm to the added to current (next) layer [$g N/m^2$],


   %if(amm\_mob.n<-1E-6  or  amm\_imob.n<-1E-6  or  nit\_mob.n<-1E-6  or  nit\_imob.n<-1E-6)
%\subsection{   %	theMessage.WarningWithDisplay("soilLayer.AddWater - negative n concs")}
%	if (surplus<0)
%\subsection{   %	theMessage.WarningWithDisplay("soilLayer.AddWater - amount of water to add can not be negative")}

   $waterContent  = waterContent + surplus$,
   WaterBudget.AddInput(surplus)

  $ fracOfSolutesToMobile = $

$\begin{cases}
                   0 & waterContent<fieldCapacity    \\
          \max(0,\min(1,\tfrac{waterContent-fieldCapacity}{surplus}) & waterContent-surplus<fieldCapacity 
        \\
               1    & \text{ otherwise}   
   \end{cases}$

(all solutes go into imobile domain |
 parts go into mobile |
 all go into mobile)        
      \begin{align*}
      nit\_mob & = nit\_mob+nitLeached \cdot fracOfSolutesToMobile \\
      nit\_imob & = nit\_imob+nitLeached \cdot (1-fracOfSolutesToMobile) \\
            amm\_mob & = amm\_mob+ammLeached \cdot fracOfSolutesToMobile \\
      amm\_imob & = amm\_imob+ammLeached \cdot (1-fracOfSolutesToMobile) \\
          Chloride\_mob & = Chloride\_mob+ChlorideLeached \cdot fracOfSolutesToMobile \\
      Chloride\_imob & = Chloride\_imob+ChlorideLeached \cdot (1-fracOfSolutesToMobile)
     \end{align*}
 %  if(amm\_mob.n<-1E-6  or  amm\_imob.n<-1E-6  or  nit\_mob.n<-1E-6  or  nit\_imob.n<-1E-6)
%\subsection{ %  	theMessage.WarningWithDisplay("soilLayer.AddWater - negative n concs")}
   Nbudget.AddInput(nitLeached.n+ammLeached.n)

   N15budget.AddInput(nitLeached.n15+ammLeached.n15)
  % if((waterContent+0.001)<iceContent)
%\subsection{ %  	theMessage.WarningWithDisplay("soilLayer.AddWater - ice exceeds total water at the end of the function")}




\minisec{soilLayer.AddWaterAndSolutes(waterInFlow, nitrogen nitAdd, nitrogen ammAdd, chlorideAdd)}
 Adds water, nit, amm and chloride to layer.
surplus - Water to be added to current (next) layer [mm],
nitLeached/ammLeached  - Nit/Amm to the added to current (next) layer [$g N/m^2$],
 
 
   waterContent  = waterContent  + waterInFlow, 
   WaterBudget.AddInput(waterInFlow)

  $ fracOfSolutesToMobile = $

$\begin{cases}
                   0 & waterContent<fieldCapacity      \\
         \max(0,\min(1,\tfrac{waterContent-fieldCapacity}{waterInFlow})) & waterContent-waterInFlow<fieldCapacity     
  \\
                   1    & \text{ otherwise}     
   \end{cases}$  

(all solutes go into imobile domain |
 parts go into mobile |
 all go into mobile)        
     \begin{align*}
 nit\_mob & = nit\_mob+nitAdd \cdot fracOfSolutesToMobile \\
      nit\_imob & = nit\_imob+nitAdd \cdot (1-fracOfSolutesToMobile) \\
      amm\_mob & = amm\_mob+ammAdd \cdot fracOfSolutesToMobile \\
      amm\_imob & = amm\_imob+ammAdd \cdot (1-fracOfSolutesToMobile) \\
      Chloride\_mob & = Chloride\_mob+chlorideAdd \cdot fracOfSolutesToMobile \\
      Chloride\_imob & = Chloride\_imob+chlorideAdd \cdot (1-fracOfSolutesToMobile)
\end{align*}
     Nbudget.AddInput(nitAdd.n+ammAdd.n),
   N15budget.AddInput(nitAdd.n15+ammAdd.n15)
   

\minisec{soilLayer.AddProduct(organicProduct product)}
   \begin{align*}
    organicProduct p & = new organicProduct(product) \\
   nit\_mob  & = nit\_mob+p.GetNit() &
   amm\_mob & = amm\_mob+p.GetAmm()
   \end{align*}
   p.ClearNO3\_content(),    p.ClearNH4\_content()
\begin{align*}
   nitrogen nit  & = nit\_mob + nit\_imob &
   nitrogen amm & = amm\_mob + amm\_imob  \\
   nitfrac & = 0
\end{align*}
   if $nit.n>0$ then $ nitfrac = \max(0,\min(1,\tfrac{nit\_mob}{nit}))$

   ammfrac =  unprotecFrac = 0

   if $amm\_mob.n>0$  and  $amm.n >0$ then
      $ammfrac = \max(0,\min(1,\tfrac{amm\_mob}{amm}))$

        Deals with protected NH4

   if $amm.n>0$ then
    $unprotecFrac = \tfrac{AccesibleAmm()}{amm.n}$
  \begin{align*}
   nitrogen ProtectedNH4\_mob  & = amm\_mob \cdot (1-unprotecFrac) \\
   nitrogen ProtectedNH4\_imob & = amm\_imob \cdot (1-unprotecFrac) \\
   amm & = amm \cdot unprotecFrac
  \end{align*}
   OrganicMatter.SetMineralNitrogen(nit,amm),
   OrganicMatter.AddProduct(p)

   OrganicMatter.GetMineralNitrogen(nit,amm)
\begin{align*}
   nitfrac &=FractionMobileWater() &
   ammfrac &= nitfrac \\
   nit\_mob & = nit \cdot nitfrac &
   nit\_imob &= nit \cdot (1-nitfrac) 
\end{align*}
   $amm\_mob = amm \cdot ammfrac+ProtectedNH4\_mob $,

   $amm\_imob = amm \cdot (1-ammfrac)+ProtectedNH4\_imob$

   Nbudget.AddInput(product.GetTotalNitrogen().n),
   N15budget.AddInput(product.GetTotalNitrogen().n15)

   delete p



\subsection{soilLayer.HeatCapacity() kg/m3/dgC}
Calculates the volumetric heatcapacity

	return $(HeatCapacitySolid \cdot \tfrac{dryBulkDensity}{1000}$

\quad	\quad\quad	\quad			$  +HeatCapacityWater \cdot WaterDensity \cdot 
\tfrac{waterContent-iceContent}{thickness}$
			
\quad	\quad\quad	\quad		 $ +HeatCapacityIce \cdot WaterDensity \cdot \tfrac{iceContent}{thickness})$


\subsection{soilLayer.SpecWaterCapacity()}

  $ relativeWater = \tfrac{waterContent}{thickness}$
   
   pF1 = pF.GetPressurePotential(relativeWater)

   if $waterContent < porosity \cdot thickness$ then

   \quad   
  specWater = $\begin{cases} $1e10$ 
      \text{ if $pF1=pF.GetPressurePotential(relativeWater-0.01)$} \\
   \tfrac{0.01}{
     pF1-pF.GetPressurePotential(relativeWater-0.01)} & \text{ otherwise}
\end{cases}$
  
   else

   	 $ specWater = \begin{cases} 1e10  \text{ if  $pF.GetPressurePotential(relativeWater+0.01)=pF1$ }  \\
      \frac{-0.01}{pF1-pF.GetPressurePotential(relativeWater+0.01)} & \text { otherwise} \end{cases}$
   
return $1.1019e-5 \cdot specWater$   (converts m to J m-3)





\subsection{soilLayer.HeatConductivity()}
     %  Taken from SOIL-N (1996 p18 and p24)
\citep{jansson2004CoupModel} - 27 pp.

  $a1=0.7, 
  a2=0.4, 
  a3=0.6245,  
  b1=0.00158, 
  b2=1.336, 
  b3=0.00375, 
  b4=0.9118$ 

  if $clayContent>0.1$ then
b1=0.00144,
b2=1.32, 
  b3=0.0036, 
  b4=0.8743  
   \begin{align*}  
   frozenPart& = \tfrac{iceContent}{waterContent} \\
   waterConcentration & = \max(1,waterContent) \cdot \tfrac{100}{thickness}  \\
   BulkDensity & = \tfrac{dryBulkDensity}{1000000}
  \end{align*}
     converted mm to vol %. Minimum of 1% according to COUP figure p 31
%
  % if (waterConcentration\le 0)
%\subsection{  %    theMessage.FatalError("soilLayer.HeatConductivity - water content zero or below")}
 \begin{align*}
  k\_hm & = 0.1434 \cdot (a1 \cdot log10(\tfrac{waterConcentration}{BulkDensity})+a2) \cdot 10^{a3 \cdot BulkDensity} \\
   k\_hmi & = b1 \cdot 10^{b2 \cdot BulkDensity}+b3 \cdot \tfrac{waterConcentration}{BulkDensity} \cdot 10^{b4 \cdot BulkDensity} \\
   result & = (1-frozenPart) \cdot k\_hm+frozenPart \cdot k\_hmi \\
   result & = \max(0.05,result)   
\end{align*}                                            
%According to COUP figure p 31
  % if (result>10)
%\subsection{   %	theMessage.WarningWithDisplay("soilLayer.HeatConductivity - conductivity to high")}
	return result





\subsection{Subtract}

\minisec{soilLayer.SubtractEvap(Evap)}
Subtracts Evap [mm] from layer.
 \begin{align*}
 draw &= \max(0,\min(waterContent-iceContent-1E-10,Evap)) \\
 waterContent & = waterContent - draw
 \end{align*}
 WaterBudget.AddOutput(draw)


\minisec{soilLayer.SubtractTransp(aWaterUptake)}
Subtracts estimated transp from the water in the layer.
aWaterUptake  - water uptake [mm]

   if (waterContent<aWaterUptake)
   	aWaterUptake=waterContent
  
   waterUptake  = aWaterUptake,
   waterContent  = waterContent - waterUptake
  
   WaterBudget.AddOutput(aWaterUptake)

\minisec{nitrogen soilLayer.SubtractNitUptake(nitUptake)}
  $
   nitrogen nit = nit\_mob+nit\_imob,
   N15RatioNit = nit.Get15NRatio()$
   
   %nitrogen NO3\_Uptake

   if $nitUptake>0$  and  $nit.n>0$ then
 	
   \quad   NO3\_Uptake.SetBoth(nitUptake,N15RatioNit  nitUptake)

  		$frac  = \max(0,\min(1,\tfrac{nit\_mob.n}{nit.n})$,  
 		if $frac<1e-10$ then $frac  = 0 $
\begin{align*}
 		nit\_mob.n  & = nit\_mob.n - nitUptake \cdot frac \\
  		nit\_imob.n & = nit\_imob.n - nitUptake \cdot (1-frac) \\
\end{align*}
          % 15N
  		frac = 1,
      if $nit.n15>0$ then $frac = \max(0,\min(1,\tfrac{nit\_mob.n15}{nit.n15}))$
      
     if $frac<1e-10$ then $frac = 0$
 		\begin{align*}
                nit\_mob.n15 & =  nit\_mob.n15 - N15RatioNit \cdot nitUptake \cdot frac \\
  		nit\_imob.n15 & = nit\_imob.n15 -N15RatioNit \cdot nitUptake \cdot (1-frac)
               \end{align*}     	
     \quad  Nbudget.AddOutput(NO3\_Uptake.n),
 		N15budget.AddOutput(NO3\_Uptake.n15)
 	
   return NO3\_Uptake


\minisec{nitrogen soilLayer.SubtractAmmUptake(ammUptake)}
\begin{align*}
	nitrogen amm & = amm\_mob+amm\_imob &
   N15RatioAmm & = amm.Get15NRatio()
\end{align*}  
 %	nitrogen NH4\_Uptake

   if $ammUptake>0$  and  $amm.n>0$ then
 	
  	\quad	$NH4\_Uptake.SetBoth(ammUptake,N15RatioAmm \cdot ammUptake)$
                \begin{align*}
  		frac & = \max(0,\min(1,\tfrac{amm\_mob.n}{amm.n}) \\
                %
 		amm\_mob.n   & = amm\_mob.n  - ammUptake \cdot frac \\
  		amm\_imob.n & = amm\_imob.n -  ammUptake \cdot (1-frac)
                \end{align*}
         %        15N
      frac = 1,
      %if (amm\_mob.n15<-1e-10  or  amm\_imob.n15<-1e-10)
%\subsection{     % 	theMessage.WarningWithDisplay("SoilLayer. SubtractAmm amm-N15 less than zero")}
	if $amm.n15>0$ then $frac = \max(0,\min(1,\tfrac{amm\_mob.n15}{amm.n15})$
 		\begin{align*}
                amm\_mob.n15 & = amm\_mob.n15 - N15RatioAmm \cdot ammUptake \cdot frac \\
  		amm\_imob.n15 & = amm\_imob.n15 - N15RatioAmm \cdot ammUptake \cdot (1-frac)
                \end{align*}      
      Nbudget.AddOutput(NH4\_Uptake.n),
		N15budget.AddOutput(NH4\_Uptake.n15)

 	
   return NH4\_Uptake










\subsection{soilLayer.FractionMobileWater()}
Estimate the frac of mobile water (SLIM concept \citep{addiscott1991simulation})

	return $\tfrac{\max(waterContent-fieldCapacity,0)}{waterContent}$


\subsection{Remove}


\minisec{soilLayer.RemoveWaterAndSolutes(waterOutflow,
  nitrogen nitRemove, nitrogen ammRemove, ChlorideRemove)}
Perform infiltration of water, nit and amm in layer.
Used with the convection dispersion equation.
The fluxes of nit amm and chloride can be negative.
waterOutFlow   - Water to be revoved from current layer [mm],
nitRemove/ammRemove  - Nit/Amm to be revomed from current layer [$g N/m^2$]

   waterContent = waterContent- waterOutflow,
   WaterBudget.AddOutput(waterOutflow)
\begin{align*}
   nit\_mob & = nit\_mob-nitRemove \\
   amm\_mob  & = amm\_mob-ammRemove \\
   Chloride\_mob &  =  Chloride\_mob  - ChlorideRemove
\end{align*}
   Nbudget.AddOutput(ammRemove.n+nitRemove.n)

   N15budget.AddOutput(ammRemove.n15+nitRemove.n15)
 












\subsection{Max}


\minisec{soilLayer.MaxTransp(rootStructure root)}
Calculates the maximum transpiration [mm] from the soil layer
   \begin{align*}
radius & = root.rootRadius \\
   pF     & = root.rootpF \\
   rootLength & = root.rootLengthList[Idx] 
 \end{align*}
   return MaxTransp(radius,pF,rootLength)

\minisec{soilLayer.MaxTransp(rootRadius,  rootpF,  rootLength)}
 \begin{align*}
rootDens & = \tfrac{rootLength}{thickness} [m m-1] \\
  relWater & = \tfrac{waterContent}{thickness} \\
UptakeRate & = -4 \cdot \pi \cdot 
   \tfrac{pF.GetFluxPotential(pF.GetpF(relWater))-pF.GetFluxPotential(rootpF)}
   {-log(rootRadius \cdot rootRadius \cdot \pi  \cdot rootDens)} \\
   availableWater & = waterContent-thickness \cdot pF.GetRelativeWater(rootpF)
  \end{align*}
 return $ \min(\max(0,availableWater),1000. \cdot 86400. \cdot UptakeRate \cdot rootLength
         \cdot \tfrac{relWater}{porosity}) $




\minisec{soilLayer.MaxNitrogenFlux(bool nit, rootStructure  root)}  
Calculates the maximum flux of nit or amm into roots in the layer.
returns maximum nitrogen uptake in layer [g N/m/d].
   StdDiff = 0

   if nit   then
      \begin{align*}
      StdDiff & = 3600 \cdot 2.0e-5 \cdot 24 \cdot 1e-4             
\text{  (convert to m2 d-1)} \\
      uptake  & = root.NitUptakeRate \\
   	N\_amount & = nit\_mob.n+nit\_imob.n \\
      conc & = 0
      \end{align*}
      if $waterContent>0$ then

    $     conc=\max(0,1000. \cdot \tfrac{N\_amount}{waterContent}-root.MinimumSoilNit)$
 
\quad		$bufferingCoef = 1   \tfrac{ waterContent}{thickness} $
  
 \text{NOT USED IN CURRENT DAISY CONCEPT?}
   else     amm
      \begin{align*}
      StdDiff & = 3600 \cdot 1.8e-5 \cdot 24 \cdot 1e-4        convert to m2 d-1 \\
   	uptake & = root.AmmUptakeRate \\
      N\_amount & = amm\_mob.n+amm\_imob.n \\
      conc & = 0
     \end{align*}
      if $waterContent>0$ then
      \begin{align*}
       conc & = \max(0,1000 \cdot \tfrac{N\_amount}{waterContent} \cdot FracAmmInWater()-root.MinimumSoilAmm) \\
  		bufferingCoef & = 1    AmmBuffering()   \\
                & \text{NOT USED IN CURRENT DAISY CONCEPT?}
        \end{align*}
 \begin{align*}
   diffusionCoef & = StdDiff \cdot 1.03^{temp} \cdot Tortuosity()  \\
                 & \cdot \tfrac{waterContent}{thickness}        m2 day-1 \\
   rootLength & = root.rootLengthList[Idx] \\
    maxFlux & = MaxNitrogenFlux(root.rootRadius,uptake,rootLength, \\
               &  conc,diffusionCoef,bufferingCoef) \\
   Nuptake & = rootLength \cdot  maxFlux
   \end{align*}
   $return \min(N\_amount,Nuptake)$

\minisec{soilLayer.Tortuosity()}
Calculate the tortuosity factor. Daisy Reference 347

   $relWat = \tfrac{waterContent}{thickness} $

   return $\tfrac{relWat^{\tfrac{7}{3}}}{pF.GetRelativeWater(0)^2} $
                (Millington-Quirk)


\minisec{soilLayer.MaxNitrogenFlux(rootRadius, uptakeRate, rootLength, conc, diffusionCoef, bufferingCoef)}
  \citep{olesen2002comparison,hansen1991simulation}
  adjusted for wheat?

   if $rootLength<1e-15$ return 0
  
   \quad $q  = \tfrac{waterUptake}{1000 \cdot rootLength}, 
   rootDens  =\tfrac{ 1000 \cdot  rootLength}{thickness}, 
   \alpha  = 0$
   
   if $q>0$  and $ diffusionCoef>0$ then
  	$\alpha = \tfrac{q}{2 \cdot \pi  \cdot diffusionCoef \cdot bufferingCoef}$
   \begin{align*}
   \beta & = \tfrac{1}{rootRadius \cdot sqrt(\pi  \cdot rootDens)} \\
   \beta_2 &= \beta \cdot \beta \\
   \beta_1 &= \beta_2-1 \\
   \beta_3 &= \beta_1 \cdot (1-0.5 \cdot \alpha) \\
   I &= 0
   \end{align*}
  $I = \begin{cases}
       q \cdot \tfrac{\beta_1 \cdot conc}{\beta_1-log(\beta_2)}        & \text{ if  $\alpha>1E-13$ and $\alpha=2$} \\
      q \cdot \beta_3 \cdot \tfrac{conc}{(\beta_3-\beta^{2-\alpha}+1)}   & \text{ if  $\alpha>1E-13$ and $\alpha \ne 2$} \\
     4. \cdot \pi  \cdot diffusionCoef \cdot \tfrac{conc}{(\beta_2 
\cdot \tfrac{log(\beta_2)}{\beta_1-1}} & \text{ otherwise}
      \end{cases}$


 $  return \min(I,uptakeRate)$

\subsection{soilLayer.FracAmmInWater()}
Calculate the frac of amm that is in the water/soluble phase.

Vp =5.96E-3, 
 Ve =0.308E-3, 
 Kp  =630, 
 Ke  =13.7 

return $\min\left(1,\tfrac{\tfrac{waterContent}{thickness}}{clayContent \cdot dryBulkDensity \cdot (Vp/Kp+Ve/Ke)+\tfrac{waterContent}{thickness}}\right)$



\subsection{soilLayer.AccesibleAmm()}
Calculates the amm (g/m2) which is accesible for nitrification and mineralisation
The rest is assumed to be physically-chemically protected from nitrifying
bacteria by clay minerals allthough it is still chemically exchangeable.
This protected part is only assumed to constitute part of the total
adsorbed but exchangeable amm.
	\begin{align*}
        AmmSum         & = amm\_imob.n+amm\_mob.n  \\
	adsorbedAmm   &  = AmmSum \cdot (1-FracAmmInWater()) \\
	protectionCoeff & = 1E-9  \\   
% optimised on organic crop rotation experiment by JB (24\_3\_3) \\
	maxProtectedAmm & = thickness \cdot clayContent \cdot dryBulkDensity \cdot protectionCoeff \\
	protectedAmm    & = \min(maxProtectedAmm,adsorbedAmm)	
  \end{align*}
return AmmSum-protectedAmm





  




\subsection{Budget}

\minisec{soilLayer.StartBudget()}
\begin{align*}
   nitrogen NO  & =OrganicMatter.GetNitrogen() \\
   nitrogen N & = nit\_mob+nit\_imob + amm\_mob+amm\_imob + NO
\end{align*}
   %Nbudget.SetInput(N.n)
   %N15budget.SetInput(N.n15)
   %WaterBudget.SetInput(waterContent)

\minisec{soilLayer.EndBudget()}

   $nitrogen NO=OrganicMatter.GetNitrogen()$,
   $nitrogen NRemain = nit\_mob+nit\_imob+amm\_mob+amm\_imob+NO$
  
 if $!Nbudget.Balance(NRemain.n)$ then

     cout << "soilLayer.EndBudget() - Name path " << GetLongName() << endl      !!!

   N15budget.Balance(NRemain.n15),
   WaterRemain = waterContent

   WaterBudget.Balance(WaterRemain)

   rrent.Next()




\subsection{Getter}

\minisec{soilLayer.GetCenterDepth(), GetPoolNitrogen(char Name,startDepth,thickness), GetWaterFilledPorosity()}

  return $startDepth+\tfrac{thickness}{2}$
   
  return $OrganicMatter.GetPoolNitrogen(Name) \cdot IntervalFraction(startDepth,thickness)$

  return $\min(1,\max(0,\tfrac{waterContent}{(porosity \cdot thickness)}))$


\minisec{nitrogen soilLayer.GetAmm(startDepth,thickness), GetNit(startDepth,thickness), 
  GetChloride(startDepth,thickness), GetOrganicNitrogen(startDepth, thickness)}

  return $(amm\_mob+amm\_imob) \cdot IntervalFraction(startDepth,thickness)$

  return $(nit\_mob+nit\_imob) \cdot IntervalFraction(startDepth,thickness)$

  return $(Chloride\_mob+Chloride\_imob) \cdot IntervalFraction(startDepth,thickness)$

  return $GetOrganicNitrogen() \cdot IntervalFraction(startDepth,thickness)$

\section{Long functions}

\subsection{soilProfile.Tridag(a[], b[], c[], r[], u[], unsigned long n)}
Solves for a vector u[0..n-1] the tridiagonal linear set. The input vectors
a,b,c and r are unmodified, and u contains the result
Slightly modified after Numerical Recipies in C (Press, 1992), main modification
is shift from [1..n] to [0..n-1], in concordance with mainstream C++ conventions.
Used to calculate the Finite difference solution for temp.
   
   gam=new[n],  
   $bet=b_0$,
   $u_0=\tfrac{r_0}{bet}$

   for all j in 1..n-1 do (Decomposition and forward substitution)

   \quad   $gam_{j}=\tfrac{c_{j-1}}{bet}  $        $ e_{j}$

   \quad   $bet=b_{j}-a_{j} \cdot gam_{j} $      $ g_{j}$

  \quad    if $0=bet$ then

   \quad\quad      Terminate("Error 2 in function Tridag")

   \quad   $u_{j}=\tfrac{r_{j}-a_{j} \cdot u_{j-1}}{bet}$   

   for (j=n-2;j+1>0;j--) do

  \quad    $u_{j}-=gam[j+1] \cdot u[j+1] $     Backsubstitution

   delete[] gam



\subsection{soilProfile.UpdateFinDiffTemp(AirTemp, snow)}
Alternative 3 (MEL 2007): UpdateFinDiffTemp calculates
the soil temp according the finite difference sceme
The upper boundary condition consist terms with
both airtemp, radiation and snow cover.
fracEvap is used to correct radiation for
reflection by leaves.


   if (theControlParameters.GetSimpleTempModel()  
     or  theControlParameters.GetMeasuredSoilTemp() 
     or theControlParameters.GetAirToSoilTemp()) then
  
   	if theControlParameters.GetSimpleTempModel() then
     

   	   for all i in 1..NumOfSoilLayers-1 do      
        	$soilTemp=soilLayerArray_{i}.GetTemp()$

	         %if ((soilTemp<-50)  or  (soilTemp>50))
%\subsection{	          %  Terminate("soilProfile.UpdateFinDiffTemp - soil temp not within realistic range")}
	        
         \quad        $depth = soilLayerArray_{i}.GetStartDepth()+ soilLayerArray_{i}.GetThickness() \cdot 0.5$

	 \quad        UpdateT(soilTemp,AirTemp,depth,8.0)
	     

	      $soilLayerArray_0.PutTemp(AirTemp)$
     
      else
	\quad      if  theControlParameters.GetAirToSoilTemp() then
	      

             set soil temp to the same as air temp - special for laboratory conditions
      		
        \quad\quad      for all i in 0..NumOfSoilLayers-1 do
                     $soilLayerArray_{i}.PutTemp(AirTemp)$

      	
      	else
    

      	\quad for all i in 0..NumOfSoilLayers-1 do
        $soilLayerArray_{i}.PutTemp(theClimate.GetNextSoilTemp())$
     
   
   else

  %\quad    i

  %\quad    soilLayer currentLayer

      deltaTime  =        currentTime  = 0,
%	moreFrozen_{MaxSoilLayers}
 %     waterFlux_{MaxSoilLayers}, waterFluxMean_{MaxSoilLayers}, waterFluxSlope_{MaxSoilLayers}
%
%    	heatConductivity_{MaxSoilLayers}, heatConductivityMean_{MaxSoilLayers}, heatConductivitySlope_{MaxSoilLayers},  SpecWaterCapacity_{MaxSoilLayers}
%    	UpperVector_{MaxSoilLayers},
%    	MidVector_{MaxSoilLayers},
% 	LowerVector_{MaxSoilLayers},
% 	TOldArray_{MaxSoilLayers},
% 	TNew_{MaxSoilLayers},
% 	TOld_{MaxSoilLayers},
%    	A_{MaxSoilLayers},
%    	B_{MaxSoilLayers},
%    	Ka_{MaxSoilLayers},
%    	Kb_{MaxSoilLayers},
%    	Kc_{MaxSoilLayers}
   %
      bool reduceTimestep=false     

      $WaterInFlux=WaterDensity \cdot \tfrac{infiltration}{86400}/1000 $  unit kg m-2 s-1

      numberOfTotalLayers = NumOfSoilLayers + ExtraLayers

   	for all i in 0..numberOfTotalLayers-1 do
         $moreFrozen_{i}=0$


     for all i in 0..numberOfTotalLayers-1 do 

\quad  
         $waterFlux_{i} = WaterDensity \cdot \tfrac{soilLayerArray_{i}.GetWaterFlux()}{86400}/1000   $              unit kg m-2 s-1
        	%if (fabs(waterFlux_{i})>0.01)
%\subsection{         	%theMessage.WarningWithDisplay("soilProfile.UpdateFinDiffTemp waterflux is larger than 0.01 kg m-2 s-1")}
        	%if (waterFlux_{i}<0)
%\subsection{         	%theMessage.WarningWithDisplay("soilProfile.UpdateFinDiffTemp waterflux is negative")}
    
      %for all i in 0..numberOfTotalLayers-1 do     
      	%if (waterFlux_{i}> 10)
%\subsection{         	%theMessage.WarningWithDisplay("soilProfile.UpdateFinDiffTemp - Waterflux unrealistic high")}
    
      $waterFluxMean_0= (WaterInFlux+waterFlux_0) \cdot 0.5$

      for all i in 1..numberOfTotalLayers-2 do

     \quad 	$waterFluxMean_{i}= (waterFlux_{i-1}+waterFlux_{i}) \cdot 0.5$

     $ waterFluxMean_{numberOfTotalLayers-1}=
 (waterFlux_{numberOfTotalLayers-1}$
  
\quad $+ waterFlux_{numberOfTotalLayers-2}) \cdot 0.5$

           (calculate the slope of the waterfluxes)

      $waterFluxSlope_0 =  (waterFlux_0-WaterInFlux) \cdot DistPreviousLayerInv_0$
    
  for all i in 1..numberOfTotalLayers-2 do

 \quad    	$waterFluxSlope_{i} = (waterFlux_{i}-waterFlux_{i-1}) \cdot DistPreviousLayerInv_{i}$
     
 $waterFluxSlope_{numberOfTotalLayers-1}
= (waterFlux_{numberOfTotalLayers-1}$

\quad$-waterFlux_{numberOfTotalLayers-2})
      \cdot DistPreviousLayerInv_{numberOfTotalLayers-1}
$          

get new LB and UB conditions,
          LB from analytical solution (as in DAISY)     

$ maxDepth = $

\quad $
\tfrac{GetMaximumDepth() 
          +soilLayerArray_{numberOfTotalLayers-2}.GetThickness() 
         +soilLayerArray_{numberOfTotalLayers-1}.GetThickness()}{1000}$   
   
(change to actual depth!)
      
if calculateAverage  then  
      
         averageThermalCond  =
         averageHeatCapacity =0
      
      	for all i in 0..numberOfTotalLayers-1 do  
           \begin{align*}     
      		averageThermalCond &= averageThermalCond \\
                    & + soilLayerArray_{i}.HeatConductivity() \cdot \tfrac{GetLayerThickness(i)}{1000}/maxDepth \\
      		averageHeatCapacity &= averageHeatCapacity  \\
                    & + soilLayerArray_{i}.HeatCapacity() \cdot \tfrac{GetLayerThickness(i)}{1000}/maxDepth
      	  \end{align*}
         calculateAverage =false

      $depthDampning = pow(2 \cdot averageThermalCond/(averageHeatCapacity \cdot \tfrac{0.0172142}{3600 \cdot 24},0.5)$

                     approximately 2.4      
      \begin{align*}
       averageAirTemp & = 7.7 \\
      amplitudeAirTemp & = 8 \\      
      dayInYear & =theTime.GetDayInYear() \\
      dayZero & =212       august 1th the hottest day january 30th the coldest day \\
      daysFromZero & =dayInYear-dayZero \\
      TLB & =averageAirTemp + amplitudeAirTemp 
             \cdot \exp(-\tfrac{maxDepth}{depthDampning}) \\
          &   \cdot cos(0.0172142 \cdot (daysFromZero)-\tfrac{maxDepth}{depthDampning})
\end{align*}
      if $snow>0$ then

         if $AirTemp>0$ then       (snow contains liquid water)
         
   	TUB = 0
				else                             no water in snowpack
           
 	$TUB = (AirTemp \cdot \tfrac{snowConductivity}{snow/1000}+
               		temp_0 \cdot soilLayerArray_0.HeatConductivity()
        $

\quad $ \cdot DistPreviousLayerInv_0)$
                       
\quad $   /\left(\tfrac{snowConductivity}{snow/1000}
  +soilLayerArray_0.HeatConductivity() \cdot DistPreviousLayerInv_0\right)
        $

       else
         TUB = AirTemp
   
        Initialise starting condition
   
	for all i in 0.. numberOfTotalLayers-1     

        $ TNew_{i}=temp_{i}$

         if $temp_{i}<0$  and  $reduceTimestep=false$ then
         	reduceTimestep=true

         if $soilLayerArray_{i}. GetIce()>0$  and  $reduceTimestep=false$
            reduceTimestep=true 
    
      if AirTemp < 0  and   reduceTimestep=false then
      	reduceTimestep=true

      deltaTime=3600 (timestep  1 hour)

      if reduceTimestep then
         deltaTime=100 (timestep 5 min)

      bool firstTime=true

           simulate 24 hours

      while $currentTime<86400$ do  

  			    transfer of new temp to old temp

for all i in 0..numberOfTotalLayers-1 do % (i=0i<numberOfTotalLayers++i)

        $TOld_{i}=TNew_{i}$,
            $frostChange_{i}=fabs(moreFrozen_{i})>1E-30$      
          True if change in icecontent

                calculate heatconductivity etc.

         for all i in 0..numberOfTotalLayers-1 do

            if $frostChange_{i}$  or $ firstTime$ then
               \begin{align*}
                heatConductivity_{i} & = soilLayerArray_{i}.HeatConductivity() \\
               SpecWaterCapacity_{i} & = soilLayerArray_{i}.SpecWaterCapacity()
             \end{align*}
        for all i in 0..numberOfTotalLayers-1 do          calculates the slope of the heatconductivity
            
         \quad if $frostChange_{i}$  or  $firstTime$ then
          
          \quad  	if i=0 then
                       (mean between soil and air)
               \begin{align*}
                  heatConductivitySlope_{i} & = (0.025+(heatConductivity_{i+1}-heatConductivity_{i}) \\
                             & \cdot DistNextLayerInv_{i}) \cdot 0.5  \\            
                  heatConductivityMean_{i} & =  (heatConductivity_{i+1}+heatConductivity_{i}) \cdot 0.5
              \end{align*}
               else             
                  if $i=numberOfTotalLayers-1$ then
                     \begin{align*}             
                     heatConductivitySlope_{i} & =  (heatConductivity_{i}-heatConductivity_{i-1}) \cdot DistPreviousLayerInv_{i} \\
                     heatConductivityMean_{i} & =  heatConductivity_{i}   
                     \end{align*}             
                  else  
                     \begin{align*}             
                     heatConductivitySlope_{i} & =  ((heatConductivity_{i+1}-heatConductivity_{i}) \cdot DistNextLayerInv_{i} \\
                                                & \quad +
                                                  (heatConductivity_{i}-heatConductivity_{i-1}) 
                        \\ & \quad \cdot DistPreviousLayerInv_{i}) \cdot 0.5 \\
                     heatConductivityMean_{i} & =  (heatConductivity_{i+1}+heatConductivity_{i}) \cdot 0.5
                    \end{align*}   
         for all i in 0..numberOfTotalLayers-1 do

            if $frostChange_{i}$  or  $firstTime$ then
            
             \quad $currentLayer = soilLayerArray_{i}$
         
             if ($TNew_{i} < 0$  and  
currentLayer.GetIce()<currentLayer.GetTotalWater()) freezing
 or  ($TNew_{i} > 0$)  and  (currentLayer.GetIce()>0)  thawing then  
                  \begin{align*}
                  A_{i} & =specCapacity \cdot \tfrac{specCapacity}{273} \cdot SpecWaterCapacity_{i}+currentLayer.HeatCapacity() \\
                  B_{i} & =specCapacity \cdot waterFluxSlope_{i}
                  \end{align*}
              else
                  \begin{align*}
                  A_{i} & =currentLayer.HeatCapacity() &
                  B_{i} & =0
                  \end{align*}              
             \begin{align*}
               Ka_{i} & = \tfrac{A_{i}}{deltaTime }\\
               Kb_{i} & = \tfrac{heatConductivityMean_{i}}
                     {distanceToPreviousLayer_{i}+distanceToNextLayer_{i}} \\
               Kc_{i} & =\tfrac{heatConductivitySlope_{i}-HeatCapacityWater \cdot waterFluxMean_{i}}
                  {2 \cdot (distanceToPreviousLayer_{i}+distanceToNextLayer_{i})}
             \end{align*}
             Initialization of diagonal vectors

         for(i=0 i<numberOfTotalLayers ++i)

            if $frostChange_{i} $ or  firstTime then       
               \begin{align*}
               UpperVector_{i} &=-Kb_{i} \cdot DistNextLayerInv_{i}-Kc_{i}  \\   
               MidVector_{i} &=Ka_{i}+Kb_{i} \cdot (DistNextLayerInv_{i}+DistPreviousLayerInv_{i})     \\
               LowerVector_{i} &= -Kb_{i} \cdot DistPreviousLayerInv_{i}+Kc_{i}                       \\
               LowerVector_0 &= 0 \\
               UpperVector_{numberOfTotalLayers-1}& =0
               \end{align*}
               %if (MidVector_0=0)
%\subsection{               %	theMessage.WarningWithDisplay("soilProfile.UpdateFinDiffTemp - Error 1 in tridag vectors")}
         
   		    Initialization of vector with old values

			i=0
        \begin{align*}
         TOldArray_{i}&=Ka_{i} \cdot TOld_{i} 
            +Kb_{i} \cdot ( \\
            &  (TOld_{i+1}-TOld_{i}) \cdot DistNextLayerInv_{i} \\
            &   -(TOld_{i}-TUB) \cdot DistPreviousLayerInv_{i} 
             ) \\
            & +Kc_{i} \cdot (TOld_{i+1}-TUB)-B_{i} \\
            & -(-Kb_{i} \cdot DistPreviousLayerInv_{i}+Kc_{i}) \cdot TUB
        \end{align*}
      	for all i in 1..numberOfTotalLayers-2 do
        \begin{align*}
         	TOldArray_{i}&=Ka_{i} \cdot TOld_{i} 
            	                +Kb_{i} \cdot ( \\
                               & (TOld_{i+1}-TOld_{i}) \cdot DistNextLayerInv_{i} \\
                               & -(TOld_{i}-TOld_{i-1}) \cdot DistPreviousLayerInv_{i} \\
                               & )  +Kc_{i} \cdot (TOld_{i+1}-TOld_{i-1})-B_{i}
        \end{align*}
$TOldArray_{numberOfTotalLayers-1}$
         \begin{align*}
         & = Ka_{numberOfTotalLayers-1}\cdot TOld_{numberOfTotalLayers-1}
                +Kb_{numberOfTotalLayers-1}
               & \cdot ( 
                (TLB-TOld_{numberOfTotalLayers-1}) \cdot DistNextLayerInv_{numberOfTotalLayers-1}\\
               & -(TOld_{numberOfTotalLayers-1}-TOld_{numberOfTotalLayers-2}) \\
              &   \cdot DistPreviousLayerInv_{numberOfTotalLayers-1}\\
              & ) 
               %
              +Kc_{numberOfTotalLayers-1}\cdot (TLB-TOld_{numberOfTotalLayers-2}) \\
              & -B_{numberOfTotalLayers-1}
               +(Kb_{numberOfTotalLayers-1}\cdot DistNextLayerInv_{numberOfTotalLayers-1}\\
              & +Kc_{numberOfTotalLayers-1}) \cdot TLB
        \end{align*}
             Solve with the sweep method (function tridag)
     
          Tridag(LowerVector, MidVector, UpperVector, TOldArray, TNew, numberOfTotalLayers)

             update IceContent
			for all i in 0..numberOfTotalLayers-1 do
   		

           $ currentLayer = soilLayerArray_{i}$

                    is there freezing or thawing

            if ($TNew_{i} $< 0
      and  $currentLayer.GetIce()<currentLayer.GetTotalWater()$)  
freezin

\quad or    
                  ($TNew_{i}$ > 0  and  $currentLayer.GetIce()>0$)   
    thawing then
                     	\begin{align*}
               TimeDerivTemp&=\tfrac{(TNew_{i}-TOld_{i})}{deltaTime} \\
               TimeDerivIce &= \tfrac{WaterDensity}{IceDensity} \cdot (-\tfrac{1}{WaterDensity} \cdot waterFluxSlope_{i} \\
                     & - (specCapacity \cdot \tfrac{SpecWaterCapacity_{i}}{273} \cdot TimeDerivTemp)) \\
               moreFrozen_{i} &=  deltaTime \cdot TimeDerivIce currentLayer.GetThickness()         in mm
               \end{align*}
                if ($moreFrozen_{i}$+ currentLayer.GetIce() > currentLayer.GetTotalWater())          frozen water exceeds water
               \begin{align*}
               	TNew_{i} & =-0.001 \\
                moreFrozen_{i} & =currentLayer.GetTotalWater()-currentLayer.GetIce()
                \end{align*}            
               else if ($moreFrozen_{i}+ currentLayer.GetIce()<0$)                                 frozen water negative
               \begin{align*}
               	TNew_{i} & =0.001 \\
                  moreFrozen_{i} & = -currentLayer.GetIce()
               \end{align*}
               $ice=currentLayer.GetIce()$

               $currentLayer.SetIceContent(\max(0,\min(currentLayer.GetTotalWater(),ice+ moreFrozen_{i})))$

            
            else  

$moreFrozen_{i}=0$,           
         $currentTime  = currentTime + deltaTime $,
         $firstTime =false$
        	
      for all i in 0..numberOfTotalLayers-1 do
         $temp_{i}=TNew_{i}$

          check if the temp is witin realistic boundaries

      $month=theTime.GetMonth(),
      year=theTime.GetYear()$

     for all i in 0..numberOfTotalLayers-1 do    
         %if (i>9  and  i<19  and  temp_{i}<minTemp[month-1]  and  year \ne lastErrorYear)
         %   lastErrorYear=year
         %   cout << "Layer no " << i << " temp is " << temp_{i}
%\subsection{         %   theMessage.WarningWithDisplay("soilProfile.UpdateFinDiffTemp - soil layer temp below \expert opinion monthly threshold")}
      $   soilLayerArray_{i}.PutTemp(temp_{i})$
    
      if theControlParameters.GetWriteSoilTemp() then

     \quad 	SoilTempFile << theTime

      \quad   for all i in 0..numberOfTotalLayers-1 do
      	  $ SoilTempFile < <  temp_{i}$

         SoilTempFile << endl


\subsection{SWAT}



\minisec{soilLayer.RemoveWaterHourly(moveWater,
      nitrogen nitLeached, nitrogen ammLeached,
   ChlorideLeached, bool downwardMovement)}
Perform infiltration of water, nit and amm in layer. Used for SWAT
water model with hourly timestep.
surplus         - Water to be added to current (next) layer [mm],
nitLeached/ammLeached  - Nit/Amm to the added to current (next) layer [$g N/m^2$]


   nitLeached.Clear(),
   ammLeached.Clear()

   if $moveWater>0$ then

   \quad   Default value 0.5. BMP and JBE changed this - very important difference!!!
      \begin{align*}
      FirstMoveFraction & = 0  \\
        mobileWater & = \max(waterContent-fieldCapacity,0) \\
      \alpha &  = 0
      \end{align*}
      if $mobileWater>0$ then

   \quad      $\alpha = \min(1,\tfrac{moveWater}{mobileWater}) $      As defined in original SLIM

           Leach first proportion of the water using current mobile solute concs
      \begin{align*}
       \cdot nitLeached  & = nit\_mob \cdot \alpha \cdot FirstMoveFraction \\
      nit\_mob      & = nit\_mob -  \cdot nitLeached \\
      %
       \cdot ammLeached & = amm\_mob \cdot \alpha \cdot FirstMoveFraction \\
      amm\_mob     & = amm\_mob -  \cdot ammLeached \\
      %
       \cdot ChlorideLeached & = Chloride\_mob \cdot \alpha \cdot FirstMoveFraction \\
      Chloride\_mob  &= Chloride\_mob -  \cdot ChlorideLeached
      \end{align*}
      EquilibrateNitrogen(mobileWater)
\begin{align*}
      waterContent & = waterContent - FirstMoveFraction \cdot moveWater \\
      mobileWater  & = mobileWater  - FirstMoveFraction \cdot moveWater
\end{align*}
           Leach second proportion of the water using new mobile solute concs
      \begin{align*}
      nitrogen N       & = nit\_mob \cdot \alpha \cdot (1-FirstMoveFraction) \\
       \cdot nitLeached  & =  \cdot nitLeached + N \\
      nit\_mob      & = nit\_mob - N \\
%
      N                & = amm\_mob \cdot \alpha \cdot (1-FirstMoveFraction) \\
       \cdot ammLeached & =  \cdot ammLeached + N \\
      amm\_mob     & = amm\_mob - N \\
%
       \cdot ChlorideLeached  &=   \cdot ChlorideLeached +  Chloride\_mob \cdot \alpha \cdot (1-FirstMoveFraction) \\
      Chloride\_mob      &= Chloride\_mob     - Chloride\_mob \cdot \alpha \cdot (1-FirstMoveFraction) \\
%
      waterContent & = waterContent - (1-FirstMoveFraction) \cdot moveWater
      \end{align*}
      WaterBudget.AddOutput(moveWater)
   
   else

    if downwardMovement then
     \begin{align*}
       ammLeaching & = ammLeaching +  \cdot ammLeached &
      nitLeaching & = nitLeaching +  \cdot nitLeached   
    \end{align*}
   Nbudget.AddOutput(ammLeached.n+nitLeached.n)

   N15budget.AddOutput(ammLeached.n15+nitLeached.n15)


   


\subsection{soilLayer.EquilibrateNitrogen(mobileWater)}
Intra-layer movement of water and solute
   %if (waterContent<0)
%\subsection{   %	theMessage.FatalError("soilLayer.EquilibrateNitrogen - water content below zero")}

   rateCoefficient=0.9,
   if (startDepth>199)
   	rateCoefficient=0.7,        holdbackconstant =0.3

   if $mobileWater > 0$ then
            
      \begin{align*}
      concDiffChloride & = \tfrac{Chloride\_mob}{mobileWater} 
         -\tfrac{Chloride\_imob+Chloride\_mob}{waterContent} \\
      moveChlorideFromMobile & = concDiffChloride \cdot mobileWater \cdot (rateCoefficient) \\
      Chloride\_mob   &= Chloride\_mob - moveChlorideFromMobile \\
      Chloride\_imob  &= Chloride\_imob + moveChlorideFromMobile \\      
      nitrogen concDiffNit & = \tfrac{nit\_mob}{mobileWater}
             -\tfrac{nit\_imob+nit\_mob}{waterContent} \\
      nitrogen moveNitFromMobile & = concDiffNit \cdot mobileWater
            \cdot (rateCoefficient) \\
      nit\_mob  & = nit\_mob - moveNitFromMobile \\
      nit\_imob  & = nit\_imob + moveNitFromMobile \\    
   	fAm   & = FracAmmInWater() \\
      nitrogen concDiffAmm  & = \tfrac{amm\_mob}
         {mobileWater}-\tfrac{amm\_imob+amm\_mob}{waterContent} \\
      nitrogen moveAmmFromMobile & = concDiffAmm \cdot mobileWater
              \cdot (rateCoefficient) \\
      amm\_mob  & = amm\_mob - moveAmmFromMobile \cdot fAm \\
      amm\_imob & = amm\_imob + moveAmmFromMobile \cdot fAm
      \end{align*}
   else  
 (if no mobile phase is present the mobile solutes are moved to 
the immobile phase)
\begin{align*}
   Chloride\_imob & =Chloride\_mob+Chloride\_imob &
   Chloride\_mob & =0 \\
   nit\_imob &=nit\_mob+nit\_imob &
   nit\_mob & =0 \\
   amm\_imob & =amm\_mob+amm\_imob &
   amm\_mob & =0
\end{align*}
%/*         Chloride
%   \begin{align*}
%AddChloride  & = (Chloride\_mob+Chloride\_imob) \cdot (1.-holdbackConstant)   \\    
%      Holdback constant burde afh�nge af antallet af lag!
%   Chloride\_mob  & = Chloride\_mob \cdot holdbackConstant + AddChloride \cdot f \\
%   Chloride\_imob  & = Chloride\_imob \cdot holdbackConstant + AddChloride \cdot (1.-f) */
% \end{align*}
   equilibrated  = true


      

\subsection{soilProfile.UpdateInfiltrationSWAT( surplus, nitrogen surfaceNit,
                                     nitrogen surfaceAmm, nitrogen NitLeached,
                                     nitrogen  AmmLeached, Chloride)}
Performs infiltration of water, nit and amm in profile using the SWAT
water model. Simulates the transport of water and solutes with timestep one hour.
Solutes are equilibrated between mobile and immobile domain only once a day.
Optional use of convection-dispersion equation for transport in mobile domain.
(MEL 2007)

\begin{tabular}{ll}
   surplus         &  Water to be added to/returned from profile [mm] \\
   nitLeached/ammLeached  & Nit/Amm  to the added to/returned from profile [$g N/m^2$] \\
 \end{tabular}
%    double
%       totalFlux_{MaxSoilLayers},
%       maxContent_{MaxSoilLayers},
%       waterContent_{MaxSoilLayers},
%       FC_{MaxSoilLayers},
%       satHydraulicConductivity_{MaxSoilLayers},
%       layerThickness_{MaxSoilLayers},
%       \theta_{MaxSoilLayers},
%       \theta Old_{MaxSoilLayers},
%       \thetaBack_{MaxSoilLayers},
%       \thetaForward_{MaxSoilLayers},
%       dispersion_{MaxSoilLayers},
%       dispersionBack_{MaxSoilLayers},
%       dispersionForward_{MaxSoilLayers},
%       flowBack_{MaxSoilLayers},
%       flowForward_{MaxSoilLayers},
%       upperVector_{MaxSoilLayers},
%       midVector_{MaxSoilLayers},
%       lowerVector_{MaxSoilLayers},
%       soluteOldVector_{MaxSoilLayers},
%       soluteOld_{MaxSoilLayers},
%       soluteNew_{MaxSoilLayers},
%       mobileNitFlux_{MaxSoilLayers},
%       mobileAmmFlux_{MaxSoilLayers},
%       mobileNitN15Flux_{MaxSoilLayers},
%       mobileAmmN15Flux_{MaxSoilLayers},
%       mobileChlorideFlux_{MaxSoilLayers},
%       mobileWaterNew_{MaxSoilLayers},
%       mobileWaterOld_{MaxSoilLayers},
%    	immobileWater_{MaxSoilLayers}


   molecularDiff =
   soluteInFlux =
   soluteIn =
   nitIn =
   nitN15In =
   ammIn =
   ammN15In =
   chlorideIn =
   standingWater = 0


%    bool IsThereSoluteFlux_{MaxSoilLayers}
% 
%    nitrogen surplusNH4_{MaxSoilLayers}
% 
%    nitrogen surplusNO3_{MaxSoilLayers}


   simulationTime  = 0,
   deltaT  =3600


   TotalNumberOfLayers = NumOfSoilLayers + ExtraLayers

   %$pF\_Curve pF_{MaxSoilLayers}$

 
	for all i in 0..numberOfTotalLayers-1 do
		$soilLayerArray_{i}.ClearLeaching() $                 

set leaching to zero
  \begin{align*}  
  waterContent_{i} & = soilLayerArray_{i}.GetTotalWater() \\
   %
   pF_{i} & = soilLayerArray_{i}.GetpF\_Curve() \\
%
%
      maxContent_{i} & =(soilLayerArray_{i}.GetPorosity() \cdot 
    soilLayerArray_{i}.GetThickness()) \\
      totalFlux_{i} & =0 \\
      FC_{i} & = soilLayerArray_{i}.GetFieldCapacity()
 \end{align*}    
 $surplusNH4_{i}.Clear()$,
      $surplusNO3_{i}.Clear() $  

   standingWater= surplus,
   soluteTransportMethod=1         

if 1 use SLIM transport option

   for all i in 0..TotalNumberOfLayers-1 do 
     \begin{align*}
      satHydraulicConductivity_{i} & = pF_{i}.GetConductivity(pF_{i}.GetpF(
          \tfrac{maxContent_{i}}{soilLayerArray_{i}.GetThickness()}))  \\
      layerThickness_{i} & =soilLayerArray_{i}.GetThickness()
    \end{align*}
 satHydraulicConductivity in m/s


   while $simulationTime<86400$ do (simulates 24 hours)
  
 % \quad    $fluxInTimeStep_{MaxSoilLayers}$

  \quad         Perform infiltration of top layer.

  \quad    $infiltration=\max(0,\min(standingWater,maxContent_0-waterContent_0))$ in mm

  \quad    if $infiltration>0$ then
         \begin{align*}
         standingWater &= standingWater- infiltration \\
         waterContent_0 &= waterContent_0+ infiltration
         \end{align*}
         %if (waterContent_0>maxContent_0+0.0001)
%\subsection{         %   theMessage.WarningWithDisplay("soilProfile.UpdateInfiltration - top layer contains more water than physically possible")}
\quad     else
         %if (infiltration<(-1E-10))
%\subsection{         %   theMessage.WarningWithDisplay("soilProfile.UpdateInfiltrationSWAT - top layer contains more water than physically possible")}

 \quad    for all i in 0..TotalNumberOfLayers-1 do
         $fluxInTimeStep_{i}=0$

  \quad    for all i in 0..numberOfTotalLayers-1 do  
                
$NotFrozenWater = waterContent_{i}-soilLayerArray_{i}.GetIce() $

    in mm
         \begin{align*}  						
         travelTime & = \tfrac{maxContent_{i}-FC_{i}}{satHydraulicConductivity_{i}/1000} \\			    in s
         soilWaterExcess & =\max(0,\min(NotFrozenWater,waterContent_{i}- FC_{i})) 	
         \end{align*}				    in mm
         if $soilWaterExcess>0$ then        drainage
            
      \quad      $fluxInTimeStep_{i} = soilWaterExcess \cdot (1-\exp(-\tfrac{deltaT}{travelTime})) $      in mm per timestep
            
            if $i<(TotalNumberOfLayers-1)$ then 
            \begin{align*}        
               fluxInTimeStep_{i} & = \max(0,\min(fluxInTimeStep_{i},maxContent_{i+1}-waterContent_{i+1})) \\       
                       &   reduce flux according to capacity in next layer \\
               waterContent_{i+1} & =  waterContent_{i+1} +fluxInTimeStep_{i}  \\ 		
                     & \quad add flux to next layer
            \end{align*}
            \begin{align*}
            waterContent_{i}  & = waterContent_{i} -  fluxInTimeStep_{i}  \\
                              & \text{  remove flux from present layer} \\
            totalFlux_{i} & =  totalFlux_{i}+  fluxInTimeStep_{i}  
            \end{align*}      
         else

        $ fluxInTimeStep_{i}=0$

         %if (waterContent_{i}>maxContent_{i}+0.0001)
%\subsection{         %   theMessage.WarningWithDisplay("soilProfile.UpdateInfiltrationSWAT - soil layer contains more than physically possible")}
    

           Transport of water and solutes

      $f=1$      f is frac of surface water that enter the soil

 \quad    if $surplus>0$ then

         f= $\tfrac{surplus-standingWater}{surplus}$

  \quad    if $soluteTransportMethod=1$ then
       
\quad    \quad    $ nitrogen Nadd =  surfaceNit \cdot f + (  surfaceAmm \cdot f)$

       $ soilLayerArray_0.AddWater( surplus \cdot f,  surfaceNit \cdot f,  surfaceAmm \cdot f,Chloride)$
        
        for all i in 0..numberOfTotalLayers-1 do      
            %if (waterContent_{i}>(maxContent_{i}+0.1))
%\subsection{            %   theMessage.WarningWithDisplay("soilProfile.UpdateInfiltrationSWAT - water content in soil layer exceeds the possible")}
          
 $ soilLayerArray_{i}.SetWaterFlux(fluxInTimeStep_{i})$,          
  $soilLayerArray_{i}.SetEquilibrated(false)$
         
   nitrogen NO3, nitrogen NH4
         
   if $fluxInTimeStep_{i}>0$ then (Downward movement)

  \quad            $ soilLayerArray_{i}.RemoveWaterHourly(fluxInTimeStep_{i},NO3,NH4,Chloride)$
             
  if $i=TotalNumberOfLayers-1$ then 
                 \begin{align*}           
                    NitLeached & = NO3 &
                   AmmLeached  & = NH4 
                \end{align*}              
              if $i<TotalNumberOfLayers-1$ then

                 \quad  $soilLayerArray_{i+1}.AddWater(fluxInTimeStep_{i},NO3,NH4,Chloride)$           
        \begin{align*}
          surplus & = surplus \cdot (1-f) \\
           surfaceNit & =  surfaceNit \cdot (1-f) \\
           surfaceAmm & =  surfaceAmm \cdot (1-f)
        \end{align*}        
        what about chloride?
        
      else
 
          Use convection-dispersion equation on mobile domain
     
%\subsection{     % theMessage.FatalError("soilProfile.UpdateInfiltrationSWAT - Convection-dispersion equation for solute transport not properly implemented")}

         for all i in 0..numberOfTotalLayers-1 do
\begin{align*}     
            mobileWaterOld_{i}&=\max(0,soilLayerArray_{i}.GetTotalWater()-FC_{i}) \\
            immobileWater_{i}&= soilLayerArray_{i}.GetTotalWater()-mobileWaterOld_{i} \\
            mobileWaterNew_{i}&=\max(0,waterContent_{i}-immobileWater_{i}) 
\end{align*}        
             solve individually for nit, amm and chloride
    
     for ($soluteType=0soluteType<5soluteType++$)
       
  \quad          if $soluteType=0$  then        (nit N)
           
 \quad              molecularDiff=1.902e-9            (m2 s-1)                   Change to diffusion coefficient in dilute solutions  !!!!!!!!!!!!
               
              for all i in 0..TotalNumberOfLayers-1 do

   \quad \quad            $ soluteOld_{i}= \begin{cases}
                \tfrac{ soilLayerArray_{i}.GetMobileNit().n}{\tfrac{mobileWaterOld_{i}}{1000}} & \text{ if } (mobileWaterOld_{i}>0) \\
                 0 & \text{ otherwise} 
                \end{cases}$

    \quad        $   soluteIn= surfaceNit.n \cdot f$

     \quad          nitIn=soluteIn  
          
            else if $soluteType=1$ then       (amm N)            
          
   \quad  molecularDiff=1.957e-9            (m2 s-1)
               
for all i in 0..TotalNumberOfLayers-1 do
            
\quad $soluteOld_{i}= \begin{cases}     
                    \tfrac{ soilLayerArray_{i}.GetMobileAmm().n}{\tfrac{mobileWaterOld_{i}}{1000}} & \text{ if } (mobileWaterOld_{i}>0) \\
                  0 & \text{ otherwise}
\end{cases}$
\begin{align*}
               soluteIn&= surfaceAmm.n \cdot f \\
                ammIn&=soluteIn
\end{align*}            
            else if  $soluteType=2$ then   (chloride)    
     
            \quad   molecularDiff=2.032e-9            m2 s-1

             \quad  for all i in 0..TotalNumberOfLayers-1 do
                        
      \quad \quad       $ soluteOld_{i}=\begin{cases}
   \tfrac{soilLayerArray_{i}.GetMobileChloride()}
{\tfrac{mobileWaterOld_{i}}{1000}} & \text{  if $mobileWaterOld_{i}>0$} \\
 0 & \text{ otherwise}                           
                            \end{cases}     $      
              \begin{align*}
               soluteIn&=Chloride \cdot f \\
               chlorideIn&=soluteIn
              \end{align*}           
            else if  $soluteType=3$ then       (nit 15N) 
        
            \quad   molecularDiff=1.902e-9            (m2 s-1)

           \quad    for all i in 0..TotalNumberOfLayers-1 do
                 
                      $ soluteOld_{i}= \begin{cases}
\tfrac{soilLayerArray_{i}.GetMobileNit().n15}{\tfrac{mobileWaterOld_{i}}{1000}} & \text{ if } (mobileWaterOld_{i}>0) \\
               0
\end{cases} $
             \begin{align*}
  soluteIn& =surfaceNit.n15 \cdot f \\
               nitN15In& =soluteIn
\end{align*}           
            else if  $soluteType=4$ then       (amm 15N)   
         
           \quad    molecularDiff=1.957e-9      
  										      m2 s-1
          \quad     for all i in 0..TotalNumberOfLayers-1 do               

                    $ soluteOld_{i}= \begin{cases}
\tfrac{soilLayerArray_{i}.GetMobileAmm().n15}{\tfrac{mobileWaterOld_{i}}{1000}} &      g/m2  if (mobileWaterOld_{i}>0) \\
                     0 &
                  \end{cases} $
              \begin{align*}
               soluteIn & =surfaceAmm.n15 \cdot f \\
               ammN15In & =soluteIn
             \end{align*}    
            bool nonZeroConcentration=false

           for all i in 0..TotalNumberOfLayers-1 do
               if ($soluteOld_{i}>0$)  nonZeroConcentration=true

            if $soluteIn>0$ then

       \quad        nonZeroConcentration=true, tortuosity=0.6, dispersionLength=0.02       									      m

         \quad      for all i in 0..TotalNumberOfLayers-1 do   
                  \begin{align*}           
                  \theta_{i}&=\min(1, \tfrac{mobileWaterNew_{i}}{soilLayerArray_{i}.GetThickness()}) \\
                  \theta Old_{i}&=\min(1,\tfrac{mobileWaterOld_{i}}{soilLayerArray_{i}.GetThickness()}) \\
                  soluteNew_{i}&= soluteOld_{i}
                  \end{align*}              
      \quad      if(nonZeroConcentration)      bypass if no conc is above zero
            

         \quad       determine where there is a solute flux (calculated out of the layer)

        \quad      for all i in 0..TotalNumberOfLayers-1 do
                 
                 
            $   IsThereSoluteFlux_{i}= $

$\begin{cases}
                       true & \text{ if } fluxInTimeStep_{i}>0 or
                  (mobileWaterOld_{i+1}>0  and 
                  mobileWaterOld_{i}>0) \\
                         false & \text{ otherwise} 
                                     \end{cases} $

               

        \quad           calculate centered values

        \quad       for all i in 0..TotalNumberOfLayers-1 do

              
        \quad          if $i=0$   then             
                    \begin{align*}
                     \theta Back_{i}&=0.5 \cdot (\theta_{i}+\theta Old_{i}) \\ 
                     \theta Forward_{i}&=0.25 \cdot (\theta_{i}+\theta_{i+1}+\theta Old_{i}+\theta Old_{i+1}) \\
                     flowBack_{i}&=infiltration/deltaT/1000 \\
                     flowForward_{i}&=fluxInTimeStep_{i}/deltaT/1000
                     \end{align*}
     \quad                if $\theta_{i}=0$ or $\theta Old_{i}=0$ then
                        $dispersion_{i}=0$
                    
\quad  else
    \quad                   $ dispersion_{i}=dispersionLength \cdot 0.5 \cdot \tfrac{flowForward_{i}+flowBack_{i}}{0.5 \cdot (\theta_{i}+\theta Old_{i})}+molecularDiff \cdot tortuosity
$
                         if $\theta_{i}=0$ or $\theta_{i-1}=0$ then
 \quad                     $ dispersionBack_{i}=0$
                        
\quad else
                           $ dispersionBack_{i}=dispersionLength \cdot \tfrac{flowBack_{i}}{\theta Back_{i}}+molecularDiff \cdot tortuosity
  $

 \quad                  if $\theta_{i}=0$  or $\theta Old_{i}=0$
            or $\theta_{i+1}=0$ or $\theta Old_{i+1}=0$ then
                       $ dispersionForward_{i}=0$
                     
\quad else
     \quad                   $ dispersionForward_{i}=dispersionLength \cdot \tfrac{flowForward_{i}}{\theta Forward_{i}}
+molecularDiff \cdot tortuosity $
                 
                
             else if $i=TotalNumberOfLayers-1$ then                
                     \begin{align*}
                     \theta Back_{i}&=0.25 \cdot (\theta_{i-1}+\theta_{i}+\theta Old_{i-1}+\theta Old_{i}) \\
                     \theta Forward_{i}&=0.5 \cdot (\theta_{i}+\theta Old_{i}) \\
                     flowBack_{i}&=fluxInTimeStep_{i-1}/deltaT/1000 \\
                     flowForward_{i}&=fluxInTimeStep_{i}/deltaT/1000
                     \end{align*}
      \quad               if $\theta_{i}=0$ or $ \theta Old_{i}=0$ then
                       $ dispersion_{i}=0$

    \quad                 else
                        $dispersion_{i}=dispersionLength \cdot 0.5 \cdot (flowForward_{i}+
   \tfrac{flowBack_{i})}{0.5 \cdot (\theta_{i}+\theta Old_{i})}+molecularDiff \cdot tortuosity
                 $

    \quad if $\theta_{i-1}=0$  or $\theta Old_{i-1}=0$ or 
    $\theta_{i}=0$ or $\theta Old_{i}=0$ then
                        $dispersionBack_{i}=0$
                   
  \quad else
                        $dispersionBack_{i}=dispersionLength \cdot 
 \tfrac{flowBack_{i}}{\theta Back_{i}}+molecularDiff \cdot tortuosity$
                    

 \quad if $\theta_{i}=0$  or  $\theta Old_{i}=0)$ then

    \quad  \quad       then              $ dispersionForward_{i}=0$

   \quad     \quad               else
                      $ dispersionForward_{i}=dispersionLength \cdot 
\tfrac{flowForward_{i}}{\theta Forward_{i}}+molecularDiff \cdot tortuosity$
                  
\quad else  
                      \begin{align*}
                     \theta Back_{i}&=0.25 \cdot (\theta_{i-1}+\theta_{i}+\theta Old_{i-1}+\theta Old_{i}) \\
                     \theta Forward_{i}&=0.25 \cdot (\theta_{i}+\theta_{i+1}+\theta Old_{i}+\theta Old_{i+1}) \\
                     flowBack_{i}&=fluxInTimeStep_{i-1}/deltaT/1000 \\
                     flowForward_{i}&=fluxInTimeStep_{i}/deltaT/1000
                     \end{align*}
                \quad     if $\theta_{i}=0$ or $\theta Old_{i}=0$ then
                        $dispersion_{i}=0$
                     
            \quad          else
                        $dispersion_{i}=dispersionLength \cdot 0.5 \cdot 
\tfrac{(flowForward_{i}+flowBack_{i})}{(0.5 \cdot (\theta_{i}+\theta Old_{i}))}+molecularDiff \cdot tortuosity $
                    
        \quad             if $\theta_{i-1}=0$ 
 or  $\theta Old_{i-1}=0$ or $\theta_{i}=0$ or $\theta Old_{i}=0)$
then
                        $dispersionBack_{i}=0$
                     
       \quad              else
                        $dispersionBack_{i}=dispersionLength \cdot 
\tfrac{flowBack_{i}}{\theta Back_{i}}+molecularDiff \cdot tortuosity$
                   
        \quad             if $\theta_{i}=0$  or $\theta Old_{i}=0$ or 
$\theta_{i+1}=0$ or $\theta Old_{i+1}=0$ then
                        $dispersionForward_{i}=0$
                     
          \quad           else
                       $ dispersionForward_{i}=dispersionLength \cdot 
\tfrac{flowForward_{i}}{\theta Forward_{i}}+molecularDiff \cdot tortuosity $
                
               
             
      \quad         $a_{MaxSoilLayers},b_{MaxSoilLayers},c_{MaxSoilLayers},
                        da_{MaxSoilLayers},db_{MaxSoilLayers},dc_{MaxSoilLayers}$

        \quad       for all i in 0..TotalNumberOfLayers-1 do
               \begin{align*}
                  a_{i} & = dispersionBack_{i} \cdot \tfrac{DistPreviousLayerInv_{i}}{2 \cdot layerThickness_{i}} 
                         +\tfrac{flowBack_{i}}{(4 \cdot layerThickness_{i})} \\
                  b_{i} & = -\tfrac{\theta_{i}}{deltaT} 
                        -dispersionForward_{i} \cdot \tfrac{DistNextLayerInv_{i}}{2 \cdot layerThickness_{i}} 
                         -dispersionBack_{i} \cdot \tfrac{DistPreviousLayerInv_{i}}{2 \cdot layerThickness_{i}} \\
                        & -\tfrac{flowForward_{i}}{4 \cdot layerThickness_{i}}
                        +\tfrac{flowBack_{i}}{4 \cdot layerThickness_{i}} \\
                  c_{i} & = dispersionForward_{i} \cdot \tfrac{DistNextLayerInv_{i}}{2 \cdot layerThickness_{i}} 
                        -\tfrac{flowForward_{i}}{4 \cdot layerThickness_{i}} \\
                  da_{i} & =-dispersionBack_{i} \cdot \tfrac{DistPreviousLayerInv_{i}}{2 \cdot layerThickness_{i}} 
                       -\tfrac{flowBack_{i}}{4 \cdot layerThickness_{i}} \\
                  db_{i}& =-\tfrac{\theta Old_{i}}{deltaT}
                         +dispersionForward_{i} \cdot \tfrac{DistNextLayerInv_{i}}{2 \cdot layerThickness_{i}} \\
                        & +dispersionBack_{i} \cdot \tfrac{DistPreviousLayerInv_{i}}{2 \cdot layerThickness_{i}} 
                        +\tfrac{flowForward_{i}}{4 \cdot layerThickness_{i}} 
                        -\tfrac{flowBack_{i}}{4 \cdot layerThickness_{i}} \\
                  dc_{i} & = -dispersionForward_{i} \cdot \tfrac{DistNextLayerInv_{i}}{2 \cdot layerThickness_{i}} 
                          +\tfrac{flowForward_{i}}{4 \cdot layerThickness_{i}}
             \end{align*}
                   Assign values to vectors

\quad               for all i in 0..TotalNumberOfLayers-1 do   
           
     \quad\quad             if $i=0$ then      
            
         \quad\quad\quad            b\_null,d\_null

         \quad\quad\quad            if $soluteIn>0$ then        
                        \begin{align*}             
                        soluteInFlux & = \tfrac{soluteIn}{deltaT} \\
                        b\_null &=\tfrac{dispersion_{i} \cdot DistPreviousLayerInv_{i}-0.5 \cdot flowBack_{i}}
                                {2 \cdot (dispersion_{i} \cdot DistPreviousLayerInv_{i}+0.5 \cdot flowBack_{i})} \\
                        d\_null &=(soluteInFlux+(dispersion_{i} \cdot DistPreviousLayerInv_{i}-0.5 \cdot flowBack_{i})) \\
                                &  \cdot \tfrac{soluteOld_{i}}{2 \cdot (dispersion_{i} \cdot DistPreviousLayerInv_{i}+0.5 \cdot flowBack_{i})}
                       \end{align*}
      \quad\quad               else  
                                          
                        b\_null=0, 
                        d\_null=0                         
                   \begin{align*}            
                     lowerVector_{i} & = 0 \\
                     midVector_{i} & = b_{i}+ b\_null \cdot a_{i} \\
                     upperVector_{i} & = c_{i} \\
                     soluteOldVector_{i} & = soluteOld_{i} \cdot (da_{i}+db_{i}) + soluteOld_{i+1} \cdot  dc_{i}
                                 -d\_null \cdot a_{i} 
                   \end{align*}               
     \quad\quad             else if $i=TotalNumberOfLayers-1$ then  
                    \begin{align*}              
                     lowerVector_{i}&= a_{i} \\
                     midVector_{i}&=b_{i}+c_{i} \\
                     upperVector_{i}&= 0 \\
                     soluteOldVector_{i}&= soluteOld_{i-1} \cdot  da_{i}+ soluteOld_{i} \cdot (db_{i}+dc_{i})
                    \end{align*}
       \quad\quad           else 
                    \begin{align*}                 
                     lowerVector_{i}&= a_{i} \\
                     midVector_{i}&= b_{i} \\
                     upperVector_{i}&= c_{i} \\
                     soluteOldVector_{i}&= soluteOld_{i-1} \cdot  da_{i}+ soluteOld_{i} \cdot db_{i} + soluteOld_{i+1} \cdot  dc_{i}
                  \end{align*}              
                   correct vectors for start of mobile domain
              
               k=0, 
               l=0
               
               while $IsThereSoluteFlux_l=false$   do
           
        \quad          k=l+1,  l=1+l               

               if $k>0$ then
             
                  for ($i=0i<TotalNumberOfLayers-ki++$)
                 

           \quad\quad\quad          if $i=0$ then  
                       \begin{align*}                 
                        midVector_{i}&=midVector_{i+k}+dispersion_{i+k} \cdot \tfrac{DistPreviousLayerInv_{i+k}}{2 \cdot layerThickness_{i+k}} \\
                        lowerVector_{i}&=0 \\
                        soluteOldVector_{i}&=soluteOldVector_{i+k}+dispersion_{i+k} \cdot \tfrac{DistPreviousLayerInv_{i+k}}{2 \cdot layerThickness_{i+k}} \cdot soluteOld_{i+k} \\
                        upperVector_{i}&= upperVector_{i+k}
                       \end{align*}
            \quad\quad\quad       else                   
                        if $IsThereSoluteFlux[i+k-1]=false$ and 
                           $IsThereSoluteFlux_{i+k}=false$ then                       
                          
                           $  lowerVector_{i} =
                           midVector_{i}= 
                           upperVector_{i}= 
                           soluteOldVector_{i}=0$
                       
            \quad\quad            else if 
$IsThereSoluteFlux[i+k-1]=false$ and 
                           $IsThereSoluteFlux_{i+k}=true$ then                       
                           \begin{align*}
                             lowerVector_{i} & =0 \\
                           midVector_{i}& = midVector_{i+k}+dispersion_{i+k} \cdot \tfrac{DistPreviousLayerInv_{i+k}}{2 \cdot layerThickness_{i+k}} \\
                                    & +\tfrac{flowBack_{i}}{4 \cdot layerThickness_{i}} \\
                           soluteOldVector_{i}& =soluteOldVector_{i+k}+soluteOld_{i} \\  & \cdot 
                         \left(dispersion_{i+k} 
                               \cdot \tfrac{DistPreviousLayerInv_{i+k}}{2 \cdot layerThickness_{i+k}} 
                                    +\tfrac{flowBack_{i}}{4 \cdot layerThickness_{i}} \right) \\
                           upperVector_{i}& = upperVector_{i+k}
                           \end{align*}                      
          \quad\quad              else if  
$IsThereSoluteFlux[i+k-1]=true$ and 
                           $IsThereSoluteFlux_{i+k}=false$                      
                         \begin{align*}
                           lowerVector_{i}&=lowerVector_{i+k} \\
                           midVector_{i}&= midVector_{i+k}+dispersion_{i+k} \cdot \tfrac{DistNextLayerInv_{i+k}}{2 \cdot layerThickness_{i+k}} 
                                    -\tfrac{flowForward_{i+k}}{4 \cdot layerThickness_{i+k}} \\
                           soluteOldVector_{i}&=soluteOldVector_{i+k}+soluteOld_{i+k} \cdot \\ & \left(dispersion_{i+k}
                                      \cdot \tfrac{DistNextLayerInv_{i+k}}{2 \cdot layerThickness_{i+k}} 
                                    -\tfrac{flowForward_{i+k}}{4 \cdot layerThickness_{i+k}} \right) \\
                           upperVector_{i}&= 0
                        \end{align*}                        
         \quad\quad               else   
                           \begin{align*}                     
                           lowerVector_{i}&=lowerVector_{i+k} \\
                           midVector_{i}&= midVector_{i+k} \\
                           soluteOldVector_{i}&=soluteOldVector_{i+k} \\
                           upperVector_{i}&= upperVector_{i+k}     
                           \end{align*}  
                   Solve with the sweep method (function tridag)
               Tridag(lowerVector, midVector, upperVector, soluteOldVector, soluteNew, TotalNumberOfLayers-k)

               if $k>0$ then     
       
           \quad       for ($i=TotalNumberOfLayers-1i>k-1i--$)
                     $soluteNew_{i}=soluteNew[i-k]$

            \quad      for all i in 0..ki-1 do %($i=0i<ki++$)
                     $soluteNew_{i}=0 $            

           for all i in 0..TotalNumberOfLayers-1 do
               %if (soluteNew_{i}<0)              
%\subsection{                %  theMessage.WarningWithDisplay("soilProfile.UpdateInfiltrationSWAT - negative conc")}
                %  cout << "solutetype " << soluteType << " in layer " << i << endl
             

                Calculate the fluxes between layers
                %$mobileXXXXFlux_{i} $ 

               refer to the flux out of the layer can be both negative and positive

      \quad\quad      if $soluteType=0$ then


      \quad\quad\quad        for all i in 0..TotalNumberOfLayers-1 do

                $  mobileNitFlux_{i}= $ 

               $  \begin{cases}                
                  nitIn-soluteNew_{i} \cdot \tfrac{mobileWaterNew_{i}}{1000}+soluteOld_{i} \cdot \tfrac{mobileWaterOld_{i}}{1000} & \text{ if }(i=0) \\
                  mobileNitFlux_{i-1}-soluteNew_{i} \cdot \tfrac{mobileWaterNew_{i}}{1000}+soluteOld_{i} \cdot \tfrac{mobileWaterOld_{i}}{1000} & \text{ otherwise}
                \end{cases}$

    \quad\quad        else if $soluteType=1$ then

      \quad\quad\quad         for all i in 0..numberOfTotalLayers-1 do     
      
             $  mobileAmmFlux_{i}= $

$\begin{cases}    
                  ammIn-soluteNew_{i} \cdot \tfrac{mobileWaterNew_{i}}{1000}+soluteOld_{i} \cdot \tfrac{mobileWaterOld_{i}}{1000} & \text{ if }(i=0) \\
                mobileAmmFlux_{i-1}-soluteNew_{i} \cdot \tfrac{mobileWaterNew_{i}}{1000}+soluteOld_{i} \cdot \tfrac{mobileWaterOld_{i}}{1000} & \text{ otherwise}
              \end{cases} $

      \quad\quad      else if  $soluteType=2$ then

           \quad \quad\quad  for all i in 0..TotalNumberOfLayers-1 do

            $     mobileChlorideFlux_{i}=$

$ \begin{cases} 
                  chlorideIn-\tfrac{soluteNew_{i} \cdot mobileWaterNew_{i}-soluteOld_{i} \cdot mobileWaterOld_{i})}{1000} & \text{ if }(i=0) \\
                    mobileChlorideFlux_{i-1}-\tfrac{soluteNew_{i} \cdot mobileWaterNew_{i}-soluteOld_{i} \cdot mobileWaterOld_{i})}{1000} & \text{ otherwise}
            \end{cases} $

       \quad\quad     else if  $soluteType=3$ then

         \quad\quad\quad     for all i in 0..TotalNumberOfLayers-1 do

                $ mobileNitN15Flux_{i}= $

$\begin{cases} 
                nitN15In-\tfrac{soluteNew_{i} \cdot mobileWaterNew_{i}-soluteOld_{i} \cdot mobileWaterOld_{i})}{1000} &  if(i=0) \\
                mobileNitN15Flux_{i-1}-\tfrac{soluteNew_{i} \cdot mobileWaterNew_{i}-soluteOld_{i} \cdot mobileWaterOld_{i})}{1000} & \text{ otherwise}
            \end{cases}$

            else if  $soluteType=4$ then
              for all i in 0..TotalNumberOfLayers-1 do
                 

                 $mobileAmmN15Flux_{i} =$

$ \begin{cases}
                     ammN15In-\tfrac{soluteNew_{i} \cdot mobileWaterNew_{i}-soluteOld_{i} \cdot mobileWaterOld_{i})}{1000} & \text{ if }(i=0) \\
                      mobileAmmN15Flux_{i-1}-\tfrac{soluteNew_{i} \cdot mobileWaterNew_{i}-soluteOld_{i} \cdot mobileWaterOld_{i})}{1000}
                                             \end{cases}    $                            

                

End bypass if nonzeroconc is false

            else      non-zero conc true 

       $     \begin{cases}        
              for all i in 0..TotalNumberOfLayers-1 do mobileNitFlux_{i}=0  & \text{ if } (soluteType=0)  \\          
               for all i in 0..TotalNumberOfLayers-1 do  mobileAmmFlux_{i}=0 & \text{ if } (soluteType=1) \\             
              for all i in 0..TotalNumberOfLayers-1 do mobileChlorideFlux_{i}=0 & \text{ if }  (soluteType=2)  \\            
              for all i in 0..TotalNumberOfLayers-1 do mobileNitN15Flux_{i}=0  & \text{ if }  (soluteType=3)  \\            
              for all i in 0..TotalNumberOfLayers-1 do  mobileAmmN15Flux_{i}=0 & \text{ if }  (soluteType=4)
           \end{cases} $
         
        all solutes

             update budgets and assign values

%          nitrogen aMobileNitFlux, aMobileAmmFlux, aNitIn, aAmmIn
%       
%          aMobileChlorideFlux, aChlorideIn

         aNitIn.SetBoth(nitIn, nitN15In),
         aAmmIn.SetBoth(ammIn, ammN15In)

         aChlorideIn=chlorideIn

         for all i in 0..numberOfTotalLayers-1 do     

    \quad\quad        aMobileNitFlux.SetBoth($mobileNitFlux_{i},mobileNitN15Flux_{i}$)

     \quad\quad       aMobileAmmFlux.SetBoth($mobileAmmFlux_{i},mobileAmmN15Flux_{i}$)

           $ aMobileChlorideFlux= mobileChlorideFlux_{i}$

     \quad\quad       if (i=0)     
     
       \quad\quad\quad        $soilLayerArray_0.AddWaterAndSolutes($

 \quad\quad\quad\quad $infiltration,aNitIn, aAmmIn,aChlorideIn)$

     \quad\quad\quad        $ soilLayerArray_1.AddWaterAndSolutes($

\quad\quad\quad\quad  $fluxInTimeStep_0,aMobileNitFlux, aMobileAmmFlux,aMobileChlorideFlux) $
                      
 Add to next

      \quad\quad\quad        $ soilLayerArray_0.RemoveWaterAndSolutes($


\quad\quad\quad\quad $ fluxInTimeStep_0,aMobileNitFlux,
                                 aMobileAmmFlux,aMobileChlorideFlux)$        



                remove from current
      
      else if $i=TotalNumberOfLayers-1$  then     
     
    $ soilLayerArray_{i}.RemoveWaterAndSolutes($

\quad $
fluxInTimeStep_{i},aMobileNitFlux, aMobileAmmFlux,aMobileChlorideFlux) $     

   remove from current,  (update leaching)
               
              $  NitLeached =  NitLeached+aMobileNitFlux$,  
              $  AmmLeached =  AmmLeached+aMobileAmmFlux $ 
          
            else          
             
              $ soilLayerArray_{i+1}.AddWaterAndSolutes($

\quad $fluxInTimeStep_{i},aMobileNitFlux,$

               \quad \quad                $aMobileAmmFlux,aMobileChlorideFlux) $                       

Add to next layer
               
\quad              $ soilLayerArray_{i}.RemoveWaterAndSolutes($

$fluxInTimeStep_{i},aMobileNitFlux, aMobileAmmFlux,aMobileChlorideFlux)   $                     

remove from current
        
        
         $ surplus= surplus \cdot (1-f)$

        $   surfaceNit=  surfaceNit \cdot (1-f)$,
        $   surfaceAmm=  surfaceAmm \cdot (1-f)$

             what about chlorid!!!
     
      simulationTime+=deltaT

   End transport of water and solute.
       Equilibrate between mobile and immobile water   (once a day)

   for all i in 0..TotalNumberOfLayers-1 do

   \quad if $waterContent_{i}<FC_{i}$ then

   \quad \quad   	$soilLayerArray_{i}.EquilibrateNitrogen(0)$

    \quad else

    \quad \quad  $soilLayerArray_{i}.EquilibrateNitrogen(waterContent_{i}-FC_{i})$

   \quad \quad $EvapContent=\min(EvapCapacity,infiltration+EvapContent)$

  for all i in 0..TotalNumberOfLayers-1 do  
    $ soilLayerArray_{i}.SetWaterFlux(totalFlux_{i})$
   
        End transport of water and solute


%Returns content of carbon/organic nitrogen in products stored on soil surface [$g C/m^2$].
%Calculates soil water content for the profile.  returns  - Water content [mm]
%Returns number of organic product objects stored on soil surface

  

  \bibliography{literature}
  \bibliographystyle{plainnat}



\end{document}
