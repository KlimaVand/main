\documentclass[%parskiphalf,%numbers=noendperiod
]{scrartcl}
\usepackage{amsmath}
\usepackage[latin1]{inputenc}
\usepackage[english]{babel}
\usepackage[T1]{fontenc} 
\usepackage{graphicx,parskip}
\usepackage{booktabs,longtable}
\usepackage{lmodern}
\usepackage[round]{natbib}

\newcommand\mymarginpar[1]{\marginpar {\flushleft\bfseries\scriptsize #1}}

\title{FASSET field}

\author{}
\date{\today}

\usepackage{lmodern}


\usepackage[backref]{hyperref}

\begin{document}

\maketitle
 \tableofcontents
\newpage

\section{Field}


\subsection{field(const char aName, const  aIndex, const base aOwner)
}   :base(aName,aIndex,aOwner)

Constructor with arguments

	Lfutr                      =new cloneList<plan>
	
	crrPlan                    =NULL
	
   thePatches                 =new cloneList<patch>
   
   Minimum                    =0.00001

   NoFixCrops =
   JBNO   =
   daysSinceIrrigation        =
	daysSinceIrrigationRequest =
   fingerFlow				      =
   daysBetweenIrrigation      =
   irrigationDelay            =
	distance                   =
   area                       = 0

   active                     =true
   
   monitorPatches             =false     
   // Read from some file later
   
\#ifdef TUPLE

   writeTuples                =
   tuple\_file\_open            =false
   
\#endif


   for ( i=0 i<BackScope i++)  
   
    \quad	 History[i]="w1"


   for ( i=0 i<BackScope i++)

   \quad    FixedCrop[i]=NULL, FixedSize[i]=0


   if (monitorPatches) then
   
       \quad sprintf(leader,globalSettings.getOutputDirectory().c\_str())

       \quad sprintf(filename,"patches%d1.xls",Index)
     
      \quad  strcat(leader,filename)
      
       \quad fPatch1.open(leader,ios::out)
       
      \quad fPatch1.precision(8)
      
      \quad fPatch1 << "Date tab "
      
      \quad for ( i=0 i<MaxWritePatches i++)
      
       \quad \quad   fPatch1 << "Area" << i << " tab "


      \quad \quad sprintf(leader,globalSettings.getOutputDirectory().c\_str())
	  
	  \quad  \quad sprintf(filename,"patches%d2.xls",Index)
      
       \quad strcat(leader,filename)
     
       \quad fPatch2.open(leader,ios::out)
     
      \quad fPatch2.precision(8)
     
      \quad fPatch2 << "Date tab "
     
      \quad for ( i=0 i<MaxWritePatches i++)
     
      \quad \quad    fPatch2 << "Area" << i << " tab "

     \quad  fPatch1 << "Date tab "
      
     \quad  for ( i=0 i<MaxWritePatches i++)
       
     \quad    \quad  fPatch1 << "N-leach" << i << " tab "
      
     \quad  fPatch1 << "Date tab "
      
      \quad for ( i=0 i<MaxWritePatches i++)
      
     \quad   \quad   fPatch1 << "Denit-" << i << " tab "
     
    \quad   fPatch1 << "Date tab "
     
     \quad  for ( i=0 i<MaxWritePatches i++)
     
     \quad  \quad    fPatch1 << "N2O-" << i << " tab "
      
     \quad  fPatch1 << endl


      \quad fPatch2 << "Date tab "
     
    \quad   for ( i=0 i<MaxWritePatches i++)
    
    \quad  \quad     fPatch2 << "DM" << i << " tab "
     
      \quad fPatch2 << "Date tab "
     
     \quad  for ( i=0 i<MaxWritePatches i++)
      
    \quad \quad      fPatch2 << "LAI" << i << " tab "
    
    \quad   fPatch2 << "Date tab "
     
    
      \quad for ( i=0 i<MaxWritePatches i++)
   
        \quad \quad  fPatch2 << "OFFT" << i << " tab "
  
      \quad fPatch2 << endl
   

   if (globalSettings.DetailsData.getPlantDetails()=true)
   
       \quad sprintf(leader,globalSettings.getOutputDirectory().c\_str())
     
      \quad sprintf(filename,"field%d.xls",Index)

     \quad  strcat(leader,filename)

     \quad  f.open(leader,ios::out)
     
     \quad  f.precision(4)
     
	 \quad	OutputPlantDetails(true)
   
   
   InitVariables()





\subsection{field(const field fl) : base(fl)}
Copy constructor, Dung, urine and their areas presently ignored not 

   for ( i=0 i<MaxFixedCrops i++)
   
  \quad    FixedCrop[i]=fl.FixedCrop[i] 
  
   // OK because reference does not change
  
  \quad    FixedSize[i]=fl.FixedSize[i]
 
   for ( i=0 i<BackScope i++)
   
	\quad   History[i]=fl.History[i]
	   
   thePatches=fl.thePatches.clone()
   
   if (fl.crrPlan) then
   
  	 \quad	   crrPlan=fl.crrPlan.clone()
  
   else
   
     \quad	  crrPlan=NULL
 
   if fl.Lfutr then
   
    \quad	   Lfutr=fl.Lfutr.clone()
  
   else
   
    \quad	   Lfutr=NULL
      
   Minimum=fl.Minimum
   
   area=fl.area

   distance=fl.distance

   JBNO=fl.JBNO
   
   daysSinceIrrigation=fl.daysSinceIrrigation
   
   daysSinceIrrigationRequest=fl.daysSinceIrrigationRequest
  
   daysBetweenIrrigation=fl.daysBetweenIrrigation
  
   fingerFlow=fl.fingerFlow
 
 	irrigationDelay=fl.irrigationDelay
  
  	NoFixCrops=fl.NoFixCrops
  
   active=fl.active

   grazingPossible=fl.grazingPossible
   
   if (globalSettings.DetailsData.getPlantDetails()=true)
   
      \quad  char filename[40]
       
        \quad
      sprintf(filename,"\\fasset\\field%d.xls",Index)

       \quad f.open(filename,ios::app)
       
       \quad f.precision(4)
   
\#ifdef TUPLE

   writeTuples=fl.writeTuples
   
\#endif

\#ifdef ARLAS

   FieldID=fl.FieldID
   
\#endif


\subsection{~field()}
Destructor

   if (globalSettings.DetailsData.getPlantDetails()=true)
   
    \quad    f.close()
      
   if monitorPatches then
   
     \quad   fPatch1.close(),
      fPatch2.close()

   if Lfutr	then delete Lfutr

	if crrPlan then delete crrPlan
  
   if thePatches then   delete thePatches


\subsection{InitVariables()}
area = 0, 

grazingPossible   =  
   monitorPatches=
   grazingShutDown   = true



\#ifdef NITROSCAPE
\subsection{UpdatePatch( farmNo)}

   patch * patchVariable
	 NumberOfPatches=thePixiData.getNumberOfEco(farmNo,Index)
	 
   $areaPerPatch = \tfrac{area}{NumberOfPatches}$
   
   string soilFileName
   
	if $NumberOfPatches>0$ then	
	
		\quad patchVariable=thePatches.ElementAtNumber(0)
		
        \quad soilFileName += patchVariable.getSoilFileName()
        
        \quad patchVariable.SetArea(areaPerPatch)
	
	for( i=0 i<NumberOfPatches i++)
	
 \quad			cout<< "Farm No " << farmNo<<" Field no "<<Index<<" patch no " << i<<endl

 \quad			if(thePixiData.getNumberOfFarms()<farmNo)
		
 \quad	 \quad				cout<<"UpdatePatch: farm does not exist in fasset.dist"<<endl

 \quad	 \quad				cout << "Farm number = " << farmNo << " number in fasset.dist " 

\quad	 \quad	 << thePixiData.getNumberOfFarms() << endl

  \quad	  \quad	       cin.get()

 \quad	 \quad				exit(99)
		
		
 \quad			if $thePixiData.getNumberOfFields(farmNo) \le Index$ then
		
 \quad	 \quad				cout<<"UpdatePatch: fields does not exist in fasset.dist"<<endl

 \quad	   \quad	       cin.get()
 
 \quad	 \quad				exit(99)
		
 \quad			 if $thePixiData.getNumberOfEco(farmNo,Index)\le i$ then
		
 \quad	 \quad						cout<<"UpdatePatch: patch does not exist in fasset.dist"<<endl

 \quad	  \quad	        		cin.get()

  \quad	\quad						exit(99)
		
 \quad			 EcosystemNumber=thePixiData.getEco(farmNo,Index, i)

 \quad			cout<< "Farm No " << farmNo<<" Field no "<<Index<<" patch no " << i << " ecosys no " << EcosystemNumber <<endl
     
   \quad	    if (i>0)
      
    \quad	     \quad	  patchVariable =new patch("patch",Index,this)
      
   \quad	     \quad	   patchVariable.Initialise(soilFileName,i,areaPerPatch)
       
  \quad	       \quad	  thePatches.InsertLast(patchVariable)
      
 \quad	//		patch * patchVariable=thePatches.ElementAtNumber(i)

 \quad		patchVariable.setEcoModel(EcosystemNumber)

\#endif


\subsection{ReadPatchParameters(patch *p, commonData * data)}

	string soilFileName

   patchArea
   
   data.setCritical(true)
   
   data.FindItem("JBNO",JBNO)
   
   data.setCritical(false)
   
   if  not data.FindItem("soilFile",soilFileName) then
   
    \quad	   convert conv
    
       \quad soilFileName="JB"+conv.IntToString(JBNO)+"-M.dat"
   
   data.FindItem("area",patchArea)
   
   area+=patchArea
   
   if $area\le 0$ then
   
   \quad  err("ReadParameters - field area cannot be zero")
   
   p.Initialise(soilFileName,0,patchArea)


\#ifdef NITROSCAPE
\subsection{zeroOutputVariables()}

	for( i=0 i<<thePatches.NumOfNodes() i++)
		
\quad		 thePatches.ElementAtNumber(i).zeroOutputVariables()
	

\#endif


\subsection{ReadParameters(commonData * file)}
   if (thePatches.NumOfNodes()>1)
   
   \quad     err("ReadParameters - only one patch allowed here")

   file.setCritical(true)
   
   if file.FindSection(Name,Index) then
   
	 \quad	  file.setCritical(false)

     \quad	  sprintf(Indexstr,"(%d)",Index)
      
     \quad	  namestr=(string) "field" +Indexstr+"."+"patch"
      
    \quad	   file.getSectionsNumbers(namestr,first,num)  
      
    \quad	    //count the number of patches
       
\#ifndef NITROSCAPE

      if $num>0$ then 
      
      \quad //may be more than one patch per field
      
	   \quad	    $Setcur\_pos(0)$
     
       \quad	   for( inx=first $inx\le (first+num)$ inx++)
         
      \quad	  \quad	    file.FindSection(namestr,inx)   // find the correct place in the input file
      
       \quad	  \quad	  
$ patch * p =new patch((char *)namestr.c\_str(),inx,this)$
      
       \quad	 \quad	    ReadPatchParameters(p,file)
     
      \quad	 \quad	     thePatches.InsertLast(p)
      
     
      else   //one patch per field
    
     \quad	    patch * p =new patch("patch",Index,this)
     
     \quad	    ReadPatchParameters(p,file)
   	
   	  \quad	   thePatches.InsertLast(p)
	

\#else


		patch p =new patch("patch",Index,this)
		
		ReadPatchParameters(p)
		
		thePatches.InsertLast(p)
	

\#endif

   	
      
      if theControlParameters.GetUniSizeFields() then
       
         \quad  area=1  // For testing purposes not 
      
       \quad file.FindItem("distance",distance)


       
	  \quad	for (k=0 k<BackScope k++)	
			
	  \quad	  \quad	itoa(k,number,10)
		  
	  \quad	  \quad	strcpy(s,"CROP")
		  
	  \quad	  \quad	strcat(s,number)
		  
	  \quad	  \quad	file.FindItem(s,History[k])
		

   
   
   
   
   
   else
   
     \quad   err("ReadParameters - section not found")

   //if we wish to simulate patches with different soil types, this will have to be modified
   
//   cloneList<patch>::PS P

//   thePatches.PeekHead(P)

//   P.element.Initialise(soilFileName,Index,area)



\subsection{PushHistory(field * fP,str * crop\_id)}
Updates cropping history when a new crop is initiated

   fP -  pointer to new field,
   crop\_id - id of new crop

	fP.History[0]=History[1]
	
	fP.History[1]=History[2]
	
	fP.History[2]=History[3]
	
	fP.History[3]=*crop\_id


\subsection{field* MakeNewField()}
	return new field(Name,Index,Owner)


 \subsection{AdjustPatchAreas(adjust,bool balance)}
   if $adjust <0$ then
   
   \quad   err("AjustPatchAreas - call parameter must be above zero")
   
   if $adjust>1$  and   not balance then
   
   \quad   err("AjustPatchAreas - call parameter must be not exceed one in this mode")

   Nbefore=NcropBefore= NsoilBefore=0

	if thePatches then
   
 \quad  	for ( i=0 i<thePatches.NumOfNodes() i++)
      
 \quad   \quad     NcropBefore+=thePatches.ElementAtNumber(i).NitrogenInCrops()
    
\quad     \quad    NsoilBefore+=thePatches.ElementAtNumber(i).NitrogenInSoil()
 
 \quad  \quad      Nbefore+=thePatches.ElementAtNumber(i).NitrogenInSystem()
  
  \quad 	\quad	 thePatches.ElementAtNumber(i).AdjustArea(adjust)
      
  
   else
   
  \quad    err("AjustPatchAreas - attempt to divide uninitialised field")

   Nafter=
   NcropAfter=
   NsoilAfter=0
   
   for ( i=0 i<thePatches.NumOfNodes() i++)
   
   \quad   NcropAfter+=thePatches.ElementAtNumber(i).NitrogenInCrops()

 \quad     NsoilAfter+=thePatches.ElementAtNumber(i).NitrogenInSoil()

 \quad     Nafter+=thePatches.ElementAtNumber(i).NitrogenInSystem()
   

   if (balance)
   
   \quad   (Adjust C and water in indicators later)
 
  \quad    AddInd(environmentalIndicator,"37.00 N difference from area adjustment","kg N",Nbefore-Nafter)

   \quad   AddInd(environmentalIndicator,"37.01 N crop difference from area adjustment","kg N",
NcropBefore-NcropAfter)

\quad      AddInd(environmentalIndicator,"37.02 N soil difference from area adjustment","kg N",NsoilBefore-NsoilAfter)
  

\subsection{AdjustFieldArea(newArea)}   // cout << "Adjusting area on field no " << Index << " from " << area << " ha to " << newArea << " ha " << theTime << endl
$   AdjustPatchAreas(\tfrac{newArea}{area},true)$
   area=newArea


\subsection{DivideThisInstance(field * fP,plan * Pl)}

Divides the present instance of the field.
Copy constructor can not be made, as "linkList" can not have a copy constructor.

   fP - pointer to new field
   Pl - new plan of field operations

   fP=MakeNewField()
	// to cope with derived fields
   
   if fP.thePatches then
   
     \quad   delete fP.thePatches
      
	fP.thePatches=thePatches.clone()
	
	fP.crrPlan = Pl
	
   fP.Index=Pl.GetFutureNumber()
   
	for( i=0 i<BackScope i++)
	
	  \quad	fP.History[i] = History[i]
		
	
	Pl.GiveCrop(cid)
	
	PushHistory(fP,cid)
	
	fP.area=Pl.GiveArea()
	
	fP.distance=distance

   fP.daysSinceIrrigation=daysSinceIrrigation

	 fP.daysSinceIrrigationRequest=daysSinceIrrigationRequest
	 
   fP.daysBetweenIrrigation=daysBetweenIrrigation
   
   fP.irrigationDelay=irrigationDelay
   
   fP.fingerFlow=fingerFlow

	fP.JBNO=JBNO
	
	ofstream * fstr=theMessage.GiveHandle()
	
	*fstr << "Divided. Area 'mother': 
" << area << "ha. Area 'child': " << Pl.GiveArea() << " ha" << "n"

   //Adjust areas
   
$   fP.AdjustPatchAreas(\tfrac{fP.area}{area},false)$
   
 $  AdjustPatchAreas(\tfrac{area-Pl.GiveArea()}{area},false)$
   
	area=area-Pl.GiveArea()
	
   for ( i=0  i<NoFixCrops i++)
   
     \quad	fP.FixedCrop[i] = FixedCrop[i]
     
      \quad  fP.FixedSize[i] = FixedSize[i]
   

\subsection{MakeInstance(field * fP,plan * Pl)}
Renew or divide field.

   fP -  pointer to new instance of field
   Pl -  new plan of field operations

	
	Ar=Pl.GiveArea()
	
	Pl.GiveCrop(cid)

   if $area=0$ then
	
		 \quad warn("MakeInstance - attempt to divide a field with an area of zero")

	if $Ar=0$ then
	
		\quad  warn("MakeInstance - attempt to divide a field with request of an area of zero")

	if $Ar>area$ then
  
\quad 	if $Ar>area+Minimum$ then // Handles rounding-errors
		
    \quad \quad      if (theControlParameters.GetFlexSizeFields())
    
    \quad \quad \quad         AdjustFieldArea(Ar)
    
     \quad \quad     else
     
   	\quad \quad \quad 		err("MakeInstance - attempt to divide a field with request of an area bigger than that of the field")
		
	\quad \quad 	Ar=area
	
	
	
	
	if $area>0$ then
   
		if $area > Ar$  and  theControlParameters.GetFlexSizeFields()) // Test if field 
		
	 \quad		size should be adjusted
		
      \quad	    AdjustFieldArea(Ar)
		
		if $area > (Ar+Minimum)$ then
		
	 \quad		 // Handles rounding-erors
       
      \quad	  	DivideThisInstance(fP,Pl)
		
		else
		
     \quad	  { // The field is not to be divided, but renewed
      
	 \quad	 		if $Lfutr.NumOfNodes()>1$ then
			
		 \quad	 \quad			 warn("MakeInstance - error in list of future plans")
				 
	 \quad			if $Pl.GetFutureNumber()\ne Index$ then
         
         \quad	     \quad	 cout << Pl.GetFutureNumber() << endl
				err("MakeInstance - inconsistence in indices")
         
			if (crrPlan)
			
			 \quad		delete crrPlan
			 
			crrPlan=Pl
			
         if (crrPlan.GetAdjustArea()>0)
         
            \quad	 AdjustFieldArea(crrPlan.GetAdjustArea())

			PushHistory(this,cid)
			
			fstr=theMessage.GiveHandle()
			
			*fstr << "Renewed. Size: " << area << " ha" << endl

	
	
\#ifdef TUPLE

   writeTuples=theControlParameters.GetWriteTuples()
   
\#endif


\subsection{SplitMe(field * fP)}
Investigates whether the field should be divided or renewed.  fP - Pointer to new instance. NULL if field is undivided.

   if active then  
   
	   \quad	 ofstream * fstr
	   
       \quad	fP=NULL
       
       \quad	if crrPlan=NULL  or  crrPlan.Finished()
       
      
       \quad	 \quad	   cloneList<plan>::PS P
       
       \quad	 \quad	   Lfutr.PeekHead(P)
       
       \quad	 \quad	   if $P\ne NULL$ then
        
         \quad	 \quad	 \quad	     year = theTime.GetYear()
         
       \quad	 \quad	 \quad	      month = theTime.GetMonth()
       
        \quad	 \quad	 \quad	      day = theTime.GetDay()
        
\#ifdef MEASCOPE

      \quad	 \quad	 \quad	      
 if ((P.element.FirstOperationToday(year,month,day) 
 
\quad and  (OperationCanBePerformed(P.element,thePatches.ElementAtNumber(0))))

 \quad            		 or  (P.element.NoOpsThisYear(year,month,day))) then
            		 
\#else

        \quad	 \quad	 \quad		     if (P.element.FirstOperationToday(year,month,day) 

\quad	 \quad	 \quad  and  (OperationCanBePerformed(P.element,thePatches.ElementAtNumber(0))))

\#endif
            

      \quad	 \quad	 \quad	 \quad	          fstr=theMessage.GiveHandle()
               *fstr << theTime << " >>> Field no " << setw(2) << Index << "  "

        \quad	 \quad	 \quad	 \quad	        MakeInstance(fP,P.element)
    
      \quad	 \quad	 \quad	 \quad	          Lfutr.Release(P)

   

\subsection{ImplementFingerFlow()}	
  if $JBNO=1$ and fingerFlow then 

 \quad	// irrigation on JB1 implies finger flow on 2/3 of the area and 1/3 do not receive any water not 
   
   
  \quad    if thePatches.NumOfNodes()>2 then
  
  \quad \quad     	 warn("ReadParameters - to many patches on field")
  
\quad //      MeltAllPatches()

 \quad   	cloneList<patch>::PS P
 
 \quad   	thePatches.PeekHead(P)
 
  \quad     patch* P2= P.element.clone()
  
 \quad      P2.SetArea(area*1.0/3.0)
 
\#ifndef NITROSCAPE

  \quad     P2.SetFingerFlow(2)
  
  \quad     P.element.SetFingerFlow(1)
  
\#endif

  \quad    $ P.element.SetArea(area*\tfrac{2}{3})$
  
 \quad      thePatches.InsertLast(P2)
   
   
   
   
   daysSinceIrrigationRequest = 0


\subsection{HandleOperation(fieldOperationFields * FOP)}
Handles a field operation (e.g. sowing)
   FOP -  object with operation characteristics

if FOP.GetCropName()[1]=1 then

	\quad 	cout<<FOP.GetCropName()<<"test 4 "<<FOP.GetCropName()[1]<<endl

	bool delayOp = false

   //NJH commented out next section because it should be transferred to management class

   if ( not delayOp)
   
\quad 	   ofstream * fstr=theMessage.GiveHandle()

  \quad     *fstr  << theTime  << " Field no " << setw(2) << Index  << " "  << setfill(' ') << setiosflags(ios::left)  << setw(20)
         
\quad 	   fstr.flush()
	   
  \quad 	*fstr  << *FOP


   \quad    cloneList<patch>::PS P
   
   \quad    thePatches.PeekHead(P)
   
   \quad    while $P\ne NULL$ do
      
   \quad     \quad   P.element.HandleOp(FOP)
   
    \quad     \quad  thePatches.OneStep(P)
      
	
   return  not delayOp



\subsection{bool SoilIsSuitableForFertilizing()}
Returns whether the soil is trafficable. Presently very simplistic, if standing water is below a threshold, it returns true.

   standingWater=
   areaSum=0
   
   cloneList<patch>::PS P
   
 	thePatches.PeekHead(P)
 
   while $P\ne NULL$ do
      
        \quad area=P.element.GetArea()
      
       \quad areaSum+=area
      
      \quad  standingWater+=P.element.StandingWater()*area
      
      \quad  thePatches.OneStep(P)
   
$   standingWater=\tfrac{standingWater}{areaSum}$
   
   return (standingWater<1.0  or  theTime.GetMonth()=5)
   
	// To ensure that fertilizing is possible later



\subsection{Irrigate( year, month, day)}
Perform irrigation of field


   if $daysSinceIrrigation\ge daysBetweenIrrigation$ then
   
	\quad  // Must be redefined to capacity of technics not 
   
    \quad   if $daysSinceIrrigationRequest\ge irrigationDelay$ then
      
	 \quad  \quad  	$char * crop\_id$
	 
	\quad \quad $PresentCrop\_(crop\_id)$
	
	\quad \quad water * aWater=new water
	
	\quad \quad	aWater.Setname("WATER")
	
\quad\quad   theProducts.GiveProductInformation(aWater)
	
\quad\quad  aWater.Setamount(30*10)  // 30mm, 1ha

\#ifdef ARLAS

		 \quad   \quad FieldSuitcase.AddIndicator(environmentalIndicator,"irrigation",30.0)

\#endif

	\quad \quad 		irrigateFields * FOP=new
	 irrigateFields
   
   \quad   \quad   $  FOP.DefineIrrigateFieldsOp(Irrigating,crop\_id,crop\_id,$

\quad   \quad $year,month,day,FieldArea_(),0,DistanceToField_(),aWater,false)
$

\quad 	\quad 	   HandleOperation(FOP)

\quad 	\quad 		theTechnics.ExecuteFields(FOP)

 \quad   \quad       delete FOP

	\quad \quad 		delete aWater
	
	\quad \quad       daysSinceIrrigation=0

 \quad  \quad        daysSinceIrrigationRequest=0
      
      
   \quad    else
   
   \quad   \quad  	daysSinceIrrigationRequest++
	
   else
   
   \quad 	daysSinceIrrigationRequest=0



\subsection{bool OperationCanBePerformed(plan* aPlan,patch* aPatch)}
Determines if the operation can be performed
Note addition of forage harvest (NJH May 2004


   if $aPlan \ne NULL$ and  $not aPlan.Finished()$ then
   
    \quad	 // Last term added 28.02.2007 by BMP for speed and transparency
   
	{
\#ifndef MEASCOPE

    	bool ReadyForHarvest
    	
      if (theControlParameters.GetForcedHarvest())

      	ReadyForHarvest = aPlan.NextIsHarvestOp() and (aPatch.ReadyForHarvestOrLater()  

  \quad     or  aPlan.NextIsForcedHarvest())
      
      else

      	ReadyForHarvest = aPlan.NextIsForageHarvestOp()  or  (aPlan.NextIsHarvestOp() 

        \quad          and aPatch.ReadyForHarvestOrLater())
      	
      bool ReadyForFertilizing = aPlan.NextIsFertilizing() and (SoilIsSuitableForFertilizing()   

\quad     or  theControlParameters.GetForcedOperations())
      
     $ bool ReadyForOtherOp = theControlParameters.GetForcedOperations() $ 

$or  ( not aPlan.NextIsHarvestOp() and  not aPlan.NextIsFertilizing())$


   	return (ReadyForHarvest  or  ReadyForFertilizing or  ReadyForOtherOp)
   	
\#else
     
      bool OpTrigger = false
      
      OpTrigger = aPlan.CheckIfToDo(0.0)    //should have real moisture from patch not  not  not 
      
      if (( not OpTrigger) and  aPlan.CheckIfMustDo()) 
     
      \quad	 //if have reached latest date at which an operation may be performed
     
    \quad	      	OpTrigger = true
         	
      return OpTrigger
      
\#endif
	
   else
   
   \quad 	return false





\subsection{UpdateField()}
Daily update of field state

    year = theTime.GetYear()
    
    month = theTime.GetMonth()
    
    day = theTime.GetDay()

   if active then

      daysSinceIrrigation++
      
      cloneList<patch>::PS P
      
      thePatches.PeekHead(P)
      
      if (P=NULL)
         err("UpdateField - no patches in field")
    
      if (P.element=NULL)
         err("UpdateField - pointer error")
      
      fieldOperationFields * FOP=NULL
     
      if $crrPlan \ne NULL$ and  not $crrPlan.Finished()$ then
       
      \quad    FOP=crrPlan.ReturnOperation(year,month,day,false)
      
      long  delay=0
    
      if $FOP\ne NULL$ then
    
\#ifndef MEASCOPE
 
  \quad        delay=theTime.GetDateValue() - FOP.GetOpTime().GetDateValue()
 
  \quad        if $delay>65$ then
         
\quad   \quad           cout << endl << GetLongName()

 \quad   \quad          err("UpdateField - field operation not performed after more than 65 days delay")
         
\quad \#else

 \quad         delay=theTime.GetDateValue() - FOP.GetopStart().GetDateValue()

\quad          if $delay>120$ then
         
  \quad \quad           cout << endl << GetLongName()

\quad\quad              err("UpdateField - field operation not performed after more than 120 days delay")
         
\#endif
      
      
      
      if (OperationCanBePerformed(crrPlan,P.element))
      
     \quad     FOP=crrPlan.ReturnOperation(year,month,day,true)
   
     \quad     if $FOP\ne NULL$ then
         
      \quad \quad       if (crrPlan.IsThisTheFirst(FOP) and fingerFlow)
  
    \quad \quad \quad            ImplementFingerFlow()
   
    \quad \quad         if ( not (FOP.IsIrrigationOp() and crrPlan.AutomaticIrrigation()))
            
      \quad \quad \quad          if (HandleOperation(FOP))
               
\#ifdef MEASCOPE
  	
  		\quad \quad \quad 				if (FOP.GetopEnd())
                  {
        \quad \quad \quad              if $theTime\ge *FOP.GetopEnd()$ then
    
      \quad \quad \quad \quad                   warn("Operation forced at end date: field no and op name = ",
     
      \quad \quad \quad                       Index,operationNameList[FOP.GetOperationId()].c\_str())
                  
       \quad \quad \quad             crrPlan.CheckIfFinished()
\#else
                  if $delay>20$ then
                  
      \quad                cout << endl << delay << " days delay of operation type " << FOP.GetOperationId() << " due for " << *FOP.GetOpTime() << " on field no " << Index << endl
 
 \quad                     if $delay>60$ then
  
   \quad \quad                      warn("UpdateField - field operation performed after more than 60 days delay")
    
    \quad                  else
    
    \quad \quad                     warn("UpdateField - field operation performed after more than 20 days delay")
                  }
\#endif
      \quad \quad             theTechnics.ExecuteFields(FOP)
      
 \quad \quad                  ofstream * file=theMessage.GiveHandle()   // write end of line marker
          
          \quad \quad         *file << endl

     
      while $P \ne NULL$ do
      
	  \quad     if (P.element.IrrigationDemand() and crrPlan.AutomaticIrrigation())
	  
   	\quad \quad       Irrigate(year,month,day)
   	
   \quad \quad    	thePatches.OneStep(P)
      

\#ifdef TUPLE

      if writeTuples then
      
       \quad   cloneList<patch>::PS P
      
       \quad   thePatches.PeekHead(P)
       
       \quad   if $P\ne NULL$ then
         
    
      \quad \quad       $fixation\_sum+=P.element.GetNFixationThisDay()$
    
     \quad \quad       $ deposition\_sum+=0.0411 $ // HACK, UPDATE! 
        
      \quad \quad    if  $day=31$ and $month=12$ then
      
        \quad    \quad \quad  $AddTuple(theTime.GetString2(),$

\quad    \quad \quad \quad  $deposition\_sum*area,0,"DEPOSITION","DEPOSITION",0)$
      
       \quad     \quad \quad  $AddTuple(theTime.GetString2(),$

\quad   \quad   \quad \quad $fixation\_sum*area,0,"FIXATION","FIXATION",0)$
     
       \quad  \quad \quad    $ deposition\_sum=0$
      
       \quad \quad \quad     $ fixation\_sum=0$
       
\#endif

      UpdatePatches()

		GiveIndicators()



\subsection{UpdatePatches()}
cloneList<patch>::PS P

  	thePatches.PeekHead(P)
  	
    k=0
    
   while $P\ne NULL$ do
   
   \quad P.element.UpdatePatches()
   
    \quad    thePatches.OneStep(P)
    
     \quad   k++
   
   
   if (globalSettings.DetailsData.getPlantDetails()=true)
    
      \quad  OutputPlantDetails(false)


 \subsection{CropHistory( nr,char * crop\_id)}
Returns the crop for a previous season
Parameters.
   nr -  season, $crop\_id$ reference for the crop id

   $crop\_id =new char[3]$
   
   $strcpy(crop\_id,History[nr].c\_str())$
   
   if  not active then
   
   \quad     err("CropHistory called for inactive field")


\subsection{PresentCrop(char cropid)}
Returns the present crop

	$crop\_id =new char[5]$
	
	$strcpy(crop\_id,History[3].c\_str())$


\subsection{DistanceToField()}
Returns the distance from farm to field (meters)

	return $distance$


\subsection{FieldArea()}
Returns the area of the field (ha)

	return $area$


\subsection{JBno()}
Returns the JB classification number of the soil

	return $JBNO$




 \subsection{PlanField(cloneList<fieldOperationFields> 
 OP,str cropid, futureNum)}

Inserts the plan in the list of plans for the following growth season

   OP - plan (list of field operations),
   $crop\_id$ - crop identification
	
	plan  pl=new plan
	
	cloneList<fieldOperationFields>::PS P
	
	OP.PeekHead(P)
	
	$pl.HandPlanOver(OP,crop\_id,futureNum)$
	
   if $pl.GiveNumOps()=0$ then
   
    \quad 	delete pl
    
   else
  
\#ifdef MEASCOPE

     \quad	pl.CheckOps()
     
\#endif

	  \quad	Lfutr.InsertFirst(pl)
   




\#ifdef TUPLE

\subsection{CloseTupleFile()}
   if $tuple\_file\_open$ then
   
      \quad  tuple.close()
      
       \quad $tuple\_file\_open=false$

\subsection{OpenTupleFile(string s)}
   $string s1="c:\\fasset\_v2\\F\_"+s+"_sim.txt"$
   
   if (theControlParameters.GetSimTupleHasBeenOpened())
   
      \quad $ tuple.open(s1.c\_str(),ios::app)$
      
   else
   
     \quad $  tuple.open(s1.c\_str(),ios::out)$
     
     \quad   tuple << "Key tab FarmId tab Fieldno tab Date tab Index tab MainCrop tab Area tab Operation tab Product tab N tab Amount tab SFU" << endl
     
     \quad   theControlParameters.SetSimTupleHasBeenOpened(true)
   
   
   $tuple\_file\_open=true$





\subsection{AddTuple(string date,n,amount,string op,string type,SFU)}

Parameters: date, N, amount, length, operation, type
NI SFU (unique) Achievement Field ID Number Date Index (by more up. Same date)

the area (ha) Operation Product Name N (kg)
length 
Example:$ 95023\_8\_15-08-95023 2001\_1\_RY 15-08-2001
 8 1 RY 6.4 Combine Harvesting RYEGRAIN 478.9 29 440 -$


   if $area\ge 0.001$ and writeTuples then
   
   \quad    string farmId=theControlParameters.GetFarmID()
      
   \quad    if $farmId=""$ then
      
   \quad     \quad   err("fieldOrder::AddTuple - attempt to add tuples but no farm ID specified")
       
    \quad   if $ not tuple_file_open$ then
      
      \quad    OpenTupleFile(farmId)
         
   \quad     idx = GetOperationIndex(date)
     
     \quad  $tuple << farmId << "\_" << Index << "\_" << 
 date << "\_" << idx << "_" << History[3].crop << " tab "$
      
      \quad  tuple << farmId << " tab " << Index << " tab "
 << date << " tab " << idx << " tab "
     
     \quad  tuple << History[3].crop << " tab " << area 
<< " tab " << op << " tab " << type << " tab " << n
     
     \quad  tuple << " tab " << amount << " tab " << SFU << endl
  
    \quad   CloseTupleFile()  // Only in field objects
   

\#endif





\subsection{OutputPlantDetails(bool header)}	 

if (header)
	
\quad       f << "date"

\quad       f << " tab crop number tab pClover"

 \quad      for ( i=0 i<MaxPlants i++)
      

\quad  \quad    		f << " tab CropName tab phase tab TempSum tab CO2conc tab GLAI tab YLAI tab DMRoot tab DMRootPL tab DMRootTubers tab DMtop tab DMStorage tab DMTotalTop"
 
\quad  \quad    	     	   f << " tab TranspirationRatio tab IrrigationDemand tab Height tab RootDepth tab TotalRootLength tab InterceptedRadiation"

\quad \quad      	   f << " tab Nitrogen tab N15 tab Nmin() tab Nmax() tab NStorage tab NTop tab NVegTop tab NRoot tab NRootPL tab NRootTubers tab fNitrogen tab fNitrogenCurve"
 
\quad  \quad     	   f << " tab RootDeposit tab RootTrans tab NRootDeposit tab TopDeposit tab NTopDeposit"

\quad \quad          f << " tab TopProd tab NUptake tab Nfix tab NfixAcc tab deltaDM tab N15Root tab AccRootResp"
 
\quad         for(unsigned  i=0 i<globalSettings.RootData.size() i++)


\quad  \quad            f << " tab RootLength-"<< globalSettings.RootData[i].getStartDepth() << "-"<< globalSettings.RootData[i].getEndDepth()

\quad  \quad         	f << " tab RootMass-"<< globalSettings.RootData[i].getStartDepth() << "-"<< globalSettings.RootData[i].getEndDepth()
 
\quad  \quad         	f << " tab RootN-"<< globalSettings.RootData[i].getStartDepth() << "-"<< globalSettings.RootData[i].getEndDepth()
     
	
   else
   
    \quad    f << theTime.GetYear() << "-" << theTime.GetMonth() << "-" << theTime.GetDay()

  \quad  cloneList<patch>::PS P
  
  \quad  thePatches.PeekHead(P)
 
 \quad   while $P\ne NULL$ then
  
 \quad    \quad    P.element.outputCropDetails(f)
   
   \quad   \quad   thePatches.OneStep(P)

 \quad   f << endl



\subsection{StartBudget()}
Initialise budget variables

	if thePatches then
	
   	  \quad for ( i=0 i<thePatches.NumOfNodes() i++)
   	
     \quad  \quad		thePatches.ElementAtNumber(i).StartBudget()


\subsection{EndBudget()}
Test if budget is valid

	bool ret\_val=true
	
   if thePatches then
   
  \quad  	for ( i=0 i<thePatches.NumOfNodes() i++)
   	
\quad \quad \quad 	   	if ( not thePatches.ElementAtNumber(i).EndBudget())

\quad \quad  ret\_val=false
	   	
  \quad \quad  if  not $ret\_val$ then
   
   \quad \quad \quad  warn("Error in budget of field number ", GetIndex())




\subsection{DepositManure(manure animalurine,
 manure animaldung, areaur, areadu)}
(needs rewriting for multiple patches)

	$prop\_lost\_volat=0.07$

 $  AddInd(environmentalIndicator,
"31.08 N in faeces from grazing",$

$"kg N",
animal\_dung.GetAmount()*animal\_dung.GetAllN().n*1000)$
  
 $  AddInd(environmentalIndicator,
"41.09 C in faeces from grazing","kg C",$


$animal\_dung.GetAmount()*animal\_dung.GetC_content()*1000)$

// $new\_area\_dung+=dung.GetAmount()*110$ 

  //OLD METHOD - area covered per tonne fresh weight from During and Weeda 73, NZJAS

$   AddInd(environmentalIndicator,"31.07 N 
in urine from grazing","kg N",$

$animal\_urine.GetAmount()
*animal\_urine.GetAllN().n *1000)$
  
  $ AddInd(environmentalIndicator,"41.08 C 
in urine from grazing","kg C",$

$animal\_urine.GetAmount()
*animal\_urine.GetC_content() *1000)$
   
      // Calculate ammonia volatilisation
 
  $ AddInd(environmentalIndicator,
"31.46 NH3-N loss during grazing","kg N",$

$prop_lost_volat 
* animal\_urine.GetAmount()
 * animal\_urine.GetNH4\_content().n * 1000)$

   size=0
   
   for( i=0 i<thePatches.NumOfNodes() i++)
   
	\quad    size+=thePatches.ElementAtNumber(i).GetArea()
   
   $totalAmountUrine=animal\_urine.GetAmount()$
   
   $totalAmountDung=animal\_dung.GetAmount()$
   
   for( i=0 i<thePatches.NumOfNodes() i++)
   
   	   \quad  $fraction=\tfrac{thePatches.ElementAtNumber(i).GetArea()}{size}$

\quad    	 $  animal\_urine.Setamount(totalAmountUrine*fraction)$

\quad    	$   animal\_dung.Setamount(totalAmountDung*fraction)$
   	   
   	   \quad $ thePatches.ElementAtNumber(i).DepositManure(
animal\_urine, animal\_dung)$
   
   $animal\_urine.Setamount(totalAmountUrine)$

  $ animal\_dung.Setamount(totalAmountDung)$


\subsection{bool GetGrazingPossible()}
//   char *cropId

   bool $ret\_val=true$


	if not grazingPossible then
	
   	\quad  return false
   	
   // Check to make sure there is no other operation that would stop grazing
   
    year = theTime.GetYear()
    
    month = theTime.GetMonth()
    
    day = theTime.GetDay()
    
   fieldOperationFields * FOP=NULL
   
   if crrPlan=NULL  or  crrPlan.Finished() then
   
   \quad    cloneList<plan>::PS P
    
    \quad   Lfutr.PeekHead(P)
  
    \quad   if P then
  
    \quad \quad      FOP=(fieldOperationFields *)
    
     \quad \quad P.element.ReturnOperation(year,month,day,false)
   
   
   else
   
    \quad   FOP=crrPlan.ReturnOperation(year,month,day,false)

   if FOP then
   
    \quad    opID=FOP.GetOperationId()
    
    \quad   switch ()
      
      \quad \quad    case opID in 0-6,12,13,15,16,17,19
        
     \quad \quad \quad        ret\_val=false,
            grazingShutDown=true
    
    \quad \quad      default:         
        ret\_val=true
  
       
   
   
   
	return ret\_val




\subsection{AutoGrazingPossible()}
Set grazable to true if crop is of type that is grazable

   char *cropId

  $ PresentCrop\_(cropId)$
   
   if "C1",cropId  or  "C2",cropId or "G1",cropId  or
   
      "G2",cropId  or  "K1",cropId  or  "G4",cropId then
      
  \quad	   for( i=0 i<thePatches.NumOfNodes() i++)
		
 \quad	 \quad			   thePatches.ElementAtNumber(i).Setgrazable(true)
   
    \quad	 \quad	      grazingPossible=true
      
   else
   
    \quad
	   for( i=0 i<thePatches.NumOfNodes() i++)
	   
	 \quad	  \quad	  	 thePatches.ElementAtNumber(i).Setgrazable(false)

    \quad	 \quad	   grazingPossible=false
   


\section{Get}

\subsection{GiveIndicators()}
   n=thePatches.NumOfNodes()
   
     for all i in 0..n-1 do
   
   \quad   patch  p = thePatches.ElementAtNumber(i)
      
   \quad   p.GiveIndicators()



\subsection{GetHerbageMasskg()}
Get total herbage mass in kg

	mass= 0,	
    n=thePatches.NumOfNodes()
    
     for all i in 0..n-1 do
   
  \quad   patch  p = thePatches.ElementAtNumber(i)
      
 \quad     mass += p.GetPatchStandingDMkg()*p.GetArea()
      
   return mass


\subsection{GetHerbageMasskgPerHa()}
Get herbage mass in kg

	mass= 0, n=thePatches.NumOfNodes()
    
     for all i in 0..n-1 do
   
  \quad    patch  p = thePatches.ElementAtNumber(i)
      
 \quad     mass += p.GetPatchStandingDMkg()*p.GetArea()
      
   return $mass/area$
   
   


\subsection{GetAvailableHerbageMasskg(residualDM)}
Get herbage mass in excess of residual (kgDM/ha),  in kg

	mass= 0,  n=thePatches.NumOfNodes()
    
   for all i in 0..n-1 do
   
  \quad   patch  p = thePatches.ElementAtNumber(i)
      
  \quad    mass += p.GetAvailablePatchStandingDMkg(residualDM)
      
   return mass

\subsection{GetAvailableHerbageMasskgPerHa(residualDM)}
Get herbage mass above cutting height in kg/ha

	mass= 0,  n=thePatches.NumOfNodes()
    
    for all i in 0..n-1 do
   
   \quad   patch p = thePatches.ElementAtNumber(i)
      
  \quad    mass += p.GetAvailablePatchStandingDMkg(residualDM)*p.GetArea()
      
      
   mass/=area
   
   return mass

\subsection{GetDMGrazed()}
ret\_val = 0, n=thePatches.NumOfNodes()

    for all i in 0..n-1 do
   
  \quad    patch  p = thePatches.ElementAtNumber(i)
      
 \quad     ret\_val+=p.GetGrazedDM()

   return ret\_val


\subsection{GetAbovegroundCropN()}
Returns above ground crop N in kg/ha

   ret\_val = 0,
    n=thePatches.NumOfNodes()
    
    for all i in 0..n-1 do
   
 \quad     patch p = thePatches.ElementAtNumber(i)
      
 \quad     ret\_val+=p.GetAbovegroundCropN()
      
   return ret\_val


\subsection{GetDailyDMProduction()}
Returns growth in kg DM

	ret\_val=0
	
   for ( i=0 i<thePatches.NumOfNodes() i++)
   
    \quad  patch p = thePatches.ElementAtNumber(i)
      
  \quad    ret\_val+=p.GetDailyDMProduction()
      
   return ret\_val


\subsection{InitExpectedGrowthRate(radiation, temperature)}
Estimate potential daily growth, given radiation and temperature

   ret\_val = 0,
    n=thePatches.NumOfNodes()
    
     for all i in 0..n-1 do
   
   \quad   patch p = thePatches.ElementAtNumber(i)
      
   \quad   ret\_val+=p.GetPotentialGrowthRate(radiation, temperature)
      
   return ret\_val



\subsection{ClearTemporaryVariables()}
This resets the variables that track grazing
Done here because we might be interested in the variables for debugging

   for ( i=0 i<thePatches.NumOfNodes() i++)
   
  \quad    patch p = thePatches.ElementAtNumber(i)
      
 \quad     p.clearPatchRecords()



\end{document}
