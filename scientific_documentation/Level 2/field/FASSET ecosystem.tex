\documentclass[%parskiphalf,%numbers=noendperiod
]{scrartcl}
\usepackage{amsmath}
\usepackage[latin1]{inputenc}
\usepackage[english]{babel}
\usepackage[T1]{fontenc} 
\usepackage{graphicx,parskip}
\usepackage{booktabs,longtable}
\usepackage{lmodern}
\usepackage[round]{natbib}

\newcommand\mymarginpar[1]{\marginpar {\flushleft\bfseries\scriptsize #1}}

\title{FASSET Ecosystem}

\author{}
\date{\today}

\usepackage{lmodern}


\usepackage[backref]{hyperref}

\begin{document}


\maketitle
\tableofcontents
\newpage

\section{CropSoilExchange}

\subsection{CheckTranspirationRatio(TranspirationRatio)}
Error check for transpirationratio

if TranspirationRatio<-1E-10 then
    
\quad warn("CheckTranspirationRatio - transpirationratio can not be negative"), TranspirationRatio=0
   
if TranspirationRatio>1+1E-10 then
      
\quad warn("CheckTranspirationRatio - transpirationratio can not be greater than one"), TranspirationRatio=1
   
\subsection{CheckNfromSoil(nitrogen NFromSoil,  NitrogenDemand)}

Error check for N from soil

if (NFromSoil.n>(NitrogenDemand+1e-6)) then
   
\quad    warn("CheckNfromSoil - nitrogen from soil should not exceed request")
    
\quad   NFromSoil.n=NitrogenDemand

if NFromSoil.n<-0.000000001 then

  \quad warn("CheckNfromSoil - nitrogen from soil can not be less than zero"), NFromSoil.n=00


\subsection{CropSoilExchange()}
\citet{berntsen2004modelling}
Handles information exchange between soil and crop objects

\begin{tabular}{ll}
   temperature &       mean air temperature, Celcius \\
   precip  &           precipitation, mm \\
   radiation    &      global radiation, MJ/m2 \\
   Epot   &            potential evaporation, mm
\end{tabular}

     precip = theClimate.precip,      
     EPot = theClimate.epot,      
     temp = theClimate.tmean,      
     radiation = theClimate.rad

   linkList<rootStructure>  roots = new linkList<rootStructure>
   
   cloneList<crop>::PS aCropElement,   
   rootStructure  cropRoot,  
   crop aCrop
   
   nFixThisDay=00

   \textbf{1. Add irrigation nitrogen and atmospheric nitrogen deposition to soil}

     NitrogenInRain = theClimate.GetNitrogenInRain()
    
 irrigation = irrigationWater.GetAmount()/10         // converts m3 to mm
    
$ irrigationN = irrigationWater.GetN\_content()$
  
 nitrogen NinWater,N1
   
NinWater.SetBoth($precip \cdot NitrogenInRain+irrigation \cdot irrigationN$,0)
 
  aSoil.AddNutrient(N1,N1,N1,NinWater)
   
WaterBudget.AddInput(precip+irrigation)
  
 AddInd(envInd,"70.01 Rainwater","cubic metre",
$precip \cdot area \cdot 10$)
   
AddInd(envInd,"31.02 N from deposition","kg N",
$NitrogenInRain \cdot precip \cdot area \cdot 10$)
  
 AddInd(envInd,"31.05 N from irrigation water","kg N",
$irrigation \cdot irrigationN \cdot area \cdot 10$)

   if globalSettings.DetailsData.getWaterDetails() then
	
	  \quad   waterDetails.WriteWithTabS(theTime.GetString2())
	 
	  \quad   waterDetails.WriteWithTab(precip+irrigation)
 
  \quad	   waterDetails.WriteWithTab(EPot)

   \textbf{2. Sum leaf areas and interception capacity, calculate
   "effective" extinction coeff.}

   totalgLAI=
     totalyLAI=
     InterceptionCapacity=
     k=0
     
   theCrops.PeekHead(aCropElement)

   while $aCropElement\ne NULL$ do
   
  \quad 	totalgLAI            += aCropElement.element.GiveLAI()
   	
\quad totalyLAI            += aCropElement.element.GiveYellowLAI()
   	
\quad InterceptionCapacity += aCropElement.element.InterceptionCapacity()
     
 \quad $k                    += aCropElement.element.GiveExtCoeff() \cdot aCropElement.element.GiveLAI()$
      
theCrops.OneStep(aCropElement)
   
   
// k calculated according to Ross et al. 1972. J. Appl. Ecol.: 535-556

if $totalgLAI>0$ then
     $k=\tfrac{k}{totalgLAI}  $
   else
  $k=0.44$
  
 // Convey information about total LAI to all crops
  
 theCrops.PeekHead(aCropElement)
  
 while $aCropElement\ne NULL$ then
   
   
\quad     aCropElement.element.SetTotalLAIStand(totalgLAI+totalyLAI)
   
\quad     theCrops.OneStep(aCropElement)
   

\#ifndef $\_\_ECOSYSTEM\_\_$
  
 if Month=6  and  Day=15  and  totalgLAI+totalyLAI<1E-10 then
   
  \quad      cout << endl << " Object path : " << GetLongName()

  \quad      err("CropSoilExchange - no leaf area present 15th June")
   
\#endif

   \textbf{3. Compute corrected Epot for canopy}

     $EpotLeaves = EPot \cdot (1-fracEPotToSoil(totalgLAI,totalyLAI))$
    
 correctedEPotLeaves= weightedMeanEvapFactor=0
   
theCrops.PeekHead(aCropElement)
  
 while $aCropElement\ne NULL$
  
 \quad    
 aCrop = aCropElement.element
    
  \quad if $totalgLAI+totalyLAI)>0$ then
      
  \quad  \quad $ correctedEPotLeaves += 
EpotLeaves \cdot aCrop.GiveEvapFactor() \cdot \tfrac{aCrop.GiveLAI()
+aCrop.GiveYellowLAI()}{totalgLAI+totalyLAI}$
     
 \quad $weightedMeanEvapFactor += aCrop.GiveEvapFactor() \cdot aCrop.GiveLAI()$
     
 \quad theCrops.OneStep(aCropElement)
  
  
 if $totalgLAI>0$ then     
   $weightedMeanEvapFactor=\tfrac{weightedMeanEvapFactor}{totalgLAI}$

    \textbf{4. Update canopy water interception and eventual snow layer}

     surplus = precip + irrigation
   
  greenLeafEpot = correctedEPotLeaves
   
  CO2conc = theClimate.GetCO2Concentration()
  
 if $fracEPotToSoil(totalgLAI,totalyLAI)<1$ then
    
    \quad $	 greenLeafEpot =
 CO2ConcentrationFactor(CO2conc) \cdot correctedEPotLeaves$

\quad\quad $ \cdot \tfrac{fracEPotToGL(totalgLAI)}{1-fracEPotToSoil(totalgLAI,totalyLAI)}$
   
   \quad	 $soilEpot = EPot \cdot 
fracEPotToSoil(totalgLAI,totalyLAI)$
   
   \quad	 greenInterceptEvap=0
 
  UpdateInterception($surplus$,
                		 $temp$,
			 	          $radiation$,
                      $greenLeafEpot$,
			 			   $ correctedEPotLeaves \cdot fracEPotToYL(totalgLAI,totalyLAI)$,
			 			    $soilEpot$,
                      $InterceptionCapacity$,
                     $ greenInterceptEvap$)

    \textbf{5. Collect root and water demand parameters}

   theCrops.PeekHead(aCropElement),    
 q=0
 
  while $aCropElement\ne NULL$ do
        
  \quad aCrop = aCropElement.element
    
   \quad cropRoot = new rootStructure
    	
 \quad aCrop.GetStatus (cropRoot.rootRadius,cropRoot.rootpF, 

 \quad  \quad cropRoot.NitrogenDemand,cropRoot.NitUptakeRate, cropRoot.AmmUptakeRate, 

 \quad  \quad cropRoot.MinimumSoilNit, cropRoot.MinimumSoilAmm)
    
   \quad cropRoot.transpirationDemand = 0
    
   \quad  if $totalgLAI>0$ then
       
 \quad     \quad  $cropRoot.transpirationDemand = 
greenLeafEpot \cdot \tfrac{aCrop.GiveEvapFactor()}{weightedMeanEvapFactor}
\cdot \tfrac{aCrop.GiveLAI()}{totalgLAI}$
     
  \quad  cropRoot.rootLengthList = aCrop.GiveRootLengthList()
     
  \quad  roots.InsertLast(cropRoot)
     
  \quad  theCrops.OneStep(aCropElement)

   \quad     q++
   

    \textbf{6. Update soil}

	 aSoil.Update(surplus,temp,soilEpot,snowContent,roots)
  	
aSoil.Incorporate(10,0.02,false,"none")   
   // Daily transfer of org. matter to top soil layer (10 mm)

   \textbf{7. Update crop}

\citep[3.3. Competition for light]{berntsen2004modelling}

   theCrops.PeekHead(aCropElement)
  
 const  CanopyIntervals = 25          // canopy divided into 25 height intervals
    
 %CropRadiation[MaxPlants]
     
MaxCanopyHeight = 0

   for ( i=0 i<theCrops.NumOfNodes() i++)
   
  \quad    	CropRadiation[i] = 0

  \quad   	MaxCanopyHeight = max(MaxCanopyHeight,theCrops.ElemAtNum(i).GiveCropHeight())
   

  %   CanopyPar[CanopyIntervals+1]

     $CanopyIntervalHeight = \tfrac{MaxCanopyHeight}{CanopyIntervals}$

   CanopyPar[CanopyIntervals] = radiation

   for ( i=CanopyIntervals-1 $i\ge 0$  i--)
   
\quad         a = 0

   \quad     $ IntervalStart = i \cdot CanopyIntervalHeight$

 \quad      $  IntervalEnd = MaxCanopyHeight-i \cdot CanopyIntervalHeight$

 \quad      for ( j=0 j<theCrops.NumOfNodes() j++)
    
%\quad \quad  //          const crop  aCrop = theCrops.ElemAtNum(j)   //Eclipse complains about the 'const' NJH, feb 2009

 \quad\quad         crop  aCrop = theCrops.ElemAtNum(j)
 
\quad \quad      	$a += aCrop.TotalLAI(IntervalStart,IntervalEnd) \cdot  (aCrop.GiveExtCoeff())
    $
     
  \quad    $ CanopyPar[i] = radiation \cdot exp(-a)$
   



   for ( i=CanopyIntervals-1 $i\ge 0$  i--)
    
    \quad	    for ( j=0 j<theCrops.NumOfNodes() j++)
      
      \quad	     \quad	    k = theCrops.ElemAtNum(j).GiveExtCoeff()

         \quad	   \quad	  $ LAIinInterval = theCrops.ElemAtNum(j).GreenLAI($

\quad	   \quad $i 
\cdot CanopyIntervalHeight,CanopyIntervalHeight)$
         
  \quad	\quad	  CropRadiation[j] += $CanopyPar[i] \cdot k \cdot LAIinInterval$
   	
   

 \textbf{8. Transfer nitrogen and water information to crop}

   nitrogen   NitrogenUptake  = aSoil.GetNitrogenUptake()
   
     TotalDMStorage = 
     TotalDMVegTop = 0
     
   for ( i=0 i<roots.NumOfNodes() i++)
  
  \quad    rootStructure   aRoot = roots.ElemAtNum(i)
   
\quad 	aCrop = theCrops.ElemAtNum(i)
   
\quad 	  TranspirationRatio = 10
   
\quad    if aRoot.transpirationDemand>0 then
   
 \quad \quad   	$TranspirationRatio = \tfrac{greenInterceptEvap+aRoot.actualTranspiration}{greenInterceptEvap+aRoot.transpirationDemand}$
  
 \quad \quad    CheckTranspirationRatio(TranspirationRatio)
    	
    	
  \quad    NFromSoil.n=
      NFromSoil.n15=0

  \quad 	if NitrogenUptake then
	   
  \quad    	  \quad	 NFromSoil = NitrogenUptake[i]

  \quad    	  \quad	 CheckNfromSoil(NFromSoil,aRoot.NitrogenDemand)

  \quad    	  \quad	 AddInd(envInd,"32.10 N uptake from soil","kg N",
$NFromSoil \cdot area \cdot 10$)
      
  \quad    aCrop.NitrogenAndTranspiration(NFromSoil,TranspirationRatio)
   
 \quad  TotalDMStorage += aCrop.GiveDMStorage()
    
 \quad TotalDMVegTop += aCrop.GiveDMVegTop()
	

  \textbf{9. Upate crop growth, 
calculate C and N from litterfall, N from fixation plus approximate Carbon content in plant}
   
  TotalCFixation = 0
   
for ( j=0 j<theCrops.NumOfNodes() j++)
   
  \quad     aCrop = theCrops.ElemAtNum(j)
 
  \quad    aCrop.SetDMVegTopTotalStand(TotalDMVegTop)
  
 \quad   $ TotalCFixation += 0.42 \cdot aCrop.Update(CropRadiation[j])$

   \quad    decomposable    rootMatter = aCrop.GetRootMatter()
  
\quad     if rootMatter then 
            	 TransferRootMatter(rootMatter,aCrop.GiveRootLengthList())
  
 \quad       aCrop.GetRootMatter().Setamount(0)
  
  %\quad      // aCrop.DeleteRootMatter()  !!!???@@@
      

   \quad    decomposable   topMatter = aCrop.GetTopMatter()
  
 \quad    if topMatter then
      
   \quad         \quad	 aSoil.AddDecomposable(topMatter)

 \quad   	    \quad	 AddInd(envInd,"32.22 N from leaf litterfall incorporated into soil","kg N",
$topMatter.GetAllN() \cdot topMatter.GetAmount() \cdot area \cdot 1000$)
		
 \quad  \quad	   AddInd(envInd,
"40.82 C from leaf litterfall incorporated into soil","kg C",
$topMatter.GetAmount() \cdot topMatter.GetC_content() \cdot area \cdot 1000$)
      
  \quad     AddInd(envInd,"31.04 N from fixation","kg N",
$aCrop.GetNFixationThisDay() \cdot area \cdot 10$)
 
 \quad     nFixThisDay+=aCrop.GetNFixationThisDay()
   

  
AddInd(envInd,"40.01 Net crop carbon fixation","kg C",
$TotalCFixation \cdot area \cdot 10$)

   \textbf{10. Remove dead crops from list}
   
theCrops.PeekHead(aCropElement)
   
while $aCropElement\ne NULL$ then
   
      \quad	  aCrop = aCropElement.element
    
\quad	    if aCrop.ReadyForRemoval()  and  theCrops.NumOfNodes()>1 then
      
       \quad	  \quad	     cloneList<crop>::PS P=aCropElement
        
 \quad	  \quad	   theCrops.OneStep(aCropElement)
       
   \quad	  \quad	  theCrops.Remove(P)
      
     \quad	   else
       
   \quad	  \quad	 theCrops.OneStep(aCropElement)
   

	if autoGraze then SimpleGraze() // Simple grazing developed by JBE+BMP
     


   /*  if FOP.GetOperationId()=MakeGrazable then
   
   	\quad $makeGrazable    graz = (makeGrazable  \cdot ) FOP$

   	\quad Setgrazable(graz.GetmakeItPossible()   

   */

   // N balance calculation

	$nitrogen N\_leached = aSoil.GetNitLeaching(aSoil.GetMaxDepth())$

$+aSoil.GetAmmLeaching(aSoil.GetMaxDepth())$
   
$nitrogen N\_den     = aSoil.GetDenitrification()$
  
 $nitrogen N\_runoff  = aSoil.GetRunOff\_N()$
   
$WaterBudget.AddOutput(aSoil.GetDrainage(aSoil.GetMaxDepth())$

$+aSoil.GetEvapotranspiration()+aSoil.GetRunOff())$

   delete roots
   
irrigationWater.Setamount(0)
   // Water calculation
     
soilEvaporation = aSoil.GetSoilSurface().GetEvaporation()
    
 totalEvaporation = aSoil.GetEvapotranspiration()

   if globalSettings.DetailsData.getWaterDetails() then
  
\quad  	   waterDetails.WriteWithTab(soilEvaporation)

\quad  	   waterDetails.WriteWithTab(totalEvaporation-soilEvaporation)

\quad  	   waterDetails.WriteWithTab(aSoil.GetDrainage(aSoil.GetMaxDepth()))

\quad  	   waterDetails.WriteNewLine()

   EndBudget()







\section{Ecosystem}

\subsection{ecosystem(const char 
 aName, const aIndex, const base aOwner)}   
:base(aName,aIndex,aOwner)

	 run=globalSettings.getCurrentRun()

	if $scenarielogNumber\ne run$ then
	
	  \quad		 currentDir=scenarielog.getCurrentPath()

	  \quad		 scenarielog.changeDir(globalSettings.getOutputDirectory())
		 
  \quad	 waterDetails.changeDir(globalSettings.getOutputDirectory())
		
  \quad	if $scenarielogNumber\ne -1$ then	

   \quad	  \quad	scenarielog.closeFile(), 	

  \quad	waterDetails.closeFile()
		
   \quad	scenarielog.openFileWriteString("scenarie.txt")
		
  \quad	 waterDetails.openFileWriteString("waterDetails.txt")
		
 \quad	scenarielog.changeDir(currentDir)
		
  \quad	 scenarielogNumber=run
	
	
	aSoil                   = new soil("Soil",aIndex,this)
	
	theCrops                = new cloneList<crop>
   
   cutPlantProducts        = new linkList<plantItem>
  
   crop    p                = new 
   
   cropbare("cropbare",0,this,"NoCrop")
  
   theCrops.InsertLast(p)
   
   irrigationWater.Setamount(0)
	
	k1                      = 0.60 
 // Ext.coeff. for leaf area, evaporation frac.


   tempSnowFall1 = -20,
   tempSnowFall2 = snowMeltRate1 = 20

   snowMeltRate2 = 0.15,  leachingDepth = 1000

   autoGraze               = false

   fingerFlow              =
   balanceDepth            =
    area                    =
   waterContent            =
   snowContent             =
   grazingEffiency         =
   nLoad                   =
    totalgLAI               =
	$totalGrazed\_DM$				= 0

   WaterBudget.SetNames("ecosystem","Water")


\subsection{ecosystem(const ecosystem p)}
Copy constructor.

 	aSoil=new soil(  p.aSoil)
   
   aSoil.SetOwner(this)
  
   theCrops = NULL
  
   if (p.theCrops)  theCrops=p.theCrops.clone()
   
   cutPlantProducts = NULL
  
   if (p.cutPlantProducts) then

   \quad  	for ( i=0 i<p.cutPlantProducts.NumOfNodes() i++)
      	
\quad  \quad   cutPlantProducts.InsertLast(p.cutPlantProducts.ElemAtNum(i))
  
   k1                  = p.k1

   area                = p.area

   irrigationWater     = p.irrigationWater

   waterContent        = p.waterContent,
   snowContent         = p.snowContent

   tempSnowFall1       = p.tempSnowFall1,
   tempSnowFall2       = p.tempSnowFall2

   snowMeltRate1       = p.snowMeltRate1,
   snowMeltRate2       = p.snowMeltRate2

   WaterBudget         = budget(p.WaterBudget)

   WaterBudget.Reset()   // History ignored

   balanceDepth        = p.balanceDepth

   leachingDepth       = p.leachingDepth

   fingerFlow          = p.fingerFlow

   totalgLAI	  = p.totalgLAI    //added by NJH 16.3.05


   if (theCrops) then // Set soil pointer in each crop
   	
\quad  for ( i=0 i<theCrops.NumOfNodes() i++)
	   	
\quad  \quad   theCrops.ElemAtNum(i).SetSoilPointer(aSoil)

	$totalGrazed\_DM				= 0$


%\subsection{~ecosystem()}
%Destructor. scenarielog.closeFile(), waterDetails.closeFile()
%
%	delete aSoil

%   if (theCrops) delete theCrops

%   if (cutPlantProducts) delete cutPlantProducts

\subsection{ecosystem  clone() const}
Returns a copy of this instance at the same inheritance level.
Used in connection with 'cloneList'

	$ecosystem   p = new ecosystem(this)$, 
	return $p$


\subsection{fPAR(k,gLAI)}	
return $1-exp(-k \cdot gLAI)$




\subsection{CO2ConcentrationFactor(  CO2conc)}
Calculates relative potential evaporation from green leaves.
Returns 1 at ppmCO2=377

   return $(1.0+0.00029 \cdot 377)-0.00029 \cdot CO2conc$






\subsection{KillAllCrops()}
Improve terminate to return products!!!
Look upon indicator update!!!!

    RootLengthList=NULL

   cloneList<crop>::PS aCropElement

   theCrops.PeekHead(aCropElement)

   while $aCropElement\ne NULL$ do
 
      aCrop = aCropElement.element

      if (aCrop) and
              (!aCrop.IsBareCrop())
         
            if (cutPlantProducts.NumOfNodes()>0)
        
 \quad    
  warn("ecosystem-KillAllCrops:: Cut plant products should not be present")
            
decomposable   Root = new decomposable
          
  plantItem   Straw = new plantItem
           
 aCrop.Terminate(Straw,Root,RootLengthList)
          
  if (Straw)
          
              \quad  $Straw.Setamount(0.01 \cdot Straw.GetAmount())$  // Convert from g/m2 to t/ha
            
 \quad   aSoil.AddDecomposable(Straw)
          
    \quad  AddInd(envInd,"32.21 N from straw incorporated into soil","kg N",
$Straw.GetAllN() \cdot Straw.GetAmount() \cdot area \cdot 1000$)
            
  \quad    $C_Amount = Straw.GetAmount() \cdot Straw.GetC_content() \cdot area \cdot 10000$
            
\quad    AddInd(envInd,"40.81 C from straw incorporated into soil","kg C",$C\_Amount$)
 
\quad //               AddInd(envInd,"40.99 Plant top res. C incorp. in org. matt.","kg C",$C\_Amount$)
              
\quad  delete Straw
              \quad  Straw = NULL
           
            if $Root.GetAmount()>0$ then

            \quad 	 TransferRootMatter(Root,RootLengthList)
          
   \quad  \quad  delete Root
      
      


 \quad  theCrops.Remove(aCropElement)
   


   aCrop=new cropbare("cropbare",0,this,"NoCrop")

   theCrops.InsertLast(aCrop)

   if $theCrops.NumOfNodes()\ne 1$ then

   \quad     err("KillAllCrops - there should only be one (bare) crop here")









\subsection{SetGrazed(grazingheight,DMGrazed)}
Sets the crop variables that record the material grazed
Called by each animal that grazes this ecosystem

     grazableDMThisCrop, grazableNThisCrop
     grazableDM= grazableN= grazedN=  0
     
$grazedC = DMGrazed  \cdot  0.46 $    //Here lies a well buried little fudge!!!
   
for( i=0 i<theCrops.NumOfNodes() i++)

    \quad 	crop   aCrop=theCrops.ElemAtNum(i)

	 \quad 	grazableDM += aCrop.GetAvailableDM(0), grazableN += aCrop.GetAvailableN(0)
  
	if (grazableDM>0)

      for( i=0 i<theCrops.NumOfNodes() i++)
    
 \quad 	   	crop    aCrop=theCrops.ElemAtNum(i)

    \quad       grazableDMThisCrop = aCrop.GetAvailableDM(0)

     \quad      grazableNThisCrop = aCrop.GetAvailableN(0)

      \quad       $grazedDMThisCrop = DMGrazed  \cdot  grazableDMThisCrop/grazableDM$
  
\quad        $ aCrop.SetGrazed(grazing\_height, grazedDMThisCrop)$
     
\quad      if (grazableDMThisCrop>0) // Added BMP
  
\quad     \quad         $grazedN+=grazedDMThisCrop  \cdot  grazableNThisCrop/grazableDMThisCrop$

   
$totalGrazed_DM+=DMGrazed$

   AddInd(envInd,"31.23 N removed by cattle","kg N",$grazedN \cdot area \cdot 10$)
   
AddInd(envInd,"40.23 C removed by cattle","kg C",$grazedC \cdot area \cdot 10$)



\subsection{Add(ecosystem  addPatch, frac)}

Add's two patches together

   if (soilManure)
   
    \quad     $ soilManure= \cdot soilManure \cdot (1-frac)$
       
   if (addPatch.soilManure)       
   $ (addPatch.soilManure)=  (addPatch.soilManure) \cdot frac$
  
  
   if (soilManure  and  addPatch.soilManure)
       $ soilManure=  soilManure+ \cdot (addPatch.soilManure)$
 
 
   if (!soilManure  and  addPatch.soilManure)
       soilManure=  (addPatch.soilManure)
  
  
   if (cropManure)
       $ cropManure= cropManure \cdot (1-frac)$
  
  
   if (addPatch.cropManure)
    $  (addPatch.cropManure)=  (addPatch.cropManure) \cdot (1-frac)$
  
  
   if (cropManure  and  addPatch.cropManure)
        cropManure=  cropManure+ addPatch.cropManure
  
  
   if (!cropManure  and  addPatch.cropManure)
        cropManure=  (addPatch.cropManure)

  $ irrigationWater=irrigationWater \cdot (1-frac)$

  $ addPatch.irrigationWater=addPatch.irrigationWater \cdot frac$

   irrigationWater=irrigationWater+addPatch.irrigationWater

   if (theCrops)

   	 \quad  for ( i=0 i<theCrops.NumOfNodes() i++)

	  \quad  \quad    	 theCrops.ElemAtNum(i).Add(addPatch.GetCrops().ElemAtNum(i),frac)
  
  
   if (cutPlantProducts or addPatch.cutPlantProducts)

  \quad 	theMessage.Warning("Patch::Add - there should not be straw on ecosystem")
  
  
   if (cutPlantProducts)
     
 \quad   for ( i=0 i<cutPlantProducts.NumOfNodes() i++)
      
         \quad  \quad    $(cutPlantProducts.ElemAtNum(i))=  (cutPlantProducts.ElemAtNum(i)) \cdot (1-frac)$
  
  
   if (addPatch.cutPlantProducts)

    \quad 	for ( i=0 i<addPatch.cutPlantProducts.NumOfNodes() i++)
      

    \quad    \quad     $ addPatch.cutPlantProducts.ElemAtNum(i)= $

\quad  \quad  \quad  $addPatch.cutPlantProducts.ElemAtNum(i) \cdot frac$
      
 \quad  \quad  	 cutPlantProducts.InsertLast(addPatch.cutPlantProducts.ElemAtNum(i))
      
 
  aSoil.Add(addPatch.GetSoil(),frac)
   
$waterContent = (1-frac) \cdot waterContent+frac \cdot addPatch.waterContent$
   
$snowContent = (1-frac) \cdot snowContent+frac \cdot addPatch.snowContent$
   
$tempSnowFall1 = (1-frac) \cdot tempSnowFall1+frac \cdot addPatch.tempSnowFall1$
  
 $tempSnowFall2 = (1-frac) \cdot tempSnowFall2+frac \cdot addPatch.tempSnowFall2$
  
 $snowMeltRate1 = (1-frac) \cdot snowMeltRate1+frac \cdot addPatch.snowMeltRate1$
 
  $snowMeltRate2 = (1-frac) \cdot snowMeltRate2+frac \cdot addPatch.snowMeltRate2$
   
$totalgLAI = (1-frac) \cdot totalgLAI+frac \cdot addPatch.totalgLAI$
   
$leachingDepth = (1-frac) \cdot leachingDepth+frac \cdot addPatch.leachingDepth$
  
 $balanceDepth = (1-frac) \cdot balanceDepth+frac \cdot addPatch.balanceDepth$
  
 $k1 = (1-frac) \cdot k1+frac \cdot addPatch.k1$
  
 if (!overlap=addPatch.overlap) then
   	overlap = false
 
 
   if (!dung=addPatch.dung) then
   	dung = false
  
  
   if (!grazable=addPatch.grazable) then
   	grazable = false
  
  
   if (!fingerFlow=addPatch.fingerFlow) then
   	fingerFlow = 0
 
 
   WaterBudget.Add(addPatch.WaterBudget,frac)
  
   area += addPatch.GetArea()
 
   birthday.SetTime(theTime.GetDay(),theTime.GetMonth(),theTime.GetYear())





\subsection{SimpleGraze()}
Developed by JBE + BMP

   crop  aCrop=NULL
   
   if ($theCrops.NumOfNodes()=0$)
   
    \quad   err("Graze - attempt to graze on field without plants")

     totalDM= density=0

   cloneList<crop>::PS aCropElement

   theCrops.PeekHead(aCropElement)

   while ($aCropElement\ne NULL$)

    \quad   aCrop = aCropElement.element,      
      totalDM+=aCrop.GiveDMVegTop()
      
     \quad  $density+=aCrop.GetBulkDensity() \cdot aCrop.GiveDMVegTop()$
      
    \quad   theCrops.OneStep(aCropElement)

   height=0, frac=10, cuttingHeight = 0.09
   
   if (totalDM>0)
   
    \quad   $density=\tfrac{density}{totalDM}$  		// Obtain the weighted average
    
    \quad   $height=\tfrac{totalDM}{density}  $		// Height in m
   
    \quad   $frac=\tfrac{cuttingHeight}{height}$
   
   else
   
    \quad   density=00
    
   if (frac>1) frac=10

   \quad   Nremove = 0            // used to simulate heifers

   theCrops.PeekHead(aCropElement)
   
   while $aCropElement\ne NULL$ do
   
      plantItem   Hay = new plantItem
   
   aCrop = aCropElement.element
      aCrop.Graze(Hay,frac)
        Wastefrac = 0.2                    // This is a "harvest" waste, covering a number of different effects. Improve in later versions!
      if (Hay.GetAmount()>0)
      
    \quad      $ plantItem   Waste = new plantItem( \cdot Hay)$
    
 \quad     $ Nremove += (Hay.GetAllN() \cdot Hay.GetAmount()).n \cdot (1-Wastefrac)  $    // unit g-N m-2
   
\quad     	$Hay.Setamount(Hay.GetAmount() \cdot 0.01 \cdot (1-Wastefrac) \cdot area) $     // Convert t
      
\quad     AddInd(envInd,"31.23 N removed by cattle","kg N",$Hay.GetAllN() \cdot Hay.GetAmount() \cdot 1000$)
      
\quad     AddInd(envInd,"40.23 C removed by cattle","kg C",$Hay.GetC_content() \cdot Hay.GetAmount() \cdot 1000$)

     
\quad      Waste.Setamount($Waste.GetAmount() \cdot Wastefrac \cdot 0.01$)         // Convert t/ha
   
  \quad      aSoil.AddDecomposable(Waste)
 
  
\quad  	   AddInd(envInd,"20.53 Plant top res. N incorp. in org. matt.","kg N",$Waste.GetAllN() \cdot Waste.GetAmount() \cdot area \cdot 1000$)

\quad   //		   AddInd(envInd,"40.99 Plant top res. C incorp. in org. matt.","kg C",$Waste.GetAmount() \cdot Waste.GetC_content() \cdot area \cdot 1000$)
      


   	delete Hay

      theCrops.OneStep(aCropElement),    NitrogenLoad = nLoad

   if (NitrogenLoad=0)
   	$NitrogenLoad = grazingEffiency \cdot Nremove $             // assumes that standard 84% of heifers is excreted

   if (NitrogenLoad>0)
   
  \quad	  NH4Loss = 0.15
  
\quad			manure cowManure

	  \quad		cowManure.Setname("CATTLE-SLURRY-FRESH")
     
   \quad	theProducts.GiveProductInformation(cowManure)
   
 \quad	    cowManure.Setamount($area \cdot \tfrac{NitrogenLoad}{cowManure.GetAllN().n/100}$)                          // Convert g/m2 to t
 
   \quad	    theProducts.SubtractProduct(cowManure)

     \quad	   $nitrogen LostNH4 = cowManure.GetNH4_content() \cdot cowManure.GetAmount() \cdot NH4Loss  $                                      // assumes a loss of 15%
     
 \quad	 $ cowManure.SetNH4_content(cowManure.GetNH4_content() \cdot (1-NH4Loss))$

      \quad	  AddInd(envInd,"31.46 Ammonia-N loss during grazing","kg N",$LostNH4 \cdot 1000$)
      
\quad	  AddInd(envInd,"31.09 Nitrogen from grazing animals","kg N",
                             $ cowManure.GetAllN() \cdot cowManure.GetAmount() \cdot 1000.0+LostNH4 \cdot 1000$)
    
  \quad	 AddInd(envInd,"41.10 C from grazing animals","kg C",
                              $cowManure.GetC_content() \cdot cowManure.GetAmount() \cdot 1000$)
   
  \quad	   AddInd(economicIndicator,"10.21 Manure spread on field","t",cowManure.GetAmount())

      \quad	  decomposable d=(decomposable) cowManure
   
   \quad	  d.Setamount(d.GetAmount()/area)                                                               // convert back to t/ha
	 
 \quad		aSoil.AddDecomposable(d)
 



\subsection{clearGrazingRecords()}	

 $totalGrazed\_DM=0$

   for( i=0 i<theCrops.NumOfNodes() i++)
	
	 \quad	 (theCrops.ElemAtNum(i)).ClearTemporaryVariables()


\subsection{Initialise(char soilFileName,theIndex,anArea)}	

Index = theIndex

  	aSoil.Initialize(soilFileName),  balanceDepth=aSoil.GetMaxDepth()
   
   area = anArea
   
   StartBudget()


\subsection{SetirrigationWater(water  aWater)
}	irrigationWater=  aWater

   if fingerFlow=1 then
         irrigationWater.Setamount($\tfrac{3}{2} \cdot irrigationWater.GetAmount()$)
				
// This part of the field receives an excess of water
  
 if fingerFlow=2 then
         irrigationWater.Setamount(0)
				
// This part of the field receives no water




\subsection{Sow(char   cropID,   SeedDensDryMatt,  RelativeDensity,  NitrogenInSeed,   CarbonInSeed)
}
//NJH May 2009, cropName allows variety to be passed via seed name for specific operations

   crop   aCrop=NULL

   if ((theCrops.NumOfNodes()+1)>MaxPlants)
  
      \quad  //cout << "Field " << Owner.GetIndex() << endl

     \quad   err("Sow - Too many crops planted")
 
   cloneList<crop>::PS aCropElement

   theCrops.PeekHead(aCropElement)

   while $aCropElement\ne NULL$ do
    // Remove all bare "crops"

   \quad     aCrop = aCropElement.element

 \quad       if (aCrop.IsBareCrop() or aCrop.GetTerminated())

   \quad  \quad then       theCrops.Remove(aCropElement)

    \quad   \quad    else
      theCrops.OneStep(aCropElement)
   


	aCrop = AllocateCrop(cropID)  
   //changed NJH May 2009 to allow seed name to determine crop variety
	
 aCrop.Sow(SeedDensDryMatt,RelativeDensity,NitrogenInSeed)
	
theCrops.InsertLast(aCrop)

\#ifdef NITROSCAPE

       $ N=NitrogenInSeed \cdot area  \cdot  100$
     
 AddInd(envInd,"31.03 N from seed","kg N",N)
      
AddInd(envInd,"40.03 C from seed","kg C",$CarbonInSeed \cdot area \cdot 10$)

\#endif




\subsection{Gathering(harvestFields   hrv)}
   if (cutPlantProducts.NumOfNodes()>0)
   {
      for ( i=0 i<cutPlantProducts.NumOfNodes() i++)  //cutPlantProducts in tonnes/ha
      {

//         plantItem   aCutProduct = new plantItem(cutPlantProducts.ElemAtNum(i))
     
    if (i>3)
      
      err("Gathering - too many harvested products")
       
  $plantItem \cdot  aCutProduct = cutPlantProducts.ElemAtNum(i)$
        
   $Noutput = aCutProduct.GetAllN().n \cdot aCutProduct.GetAmount() \cdot area \cdot 10000$
       
  $  Coutput = aCutProduct.GetAmount() \cdot aCutProduct.GetC_content() \cdot area \cdot 10000$
       
  aCutProduct.Setamount($aCutProduct.GetAmount() \cdot area$)
        
 if (hrv.GetOperationId()=GatheringBales)
         {
      
      AddInd(envInd,"31.21 N removed in straw","kg N",Noutput)
     
       AddInd(envInd,"40.21 C removed in straw","kg C",Coutput)

\#ifdef TUPLE
           
 if (((field  )Owner).GetWriteTuples())
  
  \quad            ((field  )Owner).AddTuple(theTime.GetString2(),-Noutput

,$aCutProduct.GetAmount() \cdot 10.0$,"HARVEST","$SECONDARY\_PRODUCT$",0)

\#endif
           
 if i=0  // and !hrv.GetStrawProduct() //NICK CHECK DETTE (MEL)!!
            
          \quad  then    hrv.SetStrawProduct(aCutProduct)
            
          \quad     else   hrv.AddStrawProduct(aCutProduct)
        
       
  if (hrv.GetOperationId()=GatheringCereal)
      
       
     if (i=0)   //is grain
           

         		 AddInd(envInd,"31.20 N removed in grain","kg N",Noutput)
              
 AddInd(envInd,"40.20 C removed in grain","kg C",Coutput)

\#ifdef TUPLE
             
  if (((field)Owner).GetWriteTuples())
                
  ((field)Owner).AddTuple(theTime.GetString2(),-Noutput,

$aCutProduct.GetAmount() \cdot 10.0$,"HARVEST","$MAIN\_PRODUCT$",0)

\#endif
             
  hrv.SetMainProduct(aCutProduct)
            
          
  if (i=1) then  //is straw
           
            
   AddInd(envInd,"31.21 N removed in straw","kg N",Noutput)
              
 AddInd(envInd,"40.21 C removed in straw","kg C",Coutput)

\#ifdef TUPLE
             
  if (((field  )Owner).GetWriteTuples())
                
  ((field )Owner).AddTuple(theTime.GetString2(),-Noutput,$aCutProduct.GetAmount() \cdot 10.0$

,"HARVEST","$SECONDARY\_PRODUCT$",0)

\#endif
             
  hrv.SetStrawProduct(aCutProduct)
           

           
 if (i>1)   //is error
               err("Patch:: HandleOp - too many products in GatheringCereal")
     
   
       
  if (hrv.GetOperationId()=GatheringSilage or hrv.GetOperationId()=GatheringHay)
        
       
     AddInd(envInd,"31.22 N removed in cut","kg N",Noutput)
         
   AddInd(envInd,"40.22 C removed in cut","kg C",Coutput)

\#ifdef TUPLE
            
if (((field )Owner).GetWriteTuples())
              
 ((field) Owner).AddTuple(theTime.GetString2(),-Noutput

,$aCutProduct.GetAmount() \cdot 10.0$,"HARVEST","FORAGE/SILAGE",0)

\#endif
            if (hrv.GetOperationId()=GatheringSilage)
           
    aCutProduct.SetCode(1)
            
if (hrv.GetOperationId()=GatheringHay)
          
     aCutProduct.SetCode(2)
      
      hrv.SetForageProduct(aCutProduct,i)
     
   else
     
 theMessage.Warning("patch::HandleOperation - Straw already gathered.")
   cutPlantProducts.Reset()


\subsection{Tillage(soilTreatFields   FOP)}

	if (FOP.GetOperationId()=Ploughing)

   	
KillAllCrops()
	
	soilTreatFields   Tillage=(soilTreatFields)FOP
        Depth = 0
   
   $Depth=10.0 \cdot Tillage.GetDepth()$
     
 if $Depth\le 100$ or $ Depth>500$ then        
 Depth = 250 // Ploughing to 250 mm
   	
  Incorporation = 0.95 // Standard incorporated 95%
      
Tillage.GetfracIncorporated()
     
 if $Incorporation\le 0.001$ then      
     Incorporation = 0.95        
  // Standard incorporated 95\%
      
Incorporation=10 // Otherwise error - correct later !!!
	
  \quad	  aSoil.Incorporate(Depth,Incorporation,true,"Ploughing")
   
   if FOP.GetOperationId()=StubHarrowing then
   
   	// KillAllCrops()  // BMP added 17.11.2006 - and removed 20.11.2006, should be superflouos now!!!
   	
soilTreatFields   Tillage=(soilTreatFields )FOP
       
 Depth = 80    // harrowing to 80 mm
     
 $Depth = 10.0 \cdot Tillage.GetDepth()$
    
  if $Depth\le 10$ or $Depth>200$ then
         Depth = 80  // harrowing to 80 mm
         
   Incorporation = 0.95  // Standard incorporated 95\%
    
    Incorporation = Tillage.GetfracIncorporated()
   
   if $Incorporation\le 0.001$ then
       Incorporation = 0.95        // Standard incorporated 95%
  
   Incorporation=10  // Otherwise error - correct later !!!
     
 aSoil.Incorporate(Depth,Incorporation,true,"Harrowing")
   


\subsection{SetRUEfactor(aVal)}	
theCrops.ElemAtNum(0).SetRUEfactor(aVal)

\subsection{CloverPercentage()}
Calculate how Clover percentage of DM compared with all DM

	 TotalDM = CloverDM = 0
	 
   for ( i=0 i<theCrops.NumOfNodes() i++)
   
    \quad 	  DM = theCrops.ElemAtNum(i).GiveDMVegTop()

     \quad   if (theCrops.ElemAtNum(i).GetCropName()="Clover")
     
 \quad  \quad  	CloverDM += DM
     
  \quad  TotalDM += DM


	frac = 0
	
  if (TotalDM>0) 
  
    \quad frac = CloverDM/TotalDM

	return frac


\subsection{outputCropDetails(fstream  afile)}
    cropNumber = theCrops.NumOfNodes()
   
  afile << "tab" << cropNumber
 
    afile << "tab" << CloverPercentage()

   if (theCrops)

\quad
      cloneList<crop>::PS aCropElement
     
\quad theCrops.PeekHead(aCropElement)

   \quad   for ( i=0 i<theCrops.NumOfNodes() i++)
    
\quad\quad     crop    aCrop = theCrops.ElemAtNum(i)
    
 \quad\quad    aCrop.StreamKeyData(afile)





\subsection{Volatilise(  duration, bool kill)}
\citep{sommer2000modelling}
Volatilise ammonia (only called if manure is present)

NJH Aug 2007 - Relic code but contains some features that we might wish to use later

   return  // Code presently not used in order to get
   
    movin' !!!!!!!!!!!!!!!!!!!!!!!!!

	$  actualNH4\_loss=0$
	
  temperature, $min\_temp$, $max\_temp$, evap, infiltration, ppt, radiation,
                     windspeed, relHumidity, resistance, ra, rb, rCrop
  
 if (soilManure)
  
   	  $Nin = soilManure.GetAllN().n \cdot soilManure.GetAmount() \cdot 10000$ 

// kg N/ha as soilManure is in t/ha
     
$ manure  infiltManure = (manure  ) soilManure.clone()$      
      // Below climate module is called, but not "ready"!
      
      theClimate.GetClimate(temperature, $min\_temp$, $max\_temp$, ppt, radiation, evap, windspeed, relHumidity)
      
      infiltration = 0
      
      ppt /=240  //convert to hourly rate
    
       evap/=240  // calculate infitration rate - this is a fudge!
      
        currentWater = aSoil.GetAvailCapacity(aSoil.GetLayerStartDepth(0),aSoil.GetLayerStartDepth(1))
        
        maxWater = aSoil.GetFieldCapacity(aSoil.GetLayerStartDepth(0),aSoil.GetLayerStartDepth(1))
        
        diff = maxWater-currentWater
     
     $ infiltration = diff  \cdot  soilManure.GiveInfiltrationReduction()/240$

        resistance = 0
        
      if windspeed<0.0001 then windspeed=0.0001
     
     $ rb=6.2 \cdot pow(windspeed,-0.67)$ \citep[Eq. 16]{sommer2000modelling}

     
   l=0.071
    
$  ra=\tfrac{log(\tfrac{l}{0.01})}{0.4 \cdot windspeed}$ 
// - or -   Aerodynamic res.
     
 rCrop=0
    
 resistance=rb+ra+rCrop            
  //rC is the crusts resistance to volatization

 // actualNH4loss is in tonnes per sq metre, as soilManure is in tonnes per ha
    
   waterEvap = 0
     
$ actualNH4\_loss=soilManure.VolatAmmonia(temperature, evap, infiltration, ppt, 10000, $

$resistance, duration, waterEvap, infiltManure)$

        $remainingTAN = soilManure.GetAmount() \cdot soilManure.GetNH4_content().n \cdot 10000$
        
        //amount remaining in kg per ha
        
        $remainTotN=(soilManure.GetAmount() \cdot soilManure.GetAllN().n + infiltManure.GetAmount()$ 

$\cdot infiltManure.GetAllN().n) \cdot 10000$  //in kg per ha
      
  $test = Nin-(remainTotN+actualNH4_loss \cdot 1000)$
  
    if fabs(test)>0.1 then
            err("Volatilise - balance error")

    // If manure contains little TAN then absorb remainder of soilManure. Also absorb if soilManure killed (eg by ploughing)
    
  if $remainingTAN<1$ or $kill$ then
      
      \quad 	 infiltManure +   soilManure
    
      \quad    delete soilManure
    
      \quad    soilManure = NULL
      

      if (infiltManure.GetAmount()>0)
      
       \quad   $decomposable d=(decomposable)  \cdot infiltManure$
       
        \quad  aSoil.AddDecomposable(d)
        
         \quad aSoil.Incorporate(0.01,0.02)  // From NJH @@@ Seems a very small depth !!!???
      
   \quad delete infiltManure
      
   
   
   AddInd(envInd,"31.45 Ammonia-N loss by spreading",
"kg N",$actualNH4_loss  \cdot  area  \cdot 1000$)



\subsection{UpdateInterception(  surplus,
  temperature,  globalRadiation, ePotGL,
 ePotDeadLeaves, ePotSoil,  interceptCapacity,  greenEvap)}
																 Perform addition of water and evaporation from interception layer,

\begin{tabular}{ll}
surplus  & Water to be added to interception layer and resulting drainage from the layer [mm] \\
temperature  & Mean air temperature [\textdegree C] \\
globalRadiation & Global solar radiation [$MJ/m^2$] \\
ePotGL & Potential evapotranspiration from green leaves [mm] \\
ePotDeadLeaves  & Potential evapotranspiration from dead leaves [mm] \\
ePotSoil          & Potential evapotranspiration from soil [mm] \\
interCeptCapacity & Interception capacity [mm]	
\end{tabular}
														 
alculate snow accumulation and melting

   if $surplus>0$ then
   
     \quad     incSnow = 0
   
      \quad  if (temperature<tempSnowFall1)

     \quad      incSnow =   surplus

     \quad   else

      \quad     if (temperature<tempSnowFall2)
  	
\quad     \quad        incSnow = 
$surplus \cdot \tfrac{tempSnowFall2-temperature}{tempSnowFall2-tempSnowFall1}$
   
  \quad   snowContent += incSnow
      
 \quad  surplus -= incSnow
   
   
   
   
   if $snowContent>0$ then
   
     \quad     t = max(temperature,0)
     
       \quad   snowMelt =
$min(snowContent,max(snowMeltRate1 \cdot t+snowMeltRate2        
       \cdot globalRadiation,0))$
    
     \quad   snowContent -= snowMelt
      
      \quad   surplus += snowMelt
   
   
   
   
   waterContent += (surplus)   
   // Calculate evaporation

     ePot =  ePotGL+ePotDeadLeaves+(  ePotSoil)

     snowEvap = evap = 0

   if $ePot>0$ then
   
      snowEvap = min(snowContent,ePot)

      snowContent -= snowEvap

      WaterBudget.AddOutput(snowEvap)

      AddInd(envInd,"70.12 Evapotranspiration","cubic metre",
$snowEvap \cdot area \cdot 10$)

      evap = $min(waterContent,( ePotGL+ePotDeadLeaves)
 \cdot (1-(\tfrac{snowEvap}{ePot})))$
     
  ePotSoil =  $ePotSoil \cdot (1-\tfrac{snowEvap}{ePot})$
      
  ePotGL = $ ePotGL \cdot (1-\tfrac{snowEvap}{ePot})$

      if $evap>0$ then
     
	\quad       $greenEvap = 
evap \cdot \tfrac{ ePotGL}{ePotGL+ePotDeadLeaves}$
     
  \quad    waterContent -= evap
      
 \quad      ePotGL-=  greenEvap
      
   \quad  WaterBudget.AddOutput(evap)
      
 \quad    AddInd(envInd,"70.12 Evapotranspiration",
"cubic metre",$evap \cdot area \cdot 10$)
      
   
  
 if (globalSettings.DetailsData.getWaterDetails()=true)

  \quad	   waterDetails.WriteWithTab(snowEvap+evap)



   surplus=0

   if (waterContent>interceptCapacity)

   \quad    surplus = waterContent-interceptCapacity

  \quad    waterContent = interceptCapacity



\subsection{TransferRootMatter(decomposable  DeadRoot,  RootLengthList)}
   DeadRoot.Setamount($0.01 \cdot DeadRoot.GetAmount()$)     
    // Conversion from g/m2 to t/ha
    
   AddInd(envInd,
"32.20 N from roots incorporated into soil","kg N",

($DeadRoot.GetorgN_content()+DeadRoot.GetNO3_content()+
DeadRoot.GetNH4_content()) \cdot DeadRoot.GetAmount() \cdot area \cdot 1000$)
   
	 AddInd(envInd,
"40.80 C from roots incorporated into soil","kg C",

$DeadRoot.GetAmount() \cdot DeadRoot.GetC_content() \cdot area \cdot 1000$)

   aSoil.AddDecomposable(DeadRoot,RootLengthList)



\section{fracEPot...}

fracEPotToSoil(gLAI,yLAI) = 
    $max(0,1-fracEPotToGL(gLAI)-fracEPotToYL(gLAI,yLAI))$

fracEPotToGL(gLAI) =  $1-exp(-k1 \cdot gLAI)$

fracEPotToYL(gLAI,yLAI) = 
$max(0,1-exp(-k1 \cdot (gLAI+yLAI))-fracEPotToGL(gLAI))$




\section{Nitrogen}

\subsection{NitrogenInSoil()}	
Returns the nitrogen amount in the soil (kg/ha)
     
  return $aSoil.GetTotalNitrogen().n \cdot area \cdot 10$
  
   

\subsection{NitrogenInCrops()}	
Returns the nitrogen amount in the crops (kg/ha)

     ret=0

   if theCrops then
    
  cloneList<crop>::PS aCropElement
   	
theCrops.PeekHead(aCropElement)
	  
  cropNumber = theCrops.NumOfNodes()
  
   	for ( i=0 i<cropNumber i++)
	  
   	
\quad	crop    aCrop = theCrops.ElemAtNum(i)
      
\quad	if $aCrop.GetCropName()\ne "NoCrop"$ then
            ret += aCrop.GiveTotalNitrogen().n
      
  
  return $ ret \cdot area \cdot 10 $
  



\section{Getter}

\subsection{bool GetReadyForHarvestOrLater()}
Demands that the first crop is the main crop

   if theCrops.NumOfNodes()>0

    \quad   return (theCrops.ElemAtNum(0).ReadyForHarvest() 

 \quad \quad or theCrops.ElemAtNum(0).GetTerminated())
  
warn("ReadyForHarvestOrLater - function called but no crops are defined. May hamper the rest of the simulation.")
  
 return false


\subsection{bool GetIrrigationDemand()}
Demands that the first crop is the main crop

   if $theCrops.NumOfNodes()>0$ then
   
   \quad   return theCrops.ElemAtNum(0).IrrigationDemand()
 
   return false


\subsection{GetAbovegroundCropN()}
Get all above ground N, including N in manure, crops etc in kg

	  amountN = 0

   if theCrops then
   
     \quad	for ( i=0 i<theCrops.NumOfNodes() i++)
      
      \quad  \quad     crop  aCrop = theCrops.ElemAtNum(i)
      
        \quad  \quad   amountN += aCrop.NitrogenInVegTop().n
      
   
   return $amountN \cdot 10.0 \cdot area$


\subsection{GetMaxCropHeight()}
	Get max crop height (m)

	  maxHeight = 0

   if theCrops then
   
   \quad  	for ( i=0 i<theCrops.NumOfNodes() i++)
   
    \quad   \quad      crop    aCrop = theCrops.ElemAtNum(i)
   
\quad   \quad        ht = aCrop.GiveCropHeight()
    
\quad  \quad       if $ht>maxHeight$ then
             maxHeight=ht
    
   
   return maxHeight




\subsection{GetFeedResource(feedItem myresource, grazingheight, animalType)}
Called by the animal object to find out what is available
Returns quality and tonnes fresh weight per ha

//     $grazing_height= bite_depth_factor  \cdot  GetMaxCropHeight()$
 
   //calculate the total DM and N etc available within the ecosystem
   bool gotCrop = false
  
 for( i=0 i<theCrops.NumOfNodes() i++)
     
 $feedItem  aCropResource =$

$ theCrops.ElemAtNum(i).GetAvailability(grazing\_height, animalType)$
     
 // load the information into the grazing resource
     
 if $aCropResource\ne NULL$ then
      
  
 \quad       $ \cdot my_resource +  \cdot aCropResource$
  
  \quad      delete aCropResource
   
 \quad      gotCrop = true
      
   

   if not gotCrop then

    \quad       err("ecosystem:GetFeedResource - no grazable crop found")
    
 $FW = my\_resource.GetAmount()/1000 $ // fresh weight, converted to tonnes per ha
   
$my\_resource.Setamount(FW)$  // Is this correctly calculated ???!!!



 \subsection{GetStandingDMdensity()}

Get mass of dry matter of ecosystem ($g/m^2$)
	  
	  mass = 0
	  
   if theCrops then
   
  \quad 	for ( i=0 i<theCrops.NumOfNodes() i++)
   	
   \quad  \quad     crop  aCrop = theCrops.ElemAtNum(i)
         
    \quad \quad      mass += aCrop.GiveStandingDM()

   return mass


\subsection{GetGrazedDM()}
	Get amount grazed from this ecosystem during the current period (kg)
	
     grazedDM=  0
     
   for( i=0 i<theCrops.NumOfNodes() i++)
   
	 \quad	  grazedDM += (theCrops.ElemAtNum(i)).GetGrazedDM()
	 
	return $grazedDM  \cdot  area  \cdot  100$


\subsection{GetDailyDMProduction()}
Returns growth in kg DM

	  ret\_val=0
	  
   for( i=0 i<theCrops.NumOfNodes() i++)

    \quad	  aCrop = theCrops.ElemAtNum(i)
    
   \quad   $ ret\_val+=aCrop.GetDailyDMProduction()$

   $ret\_val  =ret\_val \cdot area  \cdot  100$
   
   return ret\_val

\subsection{GetPotentialGrowthRate(  radiation,   temperature)}
Returns mean potential growth in kg DM/ha/day
Is fairly crude - assumes total interception + even distribution between crops

	$  ret\_val=0$
	  
    numCrops=theCrops.NumOfNodes()
    
   for( i=0 i<numCrops i++)
   
   	 \quad $ crop    aCrop = theCrops.ElemAtNum(i)$
    
     \quad  $ret\_val+=aCrop.PotentialDM(radiation, temperature)  \cdot  100$

   $ret\_val/=numCrops$
   
   return $ret\_val$


\section{Harvest}

\subsection{HarvestOperations(harvestFields   hrv)
}	switch (hrv.GetOperationId())
   
	case  CombineHarvesting:     
  CombineHarvestingShred:
       BeetHarvesting:
     	BeetTopHarvesting:
       BeetTopChopping: HarvestOp(hrv)
			      
     ForageHarvesting:
    Mowing:
       CutSetAside:
      			ForageHarvestOp(hrv)
      			
      GatheringBales:
      GatheringHay:
      GatheringSilage:
     GatheringCereal:
      			Gathering(hrv)
      			break
      default:
      err("HarvestOperations - Operation not found")



\subsection{HarvestOp(harvestFields    hrv)}

    aCrop=NULL

   bool operationPerformed=false
     
      RootLengthList=NULL
   
   if cutPlantProducts.NumOfNodes()>0 then       // Check if straw has not been gathered or ploughed

  \quad   warn("patch::HandleOperation - There should not be straw on field at this point.")

    j=0

   for ( i=0 i<theCrops.NumOfNodes() i++)
   
      aCrop = theCrops.ElemAtNum(i)

      $const char   presCrop = aCrop.PlantItemName.c\_str()$

      if (!(aCrop.IsUndersown()  and  aCrop.GetTerminated()) 

or hrv.GetOperationId()=BeetHarvesting
     
           fracStrawLeft   = theControlParameters.GetStrawLeft()  
// frac of straw biomass left by combine harvesting
         
  fracStorageLeft = theControlParameters.GetHarvestWaste()  

// Covers waste by harvest and higher N-content in residues of reproductional plant parts
      
   

if $fracStorageLeft>0.25$ then
           warn("patch::patch - abnormal high waste by harvest")
      
   if theControlParameters.GetOrganicFarm() then
            fracStorageLeft = 0.25
       
  if $hrv.GetfracStrawHarvested()>0$ then
            fracStrawLeft = 1-hrv.GetfracStrawHarvested()
        
 if $hrv.GetfracGrainHarvested()>0$ then
            fracStorageLeft = 1-hrv.GetfracGrainHarvested()

         plantItem   Storage = new plantItem
         
plantItem   Straw = new plantItem
        
 decomposable   Root = new decomposable

         // Harvest crop
        
   MaxRootDepth = aCrop.GiveRootDepth()
    
             AddInd(economicIndicator,"17.04 Max root depth","m",MaxRootDepth)

      
   aCrop.Harvest(Storage,Straw)
        
 operationPerformed=true

         scenarielog.WriteWithTabS(aCrop.CropName)
      
   scenarielog.WriteWithTabS(theTime.GetString2())
     
    if $Storage.GetAmount()>0$ then

                	 scenarielog.WriteWithTab($Storage.GetAmount() \cdot (1-fracStorageLeft) \cdot Storage.GetDM()$)   
 //MEL 2009: These amounts are in DM
               
 else
               	 	
scenarielog.WriteWithTab(0)
       
         if $Straw.GetAmount() > 0$ then
               	 	
 scenarielog.WriteWithTab($Straw.GetAmount() \cdot (1-fracStrawLeft) \cdot Straw.GetDM()$)          //MEL 2009: These amounts are in DM
               
 else scenarielog.WriteWithTab(0)
         
       if $Storage.GetTotalN() > 0$ then
               	 	
 scenarielog.WriteWithTab($(Storage.GetTotalN() \cdot Storage.GetAmount() \cdot (1-fracStorageLeft)).n$)
              
  else               	 	
scenarielog.WriteWithTab(0)
       
         if $Straw.GetAllN() > 0$ then
               	 	
 scenarielog.WriteWithTab($(Straw.GetAllN() \cdot Straw.GetAmount() \cdot (1-fracStrawLeft)).n$)
                
else               	 	
scenarielog.WriteWithTab(0)
      
   scenarielog.WriteNewLine()
      
   $nitrogen StorageN = Storage.GetTotalN() \cdot Storage.GetAmount()$

       (amounts in patch.cpp are in g/sq metre. Multiply by 10 to get kg, divide by 100 to get tonnes)

          $ N_removed=StorageN.n \cdot (1-fracStorageLeft) \cdot area \cdot 100$
      
   AddInd(economicIndicator,"17.02 Grain yield at economic indicator date","t"

,$Storage.GetAmount() \cdot (1-fracStorageLeft) \cdot \frac{area}{100} $)
       
  AddInd(economicIndicator,"17.03 Straw yield at economic indicator date","t"

,$Straw.GetAmount() \cdot (1-fracStrawLeft) \cdot \frac{area}{100} $)
       
  AddInd(envInd,"31.20 N removed in grain","kg N",$N\_removed$) ;
        
 AddInd(envInd,"40.20 C removed in grain","kg C"

,$Storage.GetAmount() \cdot Storage.GetC_content() \cdot (1-fracStorageLeft) \cdot area \cdot 10$)
      
   AddInd(envInd,"38.10 N15 harvest in grain","kg N"

,$StorageN.n15 \cdot (1-fracStorageLeft) \cdot area \cdot 10$)
        
 AddInd(envInd,"38.11 Total N15 harvest","kg N"

,$StorageN.n15 \cdot (1-fracStorageLeft) \cdot area \cdot 10$)
			
 AddInd(envInd,"39.10 Grain yield at environmental indicator date","t"

,$Storage.GetAmount() \cdot (1-fracStorageLeft) \cdot \frac{area}{100} $)
			
 AddInd(envInd,"39.11 Straw yield at environmental indicator date","t"

,$Straw.GetAmount() \cdot (1-fracStrawLeft) \cdot \frac{area}{100} $)
			
if $i=0$ then
			      			      		
bsTime aDate
			   			
aDate = aCrop.GiveDateOfEmergence(),			      		
 DOY = aDate.GetDayInYear()
			      		
 AddInd(envInd,	 "39.20 Date of emergence of main crop","day of year", (DOY))
			   			
aDate = aCrop.GiveDateOfFlowering(),			      		
DOY = aDate.GetDayInYear()
							
 AddInd(envInd,	 "39.21 Date of flowering of main crop","day of year", (DOY))
			   			
aDate = aCrop.GiveDateOfEndGrainFill(),			      		
DOY = aDate.GetDayInYear()
							
 AddInd(envInd,	 "39.22 Date of end of grain filling of main crop","day of year", (DOY))
						
	aDate = aCrop.GiveDateOfRipeness(),			      		
DOY = aDate.GetDayInYear()
			         	 
AddInd(envInd,	 "39.23 Date of ripeness of main crop","day of year", (DOY))
			      	

\#ifdef TUPLE
        
 if ((field)Owner).GetWriteTuples() then
          
  ((field)Owner).AddTuple(theTime.GetString2(),$-N\_removed$,$Storage.GetAmount() \cdot area \cdot 0.1$,"HARVEST","$MAIN\_PRODUCT$",0)

\#endif
       
  AddYieldToIndicators(presCrop,$(1-fracStorageLeft) \cdot area \cdot \tfrac{Storage.GetAmount()}{100},StorageN.n \cdot (1-fracStorageLeft) \cdot area \cdot 10$)
        
 AddHarvestedAreaToIndicators(presCrop,area)

         // Harvest waste
         
$plantItem  \cdot  StorageWaste = new plantItem(Storage)$
       
  Storage.Setamount($Storage.GetAmount() \cdot (1-fracStorageLeft)$)
      
   StorageWaste.Setamount($StorageWaste.GetAmount() \cdot fracStorageLeft$)
       
  AddInd(envInd,"32.23 N from storage incorporated into soil","kg N"

,$StorageWaste.GetAllN() \cdot StorageWaste.GetAmount() \cdot area \cdot 10$)
       
  AddInd(envInd,"40.83 C from storage incorporated into soil","kg C"

,$StorageWaste.GetAmount() \cdot StorageWaste.GetC_content() \cdot area \cdot 10$)
        
 StorageWaste.Setamount($0.01 \cdot StorageWaste.GetAmount()$)  //convert to t/ha
       
  aSoil.AddDecomposable(StorageWaste)
        
 delete StorageWaste

 plantItem   StrawWaste = new plantItem(Straw)
        
 Straw.Setamount($Straw.GetAmount() \cdot (1-fracStrawLeft)$)
        
 StrawWaste.Setamount($StrawWaste.GetAmount() \cdot fracStrawLeft$)
         
AddInd(envInd,"32.21 N from straw incorporated into soil","kg N"

,$StrawWaste.GetAllN() \cdot StrawWaste.GetAmount() \cdot area \cdot 10$)
      
   AddInd(envInd,"40.81 C from straw incorporated into soil","kg C"

,$StrawWaste.GetAmount() \cdot StrawWaste.GetC_content() \cdot area \cdot 10$)
        
 StrawWaste.Setamount($0.01 \cdot StrawWaste.GetAmount()$)  //convert to t/ha
        
 aSoil.AddDecomposable(StrawWaste)
     
    delete StrawWaste

         if Storage then
        
          
  \quad Storage.Setamount($Storage.GetAmount() \cdot area \cdot 0.01$)           // Convert to tonnes
          
  \quad if j=0 then // and (!hrv.GetMainProduct()))  //NICK CHECK DETTE (MEL)!!
  
 \quad \quad             hrv.SetMainProduct(Storage)
           
  \quad else

   \quad \quad             hrv.AddMainProduct(Storage)

    \quad \quad         j++
        
       
  else
        
   \quad  theMessage.Warning("patch::HandleOperation - Harvested amount should not be zero at this point.")

    \quad      if ((hrv.GetOperationId()=CombineHarvestingShred  or hrv.GetOperationId()=BeetHarvesting))
        
       
    \quad \quad  AddInd(envInd,"32.21 N from straw incorporated into soil","kg N"

,$Straw.GetAllN() \cdot Straw.GetAmount() \cdot area \cdot 10$)
        
   \quad \quad    $C_Amount = Straw.GetAmount() \cdot Straw.GetC_content() 
\cdot area \cdot 100$
         
   \quad \quad AddInd(envInd,"40.81 C from straw incorporated into soil","kg C",$C\_Amount$)
          
  \quad \quad $Straw.Setamount(0.01 \cdot Straw.GetAmount()) $  //convert to t/ha
        
  \quad \quad   aSoil.AddDecomposable(Straw)
       
   \quad \quad   delete Straw
         


        
  \quad else  // Straw is expected to be gathered
       
 
    \quad \quad  \quad        Straw.Setamount($Straw.GetAmount() \cdot 0.01$)   //convert to tonnes/ha
    
\quad \quad   \quad       cutPlantProducts.InsertLast(Straw)
         
      
   if(!aCrop.ContinueGrowth())
              
     \quad	              	plantItem   ExtraStraw = new plantItem
    
 \quad	              	 aCrop.Terminate(ExtraStraw,Root,RootLengthList)
      
  \quad	           	if $ExtraStraw.GetAmount()>0$ then
      
   \quad	  \quad	          		warn("Harvest crop not harvested correctly")
       
 \quad	           	delete ExtraStraw
      
 \quad	            	 TransferRootMatter(Root,RootLengthList)
      
  \quad	           	delete Root,        
          	delete Storage    //NJH March 2009
              
     
      else   
   	 // frac of undersown crop that is cut by ordinary harvest
    
  \quad	     plantItem   Waste = new plantItem()
    
 \quad	      aCrop.Cut(Waste,0.25)
     
\quad	      Waste.Setamount($Waste.GetAmount() \cdot 0.01$)      // Convert to t/ha
    
 \quad	      if $Waste.GetAmount()>0$ then
         
   
 \quad	  \quad	          AddInd(envInd,"32.21 N from straw incorporated into soil","kg N"

,$Waste.GetAllN() \cdot Waste.GetAmount() \cdot area \cdot 1000$)
        
\quad	  \quad	      AddInd(envInd,"40.81 C from straw incorporated into soil","kg C"

,$Waste.GetAmount() \cdot Waste.GetC_content() \cdot area \cdot 1000$)
       
\quad	  \quad	       aSoil.AddDecomposable(Waste)
       
\quad	  \quad	       delete Waste
         
      \quad	     aCrop.SetUndersown(false)
      
   
   if $!operationPerformed$ then

      \quad	  err("HarvestOp - no crops present for harvest in harvest operation")



\subsection{ForageHarvestOp(harvestFields hrv)}

     cuttingHeight= hrv.GetCuttingHeight()
   
if $cuttingHeight>0.5$ then
   
    \quad	   warn("patch::HandleOp - cutting height in forage harvesting is above 0.5 m, cm input assumed!")
  
      \quad	$ cuttingHeight=0.01 \cdot cuttingHeight$
   
 
  if $GetMaxCropHeight()>0$ then               // This is returned in m
   
      crop  c,
	   j=0

	 // $plantItem  \cdot  straw=cutPlantProducts.ElemAtNum(0)$

	//$straw.GetAllN() \cdot straw.GetAmount()$

      for ( i=0 i<theCrops.NumOfNodes() i++)
      
         c = theCrops.ElemAtNum(i)

		   $const char  presCrop = c.PlantItemName.c\_str()$

         if not c.GetTerminated() then
         

         \quad   plantItem   forage = new plantItem()

         \quad   //used to generate indicator output concerning whole harvested material. Should be deleted before exiting this function

          \quad    Wastefrac = 0
             // This is a "harvest" waste, covering a number of different effects. Improve in later versions!

      \quad      c.Cut(forage,cuttingHeight)
          // Cutting height must be given in m

     \quad       if c.ContinueGrowth() then

       \quad\quad         forage.Setamount($forage.GetAmount() \cdot 0.01$)      // Convert to t/ha

            else            
   // This is not a crop that can be cut more than once
            
            	  plantHeight=c.GiveCropHeight()
           
 	if plantHeight>cuttingHeight then
            	
            	 \quad	  frac=1-cuttingHeight/plantHeight
            	 
\quad	Wastefrac=1-frac
            	 
\quad	decomposable  Root = new decomposable
           
  \quad		 forage.Setamount($forage.GetAmount() \cdot 0.01$)
            	
 \quad	plantItem  ExtraStraw = new plantItem
            
 \quad		  
            	 
\quad	 c.Terminate(ExtraStraw,Root,RootLengthList)
            
 \quad		if ExtraStraw.GetAmount()>0 then
          
   \quad \quad			warn("patch::Harvest crop not harvested correctly")
            
 \quad		delete ExtraStraw
            	
 \quad	 TransferRootMatter(Root,RootLengthList)
            
 \quad		delete Root
            	
            	else
            	
            	 \quad	err("patch::Harvest plantHeight is lower that cuttingHeight")
            	

           
         	scenarielog.WriteWithTabS(c.CropName)
         	
 scenarielog.WriteWithTabS(theTime.GetString2())
         	
scenarielog.WriteWithTab(0)   //no storage DM
           
 if forage.GetAmount()>0 then
        	 		
  \quad scenarielog.WriteWithTab($forage.GetAmount() \cdot forage.GetDM()$)
         	
else
        	 	
 \quad	scenarielog.WriteWithTab(0)

            scenarielog.WriteWithTab(0)    //no storage N
         	
if forage.GetTotalN() > 0 then

         \quad	 		 scenarielog.WriteWithTab($(forage.GetTotalN() \cdot forage.GetAmount()).n$)
         	
else
        	
 \quad 		scenarielog.WriteWithTab(0)
         
	scenarielog.WriteNewLine()


         	if forage.GetAmount()>0 then
         	
         	 \quad	plantItem  Waste = new plantItem( forage)
         	 
 \quad     forage.Setamount($forage.GetAmount() \cdot (1-Wastefrac)$)
         	 
\quad      Waste.Setamount($Waste.GetAmount() \cdot Wastefrac$)
         	
 \quad      aSoil.AddDecomposable(Waste)
         	
 \quad      AddInd(envInd,"32.21 N from straw incorporated into soil","kg N"

,$Waste.GetAllN() \cdot Waste.GetAmount() \cdot area \cdot 1000$)
         	
 \quad      AddInd(envInd,"40.81 C from straw incorporated into soil","kg C"

,$Waste.GetAmount() \cdot Waste.GetC_content() \cdot area \cdot 1000$)
         	 
\quad       delete Waste
         	


            if $area\le 0$ then
              warn("Patch::HandleOp - area shold be above zero")

            if hrv.GetOperationId()=CutSetAside then
             
          
 \quad  	if forage.GetAmount()>0 then
                
            \quad        \quad	aSoil.AddDecomposable(forage)
          
 \quad       \quad    AddInd(envInd,"32.21 N from straw incorporated into soil","kg N"

,$forage.GetAllN() \cdot forage.GetAmount() \cdot area \cdot 1000$)
        
\quad           \quad   AddInd(envInd,"40.81 C from straw incorporated into soil","kg C"

,$forage.GetAmount() \cdot forage.GetC_content() \cdot area \cdot 1000$)
      
\quad             \quad  delete forage
                  
            

            else
             // harvest products will be removed either now or later
         
  $ nitrogen forageN = forage.GetTotalN() \cdot forage.GetAmount()$
                  
 modpresCrop=presCrop
                
  modpresCrop.append("WHOLE")
                  
AddYieldToIndicators($modpresCrop.c\_str()$,$area \cdot forage.GetAmount(),forageN.n \cdot area \cdot 1000$)
              
    AddHarvestedAreaToIndicators(presCrop,area)
                
  if hrv.GetOperationId()=ForageHarvesting then
                  
               
   \quad 	 forage.Setamount($forage.GetAmount() \cdot area$)
                 
 \quad 	 $ n=forage.GetAllN().n \cdot forage.GetAmount() \cdot 10000$
                 
 \quad   AddInd(envInd,"31.22 N removed in cut","kg N",n)
               
  \quad    AddInd(envInd,"40.22 C removed in cut","kg C",$forage.GetAmount() \cdot forage.GetC_content() \cdot 1000$)
            	
 \quad	 AddInd(envInd,"39.12 Cut at environmental indicator date","t",forage.GetAmount())
          
  \#ifdef TUPLE
                   
 \quad           if ((field  )Owner).GetWriteTuples() then

      \quad   \quad                              ((field )Owner).AddTuple(theTime.GetString2(),-n

 \quad   \quad  ,$forage.GetAmount() \cdot 10.0$,"HARVEST","FORAGE/SILAGE",0)

            \#endif

                  \quad    if $j>10$ then
                  
  \quad  	 err("HarvestGrasslandOp - too many harvested products")
               
 \quad       forage .Setamount($forage.GetAmount() \cdot area$)
               
 \quad       hrv.SetForageProduct(forage,j)
              
  \quad       j++
             
 \quad         delete forage
                  
                 
 else  //this is a mowing operation and cut material is left on the soil surface
                  

                    
  \quad         cutPlantProducts.InsertLast(forage)

   else

    \quad  warn("patch::HandleOp - crop has zero height at a cutting date")


\section{Budget}

\minisec{StartBudget()}
Initialise budget variables

	aSoil.StartBudget()
	
   if theCrops then
   
     \quad	for ( i=0 i<theCrops.NumOfNodes() i++)
     
	  \quad  \quad   	theCrops.ElemAtNum(i).StartBudget()
	   	
   WaterBudget.SetInput(snowContent+waterContent+aSoil.GetWater()+aSoil.GetSurfaceWater())


\minisec{bool EndBudget()}
Check to see if budget is still valid

	bool ret\_val = true
	
   
   if theCrops then
   
   \quad  	for ( i=0 i<theCrops.NumOfNodes() i++)
	
   \quad  	if !theCrops.ElemAtNum(i).EndBudget(N,DM) then
         
   \quad   \quad   ret\_val=false
 
  aSoil.EndBudget(N,WaterRemain)
   
if !WaterBudget.Balance(WaterRemain+waterContent+snowContent) then
   	ret\_val=false

   if not ret\_val then
      err("EndBudget error")
   
return ret\_val



\section{Indicators}

\subsection{crop   AllocateCrop(char    crop\_id)}

Uses the crop ID  to initiate the correct crop model
modified NJH May 2009, cropName allows variety to be passed via seed name for specific operations


//	Setgrazable(false)

   crop   newCrop=NULL
   
   if (crop\_id="W1" or crop\_id="W2") then
    
      \quad  newCrop= new cropWwheat("cropWwheat",0,this,"WinterWheat")
  
   if (crop\_id="RW") then
    
      \quad  newCrop= new cropWrape("cropWrape",0,this,"WinterRape")
  
  if (crop\_id="RS") then
    
      \quad  newCrop= new cropSrape("cropSrape",0,this,"SpringRape")
   
   if (crop\_id="B1" or crop\_id="B2") then
    
      \quad  newCrop= new cropSbarley("cropSbarley",0,this,"SpringBarley")
   
   if (crop\_id="F1" or crop\_id="G1" or crop\_id="GR" or crop\_id="G4") // or crop\_id="C1")   NJH removed this Nov 2007
  
   
\quad     newCrop= new cropRyegrass("cropRyegrass",0,this,"Ryegrass")
    
\quad    if (theCrops.NumOfNodes()>0)
    
\quad  \quad   		newCrop.SetUndersown(true)
   
 \quad    else
    
 \quad  \quad      newCrop.SetUndersown(false)
  
   
   if (crop\_id="B5" or crop\_id="B6") then
    
      \quad  newCrop= new cropWbarley("cropWbarley",0,this,"WinterBarley")
 
   if (crop\_id="PE") then
    
      \quad  newCrop= new cropPea("cropPea",0,this,"Pea")
  
   if (crop\_id="W5") then
   
     \quad   newCrop= new cropSpringWheat("cropSpringWheat",0,this,"SpringWheat")
   
   if (crop\_id="BE") then
   
     \quad   newCrop= new cropBeet("cropBeet",0,this,"Beets")
   
   if (crop\_id="PO") then
   
      \quad  newCrop= new cropPotato("cropPotato",0,this,"Potato")
  
   if (crop\_id="MA") then

    \quad    newCrop= new cropMaize("cropMaize",0,this,"Maize")
   
   if (crop\_id="NO") then
   
     \quad   newCrop= new cropbare("cropbare",0,this,"NoCrop")
   
   if ((crop\_id="K1") or (crop\_id="C1"))
  
     \quad   newCrop= new cropClover("cropClover",0,this,"Clover")

    \quad    newCrop.SetUndersown(true)  // ???!!!
  
   if (crop\_id="RY") then
      newCrop= new cropRye("cropRye",0,this,"Rye")

   if (crop\_id="O1" or crop\_id="O2")
      newCrop= new cropOat("cropOat",0,this,"Oat")

   if (crop\_id="OR") then
      newCrop= new cropOilRadish("cropOilRadish",0,this,"OilRadish")

   if (crop\_id="CH") then
      newCrop= new cropChickory("cropChickory",0,this,"Chickory")

   if (crop\_id="DW") then
      newCrop= new cropDyersWoad("cropDyersWoad",0,this,"DyersWoad")

   if (crop\_id="HV") then
      newCrop= new cropHairyVetch("cropHairyVetch",0,this,"HairyVetch")

   if crop\_id = "S1") then   
      newCrop= new cropSeedGrass("cropSeedGrass",0,this,"SeedGrass")
      newCrop.SetUndersown(true)
  

   //new crops added NJH May 2009

   if crop\_id = "IR") then
      newCrop= new cropItalRyegrass("cropItalRyegrass",0,this,"ItalianRyegrass")
   
if crop\_id = "SB") then
      newCrop= new cropSoybean("cropSoybean",0,this, "Soybean")
  
if crop\_id = "SF") then
      newCrop= new cropSunflower("cropSunflower",0,this,"Sunflower")
   
if crop\_id = "L1") then   
      newCrop= new cropLucerne("cropLucerne",0,this,"Lucerne")

   //new crops added NJH for compatibility with MEL, July 2009
   
if crop\_id = "CH")
   
    \quad    newCrop= new cropChickory("cropChickory",0,this,"Chickory")
 
    \quad   newCrop.SetUndersown(true)  // ???!!!
   
   if crop\_id = "OR") then
 
   \quad     newCrop= new cropOilRadish("cropOilRadish",0,this,"OilRadish")
    
\quad    newCrop.SetUndersown(true)  // ???!!!
  
   if crop\_id = "DW") then
   
  \quad      newCrop= new cropDyersWoad("cropDyersWoad",0,this,"DyersWoad")
    
\quad    newCrop.SetUndersown(true)
  
   if crop\_id = "HV") then
	
     \quad   newCrop= new cropHairyVetch("cropHairyVetch",0,this,"HairyVetch")
     
 \quad  newCrop.SetUndersown(true)
   
    //
   if crop\_id = "F2" or crop\_id="BP" or crop\_id="C2"
       or crop\_id="B8" or crop\_id="G2" or crop\_id="G3"
       or crop\_id="K2" or crop\_id="W8"
       or crop\_id="B9"  or crop\_id="S2") then
   
  \quad   err("AllocateCrop - crop ",crop\_id," can not be sown directly")
  
 if $newCrop=NULL$ then

 \quad       err("AllocateCrop - crop ",crop\_id," not found")

   newCrop.AssignRootParameters(aSoil)

   return newCrop


\subsection{AddHarvestedAreaToIndicators(const char cid,a)}

Adding the amount a to the indicat file

   if cid="W1" or cid="W2" then
          AddInd(economicIndicator,"18.10 Exp. W.wheat yield (harv. bef. 1/8)","ha",a)
	
	if cid="W5" or cid="W6" then	
       AddInd(economicIndicator,"18.11 Exp. S.wheat yield (harv. bef. 1/8)","ha",a)
 
   if cid="RW" then      
       AddInd(economicIndicator,"18.12 Exp. W.rape yield (harv. bef. 1/8)","ha",a)
  
   if cid="RS" then      
       AddInd(economicIndicator,"18.13 Exp. S.rape yield (harv. bef. 1/8)","ha",a)
  
   if cid="B5" or cid="B6" then   
       AddInd(economicIndicator,"18.14 Exp. W.barley yield (harv. bef. 1/8)","ha",a)
  
   if cid="B1" or cid="B2" then   
       AddInd(economicIndicator,"18.15 Exp. S.barley yield (harv. bef. 1/8)","ha",a)
  
   if cid="RY" then   
       AddInd(economicIndicator,"18.16 Exp. Rye yield (harv. bef. 1/8)","ha",a)
  
   if cid="OA" then   
     AddInd(economicIndicator,"18.17 Exp. Oat yield (harv. bef. 1/8)","ha",a)
  
   if cid="PE" then   
      AddInd(economicIndicator,"18.18 Exp. Pea yield (harv. bef. 1/8)","ha",a)
  
   if cid="BP" then   
	  	 AddInd(economicIndicator,"18.26 Barley/Pea area (harv. bef. 1/9)","ha",a)
 
   if cid="PO" then   
     AddInd(economicIndicator,"18.19 Exp. Potato yield (harv. bef. 1/8)","ha",a)
  
   if cid="SG" then     
      AddInd(economicIndicator,"18.20 Exp. Grass for seed yield (harv. bef. 1/8)","ha",a)
  
   if cid="MA" then   
      AddInd(economicIndicator,"18.21 Exp. Maize yield (harv. bef. 1/8)","ha",a)
  
   if cid="BP" then     
     AddInd(economicIndicator,"18.23 Exp. Barley/Pea yield (harv. bef. 1/8)","ha",a)
  
   if cid="C1" or cid="C2" then    
      AddInd(economicIndicator,"18.24 Exp. Grass/clover yield (harv. bef. 1/8)","ha",a)
 
   if cid="BE" then    
        AddInd(economicIndicator,"18.22 Exp. Beet yield (harv. bef. 1/8)","ha",a)
  
   if cid="SB" then   
       AddInd(economicIndicator,"18.21 Exp. Soy bean yield (harv. bef. 1/8)","ha",a)
 
   if cid="SF" then   
      AddInd(economicIndicator,"18.23 Exp. Sunflower yield (harv. bef. 1/8)","ha",a)








\subsection{GiveIndicators()}
Update the inicat file

   if $balanceDepth<1E-10$ then
   
      \quad  err("GiveIndicators - soil depth not initialised")

  AddStateInd(envInd,"40.60 Total C in soil","kg C",
$aSoil.GetTotalCarbon() \cdot area \cdot 10$)   
   
   //GetPoolCarbon(const char   Name,   startDep,   thick)

	AddStateInd(envInd,"40.62 Total C in soil 0-25 cm",
"kg C",$aSoil.GetTotalCarbon(0,250) \cdot area \cdot 10$)
	 
AddStateInd(envInd,"40.63 Total C in soil 25-50 cm","kg C",

$aSoil.GetTotalCarbon(250,250) \cdot area \cdot 10$)
	 
AddStateInd(envInd,"40.64 Total C in soil 50-75 cm","kg C",

$aSoil.GetTotalCarbon(500,250) \cdot area \cdot 10$)
	
AddStateInd(envInd,"40.65 Total C in soil 75-250 cm","kg C",

$aSoil.GetTotalCarbon(750,1750) \cdot area \cdot 10$)
	 
AddStateInd(envInd,"40.66 C in AOM+SMB 0-25 cm","kg C",
	   	
 ($aSoil.GetPoolCarbon(0,0,250)+aSoil.GetPoolCarbon(1,0,250)+aSoil.GetPoolCarbon(2,0,250)$

$+
	      aSoil.GetPoolCarbon(3,0,250)+aSoil.GetPoolCarbon(4,0,250))
 \cdot area \cdot 10$)
		
AddStateInd(envInd,"40.67 C in AOM+SMB 25-50 cm","kg C",
	   	
 ($aSoil.GetPoolCarbon(0,250,250)+aSoil.GetPoolCarbon(1,250,250)+aSoil.GetPoolCarbon(2,250,250)$

$+aSoil.GetPoolCarbon(3,250,250)+aSoil.GetPoolCarbon(4,250,250)) \cdot area \cdot 10$
)
		
AddStateInd(envInd,"40.68 C in AOM+SMB 50-75 cm","kg C",
	   	
 ($aSoil.GetPoolCarbon(0,500,250)+aSoil.GetPoolCarbon(1,500,250)+aSoil.GetPoolCarbon(2,500,250)$
	      
$+aSoil.GetPoolCarbon(3,500,250)+aSoil.GetPoolCarbon(4,500,250))
 \cdot area \cdot 10$)
		 
AddStateInd(envInd,"40.69 C in AOM+SMB 75-250 cm","kg C",
	   	 
($aSoil.GetPoolCarbon(0,750,1750)+aSoil.GetPoolCarbon(1,750,1750)$

$+aSoil.GetPoolCarbon(2,750,1750)$


$+    
  aSoil.GetPoolCarbon(3,750,1750)+aSoil.GetPoolCarbon(4,750,1750)) 
\cdot area \cdot 10$)
		
AddStateInd(envInd,"40.70 C in NOM 0-25 cm","kg C",
$aSoil.GetPoolCarbon(5,0,250) \cdot area \cdot 10$)
		 
AddStateInd(envInd,"40.71 C in NOM 25-50 cm","kg C",
$aSoil.GetPoolCarbon(5,250,250) \cdot area \cdot 10$)
		 
AddStateInd(envInd,"40.72 C in NOM 50-75 cm","kg C",
$aSoil.GetPoolCarbon(5,500,250) \cdot area \cdot 10$)
		
AddStateInd(envInd,"40.73 C in NOM 75-250 cm","kg C",
$aSoil.GetPoolCarbon(5,750,1750) \cdot area \cdot 10$)
		
AddStateInd(envInd,"40.74 C in IOM 0-25 cm","kg C",
$aSoil.GetPoolCarbon(6,0,250) \cdot area \cdot 10$)
		 
AddStateInd(envInd,"40.75 C in IOM 25-50 cm","kg C",
$aSoil.GetPoolCarbon(6,250,250) \cdot area \cdot 10$)
		 
AddStateInd(envInd,"40.76 C in IOM 50-75 cm","kg C",
$aSoil.GetPoolCarbon(6,500,250) \cdot area \cdot 10$)
		 
AddStateInd(envInd,"40.77 C in IOM 75-250 cm","kg C",
$aSoil.GetPoolCarbon(6,750,1750) \cdot area \cdot 10$)

 AddStateInd(envInd,"40.90 AOM1 C in soil","kg C",
$aSoil.GetPoolCarbon(0,0,balanceDepth) \cdot area \cdot 10$)
 
 AddStateInd(envInd,"40.91 AOM2 C in soil","kg C",
$aSoil.GetPoolCarbon(1,0,balanceDepth) \cdot area \cdot 10$)
  
AddStateInd(envInd,"40.92 SMB1 C in soil","kg C",
$aSoil.GetPoolCarbon(2,0,balanceDepth) \cdot area \cdot 10$)
 
 AddStateInd(envInd,"40.93 SMB2 C in soil","kg C",
$aSoil.GetPoolCarbon(3,0,balanceDepth) \cdot area \cdot 10$)
   
AddStateInd(envInd,"40.94 SMR C in soil","kg C",
$aSoil.GetPoolCarbon(4,0,balanceDepth) \cdot area \cdot 10$)
  
AddStateInd(envInd,"40.95 NOM C in soil","kg C",
$aSoil.GetPoolCarbon(5,0,balanceDepth) \cdot area \cdot 10$)
   
AddStateInd(envInd,"40.96 IOM C in soil","kg C",
$aSoil.GetPoolCarbon(6,0,balanceDepth) \cdot area \cdot 10$)

 
  AddInd(envInd,"40.40 CO2 soil respiration","kg C",
$aSoil.GetCO2Evolution() \cdot area \cdot 10$)

  AddStateInd(envInd,"32.62 AOM1 N in soil","kg N",
$aSoil.GetPoolNitrogen(0,0,balanceDepth) \cdot area \cdot 10$)
   
  AddStateInd(envInd,"32.63 AOM2 N in soil","kg N",
$aSoil.GetPoolNitrogen(1,0,balanceDepth) \cdot area \cdot 10$)
  
  AddStateInd(envInd,"32.64 SMB1 N in soil","kg N",
$aSoil.GetPoolNitrogen(2,0,balanceDepth) \cdot area \cdot 10$)
   
  AddStateInd(envInd,"32.65 SMB2 N in soil","kg N",
$aSoil.GetPoolNitrogen(3,0,balanceDepth) \cdot area \cdot 10$)
   
  AddStateInd(envInd,"32.66 SMR N in soil","kg N",
$aSoil.GetPoolNitrogen(4,0,balanceDepth) \cdot area \cdot 10$)
   
  AddStateInd(envInd,"32.67 NOM N in soil","kg N",
$aSoil.GetPoolNitrogen(5,0,balanceDepth) \cdot area \cdot 10$)
  
  AddStateInd(envInd,"32.68 IOM N in soil","kg N",

$aSoil.GetPoolNitrogen(6,0,balanceDepth) \cdot area \cdot 10$)
  
  AddStateInd(envInd,"32.60 Organic N in soil","kg N",

$(aSoil.GetTotalNitrogen(0,balanceDepth)
-aSoil.GetMinNitrogen(0,balanceDepth)) \cdot area \cdot 10$)

	AddStateInd(envInd,"32.69 Organic N in soil 0-25 cm","kg N",

$(aSoil.GetTotalNitrogen(0,250)-aSoil.GetMinNitrogen(0,250)) \cdot area \cdot 10$)
	
AddStateInd(envInd,"32.70 Mineral N in soil 0-25 cm","kg N",
$aSoil.GetMinNitrogen(0,250) \cdot area \cdot 10$)
	
AddStateInd(envInd,"32.71 Organic N in soil 25-50 cm",

"kg N",$(aSoil.GetTotalNitrogen(250,250)-aSoil.GetMinNitrogen(250,250)) \cdot area \cdot 10$)
	
AddStateInd(envInd,"32.72 Mineral N in soil 25-50 cm",
"kg N",$aSoil.GetMinNitrogen(250,250) \cdot area \cdot 10$)
	
AddStateInd(envInd,"32.73 Organic N in soil 50-75 cm",

"kg N",$(aSoil.GetTotalNitrogen(500,250)-aSoil.GetMinNitrogen(500,250)) \cdot area \cdot 10$)
	
AddStateInd(envInd,"32.74 Mineral N in soil 50-75 cm",
"kg N",$aSoil.GetMinNitrogen(500,250) \cdot area \cdot 10$)
	 
AddStateInd(envInd,"32.75 Organic N in soil 75-250 cm",

"kg N",$(aSoil.GetTotalNitrogen(750,1750)-aSoil.GetMinNitrogen(750,1750)) \cdot area \cdot 10$)
	
AddStateInd(envInd,"32.76 Mineral N in soil 75-250 cm",

"kg N",$aSoil.GetMinNitrogen(750,1750) \cdot area \cdot 10$)
	
AddStateInd(envInd,"32.77 N in AOM+SMB 0-25 cm","kg N",
	   	
 $(aSoil.GetPoolNitrogen(0,0,250)+aSoil.GetPoolNitrogen(1,0,250)
+aSoil.GetPoolNitrogen(2,0,250)$

$+	     
 aSoil.GetPoolNitrogen(3,0,250)+aSoil.GetPoolNitrogen(4,0,250)) 
\cdot area \cdot 10$)
		
AddStateInd(envInd,"32.78 N in AOM+SMB 25-50 cm","kg N",
	   	
 $(aSoil.GetPoolNitrogen(0,250,250)+aSoil.GetPoolNitrogen(1,250,250)
+aSoil.GetPoolNitrogen(2,250,250)+$
	    
 $ aSoil.GetPoolNitrogen(3,250,250)+aSoil.GetPoolNitrogen(4,250,250)) 
\cdot area \cdot 10$)
		
AddStateInd(envInd,"32.79 N in AOM+SMB 50-75 cm","kg N",
	   	
 $(aSoil.GetPoolNitrogen(0,500,250)+aSoil.GetPoolNitrogen(1,500,250)
+aSoil.GetPoolNitrogen(2,500,250)$

$+	     
 aSoil.GetPoolNitrogen(3,500,250)+aSoil.GetPoolNitrogen(4,500,250)) 
\cdot area \cdot 10$)
		
AddStateInd(envInd,"32.80 N in AOM+SMB 75-250 cm","kg N",
	   	 
($aSoil.GetPoolNitrogen(0,750,1750)+aSoil.GetPoolNitrogen(1,750,1750)$

$+aSoil.GetPoolNitrogen(2,750,1750)$
	      
$+aSoil.GetPoolNitrogen(3,750,1750)+aSoil.GetPoolNitrogen(4,750,1750)) $

$\cdot area \cdot 10$)
		 
AddStateInd(envInd,"32.81 N in NOM 0-25 cm","kg N",
$aSoil.GetPoolNitrogen(5,0,250) \cdot area \cdot 10$)
		
AddStateInd(envInd,"32.82 N in NOM 25-50 cm","kg N",
$aSoil.GetPoolNitrogen(5,250,250) \cdot area \cdot 10$)
		
AddStateInd(envInd,"32.90 N in NOM 50-75 cm","kg N",
$aSoil.GetPoolNitrogen(5,500,250) \cdot area \cdot 10$)
		
AddStateInd(envInd,"32.91 N in NOM 75-250 cm","kg N",
$aSoil.GetPoolNitrogen(5,750,1750) \cdot area \cdot 10$)
		
AddStateInd(envInd,"32.92 N in IOM 0-25 cm","kg N",
$aSoil.GetPoolNitrogen(6,0,250) \cdot area \cdot 10$)
		
AddStateInd(envInd,"32.93 N in IOM 25-50 cm","kg N",
$aSoil.GetPoolNitrogen(6,250,250) \cdot area \cdot 10$)
		
AddStateInd(envInd,"32.94 N in IOM 50-75 cm","kg N",
$aSoil.GetPoolNitrogen(6,500,250) \cdot area \cdot 10$)
		 
AddStateInd(envInd,"32.95 N in IOM 75-250 cm","kg N",
$aSoil.GetPoolNitrogen(6,750,1750) \cdot area \cdot 10$)

  AddStateInd(envInd,"32.83 Nmin 0-500","kg N",
$aSoil.GetMinNitrogen(0,500) \cdot area \cdot 10$)
	
AddStateInd(envInd,"32.84 Nmin 0-750","kg N",
$aSoil.GetMinNitrogen(0,750) \cdot area \cdot 10$)
	
AddStateInd(envInd,"32.85 Nmin 0-1000","kg N",
$aSoil.GetMinNitrogen(0,1000) \cdot area \cdot 10$)

  AddStateInd(economicIndicator,"01.01 Total area","ha",area)

   AddInd(envInd,"70.40 Percolation from leaching depth",
"cubic meter",$aSoil.GetDrainage(leachingDepth) \cdot area \cdot 10$)
   
AddInd(envInd,"70.10 Percolation","cubic meter",
$aSoil.GetDrainage(balanceDepth) \cdot area \cdot 10$)
   
AddInd(envInd,"70.11 Surface run-off","cubic meter",
$aSoil.GetRunOff() \cdot area \cdot 10$)

	 
AddInd(envInd,"70.12 Evapotranspiration","cubic meter",
$aSoil.GetEvapotranspiration() \cdot area \cdot 10$)
	 
AddInd(envInd,"70.42 Bare soil evaporation",
"cubic meter",$aSoil.GetSoilEvaporation() \cdot area \cdot 10$)
	
AddStateInd(envInd,"70.15 Water in field to balancedepth",
"cubic meter",$(snowContent+waterContent+aSoil.GetWater(0,balanceDepth)$

$+aSoil.GetSurfaceWater()) \cdot area \cdot 10$)
	
AddStateInd(envInd,"70.41 Water in field to leaching depth","cubic meter",

$(snowContent+waterContent+aSoil.GetWater(0,leachingDepth)+aSoil.GetSurfaceWater()) $

$\cdot area \cdot 10$)
	
AddStateInd(envInd,"70.20 Water in soil","cubic meter",
$waterContent \cdot area \cdot 10$)

   AddInd(envInd,"32.40 Nit N leached from leaching 
depth","kg N",$aSoil.GetNitLeaching(leachingDepth).n \cdot area \cdot 10$)
 
  AddInd(envInd,"32.41 Amm N leached from leaching 
depth","kg N",$aSoil.GetAmmLeaching(leachingDepth).n \cdot area \cdot 10$)
   
AddInd(envInd,"31.40 Nit N leached","kg N"
,$aSoil.GetNitLeaching(balanceDepth).n \cdot area \cdot 10$)
	 
AddInd(envInd,"31.41 Amm N leached","kg N",
$aSoil.GetAmmLeaching(balanceDepth).n \cdot area \cdot 10$)
	
 AddInd(envInd,"31.47 N in surface run-off","kg N"
,$aSoil.GetRunOff_N().n \cdot area \cdot 10$)
	
 AddInd(envInd,"31.42 N2 from denitrification","kg N"
,$aSoil.GetDenitrification().n \cdot area \cdot 10$)
  
 AddInd(envInd,"31.43 N2O from denitrification","kg N"
,$aSoil.GetN2OFromDenitrification().n \cdot area \cdot 10$)
 
  AddInd(envInd,"31.44 N2O from nitrification","kg N"
,$aSoil.GetN2OFromNitrification().n \cdot area \cdot 10$)
	
	 AddInd(envInd,"32.30 Gross N-mineralisation from soil org. matter","kg"

,$aSoil.GetNitrogenNetMineralisation().n \cdot area \cdot 10$)
 
  AddStateInd(envInd,"32.61 Mineral N in soil","kg N"
,$aSoil.GetMinNitrogen().n \cdot area \cdot 10$)

   AddInd(envInd,"38.12 N15 leached from leaching depth","kg N"

,$aSoil.GetNitLeaching(leachingDepth).n15 \cdot area \cdot 10$

$+aSoil.GetAmmLeaching(leachingDepth).n15 \cdot area \cdot 10$)
	
	 AddInd(envInd,"38.21 N15-N2O from denitrification","kg N"
,$aSoil.GetN2OFromDenitrification().n15 \cdot area \cdot 10$)
 
   AddInd(envInd,"38.22 N15-N2O from nitrification","kg N"
,$aSoil.GetN2OFromNitrification().n15 \cdot area \cdot 10$)  

   AddInd(envInd,"38.23 N15-Denitrification","kg N"
,$aSoil.GetDenitrification().n15 \cdot area \cdot 10$)
 	 
AddStateInd(envInd,"38.24 Total soil/soilsurface 15N","kg N"

,$aSoil.GetTotalNitrogen(1000).n15 \cdot area \cdot 10$)

   nitrogen NCrop
     CCrop = 0
  
 if theCrops then
  
    \quad  cloneList<crop>::PS aCropElement

   \quad	theCrops.PeekHead(aCropElement)

  \quad 	while $aCropElement\ne NULL$ do
     
  \quad\quad       NCrop = NCrop + aCropElement.element.GiveTotalNitrogen()
   
\quad\quad      $CCrop += (aCropElement.element.GiveDMStorage()+aCropElement.element.GiveDMVegTop()$

\quad\quad   $+aCropElement.element.GiveDMRoot()) \cdot 0.45$
 
   \quad\quad     theCrops.OneStep(aCropElement)

 
 AddStateInd(envInd,"31.60 N in standing crop/straw","kg N",$NCrop \cdot area \cdot 10$)
 
 AddStateInd(envInd,"31.61 Total soil/soilsurface N","kg N",$aSoil.GetTotalNitrogen(balanceDepth) \cdot area \cdot 10$)
	 
AddStateInd(envInd,"31.99 Total soil/soilsurface nitrogen to 1 m depth","kg N",$aSoil.GetTotalNitrogen(1000) \cdot area \cdot 10$)
   
AddStateInd(envInd,"40.61 C in standing crop/straw","kg C"
,$CCrop \cdot area \cdot 10$)

      if Day=1  and  (Month=9 or Month=11) then
   
        CatchDM = CatchN = 0
        
      cloneList<crop>::PS aCropElement
   	theCrops.PeekHead(aCropElement)
   	
while $aCropElement\ne NULL$ then
      
   \quad            if (aCropElement.element.GetCropName()="Ryegrass" or aCropElement.element.GetCropName()="Clover"

     \quad        	 or  aCropElement.element.GetCropName()="DyersWoad" or aCropElement.element.GetCropName()="OilRadish"

    \quad           or aCropElement.element.GetCropName()="HairyWetch" 

   \quad   or aCropElement.element.GetCropName()="Chickory")
      
   \quad          \quad    	if (aCropElement.element.GetCropName()="OilRadish")
            
    \quad         \quad       \quad  	$CatchDM += aCropElement.element.GiveDMVegTop() \cdot area \cdot 10$
           
     \quad         \quad        \quad  	$CatchN += aCropElement.element.NitrogenInVegTop().n \cdot area \cdot 10$
            
      \quad        \quad      else
            
       \quad         \quad    \quad  	$CatchDM += (aCropElement.element.GiveDMVegTop()+aCropElement.element.GiveDMStorage()) \cdot area \cdot 10$
            
      \quad           \quad   \quad  	$CatchN += (aCropElement.element.NitrogenInVegTop().n+aCropElement.element.NitrogenInStorage().n) \cdot area \cdot 10$
           
        
   \quad   	\quad  		theCrops.OneStep(aCropElement)
      
        	if (theTime.GetMonth()=9  and  theTime.GetDay()=1)
     
    \quad  
   		 AddInd(envInd,"39.01 Catch crop DM (1/9)","kg DM",CatchDM)
	
\quad  		 AddInd(envInd,"39.03 Catch crop N (1/9)","kg N",CatchN)
      
      if (theTime.GetMonth()=11  and  theTime.GetDay()=1)
      
	\quad  		 AddInd(envInd,"39.02 Catch crop DM (1/11)","kg DM",CatchDM)
	
 \quad  		 AddInd(envInd,"39.04 Catch crop N (1/11)","kg N",CatchN)
      
   

   // cutPlantProducts in tonnes/ha
	if (cutPlantProducts)
   
      if (cutPlantProducts.NumOfNodes()>0)
      
        \quad   nitrogen NInStraw
        
\quad   for ( i=0 i<cutPlantProducts.NumOfNodes() i++)
         

         \quad    \quad   plantItem straw = cutPlantProducts.ElemAtNum(i)
     
   \quad    \quad    $NInStraw = NInStraw + straw.GetAllN() \cdot straw.GetAmount() \cdot 10000 // kg N/ha$
         
       \quad   AddStateInd(envInd,"31.60 N in standing crop/straw","kg N",$NInStraw \cdot area$)
      
   



\subsection{AddYieldToIndicators(const char  cid,   y,  n)}

   Adding how much is harvested.

 cid the name of the crop
y the amount of the crop in tons
  n how much N that the crop contains


   if cid="WNWHT"  then 

	\quad	 AddInd(economicIndicator,"17.10 W.wheat yield (harv. bef. 1/8)","t",y)
   	 
\quad AddInd(envInd,"36.00 W.wheat yield (harv. bef. 1/8)","kg N",n)
  
   if cid="SPWHT" then
 	
   \quad 	AddInd(economicIndicator,"17.11 S.wheat yield (harv. bef. 1/8)","t",y)
   	
 \quad AddInd(envInd,"36.01 S.wheat yield (harv. bef. 1/8)","kg N",n)
  
   if cid="WNRPE" then 
	
   	\quad AddInd(economicIndicator,"17.12 W.rape yield (harv. bef. 1/8)","t",y)
   	
 \quad AddInd(envInd,"36.02 W.rape yield (harv. bef. 1/8)","kg N",n)
   
   if cid="SPRAPE" then 
	
   \quad 	AddInd(economicIndicator,"17.13 S.rape yield (harv. bef. 1/8)","t",y)
   	 
\quad AddInd(envInd,"36.03 S.rape yield (harv. bef. 1/8)","kg N",n)
  
   if cid="WNBRL" then 
	
   \quad 	AddInd(economicIndicator,"17.14 W.barley yield (harv. bef. 1/8)","t",y)
   	
 \quad AddInd(envInd,"36.04 W.barley yield (harv. bef. 1/8)","kg N",n)
 
   if cid="BARLEY" then 
	
   \quad 	AddInd(economicIndicator,"17.15 S.barley yield (harv. bef. 1/8)","t",y)
   	
 \quad AddInd(envInd,"36.05 S.barley yield (harv. bef. 1/8)","kg N",n)
  
   if cid="RYE" then 
	
    \quad   AddInd(economicIndicator,"17.16 Rye yield (harv. bef. 1/8)","t",y)
	   
\quad AddInd(envInd,"36.06 Rye yield (harv. bef. 1/8)","kg N",n)
   
   if cid="OAT" then 
   
   \quad 	AddInd(economicIndicator,"17.17 Oat yield (harv. bef. 1/8)","t",y)
   	
 \quad AddInd(envInd,"36.07 Oat yield (harv. bef. 1/8)","kg N",n)
   
   if cid="PEA" then 

   \quad 	AddInd(economicIndicator,"17.18 Pea yield (harv. bef. 1/8)","t",y)
   	
 \quad AddInd(envInd,"36.08 Pea yield (harv. bef. 1/8)","kg N",n)
  
   if cid="POTATO" then 
	
  \quad  	AddInd(economicIndicator,"17.19 Potato yield (harv. bef. 1/8)","t",y)
   	
 \quad AddInd(envInd,"36.09 Potato yield (harv. bef. 1/8)","kg N",n)
  
   if cid="SEEDGRASS" then 
	
   \quad 	AddInd(economicIndicator,"17.20 Grass for seed yield (harv. bef. 1/8)","t",y)
   	
 \quad AddInd(envInd,"36.10 Grass for seed yield (harv. bef. 1/8)","kg N",n)
  
   if cid="MAIZE" then 
	
   \quad 	AddInd(economicIndicator,"17.21 Grain maize yield (harv. bef. 1/8)","t",y)
   	
 \quad AddInd(envInd,"36.11 Grain maize yield (harv. bef. 1/8)","kg N",n)
   
   if cid="BEET" then 
	
   \quad 	AddInd(economicIndicator,"17.22 Beet yield (harv. bef. 1/8)","t",y)
   	
 \quad AddInd(envInd,"36.12 Beet yield (harv. bef. 1/8)","kg N",n)
  
   if cid="RYEGRASS" or cid="RYEGRASSWHOLE" then 
	
   \quad 	AddInd(economicIndicator,"17.23 Ryegrass yield (harv. bef. 1/8)","t",y)
   	
 \quad AddInd(envInd,"36.13 Ryegrass yield (harv. bef. 1/8)","kg N",n)
 
   if cid="CLOVER" then 
	
   \quad 	AddInd(economicIndicator,"17.24 Clover yield (harv. bef. 1/8)","t",y)
   	
\quad  AddInd(envInd,"36.14 Clover yield (harv. bef. 1/8)","kg N",n)
   
   if cid="BP" then 
	
  \quad  	AddInd(economicIndicator,"17.25 Barley/Pea yield (harv. bef. 1/9)","t",y)
   	
\quad  AddInd(envInd,"36.15 Barley/Pea yield (harv. bef. 1/9)","kg N",n)
   
   if cid="MAIZEWHOLE" then 
	
   \quad 	AddInd(economicIndicator,"17.26 Maize wholecrop yield (harv. bef. 1/8)","t",y)
   	
 \quad AddInd(envInd,"36.16 Maize wholecrop yield (harv. bef. 1/8)","kg N",n)
   
   if cid="ITALRYEGWHOLE" then 
	
 \quad   	AddInd(economicIndicator,"17.27 Italian ryegrass yield (harv. bef. 1/8)","t",y)
   	 
\quad AddInd(envInd,"36.17 Italian ryegrass yield (harv. bef. 1/8)","kg N",n)
  
   if cid="SOYBEAN" then 
	
 \quad   	AddInd(economicIndicator,"17.28 Soybean yield (harv. bef. 1/8)","t",y)
   
\quad 	 AddInd(envInd,"36.18 Soybean yield (harv. bef. 1/8)","kg N",n)
   
   if cid="SUNFLOWER" then 
	
  \quad  	AddInd(economicIndicator,"17.29 Sunflower yield (harv. bef. 1/8)","t",y)
   	
 \quad AddInd(envInd,"36.19 Sunflower yield (harv. bef. 1/8)","kg N",n)
  
   if cid="WNWHTWHOLE" or cid="SPWHTWHOLE" then 

  \quad  
		 AddInd(economicIndicator,"17.30 Wheat silage yield (harv. bef. 1/8)","t",y)
   	
\quad  AddInd(envInd,"36.20 Wheat silage yield (harv. bef. 1/8)","kg N",n)
  
   if cid="LUCERNE" or cid="LUCERNEWHOLE" then 
	
   \quad 	AddInd(economicIndicator,"17.31 Lucerne yield (harv. bef. 1/8)","t",y)
   	 
\quad AddInd(envInd,"36.21 Lucerne yield (harv. bef. 1/8)","kg N",n)



\subsection{AddExpectedYieldToIndicators(const char  cid,   y)}

   Adding how much is harvested.
   cid - the name of the crop
  y - the amount of the crop in tons


   if cid="W1" or cid="W2" then	 AddInd(economicIndicator,"18.10 Exp. W.wheat yield (harv. bef. 1/8)","t",y)
	
if cid="W5" or cid="W6" then AddInd(economicIndicator,"18.11 Exp. S.wheat yield (harv. bef. 1/8)","t",y)
   
if cid="RW"  then     AddInd(economicIndicator,"18.12 Exp. W.rape yield (harv. bef. 1/8)","t",y)
   
if cid="RS"  then   AddInd(economicIndicator,"18.13 Exp. S.rape yield (harv. bef. 1/8)","t",y)
 
  if cid="B5" or cid="B6" then  AddInd(economicIndicator,"18.14 Exp. W.barley yield (harv. bef. 1/8)","t",y)
   
if cid="B1" or cid="B2" then  AddInd(economicIndicator,"18.15 Exp. S.barley yield (harv. bef. 1/8)","t",y)
   
if cid="RY"   then      AddInd(economicIndicator,"18.16 Exp. Rye yield (harv. bef. 1/8)","t",y)
   
if cid="OA"   then   AddInd(economicIndicator,"18.17 Exp. Oat yield (harv. bef. 1/8)","t",y)
   
if cid="PE"   then      AddInd(economicIndicator,"18.18 Exp. Pea yield (harv. bef. 1/8)","t",y)
   
if cid="PO"  then    AddInd(economicIndicator,"18.19 Exp. Potato yield (harv. bef. 1/8)","t",y)
   
if cid="SG" then     AddInd(economicIndicator,"18.20 Exp. Grass for seed yield (harv. bef. 1/8)","t",y)
   
if cid="MA"  then    AddInd(economicIndicator,"18.21 Exp. Maize yield (harv. bef. 1/8)","t",y)
  
 if cid="BE"  then     AddInd(economicIndicator,"18.22 Exp. Beet yield (harv. bef. 1/8)","t",y)
   
if cid="BP"    then                      AddInd(economicIndicator,"18.23 Exp. Barley/Pea yield (harv. bef. 1/8)","t",y)
 
  if cid="C1" or cid="C2" then  AddInd(economicIndicator,"18.24 Exp. Grass/clover yield (harv. bef. 1/8)","t",y)
   
if cid="G1" or cid="G2" then  AddInd(economicIndicator,"18.25 Exp. Grass ley yield (harv. bef. 1/8)","ha",y)

\bibliography{literature}
  \bibliographystyle{plainnat}

\end{document}
